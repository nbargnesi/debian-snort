% $Id$
% 
% BUILDING HTML VERSION:
% latex2html -info 0 -local_icons -show_section_numbers -link +2 -split +2 -noaddress snort_manual.tex
%
% BUILDING PDF VERSION:
% pdflatex snort_manual.tex

\documentclass[english]{report}
%\usepackage[T1]{fontenc}
\usepackage[latin1]{inputenc}
\usepackage{geometry}
\usepackage{longtable}
\geometry{verbose,letterpaper,tmargin=1in,bmargin=.5in,lmargin=1in,rmargin=1in}
\usepackage{url}
%\IfFileExists{url.sty}{\usepackage{url}}
%                      {\newcommand{\url}{\texttt}}

\usepackage{html}

% \makeatletter

\newcounter{slistnum}
\newcounter{subslistnum}
\newcounter{subsublistnum}

\newenvironment{slist}
{ \begin{list}{ {\bf \arabic{slistnum}.} }{\usecounter{slistnum} } }
{ \end{list} }

\newenvironment{subslist}
{ \begin{list} { {\bf \arabic{slistnum}-\Alph{subslistnum}. } }
        {\usecounter{subslistnum} }   }
{ \end{list} }

\newenvironment{subsubslist} {
    \begin{list}{
        {\bf \arabic{slistnum}-\arabic{subslistnum}-\arabic{subsublistnum}. }
    }{
        \usecounter{subsubslistnum}
    }
}{
    \end{list}
}

%\begin{latexonly}
\newsavebox{\savepar}
\newenvironment{note}{
\samepage
    \vspace{10pt}{\textsf{
        {\hspace{7pt}\Huge{$\triangle$\hspace{-12.5pt}{\Large{$^!$}}}}\hspace{5pt}
        {\Large{NOTE}}
    }
    }
   \begin{center}
    \par\vspace{-17pt}

    \begin{lrbox}{\savepar}
    \begin{minipage}[r]{6in}
}
{
    \end{minipage}
    \end{lrbox}
    \fbox{
        \usebox{
            \savepar
	}
    }
    \par\vskip10pt
    \end{center}
}
%\end{latexonly}

\begin{htmlonly}
\newenvironment{note}{
        \begin{rawhtml}
        <p><table border="1"><tr><td><b>
        Note:&nbsp;&nbsp;</b>
        \end{rawhtml}
}{
        \begin{rawhtml}
        </b></td></tr></table></p>
        \end{rawhtml}
}
\end{htmlonly}

\usepackage{babel}

% \makeatother

\addtolength{\parindent}{-5mm}
\addtolength{\parskip}{2mm}

%\renewcommand\floatpagefraction{.9}
%\renewcommand\topfraction{.9}
%\renewcommand\bottomfraction{.9}
%\renewcommand\textfraction{.1}   
%\setcounter{totalnumber}{50}
%\setcounter{topnumber}{50}
%\setcounter{bottomnumber}{50}

\begin{document}

\title{SNORT\textsuperscript{\textregistered} Users Manual\\2.9.7.3}

\author{The Snort Project}

\maketitle

\newpage

Copyright \copyright 1998-2003 Martin Roesch

Copyright \copyright 2001-2003 Chris Green

Copyright \copyright 2003-2013 Sourcefire, Inc.

Copyright \copyright 2014-2015 Cisco and/or its affiliates. All rights reserved.

\tableofcontents{}

\chapter{Snort Overview}

This manual is based on \emph{Writing Snort Rules} by Martin Roesch and further
work from Chris Green $<$cmg@snort.org$>$.  It was then maintained by Brian
Caswell $<$bmc@snort.org$>$ and now is maintained by the Snort Team.  If you
have a better way to say something or find that something in the documentation
is outdated, drop us a line and we will update it.  If you would like to submit
patches for this document, you can find the latest version of the documentation
in \LaTeX\ format in the most recent source tarball under
\verb!/doc/snort_manual.tex!.  Small documentation updates are the easiest way
to help out the Snort Project.

\section{Getting Started}

Snort really isn't very hard to use, but there are a lot of command line
options to play with, and it's not always obvious which ones go together well.
This file aims to make using Snort easier for new users.

Before we proceed, there are a few basic concepts you should understand about
Snort. Snort can be configured to run in three modes:

\begin{itemize}

\item {\em Sniffer mode,} which simply reads the packets off of the network and
displays them for you in a continuous stream on the console (screen). 

\item {\em Packet Logger mode,} which logs the packets to disk. 

\item {\em Network Intrusion Detection System (NIDS) mode,} which performs
detection and analysis on network traffic. This is the most complex and
configurable mode.

\end{itemize}

\section{Sniffer Mode}

First, let's start with the basics. If you just want to print out the TCP/IP
packet headers to the screen (i.e. sniffer mode), try this:

\begin{verbatim}
    ./snort -v
\end{verbatim}

This command will run Snort and just show the IP and TCP/UDP/ICMP headers,
nothing else. If you want to see the application data in transit, try the
following:

\begin{verbatim}
    ./snort -vd
\end{verbatim}

This instructs Snort to display the packet data as well as the headers.  If you
want an even more descriptive display, showing the data link layer headers, do
this:

\begin{verbatim}
    ./snort -vde
\end{verbatim}

As an aside, notice that the command line switches can be listed separately or
in a combined form. The last command could also be typed out as:

\begin{verbatim}
    ./snort -d -v -e
\end{verbatim}

to produce the same result.

\section{Packet Logger Mode}

OK, all of these commands are pretty cool, but if you want to record the
packets to the disk, you need to specify a logging directory and Snort will
automatically know to go into packet logger mode:

\begin{verbatim}
    ./snort -dev -l ./log
\end{verbatim}

Of course, this assumes you have a directory named \verb!log!  in the current
directory. If you don't, Snort will exit with an error message. When Snort runs
in this mode, it collects every packet it sees and places it in a directory
hierarchy based upon the IP address of one of the hosts in the datagram.

If you just specify a plain -l switch, you may notice that Snort sometimes uses
the address of the remote computer as the directory in which it places packets
and sometimes it uses the local host address. In order to log relative to the
home network, you need to tell Snort which network is the home network:

\begin{verbatim}
    ./snort -dev -l ./log -h 192.168.1.0/24
\end{verbatim}

This rule tells Snort that you want to print out the data link and TCP/IP
headers as well as application data into the directory \verb!./log!, and you
want to log the packets relative to the 192.168.1.0 class C network. All
incoming packets will be recorded into subdirectories of the log directory,
with the directory names being based on the address of the remote
(non-192.168.1) host. 

\begin{note}

Note that if both the source and destination hosts are on the home network,
they are logged to a directory with a name based on the higher of the two port
numbers or, in the case of a tie, the source address.

\end{note}

If you're on a high speed network or you want to log the packets into a more
compact form for later analysis, you should consider logging in binary mode.
Binary mode logs the packets in tcpdump format to a single binary file in the
logging directory:

\begin{verbatim}
    ./snort -l ./log -b
\end{verbatim}

Note the command line changes here. We don't need to specify a home network any
longer because binary mode logs everything into a single file, which eliminates
the need to tell it how to format the output directory structure. Additionally,
you don't need to run in verbose mode or specify the -d or -e switches because
in binary mode the entire packet is logged, not just sections of it. All you
really need to do to place Snort into logger mode is to specify a logging
directory at the command line using the -l switch---the -b binary logging
switch merely provides a modifier that tells Snort to log the packets in
something other than the default output format of plain ASCII text.

Once the packets have been logged to the binary file, you can read the packets
back out of the file with any sniffer that supports the tcpdump binary format
(such as tcpdump or Ethereal). Snort can also read the packets back by using
the -r switch, which puts it into playback mode. Packets from any tcpdump
formatted file can be processed through Snort in any of its run modes. For
example, if you wanted to run a binary log file through Snort in sniffer mode
to dump the packets to the screen, you can try something like this:

\begin{verbatim}
    ./snort -dv -r packet.log
\end{verbatim}

You can manipulate the data in the file in a number of ways through Snort's
packet logging and intrusion detection modes, as well as with the BPF interface
that's available from the command line. For example, if you only wanted to see
the ICMP packets from the log file, simply specify a BPF filter at the command
line and Snort will only see the ICMP packets in the file:

\begin{verbatim}
    ./snort -dvr packet.log icmp 
\end{verbatim}

For more info on how to use the BPF interface, read the Snort and tcpdump man
pages.

\section{Network Intrusion Detection System Mode}

To enable Network Intrusion Detection System (NIDS) mode so that you don't
record every single packet sent down the wire, try this:

\begin{verbatim}
    ./snort -dev -l ./log -h 192.168.1.0/24 -c snort.conf
\end{verbatim}

where \texttt{snort.conf} is the name of your snort configuration file. This will
apply the rules configured in the \verb!snort.conf! file to each packet to decide
if an action based upon the rule type in the file should be taken. If you don't
specify an output directory for the program, it will default to
\verb!/var/log/snort!.

One thing to note about the last command line is that if Snort is going to be
used in a long term way as an IDS, the -v switch should be left off the command
line for the sake of speed.  The screen is a slow place to write data to, and
packets can be dropped while writing to the display.

It's also not necessary to record the data link headers for most applications,
so you can usually omit the -e switch, too.

\begin{verbatim}
    ./snort -d -h 192.168.1.0/24 -l ./log -c snort.conf
\end{verbatim}

This will configure Snort to run in its most basic NIDS form, logging packets
that trigger rules specified in the \texttt{snort.conf} in plain ASCII to disk
using a hierarchical directory structure (just like packet logger mode). 


\subsection{NIDS Mode Output Options}

There are a number of ways to configure the output of Snort in NIDS mode. The
default logging and alerting mechanisms are to log in decoded ASCII format and
use full alerts. The full alert mechanism prints out the alert message in
addition to the full packet headers. There are several other alert output modes
available at the command line, as well as two logging facilities.

Alert modes are somewhat more complex. There are seven alert modes available at
the command line: full, fast, socket, syslog, console, cmg, and none. Six of
these modes are accessed with the -A command line switch.  These options are:

\begin{center}
\begin{tabular}{| l | p{5.4in} |}

\hline
{\bf Option} & {\bf Description}\\
\hline

\hline
{\tt -A fast} &

Fast alert mode. Writes the alert in a simple format with a timestamp, alert
message, source and destination IPs/ports.\\

\hline
{\tt -A full} &

Full alert mode. This is the default alert mode and will be used automatically
if you do not specify a mode.\\

\hline
{\tt -A unsock} &

Sends alerts to a UNIX socket that another program can listen on.\\

\hline
{\tt -A none} &

Turns off alerting.\\

\hline
{\tt -A console} &

Sends ``fast-style'' alerts to the console (screen).\\

\hline
{\tt -A cmg} &

Generates ``cmg style'' alerts.\\

\hline
\end{tabular}
\end{center}

Packets can be logged to their default decoded ASCII format or to a binary log
file via the -b command line switch. To disable packet logging altogether, use
the -N command line switch.

For output modes available through the configuration file, see Section
\ref{output config}.

\begin{note}

Command line logging options override any output options specified in the
configuration file. This allows debugging of configuration issues quickly via
the command line.

\end{note}

To send alerts to syslog, use the -s switch. The default facilities for the
syslog alerting mechanism are LOG\_AUTHPRIV and LOG\_ALERT.  If you want to
configure other facilities for syslog output, use the output plugin directives
in snort.conf. See Section \ref{alert syslog label} for more details on
configuring syslog output.

For example, use the following command line to log to default (decoded ASCII)
facility and send alerts to syslog: 

\begin{verbatim}
    ./snort -c snort.conf -l ./log -h 192.168.1.0/24 -s
\end{verbatim}

As another example, use the following command line to log to the default
facility in /var/log/snort and send alerts to a fast alert file: 

\begin{verbatim}
    ./snort -c snort.conf -A fast -h 192.168.1.0/24
\end{verbatim}

\subsection{Understanding Standard Alert Output}

When Snort generates an alert message, it will usually look like the following:
 
\begin{verbatim}	 
    [**] [116:56:1] (snort_decoder): T/TCP Detected [**]
\end{verbatim}
  	 
The first number is the Generator ID, this tells the user what component of
Snort generated this alert. For a list of GIDs, please read etc/generators in
the Snort source. In this case, we know that this event came from the
``decode'' (116) component of Snort.
  	 
The second number is the Snort ID (sometimes referred to as Signature ID).
For a list of preprocessor SIDs, please see etc/gen-msg.map.  Rule-based SIDs
are written directly into the rules with the \emph{sid} option. In this case,
\emph{56} represents a T/TCP event.
  	 
The third number is the revision ID. This number is primarily used when
writing signatures, as each rendition of the rule should increment this number
with the \emph{rev} option.

\subsection{High Performance Configuration}

If you want Snort to go \emph{fast} (like keep up with a 1000 Mbps connection),
you need to use unified2 logging and a unified2 log reader such as
\emph{barnyard2}.  This allows Snort to log alerts in a binary form as fast as
possible while another program performs the slow actions, such as writing to a
database.

If you want a text file that's easily parsed, but still somewhat fast, try
using binary logging with the ``fast'' output mechanism. 

This will log packets in tcpdump format and produce minimal alerts. For
example:

\begin{verbatim}
    ./snort -b -A fast -c snort.conf
\end{verbatim}

\subsection{Changing Alert Order}

The default way in which Snort applies its rules to packets may not be
appropriate for all installations.  The Pass rules are applied first, then the
Drop rules, then the Alert rules and finally, Log rules are applied. 

\begin{note}
Sometimes an errant pass rule could cause alerts to not show up, in
which case you can change the default ordering to allow Alert rules
to be applied before Pass rules.  For more information, please refer
to the \texttt{--alert-before-pass} option.
\end{note}

Several command line options are available to change the order in
which rule actions are taken.

\begin{itemize}

\item \texttt{--alert-before-pass} option forces alert rules to take
affect in favor of a pass rule.

\item \texttt{--treat-drop-as-alert} causes drop and reject rules and 
any associated alerts to be logged as alerts, rather then the normal 
action.  This allows use of an inline policy with passive/IDS mode.
The sdrop rules are not loaded.

\item \texttt{--process-all-events} option causes Snort to process
every event associated with a packet, while taking the actions based
on the rules ordering.  Without this option (default case), only the
events for the first action based on rules ordering are processed.

\end{itemize}

\begin{note}

Pass rules are special cases here, in that the event processing is terminated
when a pass rule is encountered, regardless of the use of
\texttt{--process-all-events}.

\end{note}

\section{Packet Acquisition}

Snort 2.9 introduces the DAQ, or Data Acquisition library, for packet I/O.  The
DAQ replaces direct calls to libpcap functions with an abstraction layer that
facilitates operation on a variety of hardware and software interfaces without
requiring changes to Snort.  It is possible to select the DAQ type and mode
when invoking Snort to perform pcap readback or inline operation, etc.

\begin{note}

Some network cards have features which can affect Snort. Two of these features
are named "Large Receive Offload" (lro) and "Generic Receive Offload" (gro). 
With these features enabled, the network card performs packet reassembly 
before they're processed by the kernel.

By default, Snort will truncate packets larger than the default snaplen of 1518
bytes. In addition, LRO and GRO may cause issues with Stream target-based
reassembly. We recommend that you turn off LRO and GRO. On linux systems, you can run:

\begin{verbatim}
    $ ethtool -K eth1 gro off
    $ ethtool -K eth1 lro off
\end{verbatim}

\end{note}

\subsection{Configuration}

Assuming that you did not disable static modules or change the default DAQ
type, you can run Snort just as you always did for file readback or sniffing an
interface.  However, you can select and configure the DAQ when Snort is invoked
as follows:

\begin{verbatim}
    ./snort \
        [--daq <type>] \
        [--daq-mode <mode>] \
        [--daq-dir <dir>] \
        [--daq-var <var>]

    config daq: <type>
    config daq_dir: <dir>
    config daq_var: <var>
    config daq_mode: <mode>

    <type> ::= pcap | afpacket | dump | nfq | ipq | ipfw
    <mode> ::= read-file | passive | inline
    <var> ::= arbitrary <name>=<value> passed to DAQ
    <dir> ::= path where to look for DAQ module so's
\end{verbatim}

The DAQ type, mode, variable, and directory may be specified either via the 
command line or in the conf file.  You may include as many variables and 
directories as needed by repeating the arg / config.  DAQ type may be specified
at most once in the conf and once on the command line; if configured in both
places, the command line overrides the conf.


If the mode is not set explicitly, -Q will force it to inline, and if that
hasn't been set, -r will force it to read-file, and if that hasn't been set,
the mode defaults to passive.  Also, -Q and --daq-mode inline are allowed,
since there is no conflict, but -Q and any other DAQ mode will cause a fatal
error at start-up.

Note that if Snort finds multiple versions of a given library, the most recent
version is selected.  This applies to static and dynamic versions of the same
library.

\begin{verbatim}
    ./snort --daq-list[=<dir>]
    ./snort --daq-dir=<dir> --daq-list
\end{verbatim}

The above commands search the specified directories for DAQ modules and print
type, version, and attributes of each.  This feature is not available in the
conf.  Snort stops processing after parsing --daq-list so if you want to add
one or more directories add --daq-dir options before --daq-list on the command
line.  (Since the directory is optional to --daq-list, you must use an =
without spaces for this option.)

\subsection{pcap}

pcap is the default DAQ.  if snort is run w/o any DAQ arguments, it will
operate as it always did using this module.  These are equivalent:

\begin{verbatim}
    ./snort -i <device>
    ./snort -r <file>

    ./snort --daq pcap --daq-mode passive -i <device>
    ./snort --daq pcap --daq-mode read-file -r <file>
\end{verbatim}

You can specify the buffer size pcap uses with:

\begin{verbatim}
    ./snort --daq pcap --daq-var buffer_size=<#bytes>
\end{verbatim}

Note that the pcap DAQ does not count filtered packets.

\subsection{AFPACKET}

afpacket functions similar to the memory mapped pcap DAQ but no external 
library is required:

\begin{verbatim}
    ./snort --daq afpacket -i <device>
            [--daq-var buffer_size_mb=<#MB>]
            [--daq-var debug]
\end{verbatim}

If you want to run afpacket in inline mode, you must set device to one or more
interface pairs, where each member of a pair is separated by a single colon and 
each pair is separated by a double colon like this:

\begin{verbatim}
    eth0:eth1
\end{verbatim}

or this:

\begin{verbatim}
    eth0:eth1::eth2:eth3
\end{verbatim}

By default, the afpacket DAQ allocates 128MB for packet memory.  You can change
this with:

\begin{verbatim}
    --daq-var buffer_size_mb=<#MB>
\end{verbatim}

Note that the total allocated is actually higher, here's why.  Assuming the
default packet memory with a snaplen of 1518, the numbers break down like this:

\begin{slist}
\item
The frame size is 1518 (snaplen) + the size of the AFPacket header (66
bytes) = 1584 bytes.

\item
The number of frames is 128 MB / 1518 = 84733.

\item
The smallest block size that can fit at least one frame is  4 KB = 4096 bytes
  @ 2 frames per block.

\item
As a result, we need 84733 / 2 = 42366 blocks.

\item
Actual memory allocated is 42366 * 4 KB = 165.5 MB.
\end{slist}

\subsection{NFQ}

NFQ is the new and improved way to process iptables packets:

\begin{verbatim}
    ./snort --daq nfq \
        [--daq-var device=<dev>] \
        [--daq-var proto=<proto>] \
        [--daq-var queue=<qid>] \
	[--daq-var queue_len=<qlen>]

    <dev> ::= ip | eth0, etc; default is IP injection
    <proto> ::= ip4 | ip6 | ip*; default is ip4
    <qid> ::= 0..65535; default is 0
    <qlen> ::= 0..65535; default is 0
\end{verbatim}

Notes on iptables can be found in the DAQ distro README.

\subsection{IPQ}

IPQ is the old way to process iptables packets.  It replaces the inline version
available in pre-2.9 versions built with this:

\begin{verbatim}
    ./configure --enable-inline / -DGIDS
\end{verbatim}

Start the IPQ DAQ as follows:

\begin{verbatim}
    ./snort --daq ipq \
        [--daq-var device=<dev>] \
        [--daq-var proto=<proto>] \

    <dev> ::= ip | eth0, etc; default is IP injection
    <proto> ::= ip4 | ip6; default is ip4
\end{verbatim}

\subsection{IPFW}

IPFW is available for BSD systems.  It replaces the inline version available in
pre-2.9 versions built with this:

\begin{verbatim}
    ./configure --enable-ipfw / -DGIDS -DIPFW
\end{verbatim}

This command line argument is no longer supported:

\begin{verbatim}
    ./snort -J <port#>
\end{verbatim}

Instead, start Snort like this:

\begin{verbatim}
    ./snort --daq ipfw [--daq-var port=<port>]

    <port> ::= 1..65535; default is 8000
\end{verbatim}

* IPFW only supports ip4 traffic.

\subsection{Dump}

The dump DAQ allows you to test the various inline mode features available in
2.9 Snort like injection and normalization.

\begin{verbatim}
    ./snort -i <device> --daq dump
    ./snort -r <pcap> --daq dump
\end{verbatim}

By default a file named inline-out.pcap will be created containing all packets
that passed through or were generated by snort.  You can optionally specify a
different name.

\begin{verbatim}
    ./snort --daq dump --daq-var file=<name>
\end{verbatim}

dump uses the pcap daq for packet acquisition.  It therefore does not count
filtered packets.

Note that the dump DAQ inline mode is not an actual inline mode.  Furthermore,
you will probably want to have the pcap DAQ acquire in another mode like this:

\begin{verbatim}
    ./snort -r <pcap> -Q --daq dump --daq-var load-mode=read-file
    ./snort -i <device> -Q --daq dump --daq-var load-mode=passive
\end{verbatim}

\subsection{Statistics Changes}

The Packet Wire Totals and Action Stats sections of Snort's output include
additional fields:

\begin{itemize}
\item \texttt{Filtered}
count of packets filtered out and not handed to Snort for analysis.

\item \texttt{Injected}
packets Snort generated and sent, e.g. TCP resets.

\item \texttt{Allow}
packets Snort analyzed and did not take action on.

\item \texttt{Block}
packets Snort did not forward, e.g. due to a block rule.

\item \texttt{Replace}
packets Snort modified.

\item \texttt{Whitelist}
packets that caused Snort to allow a flow to pass w/o inspection by any
analysis program.

\item \texttt{Blacklist}
packets that caused Snort to block a flow from passing.

\item \texttt{Ignore}
packets that caused Snort to allow a flow to pass w/o inspection by this
instance of Snort.
\end{itemize}

The action stats show "blocked" packets instead of "dropped" packets to avoid
confusion between dropped packets (those Snort didn't actually see) and blocked
packets (those Snort did not allow to pass).

\section{Reading pcap files}

Instead of having Snort listen on an interface, you can give it a packet
capture to read.  Snort will read and analyze the packets as if they came off
the wire.  This can be useful for testing and debugging Snort.

\subsection{Command line arguments}

Any of the below can be specified multiple times on the command line
(\texttt{-r} included) and in addition to other Snort command line options.
Note, however, that specifying \texttt{--pcap-reset} and \texttt{--pcap-show}
multiple times has the same effect as specifying them once.

\begin{center}
\begin{tabular}{| l | p{4.5in} |}

\hline
\textbf{Option} & \textbf{Description}\\
\hline 

\hline 
\texttt{-r <file>} &

Read a single pcap. \\

\hline
\texttt{--pcap-single=<file>} &

Same as -r.  Added for completeness. \\

\hline
\texttt{--pcap-file=<file>} &

File that contains a list of pcap files to read.  Can specify path to each
pcap or directory to recurse to get pcaps. \\

\hline
\texttt{--pcap-list="<list>"} &

A space separated list of pcaps to read. \\

\hline
\texttt{--pcap-dir=<dir>} &

A directory to recurse to look for pcaps.  Sorted in ASCII order. \\

\hline
\texttt{--pcap-filter=<filter>} &

Shell style filter to apply when getting pcaps from file or directory.  This
filter will apply to any \texttt{--pcap-file} or \texttt{--pcap-dir} arguments
following.  Use \texttt{--pcap-no-filter} to delete filter for following
\texttt{--pcap-file} or \texttt{--pcap-dir} arguments or specify
\texttt{--pcap-filter} again to forget previous filter and to apply to
following \texttt{--pcap-file} or \texttt{--pcap-dir} arguments. \\

\hline
\texttt{--pcap-no-filter} &

Reset to use no filter when getting pcaps from file or directory. \\

\hline
\texttt{--pcap-reset} &

If reading multiple pcaps, reset snort to post-configuration state before
reading next pcap.  The default, i.e. without this option, is not to reset
state. \\

\hline
\texttt{--pcap-show} &

Print a line saying what pcap is currently being read. \\

\hline
\end{tabular}
\end{center}

\subsection{Examples}

\subsubsection{Read a single pcap}

\begin{verbatim}
    $ snort -r foo.pcap
    $ snort --pcap-single=foo.pcap
\end{verbatim}

\subsubsection{Read pcaps from a file}

\begin{verbatim}
    $ cat foo.txt
    foo1.pcap
    foo2.pcap
    /home/foo/pcaps
\end{verbatim}

\begin{verbatim}
    $ snort --pcap-file=foo.txt
\end{verbatim}

This will read foo1.pcap, foo2.pcap and all files under /home/foo/pcaps.  Note
that Snort will not try to determine whether the files under that directory are
really pcap files or not.

\subsubsection{Read pcaps from a command line list}

\begin{verbatim}
    $ snort --pcap-list="foo1.pcap foo2.pcap foo3.pcap"
\end{verbatim}

This will read foo1.pcap, foo2.pcap and foo3.pcap.

\subsubsection{Read pcaps under a directory}

\begin{verbatim}
    $ snort --pcap-dir="/home/foo/pcaps"
\end{verbatim}

This will include all of the files under /home/foo/pcaps.

\subsubsection{Using filters}

\begin{verbatim}
    $ cat foo.txt
    foo1.pcap
    foo2.pcap
    /home/foo/pcaps
\end{verbatim}

\begin{verbatim}
    $ snort --pcap-filter="*.pcap" --pcap-file=foo.txt
    $ snort --pcap-filter="*.pcap" --pcap-dir=/home/foo/pcaps
\end{verbatim}

The above will only include files that match the shell pattern "*.pcap", in
other words, any file ending in ".pcap".

\begin{verbatim}
    $ snort --pcap-filter="*.pcap --pcap-file=foo.txt \
    > --pcap-filter="*.cap" --pcap-dir=/home/foo/pcaps
\end{verbatim}

In the above, the first filter "*.pcap" will only be applied to the pcaps in
the file "foo.txt" (and any directories that are recursed in that file).  The
addition of the second filter "*.cap" will cause the first filter to be
forgotten and then applied to the directory /home/foo/pcaps, so only files
ending in ".cap" will be included from that directory.

\begin{verbatim}
    $ snort --pcap-filter="*.pcap --pcap-file=foo.txt \
    > --pcap-no-filter --pcap-dir=/home/foo/pcaps
\end{verbatim}

In this example, the first filter will be applied to foo.txt, then no filter
will be applied to the files found under /home/foo/pcaps, so all files found
under /home/foo/pcaps will be included. 

\begin{verbatim}
    $ snort --pcap-filter="*.pcap --pcap-file=foo.txt \
    > --pcap-no-filter --pcap-dir=/home/foo/pcaps \
    > --pcap-filter="*.cap" --pcap-dir=/home/foo/pcaps2
\end{verbatim}

In this example, the first filter will be applied to foo.txt, then no filter
will be applied to the files found under /home/foo/pcaps, so all files found
under /home/foo/pcaps will be included, then the filter "*.cap" will be applied
to files found under /home/foo/pcaps2. 

\subsubsection{Resetting state}

\begin{verbatim}
    $ snort --pcap-dir=/home/foo/pcaps --pcap-reset
\end{verbatim}

The above example will read all of the files under /home/foo/pcaps, but after
each pcap is read, Snort will be reset to a post-configuration state, meaning
all buffers will be flushed, statistics reset, etc.  For each pcap, it will be
like Snort is seeing traffic for the first time.

\subsubsection{Printing the pcap}

\begin{verbatim}
    $ snort --pcap-dir=/home/foo/pcaps --pcap-show
\end{verbatim}

The above example will read all of the files under /home/foo/pcaps and will
print a line indicating which pcap is currently being read.

\section{Basic Output}

Snort does a lot of work and outputs some useful statistics when it is done.
Many of these are self-explanatory.  The others are summarized below.  This
does not include all possible output data, just the basics.

\subsection{Timing Statistics}

This section provides basic timing statistics.  It includes total seconds and
packets as well as packet processing rates.  The rates are based on whole
seconds, minutes, etc. and only shown when non-zero.

Example:

\begin{verbatim}
===============================================================================
Run time for packet processing was 175.856509 seconds
Snort processed 3716022 packets.
Snort ran for 0 days 0 hours 2 minutes 55 seconds
   Pkts/min:      1858011
   Pkts/sec:        21234
===============================================================================
\end{verbatim}

\subsection{Packet I/O Totals}

This section shows basic packet acquisition and injection peg counts obtained
from the DAQ.  If you are reading pcaps, the totals are for all pcaps combined,
unless you use --pcap-reset, in which case it is shown per pcap.

\begin{itemize}
\item Outstanding indicates how many packets are buffered awaiting processing.
The way this is counted varies per DAQ so the DAQ documentation should be
consulted for more info.

\item Filtered packets are not shown for pcap DAQs.

\item Injected packets are the result of active response which can be
configured for inline or passive modes.
\end{itemize}

Example:

\begin{verbatim}
===============================================================================
Packet I/O Totals:
   Received:      3716022
   Analyzed:      3716022 (100.000%)
    Dropped:            0 (  0.000%)
   Filtered:            0 (  0.000%)
Outstanding:            0 (  0.000%)
   Injected:            0
===============================================================================
\end{verbatim}


\subsection{Protocol Statistics}

Traffic for all the protocols decoded by Snort is summarized in the breakdown
section.  This traffic includes internal "pseudo-packets" if preprocessors such
as frag3 and stream5 are enabled so the total may be greater than the number of
analyzed packets in the packet I/O section.

\begin{itemize}
\item Disc counts are discards due to basic encoding integrity flaws that
prevents Snort from decoding the packet.

\item Other includes packets that contained an encapsulation that Snort doesn't
decode.

\item S5 G 1/2 is the number of client/server sessions stream5 flushed due to
cache limit, session timeout, session reset.
\end{itemize}

Example:

\begin{verbatim}
===============================================================================
Breakdown by protocol (includes rebuilt packets):
        Eth:      3722347 (100.000%)
       VLAN:            0 (  0.000%)
        IP4:      1782394 ( 47.884%)
       Frag:         3839 (  0.103%)
       ICMP:        38860 (  1.044%)
        UDP:       137162 (  3.685%)
        TCP:      1619621 ( 43.511%)
        IP6:      1781159 ( 47.850%)
    IP6 Ext:      1787327 ( 48.016%)
   IP6 Opts:         6168 (  0.166%)
      Frag6:         3839 (  0.103%)
      ICMP6:         1650 (  0.044%)
       UDP6:       140446 (  3.773%)
       TCP6:      1619633 ( 43.511%)
     Teredo:           18 (  0.000%)
    ICMP-IP:            0 (  0.000%)
      EAPOL:            0 (  0.000%)
    IP4/IP4:            0 (  0.000%)
    IP4/IP6:            0 (  0.000%)
    IP6/IP4:            0 (  0.000%)
    IP6/IP6:            0 (  0.000%)
        GRE:          202 (  0.005%)
    GRE Eth:            0 (  0.000%)
   GRE VLAN:            0 (  0.000%)
    GRE IP4:            0 (  0.000%)
    GRE IP6:            0 (  0.000%)
GRE IP6 Ext:            0 (  0.000%)
   GRE PPTP:          202 (  0.005%)
    GRE ARP:            0 (  0.000%)
    GRE IPX:            0 (  0.000%)
   GRE Loop:            0 (  0.000%)
       MPLS:            0 (  0.000%)
        ARP:       104840 (  2.817%)
        IPX:           60 (  0.002%)
   Eth Loop:            0 (  0.000%)
   Eth Disc:            0 (  0.000%)
   IP4 Disc:            0 (  0.000%)
   IP6 Disc:            0 (  0.000%)
   TCP Disc:            0 (  0.000%)
   UDP Disc:         1385 (  0.037%)
  ICMP Disc:            0 (  0.000%)
All Discard:         1385 (  0.037%)
      Other:        57876 (  1.555%)
Bad Chk Sum:        32135 (  0.863%)
    Bad TTL:            0 (  0.000%)
     S5 G 1:         1494 (  0.040%)
     S5 G 2:         1654 (  0.044%)
      Total:      3722347
===============================================================================
\end{verbatim}

\subsection{Snort Memory Statistics}
On systems with mallinfo (3), you will see additional statistics. Check the man 
page of mallinfo for details

Example:
\begin{verbatim}
===============================================================================
Memory usage summary:
  Total non-mmapped bytes (arena):       415481856
  Bytes in mapped regions (hblkhd):      409612288
  Total allocated space (uordblks):      92130384
  Total free space (fordblks):           323351472
  Topmost releasable block (keepcost):   3200
===============================================================================
\end{verbatim}

\subsection{Actions, Limits, and Verdicts}

Action and verdict counts show what Snort did with the packets it analyzed.
This information is only output in IDS mode (when snort is run with the
\texttt{-c <conf>} option).

\begin{itemize}
\item Alerts is the number of activate, alert, and block actions processed as
determined by the rule actions.  Here block includes block, drop, and reject
actions.
\end{itemize}

Limits arise due to real world constraints on processing time and available
memory.  These indicate potential actions that did not happen:

\begin{itemize}
\item Match Limit counts rule matches were not processed due to the
\texttt{config detection: max\_queue\_events} setting.  The default is 5.

\item Queue Limit counts events couldn't be stored in the event queue
due to the \texttt{config event\_queue: max\_queue} setting.  The default is 8.

\item Log Limit counts events were not alerted due to the
\texttt{config event\_queue: log} setting.  The default is 3.

\item Event Limit counts events not alerted due to 
\texttt{event\_filter} limits.

\item Alert Limit counts events were not alerted because they already
were triggered on the session.
\end{itemize}

Verdicts are rendered by Snort on each packet:

\begin{itemize}
\item Allow = packets Snort analyzed and did not take action on. 

\item Block = packets Snort did not forward, e.g. due to a block rule.  "Block"
is used instead of "Drop" to avoid confusion between dropped packets (those
Snort didn't actually see) and blocked packets (those Snort did not allow to
pass).

\item Replace = packets Snort modified, for example, due to normalization or
replace rules.  This can only happen in inline mode with a compatible DAQ.

\item Whitelist = packets that caused Snort to allow a flow to pass w/o
inspection by any analysis program.  Like blacklist, this is done by the DAQ or
by Snort on subsequent packets.

\item Blacklist = packets that caused Snort to block a flow from passing.  This
is the case when a block TCP rule fires.  If the DAQ supports this in hardware,
no further packets will be seen by Snort for that session.  If not, snort will
block each packet and this count will be higher.

\item Ignore = packets that caused Snort to allow a flow to pass w/o inspection
by this instance of Snort.  Like blacklist, this is done by the DAQ or by Snort
on subsequent packets.

\item Int Blklst = packets that are GTP, Teredo, 6in4 or 4in6 encapsulated that are
being blocked.  These packets could get the Blacklist verdict if \texttt{config
tunnel\_verdicts} was set for the given protocol.  Note that these counts are
output only if non-zero.  Also, this count is incremented on the first packet
in the flow that alerts.  The alerting packet and all following packets on the
flow will be counted under Block.

\item Int Whtlst = packets that are GTP, Teredo, 6in4 or 4in6 encapsulated that are
being allowed.  These packets could get the Whitelist verdict if \texttt{config
tunnel\_verdicts} was set for the given protocol.  Note that these counts are
output only if non-zero.  Also, this count is incremented for all packets on
the flow starting with the alerting packet.
\end{itemize}

Example:

\begin{verbatim}
===============================================================================
Action Stats:
     Alerts:            0 (  0.000%)
     Logged:            0 (  0.000%)
     Passed:            0 (  0.000%)
Limits:
      Match:            0
      Queue:            0
        Log:            0
      Event:            0
      Alert:            0
Verdicts:
      Allow:      3716022 (100.000%)
      Block:            0 (  0.000%)
    Replace:            0 (  0.000%)
  Whitelist:            0 (  0.000%)
  Blacklist:            0 (  0.000%)
     Ignore:            0 (  0.000%)
===============================================================================
\end{verbatim}

\section{Tunneling Protocol Support}

Snort supports decoding of many tunneling protocols, including GRE, PPTP over GRE,
MPLS, IP in IP, and ERSPAN, all of which are enabled by default.  

To disable support for any GRE related encapsulation, PPTP over GRE, IPv4/IPv6 over
GRE, and ERSPAN, an extra configuration option is necessary:

\begin{verbatim}
    $ ./configure --disable-gre
\end{verbatim}

To disable support for MPLS, an separate extra configuration option is necessary:

\begin{verbatim}
    $ ./configure --disable-mpls
\end{verbatim}

\subsection{Multiple Encapsulations}

Snort will not decode more than one encapsulation.  Scenarios such as

\begin{verbatim}
    Eth IPv4 GRE IPv4 GRE IPv4 TCP Payload
\end{verbatim}

or

\begin{verbatim}
    Eth IPv4 IPv6 IPv4 TCP Payload
\end{verbatim}

will not be handled and will generate a decoder alert.

\subsection{Logging}

Currently, only the encapsulated part of the packet is logged, e.g.

\begin{verbatim}
    Eth IP1 GRE IP2 TCP Payload
\end{verbatim}

gets logged as

\begin{verbatim}
    Eth IP2 TCP Payload
\end{verbatim}

and

\begin{verbatim}
    Eth IP1 IP2 TCP Payload
\end{verbatim}

gets logged as

\begin{verbatim}
    Eth IP2 TCP Payload
\end{verbatim}

\begin{note}

Decoding of PPTP, which utilizes GRE and PPP, is not currently supported on
architectures that require word alignment such as SPARC.

\end{note}

\section{Miscellaneous}

\subsection{Running Snort as a Daemon}

If you want to run Snort as a daemon, you can the add -D switch to any
combination described in the previous sections. Please notice that if you want
to be able to restart Snort by sending a SIGHUP signal to the daemon, you {\em
must} specify the full path to the Snort binary when you start it, for example:

\begin{verbatim}
    /usr/local/bin/snort -d -h 192.168.1.0/24 \
        -l /var/log/snortlogs -c /usr/local/etc/snort.conf -s -D
\end{verbatim}

Relative paths are not supported due to security concerns.

\subsubsection{Snort PID File}

When Snort is run as a daemon , the daemon creates a PID file in the log
directory.  In Snort 2.6, the \texttt{--pid-path} command line switch causes
Snort to write the PID file in the directory specified.

Additionally, the \texttt{--create-pidfile} switch can be used to force
creation of a PID file even when not running in daemon mode.

The PID file will be locked so that other snort processes cannot start.  Use
the \texttt{--nolock-pidfile} switch to not lock the PID file.

If you do not wish to include the name of the interface in the PID file, use
the \texttt{--no-interface-pidfile} switch.

\subsection{Running in Rule Stub Creation Mode}

If you need to dump the shared object rules stub to a directory, you must use the --dump-dynamic-rules command line option. These rule stub files are used in conjunction with the shared object rules. The path can be relative or absolute. 

\begin{verbatim}
    /usr/local/bin/snort -c /usr/local/etc/snort.conf \
        --dump-dynamic-rules=/tmp
\end{verbatim}

This path can also be configured in the snort.conf using the config option dump-dynamic-rules-path as follows:

\begin{verbatim}
    config dump-dynamic-rules-path: /tmp/sorules
\end{verbatim}

The path configured by command line has precedence over the one configured using dump-dynamic-rules-path. 

\begin{verbatim}
    /usr/local/bin/snort -c /usr/local/etc/snort.conf \
        --dump-dynamic-rules

    snort.conf:
    config dump-dynamic-rules-path: /tmp/sorules
\end{verbatim}

In the above mentioned scenario the dump path is set to /tmp/sorules.

\subsection{Obfuscating IP Address Printouts}

If you need to post packet logs to public mailing lists, you might want to use
the -O switch. This switch obfuscates your IP addresses in packet printouts.
This is handy if you don't want people on the mailing list to know the IP
addresses involved. You can also combine the -O switch with the -h switch to
only obfuscate the IP addresses of hosts on the home network.  This is useful
if you don't care who sees the address of the attacking host.  For example, you
could use the following command to read the packets from a log file and dump
them to the screen, obfuscating only the addresses from the 192.168.1.0/24
class C network:
 
\begin{verbatim}
    ./snort -d -v -r snort.log -O -h 192.168.1.0/24
\end{verbatim}

\subsection{Specifying Multiple-Instance Identifiers}

In Snort v2.4, the \texttt{-G} command line option was added that specifies an
instance identifier for the event logs.  This option can be used when running
multiple instances of snort, either on different CPUs, or on the same CPU but a
different interface.  Each Snort instance will use the value specified to
generate unique event IDs.  Users can specify either a decimal value
(\texttt{-G 1}) or hex value preceded by 0x (\texttt{-G 0x11}).  This is also
supported via a long option \texttt{--logid}.

\subsection{Snort Modes}

Snort can operate in three different modes namely tap (passive), inline, and inline-test.
Snort policies can be configured in these three modes too. 

\subsubsection{Explanation of Modes}

\begin{itemize}

\item \texttt{Inline}

When Snort is in Inline mode, it acts as an IPS allowing drop rules to trigger. Snort can be 
configured to run in inline mode using the command line argument -Q and snort config option 
\texttt{policy\_mode} as follows:

\begin{verbatim}
    snort -Q
    config policy_mode:inline
\end{verbatim}

\item \texttt{Passive}

When Snort is in Passive mode, it acts as a IDS. Drop rules are not loaded (without --treat-drop-as-alert). 
Snort can be configured to passive mode using the snort config option \texttt{policy\_mode} as follows:

\begin{verbatim}
    config policy_mode:tap
\end{verbatim}


\item \texttt{Inline-Test}

Inline-Test mode simulates the inline mode of snort, allowing evaluation of inline behavior without affecting 
traffic. The drop rules will be loaded and will be triggered as a Wdrop (Would Drop) alert. Snort can be 
configured to run in inline-test mode using the command line option (--enable-inline-test) or using the 
snort config option \texttt{policy\_mode} as follows:

\begin{verbatim}
    snort --enable-inline-test
    config policy_mode:inline_test
\end{verbatim}

\begin{note}

Please note --enable-inline-test cannot be used in conjunction with -Q.

\end{note}

\end{itemize}

\texttt{Behavior of different modes with rule options}


\begin{tabular}{|l|c|c|p{6cm}|}
\hline
Rule Option & Inline Mode & Passive Mode & Inline-Test Mode\\
\hline
\hline
\texttt{reject} & Drop + Response & Alert + Response & Wdrop + Response\\
\hline
\texttt{react} & Blocks and send notice & Blocks and send notice & Blocks and send notice\\
\hline
\texttt{normalize} & Normalizes packet & Doesn't normalize & Doesn't normalize\\
\hline
\texttt{replace} & replace content & Doesn't replace & Doesn't replace\\
\hline
\texttt{respond} & close session & close session & close session\\
\hline
\end{tabular}


\texttt{Behavior of different modes with rules actions}


\begin{tabular}{|l|c|c|c|}
\hline
Adapter Mode & Snort args & config policy\_mode & Drop Rule Handling\\
\hline
\hline
Passive & \texttt{ --treat-drop-as-alert} & tap & Alert\\
\hline
Passive & \texttt{ no args} & tap & Not Loaded\\
\hline
Passive & \texttt{ --treat-drop-as-alert} & inline\_test & Alert\\
\hline
Passive & \texttt{ no args} & inline\_test & Would Drop\\
\hline
Passive & \texttt{ --treat-drop-as-alert} & inline & Alert\\
\hline
Passive & \texttt{no args} & inline & Not loaded + warning\\
\hline
Inline Test & \texttt{ --enable-inline-test --treat-drop-as-alert} & tap & Alert\\
\hline
Inline Test & \texttt{ --enable-inline-test} & tap & Would Drop\\
\hline
Inline Test & \texttt{ --enable-inline-test --treat-drop-as-alert} & inline\_test & Alert\\
\hline
Inline Test & \texttt{ --enable-inline-test} & inline\_test & Would Drop\\
\hline
Inline Test & \texttt{ --enable-inline-test --treat-drop-as-alert} & inline & Alert\\
\hline
Inline Test  & \texttt{ --enable-inline-test} & inline & Would Drop\\
\hline
Inline & \texttt{ -Q --treat-drop-as-alert} & tap & Alert\\
\hline
Inline & \texttt{ -Q} & tap & Alert\\
\hline
Inline & \texttt{ -Q --treat-drop-as-alert} & inline\_test & Alert\\
\hline
Inline & \texttt{ -Q} & inline\_test & Would Drop\\
\hline
Inline & \texttt{ -Q --treat-drop-as-alert} & inline & Alert\\
\hline
Inline & \texttt{ -Q} & inline & Drop\\
\hline
\end{tabular}

\section{Control socket}
\label{control_socket}
Snort can be configured to provide a Unix socket that can be used to issue commands 
to the running process. You must build snort with the 
\texttt{--enable-control-socket} option.  The control socket
functionality is supported on Linux only.\\

Snort can be configured to use control socket using the command line argument \texttt{--cs-dir <path>}
 and snort config option \texttt{cs\_dir} as follows:

\begin{verbatim}
    snort --cs-dir <path>
    config cs_dir:<path>
\end{verbatim}

\texttt{<path>} specifies the directory for snort to create the socket. If relative path is used, 
the path is relative to pid path specified. If there is no pid path specified, it is relative to 
current working directory. 

A command \texttt{snort\_control} is made and installed along with snort in the same 
bin directory when configured with the \texttt{--enable-control-socket} option.

\section{Configure signal value}
\label{configure_signal}
On some systems, signal used by snort might be used by other functions. To avoid conflicts, 
users can change the default signal value through \texttt{./configure} options for non-Windows system.

These signals can be changed: 
\begin{itemize}
\item \texttt{SIGNAL\_SNORT\_RELOAD} 
\item \texttt{SIGNAL\_SNORT\_DUMP\_STATS}
\item \texttt{SIGNAL\_SNORT\_ROTATE\_STATS}
\item \texttt{SIGNAL\_SNORT\_READ\_ATTR\_TBL}
\end{itemize} 

Syntax:

\begin{verbatim}
    ./configure SIGNAL_SNORT_RELOAD=<value/name>  SIGNAL_SNORT_DUMP_STATS=<value/name>\
        SIGNAL_SNORT_READ_ATTR_TBL=<value/name> SIGNAL_SNORT_ROTATE_STATS=<value/name>
\end{verbatim}

You can set those signals to user defined values or known signal names in the system.
The following example changes the rotate stats signal to 31 and reload attribute table to
signal SIGUSR2 :

\begin{verbatim}
    ./configure SIGNAL_SNORT_ROTATE_STATS=31 SIGNAL_SNORT_READ_ATTR_TBL=SIGUSR2
\end{verbatim}

If the same signal is assigned more than once a warning will be logged 
during snort initialization. If a signal handler cannot be installed a warning
will be logged and that has to be fixed, otherwise the functionality will be lost.

\texttt{Signals used in snort}

\begin{tabular}{|l|l|l|}
\hline
Signal name & Default value & Action \\
\hline
\hline
SIGTERM & SIGTERM & exit \\
\hline
SIGINT & SIGINT & exit \\
\hline
SIGQUIT & SIGQUIT & exit \\
\hline
SIGPIPE & SIGPIPE & ignore \\
\hline
SIGNAL\_SNORT\_RELOAD & SIGHUP & reload snort \\
\hline
SIGNAL\_SNORT\_DUMP\_STATS & SIGUSR1 & dump stats \\
\hline
SIGNAL\_SNORT\_ROTATE\_STATS & SIGUSR2 & rotate stats \\
\hline
SIGNAL\_SNORT\_READ\_ATTR\_TBL & SIGURG & reload attribute table \\
\hline
SIGNAL\_SNORT\_CHILD\_READY & SIGCHLD & internal use in daemon mode \\
\hline
\end{tabular}

\section{More Information}

Chapter \ref{Configuring Snort} contains much information about many
configuration options available in the configuration file.  The Snort manual
page and the output of \texttt{snort -?} or \texttt{snort --help} contain
information that can help you get Snort running in several different modes.

\begin{note}

In many shells, a backslash (\textbackslash{}) is needed to escape the ?, so
you may have to type \texttt{snort -\textbackslash{}?} instead of \texttt{snort
-?} for a list of Snort command line options.

\end{note}

The Snort web page (\url{http://www.snort.org}) and the Snort Users mailing
list:

\url{http://marc.theaimsgroup.com/?l=snort-users}

at \verb?snort-users@lists.sourceforge.net? provide informative announcements
as well as a venue for community discussion and support. There's a lot to
Snort, so sit back with a beverage of your choosing and read the documentation
and mailing list archives.

\chapter{Configuring Snort}
\label{Configuring Snort}

\section{Includes}

The {\tt include} keyword allows other snort config files to be included within the
snort.conf indicated on the Snort command line. It works much like an \#include
from the C programming language, reading the contents of the named file and
adding the contents in the place where the include statement appears in the
file.

\subsection{Format}
\begin{verbatim}
    include <include file path/name>
\end{verbatim}

\begin{note}

Note that there is no semicolon at the end of this line. 

\end{note}

Included files will substitute any predefined variable values into their own
variable references.  See Section \ref{variables} for more information on
defining and using variables in Snort config files.

\subsection{Variables}
\label{variables}

Three types of variables may be defined in Snort:

\begin{itemize}
\item var
\item portvar
\item ipvar
\end{itemize}

These are simple substitution variables set with the {\tt var}, {\tt ipvar}, or
{\tt portvar} keywords as follows:

\begin{verbatim}
    var RULES_PATH rules/
    portvar MY_PORTS [22,80,1024:1050]
    ipvar MY_NET [192.168.1.0/24,10.1.1.0/24]
    alert tcp any any -> $MY_NET $MY_PORTS (flags:S; msg:"SYN packet";)
    include $RULE_PATH/example.rule
\end{verbatim}

\subsubsection{IP Variables and IP Lists}

IPs may be specified individually, in a list, as a CIDR block, or any
combination of the three.  IP variables should be specified using 'ipvar'
instead of 'var'.  Using 'var' for an IP variable is still allowed for backward
compatibility, but it will be deprecated in a future release.

IPs, IP lists, and CIDR blocks may be negated with '!'.  Negation is handled
differently compared with Snort versions 2.7.x and earlier.  Previously, each
element in a list was logically OR'ed together.  IP lists now OR non-negated
elements and AND the result with the OR'ed negated elements.  

The following example list will match the IP 1.1.1.1 and IP from 2.2.2.0 to
2.2.2.255, with the exception of IPs 2.2.2.2 and 2.2.2.3.

\begin{verbatim}
    [1.1.1.1,2.2.2.0/24,![2.2.2.2,2.2.2.3]] 
\end{verbatim}

The order of the elements in the list does not matter.  The element 'any' can
be used to match all IPs, although '!any' is not allowed.  Also, negated IP
ranges that are more general than non-negated IP ranges are not allowed.  

See below for some valid examples if IP variables and IP lists.

\begin{verbatim}
    ipvar EXAMPLE [1.1.1.1,2.2.2.0/24,![2.2.2.2,2.2.2.3]] 
    
    alert tcp $EXAMPLE any -> any any (msg:"Example"; sid:1;)

    alert tcp [1.0.0.0/8,!1.1.1.0/24] any -> any any (msg:"Example";sid:2;)
\end{verbatim}

The following examples demonstrate some invalid uses of IP variables and IP
lists.

Use of !any:

\begin{verbatim}
    ipvar EXAMPLE any
    alert tcp !$EXAMPLE any -> any any (msg:"Example";sid:3;)
\end{verbatim}

Different use of !any:

\begin{verbatim}
    ipvar EXAMPLE !any
    alert tcp $EXAMPLE any -> any any (msg:"Example";sid:3;)
\end{verbatim}
        
Logical contradictions:

\begin{verbatim}
    ipvar EXAMPLE [1.1.1.1,!1.1.1.1]
\end{verbatim}

Nonsensical negations:

\begin{verbatim}
    ipvar EXAMPLE [1.1.1.0/24,!1.1.0.0/16]
\end{verbatim}


\subsubsection{Port Variables and Port Lists}

Portlists supports the declaration and lookup of ports and the representation
of lists and ranges of ports.  Variables, ranges, or lists may all be negated
with '!'.  Also, 'any' will specify any ports, but '!any' is not allowed.
Valid port ranges are from 0 to 65535.

Lists of ports must be enclosed in brackets and port ranges may be specified
with a ':', such as in:

\begin{verbatim}     
    [10:50,888:900]
\end{verbatim}

Port variables should be specified using 'portvar'.  The use of 'var' to
declare a port variable will be deprecated in a future release.  For backwards
compatibility, a 'var' can still be used to declare a port variable, provided
the variable name either ends with '\_PORT' or begins with 'PORT\_'. 

The following examples demonstrate several valid usages of both port variables
and port lists.

\begin{verbatim}     
    portvar EXAMPLE1 80

    var EXAMPLE2_PORT [80:90]

    var PORT_EXAMPLE2 [1]

    portvar EXAMPLE3 any

    portvar EXAMPLE4 [!70:90]

    portvar EXAMPLE5 [80,91:95,100:200]

    alert tcp any $EXAMPLE1 -> any $EXAMPLE2_PORT (msg:"Example"; sid:1;)

    alert tcp any $PORT_EXAMPLE2 -> any any (msg:"Example"; sid:2;)

    alert tcp any 90 -> any [100:1000,9999:20000] (msg:"Example"; sid:3;)
\end{verbatim}

Several invalid examples of port variables and port lists are demonstrated
below:

Use of !any:

\begin{verbatim}     
    portvar EXAMPLE5 !any
    var EXAMPLE5 !any
\end{verbatim}

Logical contradictions:

\begin{verbatim}     
    portvar EXAMPLE6 [80,!80]
\end{verbatim}

Ports out of range:

\begin{verbatim}     
    portvar EXAMPLE7 [65536]
\end{verbatim}

Incorrect declaration and use of a port variable:

\begin{verbatim}     
    var EXAMPLE8 80 
    alert tcp any $EXAMPLE8 -> any any (msg:"Example"; sid:4;)
\end{verbatim}

Port variable used as an IP:

\begin{verbatim}     
    alert tcp $EXAMPLE1 any -> any any (msg:"Example"; sid:5;)
\end{verbatim}

\subsubsection{Variable Modifiers}

Rule variable names can be modified in several ways. You can define
meta-variables using the \$ operator. These can be used with the variable
modifier operators {\tt ?}  and {\tt -}, as described in the following table: 

\begin{center}
\begin{tabular}{| l | p{4.5in} |}

\hline
\textbf{Variable Syntax} & \textbf{Description}\\
\hline

\hline
\texttt{var} &

Defines a meta-variable.\\

\hline
\texttt{\$(var) or \$var} &

Replaces with the contents of variable \texttt{var}.\\

\hline
\texttt{\$(var:-default)} &

Replaces the contents of the variable \texttt{var} with ``default'' if
\texttt{var} is undefined.\\

\hline
\texttt{\$(var:?message)} &

Replaces with the contents of variable \texttt{var} or prints out the error
message and exits.\\

\hline
\end{tabular}
\end{center}

Here is an example of advanced variable usage in action:

\begin{verbatim}
    ipvar MY_NET 192.168.1.0/24
    log tcp any any -> $(MY_NET:?MY_NET is undefined!) 23
\end{verbatim}

\subsubsection{Limitations}

When embedding variables, types can not be mixed.  For instance, port variables
can be defined in terms of other port variables, but old-style variables (with
the 'var' keyword) can not be embedded inside a 'portvar'.

Valid embedded variable:

\begin{verbatim}
    portvar pvar1 80
    portvar pvar2 [$pvar1,90]
\end{verbatim}

Invalid embedded variable:
        
\begin{verbatim}
    var pvar1 80
    portvar pvar2 [$pvar1,90]
\end{verbatim}

Likewise, variables can not be redefined if they were previously defined as a
different type.  They should be renamed instead:

Invalid redefinition:
        
\begin{verbatim}
    var pvar 80
    portvar pvar 90
\end{verbatim}

\subsection{Config}
\label{Config}

Many configuration and command line options of Snort can be specified in the
configuration file. 

\subsubsection{Format}

\begin{verbatim}
    config <directive> [: <value>]
\end{verbatim}

\newpage
\begin{center}
\begin{longtable}[t]{| p{2.5in} | p{3.5in} |}

\hline
{\bf Config Directive} & {\bf Description}\\
\hline

% KEEP THESE IN ALPHABETICAL ORDER
\hline
\texttt{config alert\_with\_interface\_name} & Appends interface name to alert
(\texttt{snort -I}). \\

\hline
\texttt{config alertfile: <filename>} & Sets the alerts output file. \\

\hline
\texttt{config asn1: <max-nodes>} & Specifies the maximum number of nodes to track when
doing ASN1 decoding. See Section \ref{asn1} for more information and
examples.\\

\hline
\texttt{config autogenerate\_preprocessor\newline \_decoder\_rules} & If Snort was
configured to enable decoder and preprocessor rules, this option will cause
Snort to revert back to its original behavior of alerting if the decoder or
preprocessor generates an event. \\

\hline
\texttt{config bpf\_file: <filename>} & Specifies BPF filters (\texttt{snort
-F}). \\

\hline
\texttt{config checksum\_drop: <types>} & Types of packets to drop if invalid
checksums. Values: \texttt{none}, \texttt{noip}, \texttt{notcp},
\texttt{noicmp}, \texttt{noudp}, \texttt{ip}, \texttt{tcp}, \texttt{udp},
\texttt{icmp} or \texttt{all} (only applicable in inline mode and for packets
checked per \texttt{checksum\_mode} config option). \\

\hline
\texttt{config checksum\_mode: <types>} & Types of packets to calculate checksums.
Values: \texttt{none}, \texttt{noip}, \texttt{notcp}, \texttt{noicmp},
\texttt{noudp}, \texttt{ip}, \texttt{tcp}, \texttt{udp}, \texttt{icmp} or
\texttt{all}. \\

\hline
\texttt{config chroot: <dir>} & Chroots to specified dir (\texttt{snort
-t}). \\

\hline
\texttt{config classification: <class>} & See Table
\ref{Snort Default Classifications} for a list of classifications.\\

\hline
\texttt{config cs\_dir: <path>} & configure snort to provide a Unix socket in the path
that can be used to issue commands to the running process. See Section
\ref{control_socket} for more details.\\

\hline
\texttt{config daemon} & Forks as a daemon (\texttt{snort -D}). \\

\hline
\texttt{config decode\_data\_link} & Decodes Layer2 headers (\texttt{snort
-e}). \\

\hline
\texttt{config default\_rule\_state: <state>} & Global configuration directive
to enable or disable the loading of rules into the detection engine.  Default
(with or without directive) is enabled.  Specify \texttt{disabled} to disable
loading rules. \\

\hline
\texttt{config daq: <type>} & Selects the type of DAQ to instantiate.  The
DAQ with the highest version of the given type is selected if there are 
multiple of the same type (this includes any built-in DAQs).\\

\hline
\texttt{config daq\_mode: <mode>} & Select the DAQ mode: passive, inline, or
read-file.  Not all DAQs support modes.  See the DAQ distro README for
possible DAQ modes or list DAQ capabilities for a brief summary. \\

\hline
\texttt{config daq\_var: <name=value>} & Set a DAQ specific variable.  Snort
just passes this information down to the DAQ.  See the DAQ distro README for
possible DAQ variables. \\

\hline
\texttt{config daq\_dir: <dir>} & Tell Snort where to look for available
dynamic DAQ modules.  This can be repeated.  The selected DAQ will be the
one with the latest version. \\

\hline \texttt{config daq\_list: [<dir>]} & Tell Snort to dump basic DAQ
capabilities and exit.  You can optionally specify a directory to include any
dynamic DAQs from that directory.  You can also precede this option with extra
DAQ directory options to look in multiple directories. \\

\hline
\texttt{config decode\_esp: [enable | disable]} & Enable or disable the decoding of
Encapsulated Security Protocol (ESP). This is disabled by default.
Some networks use ESP for authentication without encryption, allowing their
content to be inspected. Encrypted ESP may cause some false positives if this
option is enabled.\\

\hline
\texttt{config detection: [search-method <method>]} & Select type of fast pattern
matcher algorithm to use.
\begin{itemize}
\item \texttt{search-method <method>}
\begin{itemize}
\item Queued match search methods -  Matches are queued until the fast pattern
matcher is finished with the payload, then evaluated.  This was found to generally
increase performance through fewer cache misses (evaluating each rule would
generally blow away the fast pattern matcher
state in the cache).
\begin{itemize}
\item \texttt{ac} and \texttt{ac-q} - Aho-Corasick Full (high memory, best performance).
\item \texttt{ac-bnfa} and \texttt{ac-bnfa-q} - Aho-Corasick Binary NFA (low memory, high performance)
\item \texttt{lowmem} and \texttt{lowmem-q} - Low Memory Keyword Trie (low memory, moderate performance)
\item \texttt{ac-split} - Aho-Corasick Full with ANY-ANY port group evaluated separately (low memory, high performance).  Note this is shorthand for \texttt{search-method ac, split-any-any}
\item \texttt{intel-cpm} - Intel CPM library (must have compiled Snort with location of libraries to enable this)
\end{itemize}
\end{itemize}
\begin{itemize}
\item No queue search methods - The "nq" option specifies that matches should not
be queued and evaluated as they are found.
\begin{itemize}
\item \texttt{ac-nq} - Aho-Corasick Full (high memory, best performance).
\item \texttt{ac-bnfa-nq} - Aho-Corasick Binary NFA (low memory, high performance).
This is the default search method if none is specified.
\item \texttt{lowmem-nq} - Low Memory Keyword Trie (low memory, moderate performance)
\end{itemize}
\end{itemize}
\begin{itemize}
\item Other search methods (the above are considered superior to these)
\begin{itemize}
\item \texttt{ac-std} - Aho-Corasick Standard (high memory, high performance)
\item \texttt{acs} - Aho-Corasick Sparse (high memory, moderate performance)
\item \texttt{ac-banded} - Aho-Corasick Banded (high memory, moderate performance)
\item \texttt{ac-sparsebands} - Aho-Corasick Sparse-Banded (high memory, moderate performance)
\end{itemize}
\end{itemize}
\end{itemize} \\

\hline
\texttt{config detection: [split-any-any] [search-optimize] [max-pattern-len <int>]} & Other options
that affect fast pattern matching.
\begin{itemize}
\item \texttt{split-any-any}
\begin{itemize}
\item A memory/performance tradeoff.  By default, ANY-ANY port rules are added to 
every non ANY-ANY port group so that only one port group rule evaluation needs to 
be done per packet.  Not putting the ANY-ANY port rule group into every other
port group can significantly reduce the memory footprint of the fast pattern
matchers if there are many ANY-ANY port rules.  But doing so may require two
port group evaluations per packet - one for the specific port group and one for
the ANY-ANY port group, thus potentially reducing performance.  This option is
generic and can be used with any \texttt{search-method} but was specifically
intended for use with the \texttt{ac} \texttt{search-method} where the memory
footprint is significantly reduced though overall fast pattern performance is
better than \texttt{ac-bnfa}.  Of note is that the lower memory footprint can
also increase performance through fewer cache misses.  Default is not to split
the ANY-ANY port group.
\end{itemize}
\item \texttt{search-optimize}
\begin{itemize}
\item Optimizes fast pattern memory when used with \texttt{search-method}
\texttt{ac} or \texttt{ac-split} by dynamically determining the size of a 
state based on the total number of states. When used with \texttt{ac-bnfa}, some
fail-state resolution will be attempted, potentially increasing performance.
Default is not to optimize.
\end{itemize}
\item \texttt{max-pattern-len <integer>}
\begin{itemize}
\item This is a memory optimization that specifies the maximum length of a pattern
that will be put in the fast pattern matcher.  Patterns longer than this length
will be truncated to this length before inserting into the pattern matcher.  Useful
when there are very long contents being used and truncating the pattern won't diminish
the uniqueness of the patterns.  Note that this may cause more false positive rule
evaluations, i.e. rules that will be evaluated because a fast pattern was matched,
but eventually fail, however CPU cache can play a part in performance so a smaller memory
footprint of the fast pattern matcher can potentially increase performance.  Default
is to not set a maximum pattern length.
\end{itemize}
\end{itemize} \\

\hline
\texttt{config detection: [no\_stream\_inserts] [max\_queue\_events <int>] [enable-single-rule-group] [bleedover-port-limit]} & Other detection engine options.
\begin{itemize}
\item \texttt{no\_stream\_inserts}
\begin{itemize}
\item Specifies that stream inserted packets should not be evaluated against the detection engine.
This is a potential performance improvement with the idea that the stream rebuilt packet will
contain the payload in the inserted one so the stream inserted packet doesn't need to be
evaluated.  Default is to inspect stream inserts.
\end{itemize}
\item \texttt{max\_queue\_events <integer>}
\begin{itemize}
\item Specifies the maximum number of matching fast-pattern states to queue per packet.
Default is 5 events.
\end{itemize}
\item \texttt{enable-single-rule-group}
\begin{itemize}
\item Put all rules into one port group.  Not recommended.  Default is not to
do this.
\end{itemize}
\item \texttt{bleedover-port-limit}
\begin{itemize}
\item The maximum number of source or destination ports designated in a rule
before the rule is considered an ANY-ANY port group rule.  Default is 1024.
\end{itemize}
\end{itemize} \\

\hline
\texttt{config detection: [debug] [debug-print-nocontent-rule-tests] [debug-print-rule-group-build-details] [debug-print-rule-groups-uncompiled] [debug-print-rule-groups-compiled] [debug-print-fast-pattern] [bleedover-warnings-enabled]} & Options for detection engine debugging.
\begin{itemize}
\item \texttt{debug}
\begin{itemize}
\item Prints fast pattern information for a particular port group.
\end{itemize}
\item \texttt{debug-print-nocontent-rule-tests}
\begin{itemize}
\item Prints port group information during packet evaluation.
\end{itemize}
\item \texttt{debug-print-rule-group-build-details}
\begin{itemize}
\item Prints port group information during port group compilation.
\end{itemize}
\item \texttt{debug-print-rule-groups-uncompiled}
\begin{itemize}
\item Prints uncompiled port group information.
\end{itemize}
\item \texttt{debug-print-rule-groups-compiled}
\begin{itemize}
\item Prints compiled port group information.
\end{itemize}
\item \texttt{debug-print-fast-pattern}
\begin{itemize}
\item For each rule with fast pattern content, prints information about the content
being used for the fast pattern matcher.
\end{itemize}
\item \texttt{bleedover-warnings-enabled}
\begin{itemize}
\item Prints a warning if the number of source or destination ports used in a
rule exceed the \texttt{bleedover-port-limit} forcing the rule to be moved into
the ANY-ANY port group.
\end{itemize}
\end{itemize} \\

\hline
\texttt{config disable\_decode\_alerts} & Turns off the alerts generated by the
decode phase of Snort. \\

\hline
\texttt{config disable\_inline\_init\_failopen} & Disables failopen thread that
allows inline traffic to pass while Snort is starting up.  Only useful if Snort
was configured with --enable-inline-init-failopen.  (\texttt{snort
--disable-inline-init-failopen}) \\

\hline
\texttt{config disable\_ipopt\_alerts} & Disables IP option length validation
alerts. \\

\hline
\texttt{config disable\_tcpopt\_alerts} & Disables option length validation
alerts. \\

\hline
\texttt{config\newline disable\_tcpopt\_experimental\_alerts} & Turns off
alerts generated by experimental TCP options. \\

\hline
\texttt{config disable\_tcpopt\_obsolete\_alerts} & Turns off alerts
generated by obsolete TCP options. \\

\hline
\texttt{config disable\_tcpopt\_ttcp\_alerts} & Turns off alerts generated by
T/TCP options. \\

\hline
\texttt{config disable\_ttcp\_alerts} & Turns off alerts generated by T/TCP
options. \\

\hline
\texttt{config dump\_chars\_only} & Turns on character dumps (\texttt{snort
-C}). \\

\hline
\texttt{config dump\_payload} & Dumps application layer (\texttt{snort -d}). \\

\hline
\texttt{config dump\_payload\_verbose} & Dumps raw packet starting at link
layer (\texttt{snort -X}). \\

\hline
\texttt{config enable\_decode\_drops} & Enables the dropping of bad packets
identified by decoder (only applicable in inline mode).\\

\hline
\texttt{config enable\_decode\_oversized\_alerts} & Enable alerting
on packets that have headers containing length fields for which the value is
greater than the length of the packet. \\

\hline
\texttt{config enable\_decode\_oversized\_drops} & Enable dropping
packets that have headers containing length fields for which the value is
greater than the length of the packet.
\texttt{enable\_decode\_oversized\_alerts} must also be enabled for this to be
effective (only applicable in inline mode). \\

\hline
\texttt{config enable\_deep\_teredo\_inspection} & Snort's packet decoder only
decodes Teredo (IPv6 over UDP over IPv4) traffic on UDP port 3544. This option
makes Snort decode Teredo traffic on all UDP ports. \\

\hline
\texttt{config enable\_ipopt\_drops} & Enables the dropping of bad packets with
bad/truncated IP options (only applicable in inline mode).\\

\hline
\texttt{config enable\_mpls\_multicast} & Enables support for MPLS multicast.
This option is needed when the network allows MPLS multicast traffic. When this
option is off and MPLS multicast traffic is detected, Snort will generate an
alert. By default, it is off.\\

\hline
\texttt{config enable\_mpls\_overlapping\_ip} & Enables support for overlapping
IP addresses in an MPLS network. In a normal situation, where there are no
overlapping IP addresses, this configuration option should not be turned on.
However, there could be situations where two private networks share the same IP
space and different MPLS labels are used to differentiate traffic from the two
VPNs. In such a situation, this configuration option should be turned on. By
default, it is off. \\

\hline
\texttt{config enable\_tcpopt\_drops} & Enables the dropping of bad packets
with bad/truncated TCP option (only applicable in inline mode).\\

\hline
\texttt{config\newline enable\_tcpopt\_experimental\_drops} & Enables the
dropping of bad packets with experimental TCP option.  (only applicable in
inline mode).\\

\hline
\texttt{config enable\_tcpopt\_obsolete\_drops} & Enables the
dropping of bad packets with obsolete TCP option.  (only applicable in inline
mode).\\

\hline
\texttt{config enable\_tcpopt\_ttcp\_drops} & Enables the dropping of bad packets with
T/TCP option. (only applicable in inline mode).\\

\hline
\texttt{config enable\_ttcp\_drops} & Enables the dropping of bad packets with T/TCP
option. (only applicable in inline mode).\\

\hline
\texttt{config event\_filter: memcap <bytes>} & Set global memcap in bytes for
thresholding. Default is 1048576 bytes (1 megabyte). \\

\hline
\texttt{config event\_queue: [max\_queue <num>] [log <num>] [order\_events
<order>]} & Specifies conditions about Snort's event queue. You can use the
following options:

\begin{itemize}
\item \texttt{max\_queue $<$integer$>$} (max events supported)
\item \texttt{log $<$integer$>$} (number of events to log) 
\item \texttt{order\_events [priority$|$content\_length]} (how to order events within the queue)
\end{itemize}

See Section \ref{eventqueue} for more information and examples.\\

\hline
\texttt{config flowbits\_size: <num-bits>} & Specifies the maximum number of
flowbit tags that can be used within a rule set.  The default is 1024 bits
and maximum is 2048. \\

\hline
\texttt{config ignore\_ports: <proto> <port-list>} & Specifies ports to ignore
(useful for ignoring noisy NFS traffic). Specify the protocol (TCP, UDP, IP, or
ICMP), followed by a list of ports. Port ranges are supported.\\

\hline
\texttt{config interface: <iface>} & Sets the network interface (\texttt{snort
-i}). \\

\hline
\texttt{config ipv6\_frag: [bsd\_icmp\_frag\_alert on|off] [,
bad\_ipv6\_frag\_alert on|off] [, frag\_timeout <secs>] [, max\_frag\_sessions
<max-track>]} & The following options can be used:

\begin{itemize}

\item \texttt{bsd\_icmp\_frag\_alert on|off} (Specify whether or not to alert.
Default is on)

\item \texttt{bad\_ipv6\_frag\_alert on|off} (Specify whether or not to alert.
Default is on)

\item \texttt{frag\_timeout $<$integer$>$} (Specify amount of time in seconds
to timeout first frag in hash table)

\item \texttt{max\_frag\_sessions $<$integer$>$} (Specify the number of
fragments to track in the hash table)

\end{itemize} \\

\hline
\texttt{config logdir: <dir>} & Sets the logdir (\texttt{snort -l}).
\\

\hline
\texttt{config log\_ipv6\_extra\_data} & Set Snort to log IPv6 source and destination
addresses as unified2 extra data events.  \\

\hline
\texttt{config max\_attribute\_hosts: <hosts>} & Sets a limit on the maximum
number of hosts to read from the attribute table.  Minimum value is 32 and the
maximum is 524288 (512k).  The default is 10000.  If the number of hosts in the
attribute table exceeds this value, an error is logged and the remainder of the
hosts are ignored.  This option is only supported with a Host Attribute Table
(see section \ref{targetbased}). \\

\hline
\texttt{config max\_attribute\_services\_per\_host: <hosts>} & Sets a per host
limit on the maximum number of services to read from the attribute table. 
Minimum value is 1 and the maximum is 65535.  The default is 100.  For a given
host, if the number of services in the attribute table exceeds this value, an
error is logged and the remainder of the services for that host are ignored. 
This option is only supported with a Host Attribute Table (see section
\ref{targetbased}). \\

\hline
\texttt{config max\_mpls\_labelchain\_len: <num-hdrs>} & Sets a Snort-wide
limit on the number of MPLS headers a packet can have. Its default value is -1,
which means that there is no limit on label chain length.\\

\hline
\texttt{config max\_ip6\_extensions: <num-extensions>} & Sets the maximum number
of IPv6 extension headers that Snort will decode. Default is 8. \\

\hline
\texttt{config min\_ttl: <ttl>} & Sets a Snort-wide minimum ttl to ignore all
traffic. \\

\hline
\texttt{config mpls\_payload\_type: ipv4|ipv6|ethernet} & Sets a Snort-wide
MPLS payload type. In addition to ipv4, ipv6 and ethernet are also valid
options. The default MPLS payload type is ipv4\\

\hline
\texttt{config no\_promisc} & Disables promiscuous mode (\texttt{snort -p}). \\

\hline
\texttt{config nolog} & Disables logging. Note: Alerts will still occur.
(\texttt{snort -N}). \\

\hline
\texttt{config nopcre} & Disables pcre pattern matching. \\

\hline
\texttt{config obfuscate} & Obfuscates IP Addresses (\texttt{snort -O}). \\

\hline
\texttt{config order: <order>} & Changes the order that rules
are evaluated, e.g.: pass alert log activation. \\

\hline
\texttt{config pcre\_match\_limit: $<$integer$>$} & Restricts the amount of
backtracking a given PCRE option.  For example, it will limit the number of
nested repeats within a pattern.  A value of -1 allows for unlimited PCRE, up
to the PCRE library compiled limit (around 10 million).  A value of 0 results
in no PCRE evaluation.  The snort default value is 1500.  \\

\hline
\texttt{config pcre\_match\_limit\_recursion: $<$integer$>$} & Restricts the
amount of stack used by a given PCRE option.  A value of -1 allows for
unlimited PCRE, up to the PCRE library compiled limit (around 10 million).  A
value of 0 results in no PCRE evaluation.  The snort default value is 1500.
This option is only useful if the value is less than the
\texttt{pcre\_match\_limit} \\

\hline
\texttt{config pkt\_count: <N>} & Exits after N packets (\texttt{snort -n}). \\

\hline
\texttt{config policy\_version: $<$base-version-string$>$ [$<$binding-version-string$>$]} &
Supply versioning information to configuration files.  Base version should be
a string in all configuration files including included ones.  In addition,
binding version must be in any file configured with \texttt{config binding}.
This option is used to avoid race conditions when modifying and loading a
configuration within a short time span - before Snort has had a chance to
load a previous configuration. \\

\hline
\texttt{config profile\_preprocs} & Print statistics on preprocessor
performance.  See Section \ref{preproc profiling} for more details. \\


\hline
\texttt{config profile\_rules} & Print statistics on rule performance.  See
Section \ref{rule profiling} for more details. \\

\hline
\texttt{config protected\_content:
md5|sha256|sha512} & Specifies a default algorithm to use for protected\_content rules.  \\

\hline
\texttt{config quiet}& Disables banner and status reports (\texttt{snort -q}).
NOTE: The command line switch \texttt{-q} takes effect immediately after
processing the command line parameters, whereas using \texttt{config quiet}
in snort.conf takes effect when the configuration line in snort.conf is parsed.
That may occur after other configuration settings that result in output to
console or syslog.
\\

\hline
\texttt{config reference: <ref>} & Adds a new reference system to Snort, e.g.:
myref http://myurl.com/?id=\\

\hline
\texttt{config reference\_net <cidr>} & For IP obfuscation, the obfuscated net
will be used if the packet contains an IP address in the reference net.  Also
used to determine how to set up the logging directory structure for the
\texttt{session} post detection rule option and ASCII output plugin - an
attempt is made to name the log directories after the IP address that is not in
the reference net. \\

\hline \texttt{config response: [attempts <count>] [, device <dev>]} & Set the
number of strafing attempts per injected response and/or the device, such as
eth0, from which to send responses.  These options may appear in any order but
must be comma separated.  The are intended for passive mode. \\

\hline
\texttt{config set\_gid: <gid>} & Changes GID to specified GID (\texttt{snort
-g}). \\

\hline
\texttt{config set\_uid: <uid>} & Sets UID to $<$id$>$ (\texttt{snort -u}). \\

\hline
\texttt{config show\_year} & Shows year in timestamps (\texttt{snort -y}). \\

\hline
\texttt{config snaplen: <bytes>} & Set the snaplength of packet, same effect as
\texttt{-P $<$snaplen$>$} or \texttt{--snaplen $<$snaplen$>$} options.\\

\hline
\texttt{config so\_rule\_memcap: <bytes>} & Set global memcap in bytes for
so rules that dynamically allocate memory for storing session data in the
stream preprocessor.  A value of 0 disables the memcap.  Default is 0.
Maximum value is the maximum value an unsigned 32 bit integer can hold
which is 4294967295 or 4GB.\\

\hline
\texttt{config stateful} & Sets assurance mode for stream (stream is
established). \\

\hline
\texttt{config tagged\_packet\_limit: <max-tag>} & When a metric other than
\texttt{packets} is used in a tag option in a rule, this option sets the
maximum number of packets to be tagged regardless of the amount defined by the
other metric.  See Section \ref{tag section} on using the tag option when
writing rules for more details.  The default value when this option is not
configured is 256 packets.  Setting this option to a value of 0 will disable
the packet limit. \\

\hline
\texttt{config threshold: memcap <bytes>} & Set global memcap in bytes for
thresholding. Default is 1048576 bytes (1 megabyte). (This is deprecated.
Use config event\_filter instead.)\\

\hline
\texttt{config umask: <umask>} & Sets umask when running (\texttt{snort -m}). \\

\hline
\texttt{config utc} & Uses UTC instead of local time for timestamps
(\texttt{snort -U}). \\

\hline
\texttt{config verbose} & Uses verbose logging to STDOUT (\texttt{snort -v}).
\\

\hline
\texttt{config vlan\_agnostic} & Causes Snort to ignore vlan headers for
the purposes of connection and frag tracking.  This option is only valid in the
base configuration when using multiple configurations, and the default is off.
\\

\hline
\texttt{config address\_space\_agnostic} & Causes Snort to ignore DAQ
address space ID for the purposes of connection and frag tracking.  This option
is only valid in the base configuration when using multiple configurations, and
the default is off.  \\

\hline
\texttt{config policy\_mode: tap|inline|inline\_test} & Sets the policy
mode to either \texttt{passive}, \texttt{inline} or \texttt{inline\_test}. \\

\hline \texttt{config tunnel\_verdicts: gtp|teredo|6in4|4in6} & By default, whitelist
and blacklist verdicts are handled internally by Snort for GTP, Teredo, 6in4 and 4in6
encapsulated traffic.  This means Snort actually gives the DAQ a pass or block
verdict instead.  This is to workaround cases where the DAQ would apply
the verdict to the whole tunnel instead of the individual session within the
tunnel.  If your DAQ decodes GTP, Teredo, 6in4 or 4in6 correctly, setting this config
will allow the whitelist or blacklist verdict to go to the DAQ.  There is a
modest performance boost by doing this where possible since Snort won't see the
remaining packets on the session. \\

\hline
\end{longtable}
\end{center}

\section{Preprocessors}

Preprocessors were introduced in version 1.5 of Snort. They allow the
functionality of Snort to be extended by allowing users and programmers to drop
modular plugins into Snort fairly easily.  Preprocessor code is run before the
detection engine is called, but after the packet has been decoded. The packet
can be modified or analyzed in an out-of-band manner using this mechanism.

Preprocessors are loaded and configured using the {\tt preprocessor} keyword.
The format of the preprocessor directive in the Snort config file is:

\begin{verbatim}
    preprocessor <name>: <options>
\end{verbatim}

\subsection{Frag3}
\label{frag3 section}

The frag3 preprocessor is a target-based IP defragmentation module for Snort.
Frag3 is designed with the following goals:

\begin{slist}
\item Fast execution with less complex data management.
\item Target-based host modeling anti-evasion techniques.
\end{slist}

Frag3 uses the sfxhash data structure and linked lists for data handling
internally which allows it to have much more predictable and deterministic
performance in any environment which should aid us in managing heavily
fragmented environments.

Target-based analysis is a relatively new concept in network-based intrusion
detection.  The idea of a target-based system is to model the actual targets on
the network instead of merely modeling the protocols and looking for attacks
within them.  When IP stacks are written for different operating systems, they
are usually implemented by people who read the RFCs and then write their
interpretation of what the RFC outlines into code.  Unfortunately, there are
ambiguities in the way that the RFCs define some of the edge conditions that
may occur and when this happens different people implement certain aspects of
their IP stacks differently.  For an IDS this is a big problem.

In an environment where the attacker can determine what style of IP
defragmentation is being used on a particular target, the attacker can try to
fragment packets such that the target will put them back together in a specific
manner while any passive systems trying to model the host traffic have to guess
which way the target OS is going to handle the overlaps and retransmits.  As I
like to say, if the attacker has more information about the targets on a
network than the IDS does, it is possible to evade the IDS.  This is where the
idea for ``target-based IDS'' came from.  For more detail on this issue and how
it affects IDS, check out the famous Ptacek \& Newsham paper at
\url{http://www.snort.org/docs/idspaper/}.

The basic idea behind target-based IDS is that we tell the IDS information
about hosts on the network so that it can avoid Ptacek \& Newsham style evasion
attacks based on information about how an individual target IP stack operates.
Vern Paxson and Umesh Shankar did a great paper on this very topic in 2003 that
detailed mapping the hosts on a network and determining how their various IP
stack implementations handled the types of problems seen in IP defragmentation
and TCP stream reassembly.  Check it out at
\url{http://www.icir.org/vern/papers/activemap-oak03.pdf}.

We can also present the IDS with topology information to avoid TTL-based
evasions and a variety of other issues, but that's a topic for another day.
Once we have this information we can start to really change the game for these
complex modeling problems.

Frag3 was implemented to showcase and prototype a target-based module within
Snort to test this idea.

\subsubsection{Frag 3 Configuration}

There are at least two preprocessor directives required to activate frag3,
a global configuration directive and an engine instantiation.  There can
be an arbitrary number of engines defined at startup with their own
configuration, but only one global configuration.

\textbf{Global Configuration}

\begin{itemize}

\item Preprocessor name: \texttt{frag3\_global}

\item Available options: NOTE: Global configuration options are comma
separated.

\begin{itemize}

\item \texttt{max\_frags $<$number$>$} - Maximum simultaneous fragments to
track. Default is 8192.

\item \texttt{memcap $<$bytes$>$} - Memory cap for self preservation.  Default
is 4MB.  

\item \texttt{prealloc\_memcap $<$bytes$>$} - alternate memory management mode,
use preallocated fragment nodes based on a memory cap (faster in some
situations).

\item \texttt{prealloc\_frags $<$number$>$} - Alternate memory management mode,
use preallocated fragment nodes (faster in some situations).

\item \texttt{disabled} - This optional keyword is allowed with any
policy to avoid packet processing. This option disables the preprocessor
for this config, but not for other instances of multiple configurations.
Use the disable keyword in the base configuration to specify values for the
options \texttt{memcap}, \texttt{prealloc\_memcap}, and \texttt{prealloc\_frags}
without having the preprocessor inspect traffic for traffic applying to the base
configuration.  The other options are parsed but not used. Any valid
configuration may have "disabled" added to it.

\end{itemize}                               
\end{itemize}    
 
\textbf{Engine Configuration}

\begin{itemize}

\item Preprocessor name: \texttt{frag3\_engine}

\item Available options:
  NOTE: Engine configuration options are space separated.

\begin{itemize}

\item \texttt{timeout $<$seconds$>$} - Timeout for fragments.  Fragments in the
engine for longer than this period will be automatically dropped.  Default is
60 seconds.
                        
\item \texttt{min\_ttl $<$value$>$} - Minimum acceptable TTL value for a
fragment packet.  Default is 1. The accepted range for this option is 1 - 255.
                       
\item \texttt{detect\_anomalies} - Detect fragment anomalies.
     
\item \texttt{bind\_to $<$ip\_list$>$} - IP List to bind this engine to.  This
engine will only run for packets with destination addresses contained within
the IP List.  Default value is \texttt{all}.
                         
\item \texttt{overlap\_limit <number>} - Limits the number of overlapping
fragments per packet.  The default is "0" (unlimited). This config option takes
values equal to or greater than zero. This is an optional parameter. 
detect\_anomalies option must be configured for this option to take effect.

\item \texttt{min\_fragment\_length <number>} - Defines smallest fragment size
(payload size) that should be considered valid.  Fragments smaller than or
equal to this limit are considered malicious and an event is raised, if
detect\_anomalies is also configured.  The default is "0" (unlimited), the
minimum is "0".  This is an optional parameter.  detect\_anomalies option 
must be configured for this option to take effect.

\item \texttt{policy $<$type$>$} - Select a target-based defragmentation mode.
Available types are first, last, bsd, bsd-right, linux, windows and solaris.
Default type is bsd.

The Paxson Active Mapping paper introduced the terminology frag3 is using to
describe policy types.  The known mappings are as follows.  Anyone who develops
more mappings and would like to add to this list please feel free to send us an
email!

\begin{tabular}{| l | l |}
\hline
\textbf{Platform} & \textbf{Type}\\
\hline
\hline                     
                        AIX 2  & BSD \\
                        \hline
                AIX 4.3 8.9.3  & BSD \\
                        \hline
                    Cisco IOS  & Last \\
                        \hline
                      FreeBSD  & BSD\\
                        \hline 
       HP JetDirect (printer)  & BSD-right \\
                        \hline
                HP-UX B.10.20  & BSD \\
                        \hline
                  HP-UX 11.00  & First \\
                        \hline
                  IRIX 4.0.5F  & BSD \\
                        \hline
                     IRIX 6.2  & BSD \\
                        \hline
                     IRIX 6.3  & BSD \\
                        \hline
                   IRIX64 6.4  & BSD \\
                        \hline
                 Linux 2.2.10  & linux \\
                        \hline
             Linux 2.2.14-5.0  & linux \\
                        \hline
               Linux 2.2.16-3  & linux \\
                        \hline
       Linux 2.2.19-6.2.10smp  & linux \\
                        \hline
               Linux 2.4.7-10  & linux \\
                        \hline
   Linux 2.4.9-31SGI 1.0.2smp  & linux \\
                        \hline
   Linux 2.4 (RedHat 7.1-7.3)  & linux \\
                        \hline
      MacOS (version unknown)  & First \\
                        \hline
             NCD Thin Clients  & BSD \\
                        \hline
    OpenBSD (version unknown)  & linux \\
                        \hline
    OpenBSD (version unknown)  & linux \\
                        \hline
                  OpenVMS 7.1  & BSD \\
                        \hline
       OS/2 (version unknown)  & BSD \\
                        \hline
                    OSF1 V3.0  & BSD \\
                        \hline
                    OSF1 V3.2  & BSD \\
                        \hline
            OSF1 V4.0,5.0,5.1  & BSD \\
                        \hline
                  SunOS 4.1.4  & BSD \\
                        \hline
      SunOS 5.5.1,5.6,5.7,5.8  & First \\
                        \hline
        Tru64 Unix V5.0A,V5.1  & BSD \\
                        \hline
                      Vax/VMS  & BSD \\
                        \hline
   Windows (95/98/NT4/W2K/XP)  & Windows\\
\hline
\end{tabular}

\end{itemize}
\end{itemize}

\subsubsection{Format}

Note in the advanced configuration below that there are three engines specified
running with \emph{Linux}, \texttt{first} and \texttt{last} policies assigned.
The first two engines are bound to specific IP address ranges and the last one
applies to all other traffic.  Packets that don't fall within the address
requirements of the first two engines automatically fall through to the third
one.

\paragraph{Basic Configuration}
\begin{verbatim}
    preprocessor frag3_global
    preprocessor frag3_engine
\end{verbatim}

\paragraph{Advanced Configuration}
\begin{verbatim}
    preprocessor frag3_global: prealloc_nodes 8192 
    preprocessor frag3_engine: policy linux bind_to 192.168.1.0/24
    preprocessor frag3_engine: policy first bind_to [10.1.47.0/24,172.16.8.0/24]
    preprocessor frag3_engine: policy last detect_anomalies
\end{verbatim}

\subsubsection{Frag 3 Alert Output}
\label{frag3 alert output}

Frag3 is capable of detecting eight different types of anomalies.  Its event
output is packet-based so it will work with all output modes of Snort.  Read
the documentation in the \texttt{doc/signatures} directory with filenames that
begin with ``123-'' for information on the different event types.

%%Need to doc these eight types of anomalies and truncate beginning of section.

\subsection{Session}
\label{session section}

The Session preprocessor is a global stream session management module
for Snort.  It is derived from the session management functions that
were part of the Stream5 preprocessor.

Since Session implements part of the functionality and API that was
previously in Stream5 it cannot be used with Stream5 but must be
used in conjunction with the new Stream preprocessor.  Similarly, 
due to the API changes, the other preprocessors in Snort 2.9.7 work
only with the new Session and Stream preprocessers.

\subsubsection{Session API}

Session provides an API to enable the creation and management of
the session control block for a flow and the management of data
and state that may be associated with that flow by service and
application preprocessors (most of these functions were previously
supported by the Stream5 API). These methods are called to identify
sessions that may be ignored (large data transfers, etc), and 
update the identifying information about the session (application
protocol, direction, etc) that can later be used by rules.  
API methods to enable preprocessors to register for dispatch for
ports and services for which they should be called to process 
the packet have been added to the Session API. Session is required
for the use of the 'flow' and 'flowbits' keywords.

\subsubsection{Session Global Configuration}

Global settings for the Session preprocessor.

\begin{verbatim}
    preprocessor stream5_global: \
        [track_tcp <yes|no>], [max_tcp <number>], \
        [memcap <number bytes>], \
        [track_udp <yes|no>], [max_udp <number>], \
        [track_icmp <yes|no>], [max_icmp <number>], \
        [track_ip <yes|no>], [max_ip <number>], \
        [flush_on_alert], [show_rebuilt_packets], \
        [prune_log_max <number bytes>], [disabled], \
        [enable_ha]
\end{verbatim}

\begin{center}
\begin{tabular}{| l | p{4.5in} |}

\hline
\textbf{Option} & \textbf{Description}\\
\hline 

\hline 
\texttt{track\_tcp <yes|no>} &

Track sessions for TCP.  The default is "yes".\\

\hline
\texttt{max\_tcp <num sessions>} &

Maximum simultaneous TCP sessions tracked.  The default is "262144", maximum is
"1048576", minimum is "2".\\

\hline
\texttt{memcap <num bytes>} &

Memcap for TCP packet storage.  The default is "8388608" (8MB), maximum is
"1073741824" (1GB), minimum is "32768" (32KB).\\

\hline
\texttt{track\_udp <yes|no>} &

Track sessions for UDP.  The default is "yes".\\

\hline
\texttt{max\_udp <num sessions>} &

Maximum simultaneous UDP sessions tracked.  The default is "131072", maximum is
"1048576", minimum is "1".\\

\hline
\texttt{track\_icmp <yes|no>} &

Track sessions for ICMP.  The default is "no".\\

\hline
\texttt{max\_icmp <num sessions>} &

Maximum simultaneous ICMP sessions tracked.  The default is "65536", maximum is
"1048576", minimum is "1".\\

\hline
\texttt{track\_ip <yes|no>} &

Track sessions for IP.  The default is "no".  Note that "IP" includes all
non-TCP/UDP traffic over IP including ICMP if ICMP not otherwise configured.\\

\hline
\texttt{max\_ip <num sessions>} &

Maximum simultaneous IP sessions tracked.  The default is "16384", maximum is
"1048576", minimum is "1".\\

\hline 
\texttt{disabled} &

Option to disable the stream5 tracking. By default this option is turned off.
When the preprocessor is disabled only the options memcap, max\_tcp, max\_udp 
and max\_icmp are applied when specified with the configuration.\\

\hline
\texttt{flush\_on\_alert} &

Backwards compatibility.  Flush a TCP stream when an alert is generated on that
stream.  The default is set to off.\\

\hline
\texttt{show\_rebuilt\_packets} &

Print/display packet after rebuilt (for debugging).  The default is set to
off.\\

\hline
\texttt{prune\_log\_max <num bytes>} &

Print a message when a session terminates that was consuming more than the
specified number of bytes.  The default is "1048576" (1MB), minimum can be either "0"
(disabled) or if not disabled the minimum is "1024" and maximum is "1073741824".\\

\hline
\texttt{enable\_ha} &


Enable High Availability state sharing.  The default is set to off.\\

\hline
\end{tabular}
\end{center}

\subsubsection{Session HA Configuration}

Configuration for HA session state sharing.

\begin{verbatim}
    preprocessor stream5_ha: [min_session_lifetime <num millisecs>], \
        [min_sync_interval <num millisecs>], [startup_input_file <filename>], \
        [runtime_output_file <filename>], [use_side_channel]
\end{verbatim}

\begin{center}
\begin{tabular}{| l | p{4.5in} |}

\hline
\textbf{Option} & \textbf{Description}\\
\hline 

\hline 
\texttt{min\_session\_lifetime <num millisecs>} &

Minimum session liftime in milliseconds.  HA update messages will only be generated once a session has existed for
at least this long.  The default is 0, the minimum is 0, and the maximum is 65535.\\

\hline 
\texttt{min\_sync\_interval <num millisecs>} &

Minimum synchronization interval in milliseconds.  HA update messages will not be generated more often than
once per interval on a given session.  The default is 0, the minimum is 0, and the maximum is 65535.\\

\hline
\texttt{startup\_input\_file <filename>} &

The name of a file for snort to read HA messages from at startup.\\

\hline
\texttt{runtime\_output\_file <filename>} &

The name of a file to which Snort will write all HA messages that are generated while it is running.\\

\hline
\texttt{use\_side\_channel} &

Indicates that all HA messages should also be sent to the side channel for processing.\\

\hline
\end{tabular}
\end{center}

\subsubsection{Example Configurations}

\begin{enumerate}

\item{}

This example configuration sets a maximum number of TCP session control blocks to 8192, 
enables tracking of TCP and UPD sessions, and disables tracking of ICMP sessions.  The 
number of UDP session control blocks will be set to the compiled default. 

\begin{verbatim}
    preprocessor stream5_global: \
        max_tcp 8192, track_tcp yes, track_udp yes, track_icmp no

    preprocessor stream5_tcp: \
        policy first, use_static_footprint_sizes

    preprocessor stream5_udp: \
        ignore_any_rules
\end{verbatim}

\end{enumerate}



\subsection{Stream}
\label{stream5 section}

The Stream preprocessor is a target-based TCP reassembly module for Snort.  It
is capable of tracking sessions for both TCP and UDP.  

\subsubsection{Transport Protocols}

TCP sessions are identified via the classic TCP "connection".  UDP sessions are
established as the result of a series of UDP packets from two end points via
the same set of ports.  ICMP messages are tracked for the purposes of checking
for unreachable and service unavailable messages, which effectively terminate a
TCP or UDP session.

\subsubsection{Target-Based}

Stream, like Frag3, introduces target-based actions for handling of
overlapping data and other TCP anomalies.  The methods for handling overlapping
data, TCP Timestamps, Data on SYN, FIN and Reset sequence numbers, etc. and the
policies supported by Stream are the results of extensive research with many
target operating systems.

\subsubsection{Stream API}

Stream supports the modified Stream API that is now focused on
functions specific to reassembly and protocol aware flushing
operations.  Session management functions have been moved to the
Session API.  The remaining API functions enable other protocol 
normalizers/preprocessors to dynamically configure reassembly 
behavior as required by the application layer protocol.

\subsubsection{Anomaly Detection}

TCP protocol anomalies, such as data on SYN packets, data received outside the
TCP window, etc are configured via the \texttt{detect\_anomalies} option to the
TCP configuration.  Some of these anomalies are detected on a per-target basis.
For example, a few operating systems allow data in TCP SYN packets, while
others do not.

\subsubsection{Protocol Aware Flushing}

Protocol aware flushing of HTTP, SMB and DCE/RPC can be enabled with this option:

\begin{verbatim}
config paf_max: <max-pdu>
\end{verbatim}

where \texttt{<max-pdu>} is between zero (off) and 63780.  This allows Snort to
statefully scan a stream and reassemble a complete PDU regardless of
segmentation.  For example, multiple PDUs within a single TCP segment, as well
as one PDU spanning multiple TCP segments will be reassembled into one PDU per
packet for each PDU.  PDUs larger than the configured maximum will be split
into multiple packets.

\subsubsection{Stream TCP Configuration}

Provides a means on a per IP address target to configure TCP policy.  This can
have multiple occurrences, per policy that is bound to an IP address or network.
One default policy must be specified, and that policy is not bound to an IP
address or network.

\begin{verbatim}
    preprocessor stream5_tcp: \
        [bind_to <ip_addr>], \
        [timeout <number secs>], [policy <policy_id>], \
        [overlap_limit <number>], [max_window <number>], \
        [require_3whs [<number secs>]], [detect_anomalies], \
        [check_session_hijacking], [use_static_footprint_sizes], \
        [dont_store_large_packets], [dont_reassemble_async], \
        [max_queued_bytes <bytes>], [max_queued_segs <number segs>], \
        [small_segments <number> bytes <number> [ignore_ports number [number]*]],  \
        [ports <client|server|both> <all|number|!number [number]* [!number]*>], \
        [protocol <client|server|both> <all|service name [service name]*>], \
        [ignore_any_rules], [flush_factor <number segs>]
\end{verbatim}

\begin{longtable}[h]{| p{2in} | p{4in} |}

\hline
\textbf{Option} & \textbf{Description}\\
\hline 

\hline 
\texttt{bind\_to <ip\_addr>} &

IP address or network for this policy.  The default is set to any.\\

\hline
\texttt{timeout <num seconds>} &

Session timeout.  The default is "30",  the minimum is "1", and the maximum is
"86400" (approximately 1 day).\\

\hline
\texttt{policy <policy\_id>} &

The Operating System policy for the target OS.  The policy\_id can be one of
the following:

\begin{tabular}{| l | p{2.5in} |}
\hline
Policy Name & Operating Systems.\\
\hline

\hline
\texttt{first} &

Favor first overlapped segment.\\

\hline
\texttt{last} & Favor first overlapped segment.\\

\hline
\texttt{bsd} & FresBSD 4.x and newer, NetBSD 2.x and newer, OpenBSD 3.x and newer\\

\hline
\texttt{linux} & Linux 2.4 and newer\\

\hline
\texttt{old-linux} & Linux 2.2 and earlier\\

\hline
\texttt{windows} & Windows 2000, Windows XP, Windows 95/98/ME\\

\hline
\texttt{win2003} & Windows 2003 Server\\

\hline
\texttt{vista} & Windows Vista\\

\hline
\texttt{solaris} & Solaris 9.x and newer\\

\hline
\texttt{hpux} & HPUX 11 and newer\\

\hline
\texttt{hpux10} & HPUX 10\\

\hline
\texttt{irix} & IRIX 6 and newer\\

\hline
\texttt{macos} & MacOS 10.3 and newer\\

\hline
\end{tabular}\\

\hline
\texttt{overlap\_limit <number>} &

Limits the number of overlapping packets per session.  The default is "0"
(unlimited), the minimum is "0", and the maximum is "255".\\

\hline
\texttt{max\_window <number>} &

Maximum TCP window allowed.  The default is "0" (unlimited), the minimum is
"0", and the maximum is "1073725440" (65535 left shift 14).  That is the
highest possible TCP window per RFCs.  This option is intended to prevent a DoS
against Stream by an attacker using an abnormally large window, so using a
value near the maximum is discouraged.\\

\hline
\texttt{require\_3whs [<number seconds>]} &

Establish sessions only on completion of a SYN/SYN-ACK/ACK handshake.  The
default is set to off.  The optional number of seconds specifies a startup
timeout.  This allows a grace period for existing sessions to be considered
established during that interval immediately after Snort is started.  The
default is "0" (don't consider existing sessions established), the minimum is
"0", and the maximum is "86400" (approximately 1 day).\\

\hline
\texttt{detect\_anomalies} &

Detect and alert on TCP protocol anomalies.  The default is set to off.\\

\hline
\texttt{check\_session\_hijacking} &

Check for TCP session hijacking.  This check validates the hardware (MAC)
address from both sides of the connect -- as established on the 3-way handshake
against subsequent packets received on the session.  If an ethernet layer is
not part of the protocol stack received by Snort, there are no checks
performed.  Alerts are generated (per '\texttt{detect\_anomalies}' option) for
either the client or server when the MAC address for one side or the other does
not match.  The default is set to off.\\

\hline
\texttt{use\_static\_footprint\_sizes} &

Use static values for determining when to build a reassembled packet to
allow for repeatable tests.  This option should not be used production
environments.  The default is set to off.\\

\hline
\texttt{dont\_store\_large\_packets} &

Performance improvement to not queue large packets in reassembly buffer.  The
default is set to off.  Using this option may result in missed attacks.\\

\hline
\texttt{dont\_reassemble\_async} &

Don't queue packets for reassembly if traffic has not been seen in both
directions.  The default is set to queue packets.\\

\hline
\texttt{max\_queued\_bytes <bytes>} &

Limit the number of bytes queued for reassembly on a given TCP session to
bytes.  Default is "1048576" (1MB).  A value of "0" means unlimited, with a
non-zero minimum of "1024", and a maximum of "1073741824" (1GB).  A message is
written to console/syslog when this limit is enforced.\\

\hline
\texttt{max\_queued\_segs <num>} &

Limit the number of segments queued for reassembly on a given TCP session.  The
default is "2621", derived based on an average size of 400 bytes.  A value of
"0" means unlimited, with a non-zero minimum of "2", and a maximum of
"1073741824" (1GB).  A message is written to console/syslog when this limit is
enforced.\\

\hline
\texttt{small\_segments <number> bytes <number> [ignore\_ports <number(s)> ]} &

Configure the maximum small segments queued. This feature requires that 
detect\_anomalies be enabled. The first number is the number of consecutive segments 
that will trigger the detection rule. The default value is "0" (disabled), with a 
maximum of "2048". The second number is the minimum bytes for a segment to be 
considered "small". The default value is "0" (disabled), with a maximum of "2048".  
ignore\_ports is optional, defines the list of ports in which will be ignored for 
this rule. The number of ports can be up to "65535".  A message is written to 
console/syslog when this limit is enforced.\\

\hline
\texttt{ports <client|server|both> <all|number(s)|!number(s)>} &


Specify the client, server, or both and list of ports in which to perform
reassembly.  This can appear more than once in a given config.  The default
settings are \texttt{ports client 21 23 25 42 53 80 110 111 135 136 137 139 143
445 513 514 1433 1521 2401 3306}.  The minimum port allowed is "1" and the
maximum allowed is "65535".  To disable reassembly for a port specifiy the port
number preceeded by an '!', e.g. !8080 !25\\

\hline
\texttt{protocol <client|server|both> <all|service name(s)>} &


Specify the client, server, or both and list of services in which to perform
reassembly.  This can appear more than once in a given config.  The default
settings are \texttt{ports client ftp telnet smtp nameserver dns http pop3
sunrpc dcerpc netbios-ssn imap login shell mssql oracle cvs mysql}.  The
service names can be any of those used in the host attribute table (see
\ref{targetbased}), including any of the internal defaults (see
\ref{attribute:service names}) or others specific to the network.\\

\hline
\texttt{ignore\_any\_rules} &

Don't process any \texttt{->} any (ports) rules for TCP that attempt to match
payload if there are no port specific rules for the src or destination port.
Rules that have flow or flowbits will never be ignored.  This is a performance
improvement and may result in missed attacks.  Using this does not affect rules
that look at protocol headers, only those with content, PCRE, or byte test
options.  The default is "off". This option can be used only in default
policy.\\

\hline
\texttt{flush\_factor} &

Useful in ips mode to flush upon seeing a drop in segment size after N
segments of non-decreasing size.  The drop in size often indicates an
end of request or response.\\

\hline
\end{longtable}

\begin{note}

If no options are specified for a given TCP policy, that is the default TCP
policy.  If only a bind\_to option is used with no other options that TCP
policy uses all of the default values.

\end{note}

\subsubsection{Stream UDP Configuration}

Configuration for UDP session tracking.  Since there is no target based
binding, there should be only one occurrence of the UDP configuration.

\begin{verbatim}
    preprocessor stream5_udp: [timeout <number secs>], [ignore_any_rules]
\end{verbatim}

\begin{center}
\begin{tabular}{| l | p{4.5in} |}

\hline
\textbf{Option} & \textbf{Description}\\
\hline 

\hline 
\texttt{timeout <num seconds>} &

Session timeout.  The default is "30", the minimum is "1", and the maximum is
"86400" (approximately 1 day).\\

\hline
\texttt{ignore\_any\_rules} &

Don't process any \texttt{->} any (ports) rules for UDP that attempt to match
payload if there are no port specific rules for the src or destination port.
Rules that have flow or flowbits will never be ignored.  This is a performance
improvement and may result in missed attacks.  Using this does not affect rules
that look at protocol headers, only those with content, PCRE, or byte test
options.  The default is "off".\\

\hline
\end{tabular}
\end{center}

\begin{note}

With the ignore\_any\_rules option, a UDP rule will be ignored except when
there is another port specific rule that may be applied to the traffic.  For
example, if a UDP rule specifies destination port 53, the 'ignored' any
\texttt{->} any rule will be applied to traffic to/from port 53, but NOT to any
other source or destination port.  A list of rule SIDs affected by this option
are printed at Snort's startup.

\end{note}

\begin{note}

With the ignore\_any\_rules option, if a UDP rule that uses any \texttt{->} any
ports includes either flow or flowbits, the ignore\_any\_rules option is
effectively pointless.  Because of the potential impact of disabling a flowbits
rule, the ignore\_any\_rules option will be disabled in this case.

\end{note}

\subsubsection{Stream ICMP Configuration}

Configuration for ICMP session tracking.  Since there is no target based
binding, there should be only one occurrence of the ICMP configuration.

\begin{note}

ICMP is currently untested, in minimal code form and is NOT ready for use in
production networks.  It is not turned on by default.

\end{note}

\begin{verbatim}
    preprocessor stream5_icmp: [timeout <number secs>]
\end{verbatim}

\begin{center}
\begin{tabular}{| l | p{4.5in} |}

\hline
\textbf{Option} & \textbf{Description}\\
\hline 

\hline 
\texttt{timeout <num seconds>} &

Session timeout.  The default is "30", the minimum is "1", and the maximum is
"86400" (approximately 1 day).\\

\hline
\end{tabular}
\end{center}

\subsubsection{Stream IP Configuration}

Configuration for IP session tracking.  Since there is no target based
binding, there should be only one occurrence of the IP configuration.

\begin{note}

"IP" includes all non-TCP/UDP traffic over IP including ICMP if ICMP
not otherwise configured.  It is not turned on by default.

\end{note}

\begin{verbatim}
    preprocessor stream5_ip: [timeout <number secs>]
\end{verbatim}

\begin{center}
\begin{tabular}{| l | p{4.5in} |}

\hline
\textbf{Option} & \textbf{Description}\\
\hline 

\hline 
\texttt{timeout <num seconds>} &

Session timeout.  The default is "30", the minimum is "1", and the maximum is
"86400" (approximately 1 day).\\

\hline
\end{tabular}
\end{center}

\subsubsection{Example Configurations}

\begin{enumerate}

\item{}

This example configuration is the default configuration in snort.conf and
can be used for repeatable tests of stream reassembly in readback mode.

\begin{verbatim}
    preprocessor stream5_global: \
        max_tcp 8192, track_tcp yes, track_udp yes, track_icmp no

    preprocessor stream5_tcp: \
        policy first, use_static_footprint_sizes

    preprocessor stream5_udp: \
        ignore_any_rules
\end{verbatim}

\item{}

This configuration maps two network segments to different OS policies, one for
Windows and one for Linux, with all other traffic going to the default policy
of Solaris.

\begin{verbatim}
    preprocessor stream5_global: track_tcp yes
    preprocessor stream5_tcp: bind_to 192.168.1.0/24, policy windows
    preprocessor stream5_tcp: bind_to 10.1.1.0/24, policy linux
    preprocessor stream5_tcp: policy solaris
\end{verbatim}

\end{enumerate}

\subsection{sfPortscan}

The sfPortscan module, developed by Sourcefire, is designed to detect the first
phase in a network attack: Reconnaissance. In the Reconnaissance phase, an
attacker determines what types of network protocols or services a host
supports. This is the traditional place where a portscan takes place. This
phase assumes the attacking host has no prior knowledge of what protocols or
services are supported by the target; otherwise, this phase would not be
necessary.

As the attacker has no beforehand knowledge of its intended target, most
queries sent by the attacker will be negative (meaning that the service ports
are closed). In the nature of legitimate network communications, negative
responses from hosts are rare, and rarer still are multiple negative responses
within a given amount of time.  Our primary objective in detecting portscans is
to detect and track these negative responses.

One of the most common portscanning tools in use today is Nmap. Nmap
encompasses many, if not all, of the current portscanning techniques.
sfPortscan was designed to be able to detect the different types of scans Nmap
can produce.

sfPortscan will currently alert for the following types of Nmap scans:

\begin{itemize}
\item TCP Portscan
\item UDP Portscan
\item IP Portscan
\end{itemize}

These alerts are for one$\rightarrow$one portscans, which are the traditional
types of scans; one host scans multiple ports on another host. Most of the port
queries will be negative, since most hosts have relatively few services
available.

sfPortscan also alerts for the following types of decoy portscans:

\begin{itemize}
\item TCP Decoy Portscan
\item UDP Decoy Portscan
\item IP Decoy Portscan
\end{itemize}

Decoy portscans are much like the Nmap portscans described above, only the
attacker has a spoofed source address inter-mixed with the real scanning
address. This tactic helps hide the true identity of the attacker.

sfPortscan alerts for the following types of distributed portscans:

\begin{itemize}
\item TCP Distributed Portscan
\item UDP Distributed Portscan
\item IP Distributed Portscan
\end{itemize}

These are many$\rightarrow$one portscans. Distributed portscans occur when
multiple hosts query one host for open services. This is used to evade an IDS
and obfuscate command and control hosts.

\begin{note}

Negative queries will be distributed among scanning hosts, so we track this
type of scan through the scanned host.

\end{note}

sfPortscan alerts for the following types of portsweeps:

\begin{itemize}
\item TCP Portsweep
\item UDP Portsweep
\item IP Portsweep
\item ICMP Portsweep
\end{itemize}

These alerts are for one$\rightarrow$many portsweeps. One host scans a single
port on multiple hosts. This usually occurs when a new exploit comes out and
the attacker is looking for a specific service. 

\begin{note}

The characteristics of a portsweep scan may not result in many negative
responses. For example, if an attacker portsweeps a web farm for port 80, we
will most likely not see many negative responses.

\end{note}

sfPortscan alerts on the following filtered portscans and portsweeps:

\begin{itemize}
\item TCP Filtered Portscan
\item UDP Filtered Portscan
\item IP Filtered Portscan

\item TCP Filtered Decoy Portscan
\item UDP Filtered Decoy Portscan
\item IP Filtered Decoy Portscan

\item TCP Filtered Portsweep
\item UDP Filtered Portsweep
\item IP Filtered Portsweep
\item ICMP Filtered Portsweep

\item TCP Filtered Distributed Portscan
\item UDP Filtered Distributed Portscan
\item IP Filtered Distributed Portscan
\end{itemize}

``Filtered'' alerts indicate that there were no network errors (ICMP
unreachables or TCP RSTs) or responses on closed ports have been suppressed.
It's also a good indicator of whether the alert is just a very active
legitimate host. Active hosts, such as NATs, can trigger these alerts because
they can send out many connection attempts within a very small amount of time.
A filtered alert may go off before responses from the remote hosts are
received.

sfPortscan only generates one alert for each host pair in question during the
time window (more on windows below). On TCP scan alerts, sfPortscan will also
display any open ports that were scanned. On TCP sweep alerts however,
sfPortscan will only track open ports after the alert has been triggered. Open
port events are not individual alerts, but tags based on the original scan
alert.

\subsubsection{sfPortscan Configuration}

Use of the Stream preprocessor is required for sfPortscan. Stream gives
portscan direction in the case of connectionless protocols like ICMP and UDP.
You should enable the Stream preprocessor in your \texttt{snort.conf}, as
described in Section \ref{stream5 section}.

The parameters you can use to configure the portscan module are:

\begin{slist}
\item \textbf{proto $<$protocol$>$}

Available options:

\begin{itemize}
\item \texttt{TCP}
\item \texttt{UDP}
\item \texttt{ICMP}
\item \texttt{ip\_proto}
\item \texttt{all}
\end{itemize}

\item \textbf{scan\_type $<$scan\_type$>$}

Available options: 

\begin{itemize}
\item \texttt{portscan} 
\item \texttt{portsweep} 
\item \texttt{decoy\_portscan}
\item \texttt{distributed\_portscan}
\item \texttt{all}
\end{itemize}

\item \textbf{sense\_level $<$level$>$}

Available options:

\begin{itemize}

\item \texttt{low} - ``Low'' alerts are only generated on error packets sent
from the target host, and because of the nature of error responses, this
setting should see very few false positives. However, this setting will never
trigger a Filtered Scan alert because of a lack of error responses. This
setting is based on a static time window of 60 seconds, after which this window
is reset.

\item \texttt{medium} - ``Medium'' alerts track connection counts, and so will
generate filtered scan alerts. This setting may false positive on active hosts
(NATs, proxies, DNS caches, etc), so the user may need to deploy the use of
Ignore directives to properly tune this directive.

\item \texttt{high} - ``High'' alerts continuously track hosts on a network
using a time window to evaluate portscan statistics for that host. A "High"
setting will catch some slow scans because of the continuous monitoring, but is
very sensitive to active hosts. This most definitely will require the user to
tune sfPortscan.

\end{itemize}

\item \textbf{watch\_ip $<$ip1$|$ip2/cidr[ [port$|$port2-port3]]$>$ }

Defines which IPs, networks, and specific ports on those hosts to watch.  The
list is a comma separated list of IP addresses, IP address using CIDR notation.
Optionally, ports are specified after the IP address/CIDR using a space and can
be either a single port or a range denoted by a dash.  IPs or networks not
falling into this range are ignored if this option is used.

\item \textbf{ignore\_scanners $<$ip1$|$ip2/cidr[ [port$|$port2-port3]]$>$ }

Ignores the source of scan alerts.  The parameter is the same format as that of
\texttt{watch\_ip}.

\item \textbf{ignore\_scanned $<$ip1$|$ip2/cidr[ [port$|$port2-port3]]$>$ }

Ignores the destination of scan alerts.  The parameter is the same format as
that of \texttt{watch\_ip}.

\item \textbf{logfile $<$file$>$ } 

This option will output portscan events to the file specified. If \texttt{file}
does not contain a leading slash, this file will be placed in the Snort config
dir.

\item \textbf{include\_midstream}

This option will include sessions picked up in midstream by Stream.
This can lead to false alerts, especially under heavy load with dropped
packets; which is why the option is off by default.

\item \textbf{detect\_ack\_scans}

This option will include sessions picked up in midstream by the stream module,
which is necessary to detect ACK scans.  However, this can lead to false
alerts, especially under heavy load with dropped packets; which is why the
option is off by default.

\item \textbf{disabled}

This optional keyword is allowed with any policy to avoid packet processing.
This option disables the preprocessor. When the preprocessor is disabled
only the memcap option is applied when specified with the configuration.
The other options are parsed but not used. Any valid configuration may have
"disabled" added to it.

\end{slist}

\subsubsection{Format}

\begin{verbatim}
    preprocessor sfportscan: proto <protocols> \
        scan_type <portscan|portsweep|decoy_portscan|distributed_portscan|all> \
        sense_level <low|medium|high> \
        watch_ip <IP or IP/CIDR> \
        ignore_scanners <IP list> \
        ignore_scanned <IP list> \
        logfile <path and filename> \
        disabled 
\end{verbatim}

\subsubsection{Example}

\begin{verbatim}
    preprocessor flow: stats_interval 0 hash 2
    preprocessor sfportscan:\
        proto { all } \
        scan_type { all } \
        sense_level { low }
\end{verbatim}

\subsubsection{sfPortscan Alert Output}

\paragraph{Unified Output}

In order to get all the portscan information logged with the alert, snort
generates a pseudo-packet and uses the payload portion to store the additional
portscan information of priority count, connection count, IP count, port count,
IP range, and port range.  The characteristics of the packet are:

\begin{verbatim}
    Src/Dst MAC Addr == MACDAD
    IP Protocol == 255
    IP TTL == 0
\end{verbatim}

Other than that, the packet looks like the IP portion of the packet that caused
the portscan alert to be generated.  This includes any IP options, etc.  The
payload and payload size of the packet are equal to the length of the
additional portscan information that is logged.  The size tends to be around
100 - 200 bytes.

Open port alerts differ from the other portscan alerts, because open port
alerts utilize the tagged packet output system.  This means that if an output
system that doesn't print tagged packets is used, then the user won't see open
port alerts.  The open port information is stored in the IP payload and
contains the port that is open.

The sfPortscan alert output was designed to work with unified2 packet logging,
so it is possible to extend favorite Snort GUIs to display portscan alerts and
the additional information in the IP payload using the above packet
characteristics.

\paragraph{Log File Output}

Log file output is displayed in the following format, and explained further
below:

\begin{verbatim}
    Time: 09/08-15:07:31.603880
    event_id: 2
    192.168.169.3 -> 192.168.169.5 (portscan) TCP Filtered Portscan
    Priority Count: 0
    Connection Count: 200
    IP Count: 2
    Scanner IP Range: 192.168.169.3:192.168.169.4
    Port/Proto Count: 200
    Port/Proto Range: 20:47557
\end{verbatim}

If there are open ports on the target, one or more additional tagged packet(s)
will be appended:

\begin{verbatim}
    Time: 09/08-15:07:31.603881
    event_ref: 2
    192.168.169.3 -> 192.168.169.5 (portscan) Open Port
    Open Port: 38458
\end{verbatim}

\begin{slist}

\item \textbf{Event\_id/Event\_ref}

These fields are used to link an alert with the corresponding \texttt{Open
Port} tagged packet

\item \textbf{Priority Count}

\texttt{Priority Count} keeps track of bad responses (resets, unreachables).
The higher the priority count, the more bad responses have been received.

\item \textbf{Connection Count}
     
\texttt{Connection Count} lists how many connections are active on the hosts
(src or dst). This is accurate for connection-based protocols, and is more of
an estimate for others. Whether or not a portscan was filtered is determined
here. High connection count and low priority count would indicate filtered (no
response received from target).

\item \textbf{IP Count}

IP Count keeps track of the last IP to contact a host, and increments the count
if the next IP is different. For one-to-one scans, this is a low number. For
active hosts this number will be high regardless, and one-to-one scans may
appear as a distributed scan.

\item \textbf{Scanned/Scanner IP Range}

This field changes depending on the type of alert. Portsweep (one-to-many)
scans display the scanned IP range; Portscans (one-to-one) display the scanner
IP. 

\item \textbf{Port Count}

Port Count keeps track of the last port contacted and increments this number
when that changes. We use this count (along with IP Count) to determine the
difference between one-to-one portscans and one-to-one decoys.  \end{slist}

\subsubsection{Tuning sfPortscan}
\label{tuning sfportscan}

The most important aspect in detecting portscans is tuning the detection engine
for your network(s).  Here are some tuning tips:

\begin{slist}

\item \textbf{Use the watch\_ip, ignore\_scanners, and ignore\_scanned options.}
  
It's important to correctly set these options.  The \texttt{watch\_ip} option
is easy to understand.  The analyst should set this option to the list of CIDR
blocks and IPs that they want to watch.  If no \texttt{watch\_ip} is defined,
sfPortscan will watch all network traffic.
     
The \texttt{ignore\_scanners} and \texttt{ignore\_scanned} options come into
play in weeding out legitimate hosts that are very active on your network.
Some of the most common examples are NAT IPs, DNS cache servers, syslog
servers, and nfs servers.  sfPortscan may not generate false positives for
these types of hosts, but be aware when first tuning sfPortscan for these IPs.
Depending on the type of alert that the host generates, the analyst will know
which to ignore it as.  If the host is generating portsweep events, then add it
to the \texttt{ignore\_scanners} option.  If the host is generating portscan
alerts (and is the host that is being scanned), add it to the
\texttt{ignore\_scanned} option.
  
\item \textbf{Filtered scan alerts are much more prone to false positives.}
  
When determining false positives, the alert type is very important.  Most of
the false positives that sfPortscan may generate are of the filtered scan alert
type.  So be much more suspicious of filtered portscans.  Many times this just
indicates that a host was very active during the time period in question.  If
the host continually generates these types of alerts, add it to the
\texttt{ignore\_scanners} list or use a lower sensitivity level.
     
\item \textbf{Make use of the Priority Count, Connection Count, IP Count, Port
Count, IP Range, and Port Range to determine false positives.}
     
The portscan alert details are vital in determining the scope of a portscan and
also the confidence of the portscan.  In the future, we hope to automate much
of this analysis in assigning a scope level and confidence level, but for now
the user must manually do this.  The easiest way to determine false positives
is through simple ratio estimations.  The following is a list of ratios to
estimate and the associated values that indicate a legitimate scan and not a
false positive.
     
\textbf{Connection Count / IP Count:}  This ratio indicates an estimated
average of connections per IP.  For portscans, this ratio should be high, the
higher the better.  For portsweeps, this ratio should be low.
     
\textbf{Port Count / IP Count:}  This ratio indicates an estimated average of
ports connected to per IP.  For portscans, this ratio should be high and
indicates that the scanned host's ports were connected to by fewer IPs.  For
portsweeps, this ratio should be low, indicating that the scanning host
connected to few ports but on many hosts.
     
\textbf{Connection Count / Port Count:}  This ratio indicates an estimated
average of connections per port.  For portscans, this ratio should be low.
This indicates that each connection was to a different port.  For portsweeps,
this ratio should be high.  This indicates that there were many connections to
the same port.
     
The reason that \texttt{Priority Count} is not included, is because the
priority count is included in the connection count and the above comparisons
take that into consideration.  The Priority Count play an important role in
tuning because the higher the priority count the more likely it is a real
portscan or portsweep (unless the host is firewalled).
     
\item \textbf{If all else fails, lower the sensitivity level.}
     
If none of these other tuning techniques work or the analyst doesn't have the
time for tuning, lower the sensitivity level.  You get the best protection the
higher the sensitivity level, but it's also important that the portscan
detection engine generate alerts that the analyst will find informative.  The
low sensitivity level only generates alerts based on error responses.  These
responses indicate a portscan and the alerts generated by the low sensitivity
level are highly accurate and require the least tuning.  The low sensitivity
level does not catch filtered scans; since these are more prone to false
positives.  \end{slist}

\subsection{RPC Decode}
\label{sub:rpc-decoder}

The rpc\_decode preprocessor normalizes RPC multiple fragmented records into a
single un-fragmented record.  It does this by normalizing the packet into the
packet buffer.  If stream5 is enabled, it will only process client-side
traffic.  By default, it runs against traffic on ports 111 and 32771.

\subsubsection{Format}

\begin{verbatim}
    preprocessor rpc_decode: \
        <ports> [ alert_fragments ] \
        [no_alert_multiple_requests] \
        [no_alert_large_fragments] \
        [no_alert_incomplete]
\end{verbatim}

\begin{table}[h]
\begin{center}
\begin{tabular}{| l | l |}

\hline 
\textbf{Option}& \textbf{Description}\\
\hline

\hline 
\texttt{alert\_fragments}&

Alert on any fragmented RPC record.\\

\hline 
\texttt{no\_alert\_multiple\_requests}&

Don't alert when there are multiple records in one packet.\\

\hline 
\texttt{no\_alert\_large\_fragments}&

Don't alert when the sum of fragmented records exceeds one packet.\\

\hline 
\texttt{no\_alert\_incomplete}&

Don't alert when a single fragment record exceeds the size of one packet.\\

\hline
\end{tabular}
\end{center}
\end{table}

\subsection{Performance Monitor}
\label{sub:perfmonitor}

This preprocessor measures Snort's real-time and theoretical maximum
performance.  Whenever this preprocessor is turned on, it should have an output
mode enabled, either ``console'' which prints statistics to the console window
or ``file'' with a file name, where statistics get printed to the specified
file name. By default, Snort's real-time statistics are processed. This
includes:

\begin{itemize}
\item Time Stamp
\item Drop Rate
\item Mbits/Sec (wire) [duplicated below for easy comparison with other rates]
\item Alerts/Sec 
\item K-Pkts/Sec (wire) [duplicated below for easy comparison with other rates]
\item Avg Bytes/Pkt (wire) [duplicated below for easy comparison with other rates]
\item Pat-Matched [percent of data received that Snort processes in pattern matching]
\item Syns/Sec
\item SynAcks/Sec
\item New Sessions Cached/Sec
\item Sessions Del fr Cache/Sec
\item Current Cached Sessions
\item Max Cached Sessions
\item Stream Flushes/Sec
\item Stream Session Cache Faults
\item Stream Session Cache Timeouts
\item New Frag Trackers/Sec
\item Frag-Completes/Sec
\item Frag-Inserts/Sec
\item Frag-Deletes/Sec
\item Frag-Auto Deletes/Sec [memory DoS protection]
\item Frag-Flushes/Sec
\item Frag-Current [number of current Frag Trackers]
\item Frag-Max [max number of Frag Trackers at any time]
\item Frag-Timeouts
\item Frag-Faults
\item Number of CPUs [*** Only if compiled with LINUX\_SMP ***, the next three appear for each CPU]
\item CPU usage (user)
\item CPU usage (sys)
\item CPU usage (Idle)
\item Mbits/Sec (wire) [average mbits of total traffic]
\item Mbits/Sec (ipfrag) [average mbits of IP fragmented traffic]
\item Mbits/Sec (ipreass) [average mbits Snort injects after IP reassembly]
\item Mbits/Sec (tcprebuilt) [average mbits Snort injects after TCP reassembly]
\item Mbits/Sec (applayer) [average mbits seen by rules and protocol decoders]
\item Avg Bytes/Pkt (wire)
\item Avg Bytes/Pkt (ipfrag)
\item Avg Bytes/Pkt (ipreass)
\item Avg Bytes/Pkt (tcprebuilt)
\item Avg Bytes/Pkt (applayer)
\item K-Pkts/Sec (wire)
\item K-Pkts/Sec (ipfrag)
\item K-Pkts/Sec (ipreass)
\item K-Pkts/Sec (tcprebuilt)
\item K-Pkts/Sec (applayer)
\item Total Packets Received
\item Total Packets Dropped (not processed)
\item Total Packets Blocked (inline)
\item Percentage of Packets Dropped
\item Total Filtered TCP Packets
\item Total Filtered UDP Packets
\item Midstream TCP Sessions/Sec
\item Closed TCP Sessions/Sec
\item Pruned TCP Sessions/Sec
\item TimedOut TCP Sessions/Sec
\item Dropped Async TCP Sessions/Sec
\item TCP Sessions Initializing
\item TCP Sessions Established
\item TCP Sessions Closing
\item Max TCP Sessions (interval)
\item New Cached UDP Sessions/Sec
\item Cached UDP Ssns Del/Sec
\item Current Cached UDP Sessions
\item Max Cached UDP Sessions
\item Current Attribute Table Hosts (Target Based)
\item Attribute Table Reloads (Target Based)
\item Mbits/Sec (Snort)
\item Mbits/Sec (sniffing)
\item Mbits/Sec (combined)
\item uSeconds/Pkt (Snort)
\item uSeconds/Pkt (sniffing)
\item uSeconds/Pkt (combined)
\item KPkts/Sec (Snort)
\item KPkts/Sec (sniffing)
\item KPkts/Sec (combined)
\end{itemize}

There are over 100 individual statistics included.  A header line is output at startup and
rollover that labels each column.

The following options can be used with the performance monitor:

\begin{itemize}

\item \texttt{flow} - Prints out statistics about the type and amount of traffic
and protocol distributions that Snort is seeing. This option can produce large
amounts of output.

\item \texttt{flow-file} - Prints \texttt{flow} statistics in a comma-delimited
format to the file that is specified.
\begin{itemize}
\item Timestamp
\item Total \% TCP bytes
\item Total \% UDP bytes
\item Total \% ICMP bytes
\item Total \% OTHER bytes
\item Number of Packet length entries
\item Packet length entries - bytes,\%total
\item Number of TCP port flow entries
\item TCP port flow entries : port,\%total,\%src,\%dst
\item \% TCP high port to high port
\item Number of UDP port flow entries
\item UDP port flow entries : port,\%total,\%src,\%dst
\item \% UDP high port to high port
\item Number of ICMP type entries
\item ICMP type entries : type,\%total
\end{itemize}
Specifying this option implicitly enables \texttt{flow} statistics.

\item \texttt{events} - Turns on event reporting.  This prints out statistics
as to the number of rules that were evaluated and didn't match
(\textit{non-qualified events}) vs. the number of rules that were evaluated and
matched (\textit{qualified events}).  A high \textit{non-qualified event} to
\textit{qualified event} ratio can indicate there are many rules with either
minimal content or no content that are being evaluated without success.  The
fast pattern matcher is used to select a set of rules for evaluation based on
the longest \texttt{content} or a \texttt{content} modified with the
\texttt{fast\_pattern} rule option in a rule.  Rules with short, generic
contents are more likely to be selected for evaluation than those with
longer, more unique contents.  Rules without \texttt{content} are not
filtered via the fast pattern matcher and are always evaluated, so if
possible, adding a \texttt{content} rule option to those rules can decrease the
number of times they need to be evaluated and improve performance.

\item \texttt{max} - Turns on the theoretical maximum performance that Snort
calculates given the processor speed and current performance.  This is only
valid for uniprocessor machines, since many operating systems don't keep
accurate kernel statistics for multiple CPUs.  

\item \texttt{console} - Prints statistics at the console.

\item \texttt{file} - Prints statistics in a comma-delimited format to the file
that is specified.  Not all statistics are output to this file.  You may also
use \texttt{snortfile} which will output into your defined Snort log directory.
Both of these directives can be overridden on the command line with the
\texttt{-Z} or \texttt{--perfmon-file} options.  At startup, Snort will log
a distinctive line to this file with a timestamp to all readers to easily identify
gaps in the stats caused by Snort not running.

\item \texttt{pktcnt} - Adjusts the number of packets to process before
checking for the time sample.  This boosts performance, since checking the time
sample reduces Snort's performance.  By default, this is 10000.  

\item \texttt{time} - Represents the number of seconds between intervals.

\item \texttt{accumulate} or \texttt{reset} - Defines which type of drop
statistics are kept by the operating system. By default, \texttt{reset} is
used.

\item \texttt{atexitonly} - Dump stats for entire life of Snort.
One or more of the following arguments can be given to specify specific
statistic types to dump at exit:
\begin{itemize}
\item \texttt{base-stats}
\item \texttt{flow-stats}
\item \texttt{flow-ip-stats}
\item \texttt{events-stats}
\end{itemize}
Without any arguments, all enabled stats will be dumped only when Snort exits.

\item \texttt{max\_file\_size} - Defines the maximum size of the
comma-delimited file.  Before the file exceeds this size, it will be rolled
into a new date stamped file of the format YYYY-MM-DD, followed by
YYYY-MM-DD.x, where x will be incremented each time the comma delimited file
is rolled over.  The minimum is 4096 bytes and the maximum is 2147483648 bytes
(2GB).  The default is the same as the maximum.

\item \texttt{flow-ip} - Collects IP traffic distribution statistics based on
host pairs.  For each pair of hosts for which IP traffic has been seen, the
following statistics are collected for both directions (A to B and B to A):
\begin{itemize}
\item TCP Packets
\item TCP Traffic in Bytes
\item TCP Sessions Established
\item TCP Sessions Closed
\item UDP Packets
\item UDP Traffic in Bytes
\item UDP Sessions Created
\item Other IP Packets
\item Other IP Traffic in Bytes
\end{itemize}
These statistics are printed and reset at the end of each interval.

\item \texttt{flow-ip-file} - Prints the flow IP statistics in a
comma-delimited format to the file that is specified.  All of the statistics
mentioned above, as well as the IP addresses of the host pairs in
human-readable format, are included.

Each line in the file will have its values correspond (in order) to those below:
\begin{itemize}
\item IP Address A (String)
\item IP Address B (String)
\item TCP Packets from A to B
\item TCP Traffic in Bytes from A to B
\item TCP Packets from B to A
\item TCP Traffic in Bytes from B to A
\item UDP Packets from A to B
\item UDP Traffic in Bytes from A to B
\item UDP Packets from B to A
\item UDP Traffic in Bytes from B to A
\item Other IP Packets from A to B
\item Other IP Traffic in Bytes from A to B
\item Other IP Packets from B to A
\item Other IP Traffic in Bytes from B to A
\item TCP Sessions Established
\item TCP Sessions Closed
\item UDP Sessions Created
\end{itemize}

\item \texttt{flow-ip-memcap} - Sets the memory cap on the hash table used to
store IP traffic statistics for host pairs.  Once the cap has been reached, the
table will start to prune the statistics for the least recently seen host pairs
to free memory.  This value is in bytes and the default value is
52428800 (50MB).

\end{itemize}
\subsubsection{Examples}

\begin{verbatim}
    preprocessor perfmonitor: \
        time 30 events flow file stats.profile max console pktcnt 10000 

    preprocessor perfmonitor: \
        time 300 file /var/tmp/snortstat pktcnt 10000

    preprocessor perfmonitor: \
        time 30 flow-ip flow-ip-file flow-ip-stats.csv pktcnt 1000

    preprocessor perfmonitor: \
        time 30 pktcnt 1000 snortfile base.csv flow-file flows.csv atexitonly flow-stats

    preprocessor perfmonitor: \
        time 30 pktcnt 1000 flow events atexitonly base-stats flow-stats console
\end{verbatim}

\subsection{HTTP Inspect}
\label{sub:http-inspect}

HTTP Inspect is a generic HTTP decoder for user applications.  Given a data
buffer, HTTP Inspect will decode the buffer, find HTTP fields, and normalize
the fields.  HTTP Inspect works on both client requests and server responses.

HTTP Inspect has a very ``rich'' user configuration.  Users can configure
individual HTTP servers with a variety of options, which should allow the user
to emulate any type of web server. Within HTTP Inspect, there are two areas of
configuration: global and server.

\subsubsection{Global Configuration}

The global configuration deals with configuration options that determine the
global functioning of HTTP Inspect.  The following example gives the generic
global configuration format:

\subsubsection{Format}
\begin{verbatim}
    preprocessor http_inspect: \
        global \
        iis_unicode_map <map_filename> \
        codemap <integer> \
        [detect_anomalous_servers] \
        [proxy_alert] \
        [max_gzip_mem <num>] \
        [compress_depth <num>] [decompress_depth <num>] \
        [memcap <num>] \
        disabled
\end{verbatim}

You can only have a single global configuration, you'll get an error if you try
otherwise.

\paragraph{Configuration}
\begin{slist}
\item \texttt{iis\_unicode\_map $<$map\_filename$>$ [codemap $<$integer$>$]}

This is the global \texttt{iis\_unicode\_map} file.  The
\texttt{iis\_unicode\_map} is a required configuration parameter.  The map file
can reside in the same directory as \texttt{snort.conf} or be specified via a
fully-qualified path to the map file.

The \texttt{iis\_unicode\_map} file is a Unicode codepoint map which tells HTTP
Inspect which codepage to use when decoding Unicode characters.  For US
servers, the codemap is usually 1252.

A Microsoft US Unicode codepoint map is provided in the Snort source
\texttt{etc} directory by default.  It is called \texttt{unicode.map} and
should be used if no other codepoint map is available.  A tool is supplied with
Snort to generate custom Unicode \texttt{maps--ms\_unicode\_generator.c}, which
is available at \url{http://www.snort.org/dl/contrib/}.

\begin{note}

Remember that this configuration is for the global IIS Unicode map, individual
servers can reference their own IIS Unicode map.

\end{note}

\item \texttt{detect\_anomalous\_servers}

This global configuration option enables generic HTTP server traffic inspection
on non-HTTP configured ports, and alerts if HTTP traffic is seen.  Don't turn
this on if you don't have a default server configuration that encompasses all
of the HTTP server ports that your users might access.  In the future, we want
to limit this to specific networks so it's more useful, but for right now, this
inspects all network traffic. This option is turned off by default.

\item \texttt{proxy\_alert}

This enables global alerting on HTTP server proxy usage.  By configuring HTTP
Inspect servers and enabling \texttt{allow\_proxy\_use}, you will only receive
proxy use alerts for web users that aren't using the configured proxies or are
using a rogue proxy server.

Please note that if users aren't required to configure web proxy use, then you
may get a lot of proxy alerts.  So, please only use this feature with
traditional proxy environments. Blind firewall proxies don't count.

\item \texttt{compress\_depth $<$integer$>$}
This option specifies the maximum amount of packet payload to decompress. This
value can be set from 1 to 65535. The default for this option is 1460.

\begin{note}

Please note, in case of multiple policies, the value specified in the default policy
is used and this value overwrites the values specified in the other policies. In case
of \texttt{unlimited\_decompress} this should be set to its max value. This value should 
be specified in the default policy even when the HTTP inspect preprocessor is turned off 
using the \texttt{disabled} keyword.

\end{note}

\item \texttt{decompress\_depth $<$integer$>$}
This option specifies the maximum amount of decompressed data to obtain from the
compressed packet payload. This value can be set from 1 to 65535. The default for
this option is 2920.

\begin{note}

Please note, in case of multiple policies, the value specified in the default policy
is used and this value overwrites the values specified in the other policies. In case
of \texttt{unlimited\_decompress} this should be set to its max value. This value should 
be specified in the default policy even when the HTTP inspect preprocessor is turned off 
using the \texttt{disabled} keyword.

\end{note}

\item \texttt{max\_gzip\_mem $<$integer$>$}

This option determines (in bytes) the maximum amount of memory the HTTP Inspect preprocessor 
will use for decompression. The minimum allowed value for this option is 3276 bytes. This option
determines the number of concurrent sessions that can be decompressed at any given instant.
The default value for this option is 838860.

This value is also used for the optional SWF/PDF file decompression.  If these modes are enabled
this same value sets the maximum about of memory used for file decompression session state
information.

\begin{note}

This value should be specified in the default policy even when the HTTP inspect preprocessor is 
turned off using the \texttt{disabled} keyword.

\end{note}

\item \texttt{memcap $<$integer$>$}

This option determines (in bytes) the maximum amount of memory the HTTP Inspect preprocessor
will use for logging the URI and Hostname data. This value can be set from 2304 to 603979776 (576 MB).
This option along with the maximum uri and hostname logging size (which is defined in snort) will
determine the maximum HTTP sessions that will log the URI and hostname data at any given instant. The
maximum size for logging URI data is 2048 and for hostname is 256. The default value for this
option is 150994944 (144 MB).

\begin {note}

This value should be specified in the default policy even when the HTTP inspect preprocessor is turned off 
using the \texttt{disabled} keyword. In case of multiple policies, the value specified in the
default policy will overwrite the value specified in other policies.

max http sessions logged = memcap /( max uri logging size + max hostname logging size )
max uri logging size defined in snort : 2048
max hostname logging size defined in snort : 256

\end{note}

\item \texttt{disabled}

This optional keyword is allowed with any policy to avoid packet processing.
This option disables the preprocessor. When the preprocessor is disabled
only the "memcap", "max\_gzip\_mem", "compress\_depth" and "decompress\_depth" 
options are applied when specified with the configuration. Other options are
parsed but not used. Any valid configuration may have "disabled" added to it.

\end{slist}
\subsubsection{Example Global Configuration}

\begin{verbatim}
    preprocessor http_inspect: \
        global iis_unicode_map unicode.map 1252
\end{verbatim}

\subsubsection{Server Configuration}

There are two types of server configurations: default and by IP address.

\paragraph{Default}

This configuration supplies the default server configuration for any server
that is not individually configured.  Most of your web servers will most likely
end up using the default configuration.

\subsubsection{Example Default Configuration}

\begin{verbatim}
    preprocessor http_inspect_server: \
        server default profile all ports { 80 }
\end{verbatim}

\paragraph{Configuration by IP Address}

This format is very similar to ``default'', the only difference being that
specific IPs can be configured.

\subsubsection{Example IP Configuration}

\begin{verbatim}
    preprocessor http_inspect_server: \
        server 10.1.1.1 profile all ports { 80 }
\end{verbatim}

\paragraph{Configuration by Multiple IP Addresses}

This format is very similar to ``Configuration by IP Address'', the only
difference being that multiple IPs can be specified via a space separated list.
There is a limit of 40 IP addresses or CIDR notations per
\texttt{http\_inspect\_server} line.

\subsubsection{Example Multiple IP Configuration}

\begin{verbatim}
    preprocessor http_inspect_server: \
        server { 10.1.1.1 10.2.2.0/24 } profile all ports { 80 }
\end{verbatim}

\subsubsection{Server Configuration Options}

Important: Some configuration options have an argument of `yes' or `no'.  This
argument specifies whether the user wants the configuration option to generate
an HTTP Inspect alert or not.  The `yes/no' argument does not specify whether
the configuration option itself is on or off, only the alerting functionality.
In other words, whether set to `yes' or 'no', HTTP normalization will still
occur, and rules based on HTTP traffic will still trigger.

\begin{slist}
\item \texttt{profile $<$all$|$apache$|$iis$|$iis5\_0$|$iis4\_0$>$}

Users can configure HTTP Inspect by using pre-defined HTTP server profiles.
Profiles allow the user to easily configure the preprocessor for a certain type
of server, but are not required for proper operation. 

There are five profiles available: all, apache, iis, iis5\_0, and iis4\_0.

\begin{subslist}

\item \texttt{all}

The \texttt{all} profile is meant to normalize the URI using most of the common
tricks available.  We alert on the more serious forms of evasions.  This is a
great profile for detecting all types of attacks, regardless of the HTTP
server.  \texttt{profile all} sets the configuration options described in Table
\ref{profile_all_options}.

\begin{table}[h]
\begin{center}
\caption{Options for the ``all'' Profile}
\label{profile_all_options}
\begin{tabular}{| l | p{3in} |}

\hline
\textbf{Option} & \textbf{Setting} \\
\hline

\hline
server\_flow\_depth & 300 \\
\hline
client\_flow\_depth & 300 \\
\hline
post\_depth & 0 \\
\hline
chunk encoding & alert on chunks larger than 500000 bytes \\
\hline
iis\_unicode\_map & codepoint map in the global configuration \\
\hline
ASCII decoding & on, alert off \\
\hline
multiple slash & on, alert off \\
\hline
directory normalization & on, alert off \\
\hline
apache whitespace & on, alert off \\
\hline
double decoding & on, alert on \\
\hline
\%u decoding & on, alert on \\
\hline
bare byte decoding & on, alert on \\
\hline
iis unicode codepoints & on, alert on \\
\hline
iis backslash & on, alert off \\
\hline
iis delimiter & on, alert off \\
\hline
webroot & on, alert on\\
\hline
non\_strict URL parsing & on\\
\hline
tab\_uri\_delimiter & is set\\
\hline
max\_header\_length & 0, header length not checked\\
\hline
max\_spaces & 200 \\
\hline
max\_headers & 0, number of headers not checked\\

\hline
\end{tabular}
\end{center}
\end{table}

\item \texttt{apache} 

The \texttt{apache} profile is used for Apache web servers.  This differs from
the \texttt{iis} profile by only accepting UTF-8 standard Unicode encoding and
not accepting backslashes as legitimate slashes, like IIS does.  Apache also
accepts tabs as whitespace.  \texttt{profile apache} sets the configuration
options described in Table \ref{profile_apache_options}.

\begin{table}[h]
\begin{center}
\caption{Options for the \texttt{apache} Profile}
\label{profile_apache_options}
\begin{tabular}{| l | p{3in} |}

\hline
\textbf{Option} & \textbf{Setting}\\
\hline

\hline
server\_flow\_depth & 300 \\
\hline
client\_flow\_depth & 300 \\
\hline
post\_depth & 0 \\
\hline
chunk encoding & alert on chunks larger than 500000 bytes \\
\hline
ASCII decoding & on, alert off \\
\hline
multiple slash & on, alert off \\
\hline
directory normalization & on, alert off \\
\hline
webroot & on, alert on\\
\hline
apache whitespace & on, alert on \\
\hline
utf\_8 encoding & on, alert off \\
\hline
non\_strict url parsing & on \\
\hline
tab\_uri\_delimiter & is set\\
\hline
max\_header\_length & 0, header length not checked\\
\hline
max\_spaces & 200 \\
\hline
max\_headers & 0, number of headers not checked\\
\hline

\hline
\end{tabular}
\end{center}
\end{table}

\item \texttt{iis}

The \texttt{iis} profile mimics IIS servers.  So that means we use IIS Unicode
codemaps for each server, \%u encoding, bare-byte encoding, double decoding,
backslashes, etc. \texttt{profile iis} sets the configuration options described
in Table \ref{profile_iis_options}.

\begin{table}[h]
\begin{center}
\caption{Options for the \texttt{iis} Profile}
\label{profile_iis_options}
\begin{tabular}{| l | p{3in} |}

\hline
\textbf{Option} & \textbf{Setting}\\
\hline

\hline
server\_flow\_depth & 300 \\
\hline
client\_flow\_depth & 300 \\
\hline
post\_depth & -1 \\
\hline
chunk encoding & alert on chunks larger than 500000 bytes\\
\hline
iis\_unicode\_map & codepoint map in the global configuration \\
\hline
ASCII decoding & on, alert off \\
\hline
multiple slash & on, alert off \\
\hline
directory normalization & on, alert off \\
\hline
webroot & on, alert on\\
\hline
double decoding & on, alert on \\
\hline
\%u decoding & on, alert on \\
\hline
bare byte decoding & on, alert on \\
\hline
iis unicode codepoints & on, alert on \\
\hline
iis backslash & on, alert off \\
\hline
iis delimiter & on, alert on \\
\hline
apache whitespace & on, alert on \\
\hline
non\_strict URL parsing & on\\
\hline
max\_header\_length & 0, header length not checked\\
\hline
max\_spaces & 200 \\
\hline
max\_headers & 0, number of headers not checked\\

\hline
\end{tabular}
\end{center}
\end{table}

\item \texttt{iis4\_0, iis5\_0}

In IIS 4.0 and IIS 5.0, there was a double decoding vulnerability.  These two
profiles are identical to \texttt{iis}, except they will alert by default if a
URL has a double encoding.  Double decode is not supported in IIS 5.1 and
beyond, so it's disabled by default.

\item \texttt{default, no profile}

The default options used by HTTP Inspect do not use a profile and are described
in Table \ref{default_HTTP_Inspect_options}.

\begin{table}[h]
\begin{center}
\caption{Default HTTP Inspect Options}
\label{default_HTTP_Inspect_options}
\begin{tabular}{| l | p{3in} |}

\hline
\textbf{Option} & \textbf{Setting}\\
\hline

\hline
port & 80\\
\hline
server\_flow\_depth & 300 \\
\hline
client\_flow\_depth & 300 \\
\hline
post\_depth & -1 \\
\hline
chunk encoding & alert on chunks larger than 500000 bytes\\
\hline
ASCII decoding & on, alert off \\
\hline
utf\_8 encoding & on, alert off\\
\hline
multiple slash & on, alert off \\
\hline
directory normalization & on, alert off \\
\hline
webroot & on, alert on\\
\hline
iis backslash & on, alert off \\
\hline
apache whitespace & on, alert off \\
\hline
iis delimiter & on, alert off \\
\hline
non\_strict URL parsing & on\\
\hline
max\_header\_length & 0, header length not checked\\
\hline
max\_spaces & 200 \\
\hline
max\_headers & 0, number of headers not checked\\
\hline
\end{tabular}
\end{center}
\end{table}

Profiles must be specified as the first server option and cannot be combined
with any other options except:

\begin{itemize}
\item \texttt{ports}
\item \texttt{iis\_unicode\_map}
\item \texttt{allow\_proxy\_use}
\item \texttt{server\_flow\_depth}
\item \texttt{client\_flow\_depth}
\item \texttt{post\_depth}
\item \texttt{no\_alerts}
\item \texttt{inspect\_uri\_only}
\item \texttt{oversize\_dir\_length} 
\item \texttt{normalize\_headers} 
\item \texttt{normalize\_cookies} 
\item \texttt{normalize\_utf}
\item \texttt{max\_header\_length} 
\item \texttt{max\_spaces} 
\item \texttt{max\_headers} 
\item \texttt{extended\_response\_inspection}
\item \texttt{enable\_cookie}
\item \texttt{inspect\_gzip}
\item \texttt{unlimited\_decompress}
\item \texttt{normalize\_javascript}
\item \texttt{max\_javascript\_whitespaces}
\item \texttt{enable\_xff}
\item \texttt{http\_methods}
\item \texttt{log\_uri}
\item \texttt{log\_hostname}
\item \texttt{small\_chunk\_length}
\item \texttt{decompress\_swf}
\item \texttt{decompress\_pdf}
\end{itemize}

These options must be specified after the \texttt{profile} option.

\end{subslist}

\subsubsection{Example}

\begin{verbatim}
    preprocessor http_inspect_server: \
        server 1.1.1.1 profile all ports { 80 3128 }
\end{verbatim}
                             
\item \texttt{ports $\{ <$port$> [<$port$> <...>] \}$}

This is how the user configures which ports to decode on the HTTP server.
However, HTTPS traffic is encrypted and cannot be decoded with HTTP Inspect.
To ignore HTTPS traffic, use the SSL preprocessor.

\item \texttt{iis\_unicode\_map $<$map\_filename$>$ codemap $<$integer$>$}

The IIS Unicode map is generated by the program ms\_unicode\_generator.c.  This
program is located on the Snort.org web site at
\url{http://www.snort.org/dl/contrib/} directory.  Executing this program
generates a Unicode map for the system that it was run on.  So, to get the
specific Unicode mappings for an IIS web server, you run this program on that
server and use that Unicode map in this configuration.

When using this option, the user needs to specify the file that contains the
IIS Unicode map and also specify the Unicode map to use.  For US servers, this
is usually 1252.  But the ms\_unicode\_generator program tells you which
codemap to use for you server; it's the ANSI code page.  You can select the
correct code page by looking at the available code pages that the
ms\_unicode\_generator outputs.

\item \texttt{extended\_response\_inspection}

This enables the extended HTTP response inspection. The default http response
inspection does not inspect the various fields of a HTTP response. By turning
this option the HTTP response will be thoroughly inspected. The different fields
of a HTTP response such as status code, status message, headers, cookie (when
enable\_cookie is configured) and body are extracted and saved into buffers.
Different rule options are provided to inspect these buffers.

This option must be enabled to make use of the decompress\_swf or decompress\_pdf
options.

\begin{note}

When this option is turned on, if the HTTP response packet has a body then any
content pattern matches ( without http modifiers ) will search the response body
((decompressed in case of gzip) and not the entire packet payload. To search for 
patterns in the header of the response, one should use the http modifiers with 
content such as \texttt{http\_header}, \texttt{http\_stat\_code}, \texttt{http\_stat\_msg} 
and \texttt{http\_cookie}.

\end{note}

\item \texttt{enable\_cookie}

This options turns on the cookie extraction from HTTP requests and HTTP response.
By default the cookie inspection and extraction will be turned off. The cookie from 
the \texttt{Cookie} header line is extracted and stored in HTTP Cookie buffer for 
HTTP requests and cookie from the \texttt{Set-Cookie} is extracted and stored in 
HTTP Cookie buffer for HTTP responses. The \texttt{Cookie:} and \texttt{Set-Cookie:} 
header names itself along with leading spaces and the CRLF terminating the header 
line are stored in the HTTP header buffer and are not stored in the HTTP cookie buffer.

\begin{verbatim}
Ex: Set-Cookie: mycookie \r\n

In this case, Set-Cookie: \r\n will be in the HTTP header buffer and the pattern
mycookie will be in the HTTP cookie buffer.

\end{verbatim}

\item \texttt{inspect\_gzip}

This option specifies the HTTP inspect module to uncompress the compressed
data(gzip/deflate) in HTTP response. You should select the config option
"extended\_response\_inspection" before configuring this option.  Decompression 
is done across packets. So the decompression will end when either the 
'compress\_depth' or 'decompress\_depth' is reached or when the compressed data ends.
When the compressed data is spanned across multiple packets, the state of the last 
decompressed packet is used to decompressed the data of the next packet. 
But the decompressed data are individually inspected. (i.e. the 
decompressed data from different packets are not combined while inspecting). 
Also the amount of decompressed data that will be inspected depends on the 
'server\_flow\_depth' configured.

Http Inspect generates a preprocessor alert with gid 120 and sid 6 when the decompression
fails. When the decompression fails due to a CRC error encountered by zlib, HTTP Inspect
will also provide the detection module with the data that was decompressed by zlib.

\item \texttt{unlimited\_decompress}

This option enables the user to decompress unlimited gzip data (across multiple 
packets).Decompression will stop when the compressed data ends or when a out of 
sequence packet is received. To ensure unlimited decompression, user should set 
the 'compress\_depth' and 'decompress\_depth' to its maximum values in the default 
policy. The decompression in a single packet is still limited by the 'compress\_depth' 
and 'decompress\_depth'.

\item \texttt{decompress\_swf $\{ mode [mode] \}$ } 

This option will enable decompression of compressed SWF (Adobe Flash content) files
encountered as the HTTP Response body in a GET transaction.  The available decompression
modes are 'deflate' and 'lzma'.  A prerequisite is enabling
extended\_response\_inspection (described above).  When enabled, the preprocessor will
examine the response body for the corresponding file signature.  'CWS' for Deflate/ZLIB
compressed and 'ZWS' for LZMA compressed.  Each decompression mode can be individually enabled.
e.g. ... { lzma } or { deflate } or { lzma deflate }.  The compressed content is decompressed
'in-place' with the content made available to the detection/rules 'file\_data' option.
If enabled and located, the compressed SWF file signature is converted to 'FWS' to indicate
an uncompressed file.

The 'decompress\_depth', 'compress\_depth', and 'unlimited\_decompress' are optionally used to 
place limits on the decompression process.  The semantics for SWF files are similar to the
gzip decompression process.

During the decompression process, the preprocessor may generate alert 120:12 if Deflate
decompression fails or alert 120:13 if LZMA decompression fails.

\begin{note}
LZMA decompression is only available if Snort is built with the liblzma package present
and functional.  If the LZMA package is not present, then the { lzma } option will indicate
a fatal parsing error.  If the liblzma package IS present, but one desires to disable LZMA
support, then the --disable-lzma option on configure will disable usage of the library.
\end{note}

\item \texttt{decompress\_pdf $\{ mode [mode] \}$ } 

This option will enable decompression of the compressed portions of PDF files encountered
as the HTTP Response body in a GET transaction.  A prerequisite is enabling
extended\_response\_inspection (described above).

When enabled, the preprocessor will examine the response body for the '%PDF-' file signature.
PDF files are then parsed, locating PDF 'streams' with a single '/FlateDecode' filter.  These
streams are decompressed in-place, replacing the compressed content.

The 'decompress\_depth', 'compress\_depth', and 'unlimited\_decompress' are optionally used to 
place limits on the decompression process.  The semantics for PDF files are similar to the
gzip decompression process.

During the file parsing/decompression process, the preprocessor may generate several alerts:

\begin{center}
\begin{tabular}{| l | p{4.5in} |}

\hline
\textbf{Alert} & \textbf{Description}\\
\hline 

\hline 
120:14 & Deflate decompression failure \\

\hline 
120:15 & Located a 'stream' with an unsupported compression ('/Filter') algorithm \\

\hline 
120:16 & Located a 'stream' with unsupported cascaded '/FlateDecode' options, e.g.: \begin{verbatim}/Filter [ /FlateDecode /FlateDecode ]\end{verbatim} \\

\hline 
120:17 & PDF File parsing error \\

\hline
\end{tabular}
\end{center}

\item \texttt{normalize\_javascript}
This option enables the normalization of Javascript within the HTTP response body.
You should select the config option \texttt{extended\_response\_inspection} before configuring 
this option. When this option is turned on, Http Inspect searches for a Javascript within the 
HTTP response body by searching for the $<$script$>$ tags and starts normalizing it. 
When Http Inspect sees the $<$script$>$ tag without a type, it is considered as a javascript.
The obfuscated data within the javascript functions such as unescape, String.fromCharCode, decodeURI, 
decodeURIComponent will be normalized. The different encodings handled within the unescape/
decodeURI/decodeURIComponent are \texttt{\%XX}, \texttt{\%uXXXX}, \texttt{\\XX} and \texttt{\\uXXXXi}. 
Apart from these encodings, Http Inspect will also detect the consecutive whitespaces and normalize 
it to a single space. Http Inspect will also normalize the plus and concatenate the strings. 
The rule option \texttt{file\_data} can be used to access this normalized buffer from the rule.
A preprocessor alert with SID 9 and GID 120 is generated when the obfuscation levels within the 
Http Inspect is equal to or greater than 2.

\begin{verbatim}

Example:

HTTP/1.1 200 OK\r\n
Date: Wed, 29 Jul 2009 13:35:26 GMT\r\n
Server: Apache/2.2.3 (Debian) PHP/5.2.0-8+etch10 mod_ssl/2.2.3 OpenSSL/0.9.8c\r\n
Last-Modified: Sun, 20 Jan 2008 12:01:21 GMT\r\n
Accept-Ranges: bytes\r\n
Content-Length: 214\r\n
Keep-Alive: timeout=15, max=99\r\n
Connection: Keep-Alive\r\n
Content-Type: application/octet-stream\r\n\r\n 
<html xmlns="http://www.w3.org/1999/xhtml">
<head>
<title>FIXME</title>
</head>
<body>
<script>document.write(unescape(unescape("%48%65%6C%6C%6F%2C%20%73%6E%6F%72%74%20%74%65%61%6D%21")));
</script>
</body>
</html>

\end{verbatim}

The above javascript will generate the preprocessor alert with SID 9 and GIDF 120 when \texttt{normalize\_javascript}
is turned on.

Http Inspect will also generate a preprocessor alert with GID 120 and SID 11 when there are more than one type 
of encodings within the escaped/encoded data.

\begin{verbatim}

For example:

unescape("%48\x65%6C%6C%6F%2C%20%73%6E%6F%72%74%20%74%65%61%6D%21");
String.fromCharCode(0x48, 0x65, 0x6c, 0x6c, 111, 44, 32, 115, 110, 111, 114, 116, 32, 116, 101, 97, 109, 33)

\\end{verbatim}

The above obfuscation will generate the preprocessor alert with GID 120 and SID 11.

This option is turned off by default in HTTP Inspect.

\item \texttt{max\_javascript\_whitespaces $<$positive integer up to 65535$>$}
This option takes an integer as an argument.  The integer determines the maximum number
of consecutive whitespaces allowed within the Javascript obfuscated data in a HTTP
response body. The config option \texttt{normalize\_javascript} should be turned on before configuring
 this config option. When the whitespaces in the javascript obfuscated data is equal to or more
than this value a preprocessor alert with GID 120 and SID 10 is generated. The default value for 
this option is 200.  To enable, specify an integer argument to \texttt{max\_javascript\_spaces} of 1 to 65535.
Specifying a value of 0 is treated as disabling the alert.

\item \texttt{enable\_xff}

This option enables Snort to parse and log the original client IP present in the
X-Forwarded-For or True-Client-IP HTTP request headers along with the generated
events. The XFF/True-Client-IP Original client IP address is logged only with
unified2 output and is not logged with console (-A cmg) output.

\item \texttt{xff\_headers}

If/When the \texttt{enable\_xff} option is present, the \texttt{xff\_headers} option specifies a set of custom 'xff'
headers.  This option allows the definition of up to six custom headers in addition to the
two default (and always present) X-Forwarded-For and True-Client-IP headers.  The option
permits both the custom and default headers to be prioritized.  The headers/priority pairs
are specified as a list.  Lower numerical values imply a higher priority.  The headers do
not need to be specified in priority order.  Nor do the priorities need to be contiguous.
Priority values can range from 1 to 255.  The priority values and header names must be unique.
The header names must not collide with known http headers such as 'host', 'cookie',
'content-length', etc.

A example of the \texttt{xff\_header} syntax is:
\begin{verbatim}
xff_headers { [ x-forwarded-highest-priority 1 ] [ x-forwarded-second-highest-priority 2 ] \
              [ x-forwarded-lowest-priority-custom 3 ] }
\end{verbatim}

The default X-Forwarded-For and True-Client-IP headers are always present.  They may be explicitly
specified in the \texttt{xff\_headers} config in order to determine their priority.  If not specified, they
will be automatically added to the xff list as the lowest priority headers.

For example, let us say that we have the following (abbreviated) HTTP request header:

\begin{verbatim}
...
Host: www.snort.org
X-Forwarded-For: 192.168.1.1
X-Was-Originally-Forwarded-From: 10.1.1.1
...
\end{verbatim}
 
With the default xff behavior (no \texttt{xff\_headers}), the 'X-Forwarded-For' header would be used to
provide a 192.168.1.1 Original Client IP address in the unified2 log.  Custom headers are not
parsed.

With:
\begin{verbatim}
xff_headers { [ x-was-originally-forwarded-from 1 ] [ x-another-forwarding-header 2 ] \
              [ x-forwarded-for 3 ] }
\end{verbatim}

The X-Was-Originally-Forwarded-From header is the highest priority present and its value
of 10.1.1.1 will be logged as the Original Client IP in the unified2 log.

But with:
\begin{verbatim}
xff_headers { [ x-was-originally-forwarded-from 3 ] [ x-another-forwarding-header 2 ] \
              [ x-forwarded-for 1 ] }
\end{verbatim}

Now the X-Forwarded-For header is the highest priority and its value of 192.168.1.1 is logged.


\begin{note}

The original client IP from XFF/True-Client-IP in unified2 logs can be viewed using 
the tool u2spewfoo. This tool is present in the tools/u2spewfoo directory of snort 
source tree.

\end{note}

\item \texttt{server\_flow\_depth $<$integer$>$}

This specifies the amount of server response payload to inspect. When
\texttt{extended\_response\_inspection} is turned on, it is applied to the HTTP response 
body (decompressed data when \texttt{inspect\_gzip} is turned on) and not the HTTP headers.
When \texttt{extended\_response\_inspection} is turned off the \texttt{server\_flow\_depth} 
is applied to the entire HTTP response (including headers). Unlike \texttt{client\_flow\_depth} 
this option is applied per TCP session. This option can be used to balance the needs of 
IDS performance and level of inspection of HTTP server response data.  Snort rules are       
targeted at HTTP server response traffic and when used with a small flow\_depth value 
may cause false negatives. Most of these rules target either the HTTP header, or 
the content that is likely to be in the first hundred or so bytes of non-header data.  
Headers are usually under 300 bytes long, but your mileage may vary. 
It is suggested to set the \texttt{server\_flow\_depth} to its maximum value.

This value can be set from -1 to 65535. A value of -1 causes Snort
to ignore all server side traffic for ports defined in \texttt{ports} when
\texttt{extended\_response\_inspection} is turned off. When the \texttt{extended\_response\_inspection}
 is turned on, value of -1 causes Snort to ignore the HTTP response body data and
 not the HTTP headers.  Inversely, a value of 0 causes Snort to inspect all HTTP server
payloads defined in "ports" (note that this will likely slow down IDS
performance).  Values above 0 tell Snort the number of bytes to
inspect of the server response (excluding the HTTP headers when \texttt{extended\_response\_inspection}
is turned on) in a given HTTP session.  Only packets payloads starting with 'HTTP' will
be considered as the first packet of a server response.  If less than flow\_depth bytes
are in the payload of the HTTP response packets in a given session, the entire payload will be
inspected.  If more than flow\_depth bytes are in the payload of the HTTP response packet in a session
only flow\_depth bytes of the payload will be inspected for that session.  Rules that are meant to
inspect data in the payload of the HTTP response packets in a session beyond 65535 bytes will be
ineffective unless flow\_depth is set to 0. The default value for \texttt{server\_flow\_depth} is 300.
Note that the 65535 byte maximum flow\_depth applies to stream reassembled packets as well. 
It is suggested to set the \texttt{server\_flow\_depth} to its maximum value.

\begin{note}

\texttt{server\_flow\_depth} is the same as the old \texttt{flow\_depth}
option, which will be deprecated in a future release.

\end{note}

\item \texttt{client\_flow\_depth $<$integer$>$}

This specifies the amount of raw client request payload to inspect. This
value can be set from -1 to 1460. Unlike \texttt{server\_flow\_depth} this value is applied
to the first packet of the HTTP request. It is not a session based flow depth.
It has a default value of 300.  It primarily eliminates Snort from inspecting
larger HTTP Cookies that appear at the end of many client request Headers.

A value of -1 causes Snort to ignore all client side traffic for ports
defined in "ports." Inversely, a value of 0 causes Snort to inspect all HTTP client
 side traffic defined in "ports" (note that this will likely slow down IDS
performance).  Values above 0 tell Snort the number of bytes to
inspect in the first packet of the client request.  If less than flow\_depth bytes
are in the TCP payload (HTTP request) of the first packet, the entire payload will be inspected.
If more than flow\_depth bytes are in the payload of the first packet only flow\_depth
bytes of the payload will be inspected.  Rules that are meant to
inspect data in the payload of the first packet of a client request beyond 1460 bytes 
will be ineffective unless flow\_depth is set to 0. Note that the 1460 byte 
maximum flow\_depth applies to stream reassembled packets as well. It is 
suggested to set the \texttt{client\_flow\_depth} to its maximum value.

\item \texttt{post\_depth $<$integer$>$}

This specifies the amount of data to inspect in a client post message. The
value can be set from -1 to 65495. The default value is -1. A value of -1 
causes Snort to ignore all the data in the post message. Inversely, a value 
of 0 causes Snort to inspect all the client post message. This increases    
the performance by inspecting only specified bytes in the post message.

\item \texttt{ascii $<$yes$|$no$>$}

The \texttt{ascii} decode option tells us whether to decode encoded ASCII
chars, a.k.a \%2f = /, \%2e = ., etc.  It is normal to see ASCII encoding usage
in URLs, so it is recommended that you disable HTTP Inspect alerting for this
option.

\item \texttt{extended\_ascii\_uri}

This option enables the support for extended ASCII codes in the HTTP request
URI. This option is turned off by default and is not supported with any of
the profiles.

\item \texttt{utf\_8 $<$yes$|$no$>$}

The \texttt{utf-8} decode option tells HTTP Inspect to decode standard UTF-8
Unicode sequences that are in the URI.  This abides by the Unicode standard and
only uses \% encoding.  Apache uses this standard, so for any Apache servers,
make sure you have this option turned on.  As for alerting, you may be
interested in knowing when you have a UTF-8 encoded URI, but this will be prone
to false positives as legitimate web clients use this type of encoding.  When
\texttt{utf\_8} is enabled, ASCII decoding is also enabled to enforce correct
functioning.  

\item \texttt{u\_encode $<$yes$|$no$>$}

This option emulates the IIS \%u encoding scheme.  How the \%u encoding scheme
works is as follows:  the encoding scheme is started by a \%u followed by 4
characters, like \%uxxxx.  The xxxx is a hex-encoded value that correlates to
an IIS Unicode codepoint.  This value can most definitely be ASCII.  An ASCII
character is encoded like \%u002f = /, \%u002e = ., etc.  If no
iis\_unicode\_map is specified before or after this option, the default codemap
is used.

You should alert on \%u encodings, because we are not aware of any legitimate
clients that use this encoding.  So it is most likely someone trying to be
covert.

\item \texttt{bare\_byte $<$yes$|$no$>$}

Bare byte encoding is an IIS trick that uses non-ASCII characters as valid
values when decoding UTF-8 values.  This is not in the HTTP standard, as all
non-ASCII values have to be encoded with a \%.  Bare byte encoding allows the
user to emulate an IIS server and interpret non-standard encodings correctly.

The alert on this decoding should be enabled, because there are no legitimate
clients that encode UTF-8 this way since it is non-standard.

\item \texttt{iis\_unicode $<$yes$|$no$>$}

The \texttt{iis\_unicode} option turns on the Unicode codepoint mapping.  If
there is no iis\_unicode\_map option specified with the server config,
\texttt{iis\_unicode} uses the default codemap.  The \texttt{iis\_unicode}
option handles the mapping of non-ASCII codepoints that the IIS server accepts
and decodes normal UTF-8 requests.

You should alert on the \texttt{iis\_unicode option}, because it is seen mainly
in attacks and evasion attempts.  When \texttt{iis\_unicode} is enabled, ASCII
and UTF-8 decoding are also enabled to enforce correct decoding.  To alert on
UTF-8 decoding, you must enable also enable \texttt{utf\_8 yes}. 

\item \texttt{double\_decode $<$yes$|$no$>$}

The \texttt{double\_decode} option is once again IIS-specific and emulates IIS
functionality.  How this works is that IIS does two passes through the request
URI, doing decodes in each one.  In the first pass, it seems that all types of
iis encoding is done: utf-8 unicode, ASCII, bare byte, and \%u.  In the second
pass, the following encodings are done:  ASCII, bare byte, and \%u.  We leave
out utf-8 because I think how this works is that the \% encoded utf-8 is
decoded to the Unicode byte in the first pass, and then UTF-8 is decoded in the
second stage.  Anyway, this is really complex and adds tons of different
encodings for one character.  When \texttt{double\_decode} is enabled, so ASCII
is also enabled to enforce correct decoding.

\item \texttt{non\_rfc\_char $\{ <$byte$> [<$byte ...$>] \}$}

This option lets users receive an alert if certain non-RFC chars are used in a
request URI.  For instance, a user may not want to see null bytes in the
request URI and we can alert on that.  Please use this option with care,
because you could configure it to say, alert on all `/' or something like that.
It's flexible, so be careful.

\item \texttt{multi\_slash $<$yes$|$no$>$}

This option normalizes multiple slashes in a row, so something like:
``foo/////////bar'' get normalized to ``foo/bar.''

If you want an alert when multiple slashes are seen, then configure with a
\texttt{yes}; otherwise, use \texttt{no}.

\item \texttt{iis\_backslash $<$yes$|$no$>$}

Normalizes backslashes to slashes.  This is again an IIS emulation.  So a
request URI of ``/foo$\backslash$bar'' gets normalized to ``/foo/bar.''

\item \texttt{directory $<$yes$|$no$>$}

This option normalizes directory traversals and self-referential directories.

The directory:

\begin{verbatim}
    /foo/fake\_dir/../bar
\end{verbatim}

gets normalized to:

\begin{verbatim}
    /foo/bar
\end{verbatim}

The directory:

\begin{verbatim}
    /foo/./bar
\end{verbatim}

gets normalized to:

\begin{verbatim}
    /foo/bar
\end{verbatim}

If you want to configure an alert, specify \texttt{yes}, otherwise, specify
\texttt{no}.  This alert may give false positives, since some web sites refer
to files using directory traversals.

\item \texttt{apache\_whitespace $<$yes$|$no$>$}

This option deals with the non-RFC standard of using tab for a space delimiter.
Apache uses this, so if the emulated web server is Apache, enable this option.
Alerts on this option may be interesting, but may also be false positive prone.

\item \texttt{iis\_delimiter $<$yes$|$no$>$}

This started out being IIS-specific, but Apache takes this non-standard
delimiter was well.  Since this is common, we always take this as standard
since the most popular web servers accept it.  But you can still get an alert
on this option.

\item \texttt{chunk\_length $<$non-zero positive integer$>$}

This option is an anomaly detector for abnormally large chunk sizes.  This
picks up the Apache chunk encoding exploits, and may also alert on HTTP
tunneling that uses chunk encoding.

\item \texttt{small\_chunk\_length \{ $<$chunk size$>$ $<$consecutive chunks$>$ \} }

This option is an evasion detector for consecutive small chunk sizes when
either the client or server use \texttt{Transfer-Encoding: chunked}.
$<$chunk size$>$ specifies the maximum chunk size for which a chunk will be
considered small.  $<$consecutive chunks$>$ specifies the number of consecutive
small chunks $<$= $<$chunk size$>$ before an event will be generated.  This option
is turned off by default.  Maximum values for each are 255 and a $<$chunk size$>$ of 0
disables.  Events generated are gid:119, sid:26 for client small
chunks and gid:120, sid:7 for server small chunks.

Example:
\begin{verbatim}
small_chunk_length { 10 5 }
\end{verbatim}
Meaning alert if we see 5 consecutive chunk sizes of 10 or less.

\item \texttt{no\_pipeline\_req}

This option turns HTTP pipeline decoding off, and is a performance enhancement
if needed.  By default, pipeline requests are inspected for attacks, but when
this option is enabled, pipeline requests are not decoded and analyzed per HTTP
protocol field.  It is only inspected with the generic pattern matching.

\item \texttt{non\_strict}

This option turns on non-strict URI parsing for the broken way in which Apache
servers will decode a URI.  Only use this option on servers that will accept
URIs like this: "get /index.html alsjdfk alsj lj aj  la jsj s$\backslash$n".
The non\_strict option assumes the URI is between the first and second space
even if there is no valid HTTP identifier after the second space.

\item \texttt{allow\_proxy\_use}

By specifying this keyword, the user is allowing proxy use on this server.
This means that no alert will be generated if the \texttt{proxy\_alert} global
keyword has been used.  If the proxy\_alert keyword is not enabled, then this
option does nothing.  The \texttt{allow\_proxy\_use} keyword is just a way to
suppress unauthorized proxy use for an authorized server.  

\item \texttt{no\_alerts}

This option turns off all alerts that are generated by the HTTP Inspect
preprocessor module.  This has no effect on HTTP rules in the rule set.  No
argument is specified.

\item \texttt{oversize\_dir\_length $<$non-zero positive integer$>$}

This option takes a non-zero positive integer as an argument.  The argument
specifies the max char directory length for URL directory.  If a url directory
is larger than this argument size, an alert is generated. A good argument value
is 300 characters.  This should limit the alerts to IDS evasion type attacks,
like whisker -i 4.

\item \texttt{inspect\_uri\_only}

This is a performance optimization.  When enabled, only the URI portion of HTTP
requests will be inspected for attacks.  As this field usually contains 90-95\%
of the web attacks, you'll catch most of the attacks.  So if you need extra
performance, enable this optimization.  It's important to note that if this
option is used without any \texttt{uricontent} rules, then no inspection will
take place.  This is obvious since the URI is only inspected with
\texttt{uricontent} rules, and if there are none available, then there is
nothing to inspect.

For example, if we have the following rule set:

\begin{verbatim}
    alert tcp any any -> any 80 ( msg:"content"; content: "foo"; )
\end{verbatim}

and the we inspect the following URI:

\begin{verbatim}
    get /foo.htm http/1.0\r\n\r\n
\end{verbatim}

No alert will be generated when \texttt{inspect\_uri\_only} is enabled.  The
\texttt{inspect\_uri\_only} configuration turns off all forms of detection
except \texttt{uricontent} inspection.

\item \texttt{max\_header\_length $<$positive integer up to 65535$>$}

This option takes an integer as an argument.  The integer is the maximum length
allowed for an HTTP client request header field.  Requests that exceed this
length will cause a "Long Header" alert.  This alert is off by default.  To
enable, specify an integer argument to max\_header\_length of 1 to 65535.
Specifying a value of 0 is treated as disabling the alert.

\item \texttt{max\_spaces $<$positive integer up to 65535$>$}

This option takes an integer as an argument.  The integer determines the maximum number
of whitespaces allowed with HTTP client request line folding. Requests headers
folded with whitespaces equal to or more than this value will cause a
"Space Saturation" alert with SID 26 and GID 119.  The default value for this
option is 200.  To enable, specify an integer argument to \texttt{max\_spaces} of 1 to 65535.
Specifying a value of 0 is treated as disabling the alert.


\item \texttt{webroot $<$yes$|$no$>$}

This option generates an alert when a directory traversal traverses past the
web server root directory.  This generates much fewer false positives than the
directory option, because it doesn't alert on directory traversals that stay
within the web server directory structure.  It only alerts when the directory
traversals go past the web server root directory, which is associated with
certain web attacks.

\item \texttt{tab\_uri\_delimiter}

This option turns on the use of the tab character (0x09) as a delimiter for a
URI.  Apache accepts tab as a delimiter; IIS does not.  For IIS, a tab in the
URI should be treated as any other character.  Whether this option is on or
not, a tab is treated as whitespace if a space character (0x20) precedes it.
No argument is specified.

\item \texttt{normalize\_headers}

This option turns on normalization for HTTP Header Fields, not including
Cookies (using the same configuration parameters as the URI normalization (i.e.,
multi-slash, directory, etc.).  It is useful for normalizing Referrer URIs that
may appear in the HTTP Header.

\item \texttt{normalize\_cookies}

This option turns on normalization for HTTP Cookie Fields (using the same
configuration parameters as the URI normalization (i.e., multi-slash, directory,
etc.).  It is useful for normalizing data in HTTP Cookies that may be encoded.

\item \texttt{normalize\_utf}

This option turns on normalization of HTTP response bodies where the Content-Type
header lists the character set as "utf-16le", "utf-16be", "utf-32le", or
"utf-32be". HTTP Inspect will attempt to normalize these back into 8-bit encoding,
generating an alert if the extra bytes are non-zero.

\item \texttt{max\_headers $<$positive integer up to 1024$>$}

This option takes an integer as an argument.  The integer is the maximum number
of HTTP client request header fields.  Requests that contain more HTTP Headers
than this value will cause a "Max Header" alert.  The alert is off by default.
To enable, specify an integer argument to max\_headers of 1 to 1024.
Specifying a value of 0 is treated as disabling the alert.

\item \texttt{http\_methods $\{ cmd [cmd] \}$ } 
This specifies additional HTTP Request Methods outside of those checked by
default within the preprocessor (GET and POST). The list should be enclosed
within braces and delimited by spaces, tabs, line feed or carriage return. 
The config option, braces and methods also needs to be separated by braces.

\begin{verbatim}
    http_methods { PUT CONNECT }
\end{verbatim}

\begin{note}

Please note the maximum length for a method name is 256.

\end{note}

\item \texttt{log\_uri}

This option enables HTTP Inspect preprocessor to parse the URI data from the
HTTP request and log it along with all the generated events for that session.
Stream reassembly needs to be turned on HTTP ports to enable the logging.
If there are multiple HTTP requests in the session, the URI data of the most recent
HTTP request during the alert will be logged. The maximum URI logged is 2048.

\begin{note}

Please note, this is logged only with the unified2 output and is not logged
with console output (-A cmg). \texttt{u2spewfoo} can be used to read this data from
the unified2.

\end{note}

\item \texttt{log\_hostname}

This option enables HTTP Inspect preprocessor to parse the hostname data from the
"Host" header of the HTTP request and log it along with all the generated events
for that session. Stream reassembly needs to be turned on HTTP ports to enable
the logging. If there are multiple HTTP requests in the session, the Hostname data
of the most recent HTTP request during the alert will be logged. In case of
multiple "Host" headers within one HTTP request, a preprocessor alert with sid 24 is
generated. The maximum hostname length logged is 256.

\begin{note}

Please note, this is logged only with the unified2 output and is not logged
with console output (-A cmg). \texttt{u2spewfoo} can be used to read this data from
the unified2.

\end{note}

\end{slist}

\subsubsection{Examples}

\begin{verbatim}
    preprocessor http_inspect_server: \
        server 10.1.1.1 \
        ports { 80 3128 8080 } \
        server_flow_depth 0 \
        ascii no \
        double_decode yes \
        non_rfc_char { 0x00 } \
        chunk_length 500000 \
        non_strict \
        no_alerts

    preprocessor http_inspect_server: \
        server default \ 
        ports  { 80 3128 }  \
        non_strict \
        non_rfc_char  { 0x00 }  \
        server_flow_depth 300  \
        apache_whitespace yes \
        directory no \
        iis_backslash no \
        u_encode yes \
        ascii no \
        chunk_length 500000 \
        bare_byte yes \
        double_decode yes \
        iis_unicode yes \ 
        iis_delimiter yes \
        multi_slash no

    preprocessor http_inspect_server: \
        server default \
        profile all \
        ports { 80 8080 }
\end{verbatim}

\subsection{SMTP Preprocessor}
\label{SMTP}

The SMTP preprocessor is an SMTP decoder for user applications.  Given a data
buffer, SMTP will decode the buffer and find SMTP commands and responses.  It
will also mark the command, data header data body sections, and TLS data.

SMTP handles stateless and stateful processing.  It saves state between
individual packets.  However maintaining correct state is dependent on the
reassembly of the client side of the stream (i.e., a loss of coherent stream data
results in a loss of state).

\subsubsection{Configuration}

SMTP has the usual configuration items, such as \texttt{port} and
\texttt{inspection\_type}.  Also, SMTP command lines can be normalized to
remove extraneous spaces.  TLS-encrypted traffic can be ignored, which improves
performance.  In addition, regular mail data can be ignored for an additional
performance boost.  Since so few (none in the current snort rule set) exploits
are against mail data, this is relatively safe to do and can improve the
performance of data inspection.

The configuration options are described below:

\begin{slist}

\item \texttt{ports \{ <port> [<port>] ... \}}

This specifies on what ports to check for SMTP data.  Typically, this will
include 25 and possibly 465, for encrypted SMTP.

\item \texttt{inspection\_type <stateful | stateless>}

Indicate whether to operate in stateful or stateless mode.

\item \texttt{normalize <all | none | cmds>}

This turns on normalization.  Normalization checks for more than one space
character after a command.  Space characters are defined as space (ASCII 0x20)
or tab (ASCII 0x09).

\texttt{all} checks all commands

\texttt{none} turns off normalization for all commands.

\texttt{cmds} just checks commands listed with the \texttt{normalize\_cmds} parameter.

\item \texttt{ignore\_data}

Ignore data section of mail (except for mail headers) when processing rules.

\item \texttt{ignore\_tls\_data}

Ignore TLS-encrypted data when processing rules.

\item \texttt{max\_command\_line\_len <int>}

Alert if an SMTP command line is longer than this value.  Absence of this
option or a "0" means never alert on command line length.  RFC 2821 recommends
512 as a maximum command line length.

\item \texttt{max\_header\_line\_len <int>}

Alert if an SMTP DATA header line is longer than this value.  Absence of this
option or a "0" means never alert on data header line length.  RFC 2821
recommends 1024 as a maximum data header line length.

\item \texttt{max\_response\_line\_len <int>}

Alert if an SMTP response line is longer than this value.  Absence of this
option or a "0" means never alert on response line length.  RFC 2821 recommends
512 as a maximum response line length.

\item \texttt{alt\_max\_command\_line\_len <int> \{ <cmd> [<cmd>] \}}

Overrides \texttt{max\_command\_line\_len} for specific commands.

\item \texttt{no\_alerts}

Turn off all alerts for this preprocessor.

\item \texttt{invalid\_cmds \{ <Space-delimited list of commands> \}}

Alert if this command is sent from client side.  Default is an empty list.

\item \texttt{valid\_cmds \{ <Space-delimited list of commands> \}}

List of valid commands.  We do not alert on commands in this list.  Default is
an empty list, but preprocessor has this list hard-coded:

\begin{itemize}
\item[]
\{ ATRN AUTH BDAT DATA DEBUG EHLO EMAL ESAM ESND ESOM ETRN EVFY EXPN
HELO HELP IDENT MAIL NOOP QUIT RCPT RSET SAML SOML SEND ONEX QUEU
STARTTLS TICK TIME TURN TURNME VERB VRFY X-EXPS X-LINK2STATE
XADR XAUTH XCIR XEXCH50 XGEN XLICENSE XQUE XSTA XTRN XUSR \}
\end{itemize}

\item \texttt{data\_cmds \{ <Space-delimited list of commands> \}}

List of commands that initiate sending of data with an end of data delimiter
the same as that of the DATA command per RFC 5321 - \texttt{"<CRLF>.<CRLF>"}.
Default is \{ DATA \}.

\item \texttt{binary\_data\_cmds \{ <Space-delimited list of commands> \}}

List of commands that initiate sending of data and use a length value after
the command to indicate the amount of data to be sent, similar to that of the
BDAT command per RFC 3030.  Default is \{ BDAT XEXCH50 \}.

\item \texttt{auth\_cmds \{ <Space-delimited list of commands> \}}

List of commands that initiate an authentication exchange between client
and server.  Default is \{ AUTH XAUTH X-EXPS \}.

\item \texttt{alert\_unknown\_cmds}

Alert if we don't recognize command.  Default is off.

\item \texttt{normalize\_cmds \{ <Space-delimited list of commands> \}}

Normalize this list of commands Default is \{ RCPT VRFY EXPN \}.

\item \texttt{xlink2state \{ enable | disable [drop] \}}

Enable/disable xlink2state alert.  Drop if alerted.  Default is
\texttt{enable}.

\item \texttt{print\_cmds}

List all commands understood by the preprocessor.  This not normally printed
out with the configuration because it can print so much data.

\item \texttt{disabled}

Disables the SMTP preprocessor in a config. This is useful when specifying
the decoding depths such as \texttt{b64\_decode\_depth}, \texttt{qp\_decode\_depth}, 
\texttt{uu\_decode\_depth}, \texttt{bitenc\_decode\_depth} or the memcap used for 
decoding \texttt{max\_mime\_mem} in default config without turning on the SMTP preprocessor.

\item \texttt{b64\_decode\_depth}

This config option is used to turn off/on or set the base64 decoding depth used to
decode the base64 encoded MIME attachments. The value ranges from -1 to 65535.
A value of -1 turns off the base64 decoding of MIME attachments. The value of 0
sets the decoding of base64 encoded MIME attachments to unlimited. A value other
than 0 or -1 restricts the decoding of base64 MIME attachments, and applies per attachment. 
A SMTP preprocessor alert with sid 10 is generated (if enabled) when the decoding fails.

Multiple MIME attachments/data in one packet are pipelined. When stateful inspection 
is turned on the base64 encoded MIME attachments/data across multiple packets are decoded too.

The decoded data is available for detection using the rule option \texttt{file\_data}. 
See \ref{sub:file_data} rule option for more details.

This option replaces the deprecated options, \texttt{enable\_mime\_decoding} and 
\texttt{max\_mime\_depth}.  It is recommended that user inputs a value that is a 
multiple of 4. When the value specified is not a multiple of 4, the SMTP preprocessor 
will round it up to the next multiple of 4.

In case of multiple configs, the value specified in the non-default config cannot
exceed the value specified in the default config.

\item \texttt{qp\_decode\_depth}

This config option is used to turn off/on or set the Quoted-Printable decoding depth
used to decode the Quoted-Printable(QP) encoded MIME attachments. The value ranges
from -1 to 65535. A value of -1 turns off the QP decoding of MIME attachments.
The value of 0 sets the decoding of QP encoded MIME attachments to unlimited. A
value other than 0 or -1 restricts the decoding of QP MIME attachments, and applies per 
attachment. A SMTP preprocessor alert with sid 11 is generated (if enabled) when the decoding fails.

Multiple MIME attachments/data in one packet are pipelined. When stateful inspection 
is turned on the QP encoded MIME attachments/data across multiple packets are decoded too.

The decoded data is available for detection using the rule option \texttt{file\_data}.
See \ref{sub:file_data} rule option for more details.

In case of multiple configs, the value specified in the non-default config cannot exceed
the value specified in the default config.

\item \texttt{bitenc\_decode\_depth}

This config option is used to turn off/on or set the non-encoded MIME extraction
depth used to extract the non-encoded MIME attachments. The value ranges from -1 
to 65535. A value of -1 turns off the extraction of these MIME attachments. 
The value of 0 sets the extraction of these MIME attachments to unlimited.
A value other than 0 or -1 restricts the extraction of these MIME attachments, and applies 
per attachment.

Multiple MIME attachments/data in one packet are pipelined. When stateful inspection 
is turned on the non-encoded MIME attachments/data across multiple packets are extracted too.

The extracted data is available for detection using the rule option \texttt{file\_data}.
See \ref{sub:file_data} rule option for more details.

In case of multiple configs, the value specified in the non-default config cannot exceed
the value specified in the default config.

\item \texttt{uu\_decode\_depth}

This config option is used to turn off/on or set the Unix-to-Unix decoding depth
used to decode the Unix-to-Unix(UU) encoded attachments. The value ranges
from -1 to 65535. A value of -1 turns off the UU decoding of SMTP attachments.
The value of 0 sets the decoding of UU encoded SMTP attachments to unlimited. A
value other than 0 or -1 restricts the decoding of UU SMTP attachments, and applies 
per attachment. A SMTP preprocessor alert with sid 13 is generated (if enabled) when the decoding fails.

Multiple UU attachments/data in one packet are pipelined. When stateful inspection
is turned on the UU encoded SMTP attachments/data across multiple packets are decoded too.

The decoded data is available for detection using the rule option \texttt{file\_data}.
See \ref{sub:file_data} rule option for more details.

In case of multiple configs, the value specified in the non-default config cannot exceed
the value specified in the default config.

\item \texttt{enable\_mime\_decoding}

Enables Base64 decoding of Mime attachments/data. Multiple base64 encoded MIME
attachments/data in one packet are pipelined. When stateful inspection is turned
on the base64 encoded MIME attachments/data across multiple packets are decoded too.
The decoding of base64 encoded attachments/data ends when either the 
\texttt{max\_mime\_depth} or maximum MIME sessions (calculated using 
\texttt{max\_mime\_depth} and \texttt{max\_mime\_mem}) is reached or when the 
encoded data ends. The decoded data is available for detection using the rule option 
\texttt{file\_data}. See \ref{sub:file_data} rule option for more details.

This option is deprecated. Use the option \texttt{b64\_decode\_depth} to turn off
or on the base64 decoding instead.

\item \texttt{max\_mime\_depth <int>}

Specifies the maximum number of base64 encoded data to decode per SMTP attachment.
The option take values ranging from 4 to 20480 bytes. The default value for this
in snort in 1460 bytes.

It is recommended that user inputs a value that is a multiple of 4. When the value 
specified is not a multiple of 4, the SMTP preprocessor will round it up to the next 
multiple of 4.

This option is deprecated. Use the option \texttt{b64\_decode\_depth} to turn off
or on the base64 decoding instead.

\item \texttt{max\_mime\_mem <int>}

This option determines (in bytes) the maximum amount of memory the SMTP preprocessor
will use for decoding base64 encoded/quoted-printable/non-encoded MIME attachments/data
or Unix-to-Unix encoded attachments. This value can be set from 3276 bytes to 100MB.

This option along with the maximum of the decoding depths will determine the SMTP
sessions that will be decoded at any given instant. The default value for this option
is 838860.

Note: It is suggested to set this value such that the max smtp session calculated as
follows is at least 1.

max smtp session = \texttt{max\_mime\_mem} /(2 * max of (\texttt{b64\_decode\_depth}, 
					\texttt{uu\_decode\_depth}, \texttt{qp\_decode\_depth}
                                        or \texttt{bitenc\_decode\_depth}))

For example, if \texttt{b64\_decode\_depth} is 0 (indicates unlimited decoding) and 
\texttt{qp\_decode\_depth} is 100, then

max smtp session = \texttt{max\_mime\_mem}/2*65535 (max value for \texttt{b64\_decode\_depth})

In case of multiple configs, the \texttt{max\_mime\_mem} of the non-default configs will be overwritten by the
default config's value. Hence user needs to define it in the default config with the new keyword
disabled (used to disable SMTP preprocessor in a config).

\item \texttt{log\_mailfrom}
This option enables SMTP preprocessor to parse and log the sender's email address extracted 
from the "MAIL FROM" command along with all the generated events for that session. The maximum
number of bytes logged for this option is 1024.

Please note, this is logged only with the unified2 output and is not logged with console output (-A cmg).
u2spewfoo can be used to read this data from the unified2.

\item \texttt{log\_rcptto}
This option enables SMTP preprocessor to parse and log the recipient's email addresses 
extracted from the "RCPT TO" command along with all the generated events for that session. 
Multiple recipients are appended with commas. The maximum number of bytes logged for this option is 1024.

Please note, this is logged only with the unified2 output and is not logged with console output (-A cmg).
 u2spewfoo can be used to read this data from the unified2.

\item \texttt{log\_filename}
This option enables SMTP preprocessor to parse and log the MIME attachment filenames extracted 
from the Content-Disposition header within the MIME body along with all the generated events 
for that session. Multiple filenames are appended with commas. The maximum number of bytes logged 
for this option is 1024.

Please note, this is logged only with the unified2 output and is not logged with the 
console output (-A cmg). u2spewfoo can be used to read this data from the unified2.

\item \texttt{log\_email\_hdrs}
This option enables SMTP preprocessor to parse and log the SMTP email headers extracted from 
SMTP data along with all generated events for that session. The number of bytes extracted and 
logged depends upon the \texttt{email\_hdrs\_log\_depth}.

Please note, this is logged only with the unified2 output and is not logged with the console output (-A cmg). 
u2spewfoo can be used to read this data from the unified2.

\item \texttt{email\_hdrs\_log\_depth <int>} 
This option specifies the depth for logging email headers. The allowed range for this option is 
0 - 20480. A value of 0 will disable email headers logging. The default value for this option is 1464. 

Please note, in case of multiple policies, the value specified in the default policy is used and the values 
specified in the targeted policies are overwritten by the default value. 
This option must be configured in the default policy even if the SMTP configuration is disabled.

\item \texttt{memcap <int>}
This option determines in bytes the maximum amount of memory the SMTP preprocessor will
use for logging of filename, MAIL FROM addresses, RCPT TO addresses and email headers. This value
along with the buffer size used to log MAIL FROM, RCPT TO, filenames and \texttt{email\_hdrs\_log\_depth} 
will determine the maximum SMTP sessions that will log the email headers at any given time. When this memcap 
is reached SMTP will stop logging the filename, MAIL FROM address, RCPT TO addresses and email headers
until memory becomes available.

Max SMTP sessions logging email headers at any given time
                = memcap/(1024 + 1024 + 1024 + \texttt{email\_hdrs\_log\_depth})

The size 1024 is the maximum buffer size used for logging filename, RCPTTO and MAIL FROM addresses.

Default value for this option is 838860. The allowed range for this option is 3276 to 104857600.
The value specified in the default config is used when this option is specified in multiple configs.
This option must be configured in the default config even if the SMTP configuration is disabled.

Please note, in case of multiple policies, the value specified in the default policy is used and the values 
specified in the targeted policies are overwritten by the default value.                                                  
This option must be configured in the default policy even if the SMTP configuration is disabled.

\end{slist}

\subsubsection{Example}

\begin{verbatim}
    preprocessor SMTP: \
        ports { 25 } \
        inspection_type stateful \
        normalize cmds \
        normalize_cmds { EXPN VRFY RCPT } \
        ignore_data \
        ignore_tls_data \
        max_command_line_len  512 \
        max_header_line_len   1024 \
        max_response_line_len 512 \
        no_alerts \
        alt_max_command_line_len 300 { RCPT } \
        invalid_cmds { } \
        valid_cmds { } \
        xlink2state { disable } \
	print_cmds \
	log_filename \
	log_email_hdrs \
	log_mailfrom \
	log_rcptto \
	email_hdrs_log_depth 2920 \
	memcap 6000

    preprocessor SMTP: \
	b64_decode_depth 0\
        max_mime_mem 4000 \
	memcap 6000 \
	email_hdrs_log_depth 2920 \
	disabled
\end{verbatim}

\subsubsection{Default}

\begin{verbatim}
    preprocessor SMTP: \
        ports { 25 } \
        inspection_type stateful \
        normalize cmds \
        normalize_cmds { EXPN VRFY RCPT } \
        alt_max_command_line_len 260 { MAIL } \
        alt_max_command_line_len 300 { RCPT } \
        alt_max_command_line_len 500 { HELP HELO ETRN } \
        alt_max_command_line_len 255 { EXPN VRFY }
\end{verbatim}

\subsubsection{Note}

\texttt{RCPT TO:} and \texttt{MAIL FROM:} are SMTP commands.  For the
preprocessor configuration, they are referred to as RCPT and MAIL,
respectively.  Within the code, the preprocessor actually maps RCPT and MAIL to
the correct command name.

\subsection{POP Preprocessor}
\label{POP}

POP is an POP3 decoder for user applications. Given a data buffer,
POP will decode the buffer and find POP3 commands and responses.
It will also mark the command, data header data body sections and
extract the POP3 attachments and decode it appropriately.

POP will handle stateful processing. It saves state between individual
packets. However maintaining correct state is dependent on the reassembly
of the server side of the stream (i.e., a loss of coherent stream data results
in a loss of state).

Stream should be turned on for POP. Please ensure that the POP ports are added
 to the stream5 ports for proper reassembly.

The POP preprocessor uses GID 142 to register events.

\subsubsection{Configuration}

The configuration options are described below:

\begin{slist}

\item \texttt{ports \{ <port> [<port>] ... \}}

This specifies on what ports to check for POP data.  Typically, this will
include 110. Default ports if none are specified are 110 .

\item \texttt{disabled}

Disables the POP preprocessor in a config. This is useful when specifying
the decoding depths such as \texttt{b64\_decode\_depth}, \texttt{qp\_decode\_depth},
\texttt{uu\_decode\_depth}, \texttt{bitenc\_decode\_depth} or the memcap used for
decoding \texttt{memcap} in default config without turning on the POP preprocessor.

\item \texttt{b64\_decode\_depth}

This config option is used to turn off/on or set the base64 decoding depth used to
decode the base64 encoded MIME attachments. The value ranges from -1 to 65535.
A value of -1 turns off the base64 decoding of MIME attachments. The value of 0
sets the decoding of base64 encoded MIME attachments to unlimited. A value other
than 0 or -1 restricts the decoding of base64 MIME attachments, and applies per 
attachment. A POP preprocessor alert with sid 4 is generated (if enabled) when the decoding fails.

Multiple MIME attachments/data in one packet are pipelined. When stateful inspection
is turned on the base64 encoded MIME attachments/data across multiple packets are decoded too.

The decoded data is available for detection using the rule option \texttt{file\_data}.
See \ref{sub:file_data} rule option for more details.

It is recommended that user inputs a value that is a multiple of 4. When the value specified 
is not a multiple of 4, the POP preprocessor will round it up to the next multiple of 4.

In case of multiple configs, the value specified in the non-default config cannot
exceed the value specified in the default config.

\item \texttt{qp\_decode\_depth}

This config option is used to turn off/on or set the Quoted-Printable decoding depth
used to decode the Quoted-Printable(QP) encoded MIME attachments. The value ranges
from -1 to 65535. A value of -1 turns off the QP decoding of MIME attachments.
The value of 0 sets the decoding of QP encoded MIME attachments to unlimited. A
value other than 0 or -1 restricts the decoding of QP MIME attachments, and applies per 
attachment. A POP preprocessor alert with sid 5 is generated (if enabled) when the decoding fails.

Multiple MIME attachments/data in one packet are pipelined. When stateful inspection
is turned on the QP encoded MIME attachments/data across multiple packets are decoded too.

The decoded data is available for detection using the rule option \texttt{file\_data}.
See \ref{sub:file_data} rule option for more details.

In case of multiple configs, the value specified in the non-default config cannot exceed
the value specified in the default config.

\item \texttt{bitenc\_decode\_depth}

This config option is used to turn off/on or set the non-encoded MIME extraction
depth used to extract the non-encoded MIME attachments. The value ranges from -1 
to 65535. A value of -1 turns off the extraction of these MIME attachments. 
The value of 0 sets the extraction of these MIME attachments to unlimited.
A value other than 0 or -1 restricts the extraction of these MIME attachments, and applies 
per attachment.

Multiple MIME attachments/data in one packet are pipelined. When stateful inspection
is turned on the non-encoded MIME attachments/data across multiple packets are extracted too.

The extracted data is available for detection using the rule option \texttt{file\_data}.
See \ref{sub:file_data} rule option for more details.

In case of multiple configs, the value specified in the non-default config cannot exceed
the value specified in the default config.

\item \texttt{uu\_decode\_depth}

This config option is used to turn off/on or set the Unix-to-Unix decoding depth
used to decode the Unix-to-Unix(UU) encoded attachments. The value ranges
from -1 to 65535. A value of -1 turns off the UU decoding of POP attachments.
The value of 0 sets the decoding of UU encoded POP attachments to unlimited. A
value other than 0 or -1 restricts the decoding of UU POP attachments, and applies per 
attachment. A POP preprocessor alert with sid 7 is generated (if enabled) when the decoding fails.

Multiple UU attachments/data in one packet are pipelined. When stateful inspection
is turned on the UU encoded POP attachments/data across multiple packets are decoded too.

The decoded data is available for detection using the rule option \texttt{file\_data}.
See \ref{sub:file_data} rule option for more details.

In case of multiple configs, the value specified in the non-default config cannot exceed
the value specified in the default config.

\item \texttt{memcap <int>}

This option determines (in bytes) the maximum amount of memory the POP preprocessor
will use for decoding base64 encoded/quoted-printable/non-encoded MIME attachments/data
or Unix-to-Unix encoded attachments. This value can be set from 3276 bytes to 100MB.

This option along with the maximum of the decoding depths will determine the POP 
sessions that will be decoded at any given instant. The default value for this option
is 838860.

Note: It is suggested to set this value such that the max pop session calculated as
follows is at least 1.

max pop session = \texttt{memcap} /(2 * max of (\texttt{b64\_decode\_depth},
                                        \texttt{uu\_decode\_depth}, \texttt{qp\_decode\_depth}
                                        or \texttt{bitenc\_decode\_depth}))

For example, if \texttt{b64\_decode\_depth} is 0 (indicates unlimited decoding) and
\texttt{qp\_decode\_depth} is 100, then

max pop session = \texttt{memcap}/2*65535 (max value for \texttt{b64\_decode\_depth})

In case of multiple configs, the \texttt{memcap} of the non-default configs will be overwritten by the
default config's value. Hence user needs to define it in the default config with the new keyword
disabled (used to disable POP preprocessor in a config).

When the memcap for decoding (\texttt{memcap}) is exceeded the POP preprocessor alert with sid 3 is
generated (when enabled).

\end{slist}

\subsubsection{Example}

\begin{verbatim}
	preprocessor pop: \
	  ports { 110 } \
	  memcap 1310700 \
	  qp_decode_depth -1 \
	  b64_decode_depth 0 \
	  bitenc_decode_depth 100


	preprocessor pop: \
	  memcap 1310700 \
	  qp_decode_depth 0 \
	  disabled
\end{verbatim}

\subsubsection{Default}

\begin{verbatim}
	preprocessor pop: \
	  ports { 110 } \
	  b64_decode_depth 1460 \
	  qp_decode_depth 1460 \
	  bitenc_decode_depth 1460 \
	  uu_decode_depth 1460
\end{verbatim}

\subsection{IMAP Preprocessor}
\label{IMAP}

IMAP is an IMAP4 decoder for user applications. Given a data buffer,
IMAP will decode the buffer and find IMAP4 commands and responses.
It will also mark the command, data header data body sections and
extract the IMAP4 attachments and decode it appropriately.

IMAP will handle stateful processing. It saves state between individual
packets. However maintaining correct state is dependent on the reassembly
of the server side of the stream (i.e., a loss of coherent stream data results
in a loss of state).

Stream should be turned on for IMAP. Please ensure that the IMAP ports are added
 to the stream5 ports for proper reassembly.

The IMAP preprocessor uses GID 141 to register events.

\subsubsection{Configuration}

The configuration options are described below:

\begin{slist}

\item \texttt{ports \{ <port> [<port>] ... \}}

This specifies on what ports to check for IMAP data.  Typically, this will
include 143. Default ports if none are specified are 143 .

\item \texttt{disabled}

Disables the IMAP preprocessor in a config. This is useful when specifying
the decoding depths such as \texttt{b64\_decode\_depth}, \texttt{qp\_decode\_depth},
\texttt{uu\_decode\_depth}, \texttt{bitenc\_decode\_depth} or the memcap used for
decoding \texttt{memcap} in default config without turning on the IMAP preprocessor.

\item \texttt{b64\_decode\_depth}

This config option is used to turn off/on or set the base64 decoding depth used to
decode the base64 encoded MIME attachments. The value ranges from -1 to 65535.
A value of -1 turns off the base64 decoding of MIME attachments. The value of 0
sets the decoding of base64 encoded MIME attachments to unlimited. A value other
than 0 or -1 restricts the decoding of base64 MIME attachments, and applies per attachment. 
A IMAP preprocessor alert with sid 4 is generated (if enabled) when the decoding fails.

Multiple MIME attachments/data in one packet are pipelined. When stateful inspection
is turned on the base64 encoded MIME attachments/data across multiple packets are decoded too.

The decoded data is available for detection using the rule option \texttt{file\_data}.
See \ref{sub:file_data} rule option for more details.

It is recommended that user inputs a value that is a multiple of 4. When the value specified 
is not a multiple of 4, the IMAP preprocessor will round it up to the next multiple of 4.

In case of multiple configs, the value specified in the non-default config cannot
exceed the value specified in the default config.

\item \texttt{qp\_decode\_depth}

This config option is used to turn off/on or set the Quoted-Printable decoding depth
used to decode the Quoted-Printable(QP) encoded MIME attachments. The value ranges
from -1 to 65535. A value of -1 turns off the QP decoding of MIME attachments.
The value of 0 sets the decoding of QP encoded MIME attachments to unlimited. A
value other than 0 or -1 restricts the decoding of QP MIME attachments, and applies per 
attachment. A IMAP preprocessor alert with sid 5 is generated (if enabled) when the decoding fails.

Multiple MIME attachments/data in one packet are pipelined. When stateful inspection
is turned on the QP encoded MIME attachments/data across multiple packets are decoded too.

The decoded data is available for detection using the rule option \texttt{file\_data}.
See \ref{sub:file_data} rule option for more details.

In case of multiple configs, the value specified in the non-default config cannot exceed
the value specified in the default config.

\item \texttt{bitenc\_decode\_depth}

This config option is used to turn off/on or set the non-encoded MIME extraction
depth used to extract the non-encoded MIME attachments. The
value ranges from -1 to 65535. A value of -1 turns off the extraction of these MIME
attachments. The value of 0 sets the extraction of these MIME attachments to unlimited.
A value other than 0 or -1 restricts the extraction of these MIME attachments, and applies 
per attachment.

Multiple MIME attachments/data in one packet are pipelined. When stateful inspection
is turned on the non-encoded MIME attachments/data across multiple packets are extracted too.

The extracted data is available for detection using the rule option \texttt{file\_data}.
See \ref{sub:file_data} rule option for more details.

In case of multiple configs, the value specified in the non-default config cannot exceed
the value specified in the default config.

\item \texttt{uu\_decode\_depth}

This config option is used to turn off/on or set the Unix-to-Unix decoding depth
used to decode the Unix-to-Unix(UU) encoded attachments. The value ranges
from -1 to 65535. A value of -1 turns off the UU decoding of IMAP attachments.
The value of 0 sets the decoding of UU encoded IMAP attachments to unlimited. A
value other than 0 or -1 restricts the decoding of UU IMAP attachments, and applies per 
attachment. A IMAP preprocessor alert with sid 7 is generated (if enabled) when the decoding fails.

Multiple UU attachments/data in one packet are pipelined. When stateful inspection
is turned on the UU encoded IMAP attachments/data across multiple packets are decoded too.

The decoded data is available for detection using the rule option \texttt{file\_data}.
See \ref{sub:file_data} rule option for more details.

In case of multiple configs, the value specified in the non-default config cannot exceed
the value specified in the default config.

\item \texttt{memcap <int>}

This option determines (in bytes) the maximum amount of memory the IMAP preprocessor
will use for decoding base64 encoded/quoted-printable/non-encoded MIME attachments/data
or Unix-to-Unix encoded attachments. This value can be set from 3276 bytes to 100MB.

This option along with the maximum of the decoding depths will determine the IMAP 
sessions that will be decoded at any given instant. The default value for this option
is 838860.

Note: It is suggested to set this value such that the max imap session calculated as
follows is at least 1.

max imap session = \texttt{memcap} /(2 * max of (\texttt{b64\_decode\_depth},
                                        \texttt{uu\_decode\_depth}, \texttt{qp\_decode\_depth}
                                        or \texttt{bitenc\_decode\_depth}))

For example, if \texttt{b64\_decode\_depth} is 0 (indicates unlimited decoding) and
\texttt{qp\_decode\_depth} is 100, then

max imap session = \texttt{memcap}/2*65535 (max value for \texttt{b64\_decode\_depth})

In case of multiple configs, the \texttt{memcap} of the non-default configs will be overwritten by the
default config's value. Hence user needs to define it in the default config with the new keyword
disabled (used to disable IMAP preprocessor in a config).

When the memcap for decoding (\texttt{memcap}) is exceeded the IMAP preprocessor alert with sid 3 is
generated (when enabled).

\end{slist}

\subsubsection{Example}

\begin{verbatim}
	preprocessor imap: \
	  ports { 110 } \
	  memcap 1310700 \
	  qp_decode_depth -1 \
	  b64_decode_depth 0 \
	  bitenc_decode_depth 100


	preprocessor imap: \
	  memcap 1310700 \
	  qp_decode_depth 0 \
	  disabled
\end{verbatim}

\subsubsection{Default}

\begin{verbatim}
	preprocessor imap: \
	  ports { 110 } \
	  b64_decode_depth 1460 \
	  qp_decode_depth 1460 \
	  bitenc_decode_depth 1460 \
	  uu_decode_depth 1460
\end{verbatim}

\subsection{FTP/Telnet Preprocessor}
\label{sub:ftptelnet}

FTP/Telnet is an improvement to the Telnet decoder and provides stateful
inspection capability for both FTP and Telnet data streams.  FTP/Telnet will
decode the stream, identifying FTP commands and responses and Telnet escape
sequences and normalize the fields.  FTP/Telnet works on both client requests
and server responses.

FTP/Telnet has the capability to handle stateless processing, meaning it only
looks for information on a packet-by-packet basis.  

The default is to run FTP/Telnet in stateful inspection mode, meaning it looks
for information and handles reassembled data correctly.

FTP/Telnet has a very ``rich'' user configuration, similar to that of HTTP
Inspect (See \ref{sub:http-inspect}).  Users can configure individual FTP
servers and clients with a variety of options, which should allow the user to
emulate any type of FTP server or FTP Client.  Within FTP/Telnet, there are
four areas of configuration: Global, Telnet, FTP Client, and FTP Server.

\begin{note}

Some configuration options have an argument of \texttt{yes} or \texttt{no}.
This argument specifies whether the user wants the configuration option to
generate a ftptelnet alert or not.  The presence of the option indicates the
option itself is on, while the \texttt{yes/no} argument applies to the alerting
functionality associated with that option.

\end{note}

\subsubsection{Global Configuration}

The global configuration deals with configuration options that determine the
global functioning of FTP/Telnet.  The following example gives the generic
global configuration format:

\subsubsection{Format}

\begin{verbatim}
    preprocessor ftp_telnet: \
        global \
        inspection_type stateful \
        encrypted_traffic yes \
        check_encrypted
\end{verbatim}

You can only have a single global configuration, you'll get an error if you try
otherwise.  The FTP/Telnet global configuration must appear before the other
three areas of configuration.

\paragraph{Configuration}
\begin{slist}
\item \texttt{inspection\_type}

This indicates whether to operate in stateful or stateless mode.

\item \texttt{encrypted\_traffic $<$yes|no$>$}

This option enables detection and alerting on encrypted Telnet and FTP command
channels.

\begin{note}

When \texttt{inspection\_type} is in stateless mode, checks for encrypted
traffic will occur on every packet, whereas in stateful mode, a particular
session will be noted as encrypted and not inspected any further.

\end{note}

\item \texttt{check\_encrypted}

Instructs the preprocessor to continue to check an encrypted session for a
subsequent command to cease encryption.

\end{slist}

\subsubsection{Example Global Configuration}

\begin{verbatim}
    preprocessor ftp_telnet: \
        global inspection_type stateful encrypted_traffic no
\end{verbatim}

\subsubsection{Telnet Configuration}

The telnet configuration deals with configuration options that determine the
functioning of the Telnet portion of the preprocessor.  The following example
gives the generic telnet configuration format:

\subsubsection{Format}
\begin{verbatim}
    preprocessor ftp_telnet_protocol: \
        telnet \
        ports { 23 } \
        normalize \
        ayt_attack_thresh 6 \
        detect_anomalies

\end{verbatim}

There should  only be a single telnet configuration, and subsequent instances
will override previously set values.

\paragraph{Configuration}
\begin{slist}
\item \texttt{ports $\{ <$port$> [<$port$> <...>] \}$}

This is how the user configures which ports to decode as telnet traffic.  SSH
tunnels cannot be decoded, so adding port 22 will only yield false positives.
Typically port 23 will be included.

\item \texttt{normalize}

This option tells the preprocessor to normalize the telnet traffic by
eliminating the telnet escape sequences.  It functions similarly to its
predecessor, the telnet\_decode preprocessor.  Rules written with 'raw' content
options will ignore the normalized buffer that is created when this option is
in use.

\item \texttt{ayt\_attack\_thresh $<$ number $>$}

This option causes the preprocessor to alert when the number of consecutive
telnet Are You There (AYT) commands reaches the number specified.  It is only
applicable when the mode is stateful.

\item \texttt{detect\_anomalies}

In order to support certain options, Telnet supports subnegotiation. Per the
Telnet RFC, subnegotiation begins with SB (subnegotiation begin) and must end
with an SE (subnegotiation end). However, certain implementations of Telnet
servers will ignore the SB without a corresponding SE. This is anomalous
behavior which could be an evasion case. Being that FTP uses the Telnet
protocol on the control connection, it is also susceptible to this behavior.
The \texttt{detect\_anomalies} option enables alerting on Telnet SB without the
corresponding SE.

\end{slist}

\subsubsection{Example Telnet Configuration}

\begin{verbatim}
    preprocessor ftp_telnet_protocol: \
        telnet ports { 23 } normalize ayt_attack_thresh 6
\end{verbatim}

\subsubsection{FTP Server Configuration}

There are two types of FTP server configurations: default and by IP address.

\paragraph{Default}

This configuration supplies the default server configuration for any FTP server
that is not individually configured.  Most of your FTP servers will most likely
end up using the default configuration.

\subsubsection{Example Default FTP Server Configuration}

\begin{verbatim}
    preprocessor ftp_telnet_protocol: \
        ftp server default ports { 21 } 
\end{verbatim}

Refer to \pageref{sub:default ftp server config} for the list of options set in default ftp server configuration.

\paragraph{Configuration by IP Address}

This format is very similar to ``default'', the only difference being that
specific IPs can be configured.

\subsubsection{Example IP specific FTP Server Configuration}

\begin{verbatim}
    preprocessor _telnet_protocol: \
        ftp server 10.1.1.1 ports { 21 } ftp_cmds { XPWD XCWD }
\end{verbatim}

\subsubsection{FTP Server Configuration Options}

\begin{slist}
\item \texttt{ports $\{ <$port$> [<$port$> <...>] \}$}

This is how the user configures which ports to decode as FTP command channel
traffic.  Typically port 21 will be included.

\item \texttt{print\_cmds}

During initialization, this option causes the preprocessor to print the
configuration for each of the FTP commands for this server.  

\item \texttt{ftp\_cmds $\{ cmd [cmd] \}$ }

The preprocessor is configured to alert when it sees an FTP command that is not
allowed by the server.

This option specifies a list of additional commands allowed by this server,
outside of the default FTP command set as specified in RFC 959.  This may be
used to allow the use of the 'X' commands identified in RFC 775, as well as any
additional commands as needed.

For example:

\begin{verbatim}
    ftp_cmds { XPWD XCWD XCUP XMKD XRMD }
\end{verbatim}

\item \texttt{def\_max\_param\_len $<$number$>$}

This specifies the default maximum allowed parameter length for an FTP command.
It can be used as a basic buffer overflow detection.

\item \texttt{alt\_max\_param\_len $<$number$>$ $\{ cmd [cmd] \}$}

This specifies the maximum allowed parameter length for the specified FTP
command(s).  It can be used as a more specific buffer overflow detection.  For
example the USER command -- usernames may be no longer than 16 bytes, so the
appropriate configuration would be:

\begin{verbatim}
    alt_max_param_len 16 { USER }
\end{verbatim}

\item \texttt{chk\_str\_fmt $\{ cmd [cmd] \}$}

This option causes a check for string format attacks in the specified commands. 

\item \texttt{cmd\_validity cmd $<$ fmt $>$}

This option specifies the valid format for parameters of a given command.

fmt must be enclosed in $<>$'s and may contain the following:

\begin{center}
\begin{tabular}{| l | p{3in} |}

\hline
\textbf{Value} & \textbf{Description} \\
\hline

\hline
int & Parameter must be an integer \\

\hline
number & Parameter must be an integer between 1 and 255 \\

\hline
char $<$chars$>$ & Parameter must be a single character, one of $<$chars$>$ \\

\hline
date $<$datefmt$>$ & Parameter follows format specified, where:

\begin{tabular}{ l l }
n & Number \\
C & Character \\
$[]$ & optional format enclosed \\
$|$ & OR \\
$\{\}$ & choice of options \\
. + - & literal \\
\end{tabular} \\

\hline
string & Parameter is a string (effectively unrestricted) \\

\hline
host\_port & Parameter must be a host/port specified, per RFC 959 \\

\hline
long\_host\_port & Parameter must be a long host port specified, per RFC 1639 \\

\hline
extended\_host\_port & Parameter must be an extended host port specified, per RFC 2428 \\

\hline
$\{\}$, $|$ & One of choices enclosed within, separated by $|$ \\

\hline
$\{\}$, $[]$ & One of the choices enclosed within $\{\}$, optional value enclosed within $[]$ \\

\hline
\end{tabular}
\end{center}

Examples of the cmd\_validity option are shown below.  These examples are the
default checks, per RFC 959 and others performed by the preprocessor.

\begin{verbatim}
    cmd_validity MODE <char SBC>
    cmd_validity STRU <char FRP>
    cmd_validity ALLO < int [ char R int ] >
    cmd_validity TYPE < { char AE [ char NTC ] | char I | char L [ number ] } >
    cmd_validity PORT < host_port >
\end{verbatim}

A cmd\_validity line can be used to override these defaults and/or add a check
for other commands.

\begin{verbatim}
    # This allows additional modes, including mode Z which allows for
    # zip-style compression.
    cmd_validity MODE < char ASBCZ >
    
    # Allow for a date in the MDTM command.
    cmd_validity MDTM < [ date nnnnnnnnnnnnnn[.n[n[n]]] ] string >
\end{verbatim}

MDTM is an off case that is worth discussing.  While not part of an established
standard, certain FTP servers accept MDTM commands that set the modification
time on a file.  The most common among servers that do, accept a format using
YYYYMMDDHHmmss[.uuu].  Some others accept a format using YYYYMMDDHHmmss[+|-]TZ
format.  The example above is for the first case (time format as specified in
http://www.ietf.org/internet-drafts/draft-ietf-ftpext-mlst-16.txt)

To check validity for a server that uses the TZ format, use the following:

\begin{verbatim}
    cmd_validity MDTM < [ date nnnnnnnnnnnnnn[{+|-}n[n]] ] string >
\end{verbatim}

\item \texttt{telnet\_cmds $<$yes$|$no$>$}

This option turns on detection and alerting when telnet escape sequences are
seen on the FTP command channel.  Injection of telnet escape sequences could be
used as an evasion attempt on an FTP command channel.

\item \texttt{ignore\_telnet\_erase\_cmds $<$yes|no$>$}

This option allows Snort to ignore telnet escape sequences for erase character
(TNC EAC) and erase line (TNC EAL) when normalizing FTP command channel.  Some FTP
servers do not process those telnet escape sequences.

\item \texttt{data\_chan}

This option causes the rest of snort (rules, other preprocessors) to ignore FTP
data channel connections.  Using this option means that \textbf{NO INSPECTION}
other than TCP state will be performed on FTP data transfers.  It can be used
to improve performance, especially with large file transfers from a trusted
source.   If your rule set includes virus-type rules, it is recommended that
this option not be used.

Use of the "data\_chan" option is deprecated in favor of the
"ignore\_data\_chan" option. "data\_chan" will be removed in a future release.

\item \texttt{ignore\_data\_chan $<$yes$|$no$>$}

This option causes the rest of Snort (rules, other preprocessors) to ignore FTP
data channel connections.  Setting this option to "yes" means that \textbf{NO
INSPECTION} other than TCP state will be performed on FTP data transfers.  It
can be used to improve performance, especially with large file transfers from a
trusted source.  If your rule set includes virus-type rules, it is recommended
that this option not be used.

\end{slist}

\subsubsection{FTP Server Base Configuration Options}
\label{sub:default ftp server config}

The base FTP server configuration is as follows.  Options specified in the
configuration file will modify this set of options.  FTP commands are added to
the set of allowed commands.  The other options will override those in the base
configuration.

\begin{verbatim}
    def_max_param_len 100
    ftp_cmds { USER PASS ACCT CWD CDUP SMNT 
	       QUIT REIN TYPE STRU MODE RETR 
	       STOR STOU APPE ALLO REST RNFR 
	       RNTO ABOR DELE RMD MKD PWD LIST 
               NLST SITE SYST STAT HELP NOOP } 
    ftp_cmds { AUTH ADAT PROT PBSZ CONF ENC } 
    ftp_cmds { PORT PASV LPRT LPSV EPRT EPSV } 
    ftp_cmds { FEAT OPTS } 
    ftp_cmds { MDTM REST SIZE MLST MLSD } 
    alt_max_param_len 0 { CDUP QUIT REIN PASV STOU ABOR PWD SYST NOOP } 
    cmd_validity MODE < char SBC > 
    cmd_validity STRU < char FRPO [ string ] > 
    cmd_validity ALLO < int [ char R int ] > 
    cmd_validity TYPE < { char AE [ char NTC ] | char I | char L [ number ] } > 
    cmd_validity PORT < host_port > 
    cmd_validity LPRT < long_host_port > 
    cmd_validity EPRT < extd_host_port > 
    cmd_validity EPSV < [ { '1' | '2' | 'ALL' } ] >
\end{verbatim}

\subsubsection{FTP Client Configuration}

Similar to the FTP Server configuration, the FTP client configurations has two
types: default, and by IP address.

\paragraph{Default}

This configuration supplies the default client configuration for any FTP client
that is not individually configured.  Most of your FTP clients will most likely
end up using the default configuration.

\subsubsection{Example Default FTP Client Configuration}
\begin{verbatim}
    preprocessor ftp_telnet_protocol: \
        ftp client default bounce no max_resp_len 200
\end{verbatim}

\paragraph{Configuration by IP Address}

This format is very similar to ``default'', the only difference being that
specific IPs can be configured.

\subsubsection{Example IP specific FTP Client Configuration}

\begin{verbatim}
    preprocessor ftp_telnet_protocol: \
        ftp client 10.1.1.1 bounce yes max_resp_len 500
\end{verbatim}

\subsubsection{FTP Client Configuration Options}

\begin{slist}
\item \texttt{max\_resp\_len $<$number$>$}

This specifies the maximum allowed response length to an FTP command accepted
by the client.  It can be used as a basic buffer overflow detection.

\item \texttt{bounce $<$yes|no$>$}

This option turns on detection and alerting of FTP bounce attacks.  An FTP
bounce attack occurs when the FTP PORT command is issued and the specified host
does not match the host of the client.

\item \texttt{bounce\_to $<$ CIDR,[port$|$portlow,porthi] $>$}

When the bounce option is turned on, this allows the PORT command to use the IP
address (in CIDR format) and port (or inclusive port range) without generating
an alert.  It can be used to deal with proxied FTP connections where the FTP
data channel is different from the client.

A few examples:

\begin{itemize}
\item Allow bounces to 192.162.1.1 port 20020 -- i.e., the use of
\texttt{PORT 192,168,1,1,78,52}.

\begin{verbatim}
    bounce_to { 192.168.1.1,20020 }
\end{verbatim}

\item Allow bounces to 192.162.1.1 ports 20020 through 20040 -- i.e., the use of
\texttt{PORT 192,168,1,1,78,xx}, where xx is 52 through 72 inclusive.

\begin{verbatim}
    bounce_to { 192.168.1.1,20020,20040 }
\end{verbatim}

\item Allow bounces to 192.162.1.1 port 20020 and 192.168.1.2 port 20030.

\begin{verbatim}
    bounce_to { 192.168.1.1,20020 192.168.1.2,20030 }
\end{verbatim}

\item Allows bounces to IPv6 address fe8::5 port 59340.  

\begin{verbatim}
    bounce_to { fe8::5,59340 }
\end{verbatim}


\end{itemize}

\item \texttt{telnet\_cmds $<$yes|no$>$}

This option turns on detection and alerting when telnet escape sequences are
seen on the FTP command channel.  Injection of telnet escape sequences could be
used as an evasion attempt on an FTP command channel.

\item \texttt{ignore\_telnet\_erase\_cmds $<$yes|no$>$}

This option allows Snort to ignore telnet escape sequences for erase character
(TNC EAC) and erase line (TNC EAL) when normalizing FTP command channel.  Some FTP
clients do not process those telnet escape sequences.

\end{slist}

\subsubsection{Examples/Default Configuration from snort.conf}
\begin{verbatim}
    preprocessor ftp_telnet: \
        global \
        encrypted_traffic yes \
        inspection_type stateful

    preprocessor ftp_telnet_protocol:\
        telnet \
        normalize \
        ayt_attack_thresh 200

    # This is consistent with the FTP rules as of 18 Sept 2004.
    # Set CWD to allow parameter length of 200
    # MODE has an additional mode of Z (compressed)
    # Check for string formats in USER & PASS commands
    # Check MDTM commands that set modification time on the file.

    preprocessor ftp_telnet_protocol: \
        ftp server default \
        def_max_param_len 100 \
        alt_max_param_len 200 { CWD } \
        cmd_validity MODE < char ASBCZ > \
        cmd_validity MDTM < [ date nnnnnnnnnnnnnn[.n[n[n]]] ] string > \
        chk_str_fmt { USER PASS RNFR RNTO SITE MKD } \
        telnet_cmds yes \
        ignore_data_chan yes

    preprocessor ftp_telnet_protocol: \
        ftp client default \
        max_resp_len 256 \
        bounce yes \
        telnet_cmds yes
\end{verbatim}

\subsection{SSH}
\label{sub:ssh}

The SSH preprocessor detects the following exploits: Challenge-Response Buffer
Overflow, CRC 32, Secure CRT, and the Protocol Mismatch exploit.

Both Challenge-Response Overflow and CRC 32 attacks occur after the key
exchange, and are therefore encrypted.  Both attacks involve sending a large
payload (20kb+) to the server immediately after the authentication challenge.
To detect the attacks, the SSH preprocessor counts the number of bytes
transmitted to the server.  If those bytes exceed a predefined limit within a
predefined number of packets, an alert is generated.  Since the
Challenge-Response Overflow only effects SSHv2 and CRC 32 only effects SSHv1,
the SSH version string exchange is used to distinguish the attacks.

The Secure CRT and protocol mismatch exploits are observable before the key
exchange.

\subsubsection{Configuration}

By default, all alerts are disabled and the preprocessor checks traffic on port
22.

The available configuration options are described below.

\begin{slist}

\item \texttt{server\_ports $\{ <$port$> [<$port$> <...>] \}$}

This option specifies which ports the SSH preprocessor should inspect traffic
to.

\item \texttt{max\_encrypted\_packets $<$ number $>$}

The number of stream reassembled encrypted packets that Snort will inspect before ignoring a given
SSH session. The SSH vulnerabilities that Snort can detect all happen at the
very beginning of an SSH session. Once max\_encrypted\_packets packets have been
seen, Snort ignores the session to increase performance. The default is set to 25.
This value can be set from 0 to 65535.

\item \texttt{max\_client\_bytes $<$ number $>$}

The number of unanswered bytes allowed to be transferred before alerting on
Challenge-Response Overflow or CRC 32. This number must be hit before
max\_encrypted\_packets packets are sent, or else Snort will ignore the traffic.
The default is set to 19600. This value can be set from 0 to 65535.

\item \texttt{max\_server\_version\_len $<$ number $>$}

The maximum number of bytes allowed in the SSH server version string before
alerting on the Secure CRT server version string overflow. The default is set to
80. This value can be set from 0 to 255.

\item \texttt{autodetect}

Attempt to automatically detect SSH.

\item \texttt{enable\_respoverflow}

Enables checking for the Challenge-Response Overflow exploit.

\item \texttt{enable\_ssh1crc32}

Enables checking for the CRC 32 exploit.

\item \texttt{enable\_srvoverflow}

Enables checking for the Secure CRT exploit.

\item \texttt{enable\_protomismatch}

Enables checking for the Protocol Mismatch exploit.

\item \texttt{enable\_badmsgdir}

Enable alerts for traffic flowing the wrong direction. For instance, if the
presumed server generates client traffic, or if a client generates server
traffic.

\item \texttt{enable\_paysize}

Enables alerts for invalid payload sizes.

\item \texttt{enable\_recognition}

Enable alerts for non-SSH traffic on SSH ports.

\end{slist}

The SSH preprocessor should work by default.  After max\_encrypted\_packets is
reached, the preprocessor will stop processing traffic for a given session.  If
Challenge-Response Overflow or CRC 32 false positive, try increasing the number
of required client bytes with max\_client\_bytes.

\subsubsection{Example Configuration from snort.conf}

Looks for attacks on SSH server port 22.  Alerts at 19600 unacknowledged bytes
within 20 encrypted packets for the Challenge-Response Overflow/CRC32 exploits.

\begin{verbatim}
    preprocessor ssh: \
        server_ports { 22 } \
        max_client_bytes 19600 \
        max_encrypted_packets 20 \
        enable_respoverflow \
        enable_ssh1crc32
\end{verbatim}

\subsection{DNS}
\label{sub:dns}

The DNS preprocessor decodes DNS Responses and can detect the following
exploits: DNS Client RData Overflow, Obsolete Record Types, and Experimental
Record Types.

DNS looks at DNS Response traffic over UDP and TCP and it requires Stream
preprocessor to be enabled for TCP decoding.

\subsubsection{Configuration}

By default, all alerts are disabled and the preprocessor checks traffic on port
53.

The available configuration options are described below.

\begin{slist}

\item \texttt{ports $\{ <$port$> [<$port$> <...>] \}$}

This option specifies the source ports that the DNS preprocessor should inspect
traffic.

\item \texttt{enable\_obsolete\_types}

Alert on Obsolete (per RFC 1035) Record Types

\item \texttt{enable\_experimental\_types}

Alert on Experimental (per RFC 1035) Record Types

\item \texttt{enable\_rdata\_overflow}

Check for DNS Client RData TXT Overflow

\end{slist}

The DNS preprocessor does nothing if none of the 3 vulnerabilities it checks
for are enabled.  It will not operate on TCP sessions picked up midstream, and
it will cease operation on a session if it loses state because of missing data
(dropped packets).

\subsubsection{Examples/Default Configuration from snort.conf}

Looks for traffic on DNS server port 53.  Check for the DNS Client RData
overflow vulnerability.  Do not alert on obsolete or experimental RData record
types.

\begin{verbatim}
    preprocessor dns: \
        ports { 53 } \
        enable_rdata_overflow
\end{verbatim}

\subsection{SSL/TLS}
\label{sub:SSL/TLS}

Encrypted traffic should be ignored by Snort for both performance reasons and
to reduce false positives.  The SSL Dynamic Preprocessor (SSLPP) decodes SSL
and TLS traffic and optionally determines if and when Snort should stop
inspection of it.

Typically, SSL is used over port 443 as HTTPS.  By enabling the SSLPP to
inspect port 443 and enabling the noinspect\_encrypted option, only the SSL
handshake of each connection will be inspected.  Once the traffic is determined
to be encrypted, no further inspection of the data on the connection is made.

By default, SSLPP looks for a handshake followed by encrypted traffic traveling
to both sides.  If one side responds with an indication that something has
failed, such as the handshake, the session is not marked as encrypted.
Verifying that faultless encrypted traffic is sent from both endpoints ensures
two things: the last client-side handshake packet was not crafted to evade
Snort, and that the traffic is legitimately encrypted.

In some cases, especially when packets may be missed, the only observed
response from one endpoint will be TCP ACKs.  Therefore, if a user knows that
server-side encrypted data can be trusted to mark the session as encrypted, the
user should use the 'trustservers' option, documented below.

\subsubsection{Configuration}

\begin{slist}

\item \texttt{ports $\{ <$port$> [<$port$> <...>] \}$}

This option specifies which ports SSLPP will inspect traffic on.

By default, SSLPP watches the following ports:
    
\begin{itemize}
    \item \texttt{443}     HTTPS   
    \item \texttt{465}     SMTPS   
    \item \texttt{563}     NNTPS                                                                
    \item \texttt{636}     LDAPS
    \item \texttt{989}     FTPS
    \item \texttt{992}     TelnetS  
    \item \texttt{993}     IMAPS                 
    \item \texttt{994}     IRCS
    \item \texttt{995}     POPS
\end{itemize}

\item \texttt{noinspect\_encrypted}

Disable inspection on traffic that is encrypted.  Default is off.

\item \texttt{max\_heartbeat\_length}

Maximum length of heartbeat record allowed.  This config option is 
used to detect the heartbleed attacks.  The allowed range is 0 to 65535. 
Setting the value to 0 turns off the heartbeat length checks. For heartbeat 
requests, if the payload size of the request record is greater than the 
max\_heartbeat\_length an alert with sid 3 and gid 137 is generated.  
For heartbeat responses, if the record size itself is greater than 
the max\_heartbeat\_length an alert with sid 4 and gid 137 is generated. Default
is off.

\item \texttt{trustservers}

Disables the requirement that application (encrypted) data must be observed on
both sides of the session before a session is marked encrypted.  Use this
option for slightly better performance if you trust that your servers are not
compromised.  This requires the \texttt{noinspect\_encrypted} option to be
useful.  Default is off.  \end{slist}

\subsubsection{Examples/Default Configuration from snort.conf}

Enables the SSL preprocessor and tells it to disable inspection on encrypted
traffic.

\begin{verbatim}
    preprocessor ssl: noinspect_encrypted
\end{verbatim}

\subsubsection{Rule Options}

The following rule options are supported by enabling the \texttt{ssl} preprocessor:

\begin{itemize}
\item[]
\begin{verbatim}
    ssl_version
    ssl_state
\end{verbatim}
\end{itemize}

\texttt{ssl\_version}
\label{ssl:ssl_version}
\begin{itemize}

\item[] The \texttt{ssl\_version} rule option tracks the version negotiated between
the endpoints of the SSL encryption.  The list of version identifiers are below, and
more than one identifier can be specified, via a comma separated list.  Lists of
identifiers are OR'ed together.

The option will match if any one of the OR'ed versions are used in the SSL
connection.  To check for two or more SSL versions in use simultaneously, multiple
\texttt{ssl\_version} rule options should be used.

\textit{Syntax}
\footnotesize
\begin{verbatim}
   ssl_version: <version-list>

   version-list = version | version , version-list
   version      = ["!"] "sslv2" | "sslv3" | "tls1.0" | "tls1.1" | "tls1.2"
\end{verbatim}

\textit{Examples}
\begin{verbatim}
   ssl_version:sslv3;
   ssl_version:tls1.0,tls1.1,tls1.2;
   ssl_version:!sslv2;
\end{verbatim}

\end{itemize}

\texttt{ssl\_state}
\label{ssl:ssl_state}
\begin{itemize}
    
\item[] The \texttt{ssl\_state} rule option tracks the state of the SSL encryption
during the process of hello and key exchange.  The list of states are below.  More than
one state can be specified, via a comma separated list, and are OR'ed together.

The option will match if the connection is currently in any one of the OR'ed states.
To ensure the connection has reached each of a set of states, multiple rules using
the \texttt{ssl\_state} rule option should be used.

\textit{Syntax}
\footnotesize
\begin{verbatim}
   ssl_state: <state-list>

   state-list = state | state , state-list
   state      = ["!"] "client_hello" | "server_hello" | "client_keyx" | "server_keyx" | "unknown"
\end{verbatim}


\textit{Examples}
\begin{verbatim}
   ssl_state:client_hello;
   ssl_state:client_keyx,server_keyx;
   ssl_state:!server_hello;
\end{verbatim}

\end{itemize}

\subsection{ARP Spoof Preprocessor}
\label{sub:arpspoof}

The ARP spoof preprocessor decodes ARP packets and detects ARP attacks, unicast
ARP requests, and inconsistent Ethernet to IP mapping.

When no arguments are specified to arpspoof, the preprocessor inspects Ethernet
addresses and the addresses in the ARP packets. When inconsistency occurs, an
alert with GID 112 and SID 2 or 3 is generated.

When "\texttt{-unicast}" is specified as the argument of arpspoof, the
preprocessor checks for unicast ARP requests. An alert with GID 112 and SID 1
will be generated if a unicast ARP request is detected.

Specify a pair of IP and hardware address as the argument to
\texttt{arpspoof\_detect\_host}.  The host with the IP address should be on the
same layer 2 segment as Snort is.  Specify one host IP MAC combo per line. The
preprocessor will use this list when detecting ARP cache overwrite attacks.
Alert SID 4 is used in this case.

\subsubsection{Format}

\begin{verbatim}
    preprocessor arpspoof[: -unicast]
    preprocessor arpspoof_detect_host: ip mac                   
\end{verbatim}

\begin{table}[h]
\begin{center}
\begin{tabular}{| l | l |}

\hline 
\textbf{Option} & \textbf{Description}\\
\hline

\hline
\texttt{ip} & IP address.\\

\hline 
\texttt{mac} & The Ethernet address corresponding to the preceding IP. \\
\hline

\end{tabular}
\end{center}
\end{table}

\subsubsection{Example Configuration}

The first example configuration does neither unicast detection nor ARP mapping
monitoring. The preprocessor merely looks for Ethernet address inconsistencies.

\begin{verbatim}
    preprocessor arpspoof
\end{verbatim}

The next example configuration does not do unicast detection but monitors ARP
mapping for hosts 192.168.40.1 and 192.168.40.2.

\begin{verbatim}
    preprocessor arpspoof
    preprocessor arpspoof_detect_host: 192.168.40.1 f0:0f:00:f0:0f:00
    preprocessor arpspoof_detect_host: 192.168.40.2 f0:0f:00:f0:0f:01
\end{verbatim}

The third example configuration has unicast detection enabled.

\begin{verbatim}
    preprocessor arpspoof: -unicast
    preprocessor arpspoof_detect_host: 192.168.40.1 f0:0f:00:f0:0f:00
    preprocessor arpspoof_detect_host: 192.168.40.2 f0:0f:00:f0:0f:01
\end{verbatim}

\subsection{DCE/RPC 2 Preprocessor}
\label{sub:dcerpc2}

The main purpose of the preprocessor is to perform SMB desegmentation and
DCE/RPC defragmentation to avoid rule evasion using these techniques.  SMB
desegmentation is performed for the following commands that can be used to
transport DCE/RPC requests and responses: \texttt{Write}, \texttt{Write Block
Raw}, \texttt{Write and Close}, \texttt{Write AndX}, \texttt{Transaction},
\texttt{Transaction Secondary}, \texttt{Read}, \texttt{Read Block Raw} and
\texttt{Read AndX}.  The following transports are supported for DCE/RPC: SMB,
TCP, UDP and RPC over HTTP v.1 proxy and server.  New rule options have been
implemented to improve performance, reduce false positives and reduce the count
and complexity of DCE/RPC based rules.

\subsubsection{Dependency Requirements}

For proper functioning of the preprocessor:

\begin{itemize}

\item Stream session tracking must be enabled, i.e. \texttt{stream5}.  The
preprocessor requires a session tracker to keep its data.

\item Stream reassembly must be performed for TCP sessions. If it is decided
that a session is SMB or DCE/RPC, either through configured ports, servers or
autodetecting, the \texttt{dcerpc2} preprocessor will enable stream reassembly
for that session if necessary.

\item IP defragmentation should be enabled, i.e. the \texttt{frag3}
preprocessor should be enabled and configured.

\end{itemize}

\subsubsection{Target Based}

There are enough important differences between Windows and Samba versions that
a target based approach has been implemented. Some important differences:\\

\textit{Named pipe instance tracking}

\begin{itemize}

\item[] A combination of valid login handle or UID, share handle or TID and
file/named pipe handle or FID must be used to write data to a named pipe.  The
binding between these is dependent on OS/software version.

\begin{itemize}

\item[] Samba 3.0.22 and earlier

\begin{itemize}

\item[] Any valid UID and TID, along with a valid FID can be used to make a
request, however, if the TID used in creating the FID is deleted (via a tree
disconnect), the FID that was created using this TID becomes invalid, i.e. no
more requests can be written to that named pipe instance.

\end{itemize}

\item[] Samba greater than 3.0.22

\begin{itemize}

\item[] Any valid TID, along with a valid FID can be used to make a request.
However, only the UID used in opening the named pipe can be used to make a
request using the FID handle to the named pipe instance. If the TID used to
create the FID is deleted (via a tree disconnect), the FID that was created
using this TID becomes invalid, i.e. no more requests can be written to that
named pipe instance. If the UID used to create the named pipe instance is
deleted (via a \texttt{Logoff AndX}), since it is necessary in making a request
to the named pipe, the FID becomes invalid.

\end{itemize}

\item[] Windows 2003
\item[] Windows XP
\item[] Windows Vista

\begin{itemize}

\item[] These Windows versions require strict binding between the UID, TID and
FID used to make a request to a named pipe instance. Both the UID and TID used
to open the named pipe instance must be used when writing data to the same
named pipe instance. Therefore, deleting either the UID or TID invalidates the
FID.

\end{itemize}

\item[] Windows 2000

\begin{itemize}

\item[] Windows 2000 is interesting in that the first request to a named pipe
must use the same binding as that of the other Windows versions. However,
requests after that follow the same binding as Samba 3.0.22 and earlier, i.e.
no binding. It also follows Samba greater than 3.0.22 in that deleting the UID
or TID used to create the named pipe instance also invalidates it.

\end{itemize}
\end{itemize}
\end{itemize}

\textit{Accepted SMB commands}
\begin{itemize}

\item[] Samba in particular does not recognize certain commands under an
\texttt{IPC\$} tree.
\begin{itemize}
\item[] Samba (all versions)
\begin{itemize}
\item[] Under an \texttt{IPC\$} tree, does not accept:
\begin{itemize}
\item[] \texttt{Open}
\item[] \texttt{Write And Close}
\item[] \texttt{Read}
\item[] \texttt{Read Block Raw}
\item[] \texttt{Write Block Raw}
\end{itemize}
\end{itemize}

\item[] Windows (all versions)
\begin{itemize}

\item[] Accepts all of the above commands under an \texttt{IPC\$} tree.
\end{itemize}
\end{itemize}
\end{itemize}

\textit{AndX command chaining}
\begin{itemize}

\item[] Windows is very strict in what command combinations it allows to be
chained. Samba, on the other hand, is very lax and allows some nonsensical
combinations, e.g. multiple logins and tree connects (only one place to return
handles for these), login/logoff and tree connect/tree disconnect.  Ultimately,
we don't want to keep track of data that the server won't accept. An evasion
possibility would be accepting a fragment in a request that the server won't
accept that gets sandwiched between an exploit.

\end{itemize}

\textit{Transaction tracking}
\begin{itemize}

\item[] The differences between a \texttt{Transaction} request and using one of
the \texttt{Write*} commands to write data to a named pipe are that (1) a
\texttt{Transaction} performs the operations of a write and a read from the
named pipe, whereas in using the \texttt{Write*} commands, the client has to
explicitly send one of the \texttt{Read*} requests to tell the server to send
the response and (2) a \texttt{Transaction} request is not written to the named
pipe until all of the data is received (via potential \texttt{Transaction
Secondary} requests) whereas with the \texttt{Write*} commands, data is written
to the named pipe as it is received by the server. Multiple Transaction
requests can be made simultaneously to the same named pipe. These requests can
also be segmented with \texttt{Transaction Secondary} commands. What
distinguishes them (when the same named pipe is being written to, i.e. having
the same FID) are fields in the SMB header representing a process id (PID) and
multiplex id (MID).  The PID represents the process this request is a part of.
An MID represents different sub-processes within a process (or under a PID).
Segments for each "thread" are stored separately and written to the named pipe
when all segments are received. It is necessary to track this so as not to
munge these requests together (which would be a potential evasion opportunity).

\begin{itemize}
\item[] Windows (all versions)
\begin{itemize}
\item[] Uses a combination of PID and MID to define a "thread".
\end{itemize}
\item[] Samba (all versions)
\begin{itemize}
\item[] Uses just the MID to define a "thread".
\end{itemize}
\end{itemize}
\end{itemize}

\textit{Multiple Bind Requests}
\begin{itemize}

\item[] A \texttt{Bind} request is the first request that must be made in a
connection-oriented DCE/RPC session in order to specify the
interface/interfaces that one wants to communicate with.

\begin{itemize}
\item[] Windows (all versions)
\begin{itemize}

\item[] For all of the Windows versions, only one \texttt{Bind} can ever be
made on a session whether or not it succeeds or fails. Any binding after that
must use the \texttt{Alter Context} request.  If another \texttt{Bind} is made,
all previous interface bindings are invalidated.

\end{itemize}
\item[] Samba 3.0.20 and earlier
\begin{itemize}
\item[] Any amount of \texttt{Bind} requests can be made.
\end{itemize}
\item[] Samba later than 3.0.20
\begin{itemize}

\item[] Another \texttt{Bind} request can be made if the first failed and no
interfaces were successfully bound to. If a \texttt{Bind} after a successful
\texttt{Bind} is made, all previous interface bindings are invalidated.

\end{itemize}
\end{itemize}
\end{itemize}

\textit{DCE/RPC Fragmented requests - Context ID}
\begin{itemize}

\item[] Each fragment in a fragmented request carries the context id of the
bound interface it wants to make the request to.

\begin{itemize}
\item[] Windows (all versions)
\begin{itemize}

\item[] The context id that is ultimately used for the request is contained in
the first fragment.  The context id field in any other fragment can contain any
value.

\end{itemize}
\item[] Samba (all versions)
\begin{itemize}

\item[] The context id that is ultimately used for the request is contained in
the last fragment.  The context id field in any other fragment can contain any
value.

\end{itemize}
\end{itemize}
\end{itemize}

\textit{DCE/RPC Fragmented requests - Operation number}
\begin{itemize}

\item[] Each fragment in a fragmented request carries an operation number
(opnum) which is more or less a handle to a function offered by the interface.

\begin{itemize}
\item[] Samba (all versions)
\item[] Windows 2000
\item[] Windows 2003
\item[] Windows XP
\begin{itemize}

\item[] The opnum that is ultimately used for the request is contained in the
last fragment.  The opnum field in any other fragment can contain any value.
\end{itemize}

\item[] Windows Vista
\begin{itemize}

\item[] The opnum that is ultimately used for the request is contained in the
first fragment.  The opnum field in any other fragment can contain any value.

\end{itemize}
\end{itemize}
\end{itemize}

\textit{DCE/RPC Stub data byte order}
\begin{itemize}

\item[] The byte order of the stub data is determined differently for Windows
and Samba.

\begin{itemize}
\item[] Windows (all versions)
\begin{itemize}

\item[] The byte order of the stub data is that which was used in the
\texttt{Bind} request.

\end{itemize}

\item[] Samba (all versions)

\begin{itemize}

\item[] The byte order of the stub data is that which is used in the request
carrying the stub data.

\end{itemize}
\end{itemize}
\end{itemize}

\subsubsection{Configuration}

The \texttt{dcerpc2} preprocessor has a global configuration and one or more
server configurations.  The global preprocessor configuration name is
\texttt{dcerpc2} and the server preprocessor configuration name is
\texttt{dcerpc2\_server}.\\

\underline{Global Configuration}

\begin{verbatim}
    preprocessor dcerpc2
\end{verbatim}

The global \texttt{dcerpc2} configuration is required.  Only one global
\texttt{dcerpc2} configuration can be specified.\\

\textit{Option syntax}
\begin{itemize}

\item[]
\begin{tabular}{|l|c|c|p{6cm}|}
\hline
Option & Argument & Required & Default\\
\hline
\hline
\texttt{memcap} & \texttt{<memcap>} & NO & \texttt{memcap 102400}\\
\hline
\texttt{disable\_defrag} & NONE & NO & OFF\\
\hline
\texttt{max\_frag\_len} & \texttt{<max-frag-len>} & NO & OFF\\
\hline
\texttt{events} & \texttt{<events>} & NO & OFF\\
\hline
\texttt{reassemble\_threshold} & \texttt{<re-thresh>} & NO & OFF\\
\hline
\texttt{disabled} & NONE & NO & OFF\\
\hline
\texttt{smb\_fingerprint\_policy} & \texttt{<fp-policy>} & NO & OFF\\
\hline
\end{tabular}
\end{itemize}

\footnotesize
\begin{verbatim}
    memcap           = 1024-4194303 (kilobytes)
    max-frag-len     = 1514-65535
    events           = pseudo-event | event | '[' event-list ']'
    pseudo-event     = "none" | "all"
    event-list       = event | event ',' event-list
    event            = "memcap" | "smb" | "co" | "cl"
    re-thresh        = 0-65535
    fp-policy        = "server" | "client" | "both"
\end{verbatim}
\normalsize

\textit{Option explanations}
\begin{itemize}
\item[] \texttt{memcap}
\begin{itemize}

\item[] Specifies the maximum amount of run-time memory that can be allocated.
Run-time memory includes any memory allocated after configuration.  Default is
100 MB.

\end{itemize}

\item[] \texttt{disabled}
\begin{itemize}

\item[] Disables the preprocessor. By default this value is turned off. When the 
preprocessor is disabled only the memcap option is applied when specified 
with the configuration.

\end{itemize}

\item[] \texttt{disable\_defrag}

\begin{itemize}

\item[] Tells the preprocessor not to do DCE/RPC defragmentation.  Default is
to do defragmentation.

\end{itemize}

\item[] \texttt{max\_frag\_len}

\begin{itemize}

\item[] Specifies the maximum fragment size that will be added to the
defragmentation module. If a fragment is greater than this size, it is truncated
before being added to the defragmentation module.  The allowed range for this
option is 1514 - 65535.

\end{itemize}

\item[] \texttt{events}

\begin{itemize}

\item[] Specifies the classes of events to enable.  (See Events section for an
enumeration and explanation of events.)

\begin{itemize}
\item[] \texttt{memcap}

\begin{itemize}
\item[] Only one event. If the memcap is reached or exceeded, alert.

\end{itemize}

\item[] \texttt{smb}

\begin{itemize}
\item[] Alert on events related to SMB processing.
\end{itemize}

\item[] \texttt{co}

\begin{itemize}

\item[] Stands for connection-oriented DCE/RPC. Alert on events related to
connection-oriented DCE/RPC processing.

\end{itemize}

\item[] \texttt{cl}
\begin{itemize}

\item[] Stands for connectionless DCE/RPC. Alert on events related to
connectionless DCE/RPC processing.

\end{itemize}
\end{itemize}
\end{itemize}

\item[] \texttt{reassemble\_threshold}
\begin{itemize}

\item[] Specifies a minimum number of bytes in the DCE/RPC desegmentation and
defragmentation buffers before creating a reassembly packet to send to the
detection engine. This option is useful in inline mode so as to potentially
catch an exploit early before full defragmentation is done.  A value of 0
supplied as an argument to this option will, in effect, disable this option.
Default is disabled.

\end{itemize}

\item[] \texttt{smb\_fingerprint\_policy}
\begin{itemize}

\item[] In the initial phase of an SMB session, the client needs to authenticate
with a SessionSetupAndX.  Both the request and response to this command
contain OS and version information that can allow the preprocessor to
dynamically set the policy for a session which allows for better
protection against Windows and Samba specific evasions.

\end{itemize}
\end{itemize}

\textit{Option examples}
\footnotesize
\begin{verbatim}
    memcap 30000
    max_frag_len 16840
    events none
    events all
    events smb
    events co
    events [co]
    events [smb, co]
    events [memcap, smb, co, cl]
    reassemble_threshold 500
    smb_fingerprint_policy both
    smb_fingerprint_policy client
\end{verbatim}
\normalsize

\textit{Configuration examples}
\footnotesize
\begin{verbatim}
    preprocessor dcerpc2
    preprocessor dcerpc2: memcap 500000
    preprocessor dcerpc2: max_frag_len 16840, memcap 300000, events smb
    preprocessor dcerpc2: memcap 50000, events [memcap, smb, co, cl], max_frag_len 14440
    preprocessor dcerpc2: disable_defrag, events [memcap, smb]
    preprocessor dcerpc2: reassemble_threshold 500
    preprocessor dcerpc2: memcap 50000, events [memcap, smb, co, cl], max_frag_len 14440, smb_fingerprint_policy both
\end{verbatim}
\normalsize

\textit{Default global configuration}
\footnotesize
\begin{verbatim}
    preprocessor dcerpc2: memcap 102400
\end{verbatim}
\normalsize

\underline{Server Configuration}

\begin{verbatim}
    preprocessor dcerpc2_server
\end{verbatim}

The \texttt{dcerpc2\_server} configuration is optional.  A
\texttt{dcerpc2\_server} configuration must start with \texttt{default} or
\texttt{net} options.  The \texttt{default} and \texttt{net} options are
mutually exclusive.  At most one default configuration can be specified. If no
\texttt{default} configuration is specified, default values will be used for
the \texttt{default} configuration.  Zero or more \texttt{net} configurations
can be specified.  For any \texttt{dcerpc2\_server} configuration, if
non-required options are not specified, the defaults will be used.  When
processing DCE/RPC traffic, the \texttt{default} configuration is used if no
net configurations match.  If a \texttt{net} configuration matches, it will
override the \texttt{default} configuration. A \texttt{net} configuration
matches if the packet's server IP address matches an IP address or net
specified in the \texttt{net} configuration.  The \texttt{net} option supports
IPv6 addresses.  Note that port and ip variables defined in \texttt{snort.conf}
\textsc{cannot} be used.

\textit{Option syntax}
\begin{itemize}
\item[]
\begin{tabular}{|l|c|c|p{6cm}|}

\hline
Option & Argument & Required & Default\\
\hline

\hline
\texttt{default} & NONE              & YES & NONE\\

\hline
\texttt{net}     & \texttt{<net>}    & YES & NONE\\

\hline
\texttt{policy}  & \texttt{<policy>} &  NO & \texttt{policy WinXP}\\

\hline
\texttt{detect}  & \texttt{<detect>} &  NO & \texttt{detect [smb [139,445], tcp
135, udp 135, rpc-over-http-server 593]}\\

\hline
\texttt{autodetect} & \texttt{<detect>} & NO & \texttt{autodetect [tcp
1025:, udp 1025:, rpc-over-http-server 1025:]}\\

\hline
\texttt{no\_autodetect\_http\_proxy\_ports} & NONE & NO & DISABLED (The
preprocessor autodetects on all proxy ports by default)\\

\hline
\texttt{smb\_invalid\_shares} & \texttt{<shares>} & NO & NONE\\

\hline
\texttt{smb\_max\_chain} & \texttt{<max-chain>} & NO & \texttt{smb\_max\_chain 3}\\

\hline
\texttt{smb\_file\_inspection} & \texttt{<file-inspect>} & NO & \texttt{smb\_file\_inspection off}\\

\hline
\end{tabular}
\end{itemize}

\footnotesize
\begin{verbatim}
    net          = ip | '[' ip-list ']'
    ip-list      = ip | ip ',' ip-list
    ip           = ip-addr | ip-addr '/' prefix | ip4-addr '/' netmask
    ip-addr      = ip4-addr | ip6-addr
    ip4-addr     = a valid IPv4 address
    ip6-addr     = a valid IPv6 address (can be compressed)
    prefix       = a valid CIDR
    netmask      = a valid netmask
    policy       = "Win2000" | "Win2003" | "WinXP" | "WinVista" |
                   "Samba" | "Samba-3.0.22" | "Samba-3.0.20"
    detect       = "none" | detect-opt | '[' detect-list ']'
    detect-list  = detect-opt | detect-opt ',' detect-list
    detect-opt   = transport | transport port-item | 
                   transport '[' port-list ']'
    transport    = "smb" | "tcp" | "udp" | "rpc-over-http-proxy" | 
                   "rpc-over-http-server"
    port-list    = port-item | port-item ',' port-list
    port-item    = port | port-range
    port-range   = ':' port | port ':' | port ':' port
    port         = 0-65535
    shares       = share | '[' share-list ']'
    share-list   = share | share ',' share-list
    share        = word | '"' word '"' | '"' var-word '"'
    word         = graphical ASCII characters except ',' '"' ']' '[' '$'
    var-word     = graphical ASCII characters except ',' '"' ']' '['
    max-chain    = 0-255
    file-inspect = file-arg | '[' file-list ']'
    file-arg     = "off" | "on" | "only"
    file-list    = file-arg [ ',' "file-depth" <int64_t> ]
\end{verbatim}
\normalsize

\begin{itemize}

\item[] Because the Snort main parser treats '\$' as the start of a variable
and tries to expand it, shares with '\$' must be enclosed quotes.

\end{itemize}

\textit{Option explanations}

\begin{itemize}
\item[] \texttt{default}

\begin{itemize}

\item[] Specifies that this configuration is for the default server
configuration.

\end{itemize}

\item[] \texttt{net}

\begin{itemize}

\item[] Specifies that this configuration is an IP or net specific
configuration. The configuration will only apply to the IP addresses and nets
supplied as an argument.

\end{itemize}

\item[] \texttt{policy}

\begin{itemize}

\item[] Specifies the target-based policy to use when processing.  Default is
"WinXP".

\end{itemize}

\item[] \texttt{detect}

\begin{itemize}

\item[] Specifies the DCE/RPC transport and server ports that should be
detected on for the transport.  Defaults are ports 139 and 445 for SMB, 135 for
TCP and UDP, 593 for RPC over HTTP server and 80 for RPC over HTTP proxy. 

\end{itemize}

\item[] \texttt{autodetect}

\begin{itemize}

\item[] Specifies the DCE/RPC transport and server ports that the preprocessor
should attempt to autodetect on for the transport. The autodetect ports are
only queried if no detect transport/ports match the packet. The order in which
the preprocessor will attempt to autodetect will be - TCP/UDP, RPC over HTTP
server, RPC over HTTP proxy and lastly SMB.  Note that most dynamic DCE/RPC
ports are above 1024 and ride directly over TCP or UDP.  It would be very
uncommon to see SMB on anything other than ports 139 and 445.  Defaults are
1025-65535 for TCP, UDP and RPC over HTTP server.

\end{itemize}

\item[] \texttt{no\_autodetect\_http\_proxy\_ports}

\begin{itemize}

\item[] By default, the preprocessor will always attempt to autodetect for
ports specified in the detect configuration for rpc-over-http-proxy.  This is
because the proxy is likely a web server and the preprocessor should not look
at all web traffic. This option is useful if the RPC over HTTP proxy configured
with the detect option is only used to proxy DCE/RPC traffic.  Default is to
autodetect on RPC over HTTP proxy detect ports.

\end{itemize}

\item[] \texttt{smb\_invalid\_shares}

\begin{itemize}

\item[] Specifies SMB shares that the preprocessor should alert on if an
attempt is made to connect to them via a \texttt{Tree Connect} or \texttt{Tree
Connect AndX}.  Default is empty.

\end{itemize}

\item[] \texttt{smb\_max\_chain}

\begin{itemize}

\item[] Specifies the maximum amount of AndX command chaining that is allowed
before an alert is generated.  Default maximum is 3 chained commands.  A value
of 0 disables this option. This value can be set from 0 to 255.

\end{itemize}

\item[] \texttt{smb\_file\_inspection}

\begin{itemize}

\item[] Instructs the preprocessor to do inspection of normal SMB file
transfers.  This includes doing file type and signature through
the file API as well as setting a pointer for the \texttt{file\_data}
rule option.  Note that the \texttt{file-depth} option only applies to the
maximum amount of file data for which it will set the pointer for the
\texttt{file\_data} rule option.  For file type and signature it will use the
value configured for the file API.  If \texttt{only} is specified, the
preprocessor will only do SMB file inspection, i.e. it will not do any DCE/RPC
tracking or inspection.  If \texttt{on} is specified with no
arguments, the default file depth is 16384 bytes.  An argument of -1 to
\texttt{file-depth} disables setting the pointer for \texttt{file\_data}, effectively
disabling SMB file inspection in rules.  An argument of 0 to
\texttt{file-depth} means unlimited.  Default is \texttt{off}, i.e. no SMB file
inspection is done in the preprocessor.

\end{itemize}

\end{itemize}

\textit{Option examples}
\footnotesize
\begin{verbatim}
    net 192.168.0.10
    net 192.168.0.0/24
    net [192.168.0.0/24]
    net 192.168.0.0/255.255.255.0
    net feab:45b3:ab92:8ac4:d322:007f:e5aa:7845
    net feab:45b3:ab92:8ac4:d322:007f:e5aa:7845/128
    net feab:45b3::/32
    net [192.168.0.10, feab:45b3::/32]
    net [192.168.0.0/24, feab:45b3:ab92:8ac4:d322:007f:e5aa:7845]
    policy Win2000
    policy Samba-3.0.22
    detect none
    detect smb
    detect [smb]
    detect smb 445
    detect [smb 445]
    detect smb [139,445]
    detect [smb [139,445]]
    detect [smb, tcp]
    detect [smb 139, tcp [135,2103]]
    detect [smb [139,445], tcp 135, udp 135, rpc-over-http-server [593,6002:6004]]
    autodetect none
    autodetect tcp
    autodetect [tcp]
    autodetect tcp 2025:
    autodetect [tcp 2025:]
    autodetect tcp [2025:3001,3003:]
    autodetect [tcp [2025:3001,3003:]]
    autodetect [tcp, udp]
    autodetect [tcp 2025:, udp 2025:]
    autodetect [tcp 2025:, udp, rpc-over-http-server [1025:6001,6005:]]
    smb_invalid_shares private
    smb_invalid_shares "private"
    smb_invalid_shares "C$"
    smb_invalid_shares [private, "C$"]
    smb_invalid_shares ["private", "C$"]
    smb_max_chain 1
    smb_file_inspection on
    smb_file_inspection off
    smb_file_inspection [ on, file-depth -1 ]
    smb_file_inspection [ on, file-depth 0 ]
    smb_file_inspection [ on, file-depth 4294967296 ]
    smb_file_inspection [ only, file-depth -1 ]
\end{verbatim}
\normalsize

\textit{Configuration examples}
\footnotesize
\begin{verbatim}
    preprocessor dcerpc2_server: \
        default

    preprocessor dcerpc2_server: \
        default, policy Win2000

    preprocessor dcerpc2_server: \
        default, policy Win2000, detect [smb, tcp], autodetect tcp 1025:, \
        smb_invalid_shares ["C$", "D$", "ADMIN$"]

    preprocessor dcerpc2_server: net 10.4.10.0/24, policy Win2000

    preprocessor dcerpc2_server: \
        net [10.4.10.0/24,feab:45b3::/126], policy WinVista, smb_max_chain 1

    preprocessor dcerpc2_server: \
        net [10.4.10.0/24,feab:45b3::/126], policy WinVista, \
        detect [smb, tcp, rpc-over-http-proxy 8081], 
        autodetect [tcp, rpc-over-http-proxy [1025:6001,6005:]], \
        smb_invalid_shares ["C$", "ADMIN$"], no_autodetect_http_proxy_ports

    preprocessor dcerpc2_server: \
        net [10.4.11.56,10.4.11.57], policy Samba, detect smb, autodetect none

    preprocessor dcerpc2_server: default, policy WinXP, \
        smb_file_inspection [ on, file-depth 0 ]
\end{verbatim}
\normalsize

\textit{Default server configuration}
\footnotesize
\begin{verbatim}
    preprocessor dcerpc2_server: default, policy WinXP, \
        detect [smb [139,445], tcp 135, udp 135, rpc-over-http-server 593], \
        autodetect [tcp 1025:, udp 1025:, rpc-over-http-server 1025:], \
        smb_max_chain 3, smb_file_inspection off
\end{verbatim}
\normalsize

\underline{Complete \texttt{dcerpc2} default configuration}
\footnotesize
\begin{verbatim}
    preprocessor dcerpc2: memcap 102400

    preprocessor dcerpc2_server: \
        default, policy WinXP, \
        detect [smb [139,445], tcp 135, udp 135, rpc-over-http-server 593], \
        autodetect [tcp 1025:, udp 1025:, rpc-over-http-server 1025:], \
        smb_max_chain 3, smb_file_inspection off
\end{verbatim}
\normalsize

\subsubsection{Events}
The preprocessor uses GID 133 to register events.\\

\textit{Memcap events}
\begin{itemize}
\item[]

\begin{longtable}{|r|p{13.5cm}|}

\hline
SID & Description\\
\hline

\hline
  1 &

If the memory cap is reached and the preprocessor is configured to alert.\\

\hline
\end{longtable}
\end{itemize}

\textit{SMB events}
\begin{itemize}

\item[]
\begin{longtable}{|r|p{13.5cm}|}

\hline
SID & Description\\
\hline

\hline
  2 & An invalid NetBIOS Session Service type was specified in the header.
      Valid types are: \texttt{Message}, \texttt{Request} (only from client),
      \texttt{Positive Response} (only from server), \texttt{Negative Response}
      (only from server), \texttt{Retarget Response} (only from server) and
      \texttt{Keep Alive}.\\
\hline
  3 & An SMB message type was specified in the header. Either a request was
      made by the server or a response was given by the client.\\
\hline
  4 & The SMB id does not equal \texttt{\textbackslash xffSMB}. Note that since
      the preprocessor does not yet support SMB2, id of
      \texttt{\textbackslash xfeSMB} is turned away before an eventable point is
      reached.\\
\hline
  5 & The word count of the command header is invalid. SMB commands have
      pretty specific word counts and if the preprocessor sees a command with
      a word count that doesn't jive with that command, the preprocessor will
      alert.\\
\hline
  6 & Some commands require a minimum number of bytes after the command
      header. If a command requires this and the byte count is less than the
      minimum required byte count for that command, the preprocessor will
      alert.\\
\hline
  7 & Some commands, especially the commands from the SMB Core implementation
      require a data format field that specifies the kind of data that will be
      coming next. Some commands require a specific format for the data. The
      preprocessor will alert if the format is not that which is expected for
      that command.\\
\hline
  8 & Many SMB commands have a field containing an offset from the beginning
      of the SMB header to where the data the command is carrying starts. If
      this offset puts us before data that has already been processed or after
      the end of payload, the preprocessor will alert.\\
\hline
  9 & Some SMB commands, such as \texttt{Transaction}, have a field containing
      the total amount of data to be transmitted. If this field is zero, the
      preprocessor will alert.\\
\hline
 10 & The preprocessor will alert if the NetBIOS Session Service length field
      contains a value less than the size of an SMB header.\\
\hline
 11 & The preprocessor will alert if the remaining NetBIOS packet length is
      less than the size of the SMB command header to be decoded.\\
\hline
 12 & The preprocessor will alert if the remaining NetBIOS packet length is
      less than the size of the SMB command byte count specified in the
      command header.\\
\hline
 13 & The preprocessor will alert if the remaining NetBIOS packet length is
      less than the size of the SMB command data size specified in the command
      header.\\
\hline
 14 & The preprocessor will alert if the total data count specified in the
      SMB command header is less than the data size specified in the SMB
      command header. (Total data count must always be greater than or equal
      to current data size.)\\
\hline
 15 & The preprocessor will alert if the total amount of data sent in a
      transaction is greater than the total data count specified in the SMB
      command header.\\
\hline
 16 & The preprocessor will alert if the byte count specified in the SMB
      command header is less than the data size specified in the SMB command.
      (The byte count must always be greater than or equal to the data size.)\\
\hline
 17 & Some of the Core Protocol commands (from the initial SMB implementation)
      require that the byte count be some value greater than the data size
      exactly. The preprocessor will alert if the byte count minus a
      predetermined amount based on the SMB command is not equal to the data
      size.\\
\hline
 18 & For the \texttt{Tree Connect} command (and not the
      \texttt{Tree Connect AndX} command), the preprocessor has to queue
      the requests up and wait for a server response to determine whether or
      not an IPC share was successfully connected to (which is what the
      preprocessor is interested in). Unlike the \texttt{Tree Connect AndX}
      response, there is no indication in the \texttt{Tree Connect} response
      as to whether the share is IPC or not. There should be under normal
      circumstances no more than a few pending tree connects at a time and
      the preprocessor will alert if this number is excessive.\\
\hline
 19 & After a client is done writing data using the \texttt{Write*} commands,
      it issues a \texttt{Read*} command to the server to tell it to send a
      response to the data it has written. In this case the preprocessor is
      concerned with the server response. The \texttt{Read*} request contains
      the file id associated with a named pipe instance that the preprocessor
      will ultimately send the data to. The server response, however, does not
      contain this file id, so it need to be queued with the request and
      dequeued with the response. If multiple \texttt{Read*} requests are sent
      to the server, they are responded to in the order they were sent. There
      should be under normal circumstances no more than a few pending
      \texttt{Read*} requests at a time and the preprocessor will alert if this
      number is excessive.\\
\hline
 20 & The preprocessor will alert if the number of chained commands in a
      single request is greater than or equal to the configured amount
      (default is 3).\\
\hline
 21 & With \texttt{AndX} command chaining it is possible to chain multiple
      \texttt{Session Setup AndX} commands within the same request. There is,
      however, only one place in the SMB header to return a login handle
      (or Uid).  Windows does not allow this behavior, however Samba does. This
      is anomalous behavior and the preprocessor will alert if it happens.\\
\hline
 22 & With \texttt{AndX} command chaining it is possible to chain multiple
      \texttt{Tree Connect AndX} commands within the same request. There is,
      however, only one place in the SMB header to return a tree handle
      (or Tid).  Windows does not allow this behavior, however Samba does. This
      is anomalous behavior and the preprocessor will alert if it happens.\\
\hline
 23 & When a \texttt{Session Setup AndX} request is sent to the server, the
      server responds (if the client successfully authenticates) which a user
      id or login handle. This is used by the client in subsequent requests to
      indicate that it has authenticated. A \texttt{Logoff AndX} request is
      sent by the client to indicate it wants to end the session and invalidate
      the login handle. With commands that are chained after a
      \texttt{Session Setup AndX} request, the login handle returned by the
      server is used for the subsequent chained commands. The combination of a
      \texttt{Session Setup AndX} command with a chained \texttt{Logoff AndX}
      command, essentially logins in and logs off in the same request and is
      anomalous behavior. The preprocessor will alert if it sees this.\\
\hline
 24 & A \texttt{Tree Connect AndX} command is used to connect to a share. The
      \texttt{Tree Disconnect} command is used to disconnect from that share.
      The combination of a \texttt{Tree Connect AndX} command with a chained
      \texttt{Tree Disconnect} command, essentially connects to a share and
      disconnects from the same share in the same request and is anomalous
      behavior. The preprocessor will alert if it sees this.\\
\hline
 25 & An \texttt{Open AndX} or \texttt{Nt Create AndX} command is used to
      open/create a file or named pipe. (The preprocessor is only interested in
      named pipes as this is where DCE/RPC requests are written to.) The
      \texttt{Close} command is used to close that file or named pipe. The
      combination of a \texttt{Open AndX} or \texttt{Nt Create AndX} command
      with a chained \texttt{Close} command, essentially opens and closes the
      named pipe in the same request and is anomalous behavior.  The
      preprocessor will alert if it sees this.\\
\hline
 26 & The preprocessor will alert if it sees any of the invalid SMB shares
      configured. It looks for a \texttt{Tree Connect} or
      \texttt{Tree Connect AndX} to the share.\\
\hline
 48 & The preprocessor will alert if a data count for a Core dialect write
      command is zero.\\
\hline
 49 & For some of the Core dialect commands such as \texttt{Write} and
      \texttt{Read}, there are two data count fields, one in the main command
      header and one in the data format section.  If these aren't the same, the
      preprocessor will alert.\\
\hline
 50 & In the initial negotiation phase of an SMB session, the server in a
      \texttt{Negotiate} response and the client in a \texttt{SessionSetupAndX}
      request will advertise the maximum number of outstanding requests
      supported.  The preprocessor will alert if the lesser of the two is
      exceeded.\\
\hline
 51 & When a client sends a request it uses a value called the MID (multiplex
      id) to match a response, which the server is supposed to echo, to a
      request.  If there are multiple outstanding requests with the same
      MID, the preprocessor will alert.\\
\hline
 52 & In the \texttt{Negotiate} request a client gives a list of SMB dialects it
      supports, normally in order from least desirable to most desirable
      and the server responds with the index of the dialect to be used on
      the SMB session.  Anything less than "NT LM 0.12" would be very odd
      these days (even Windows 98 supports it) and the preprocessor will
      alert if the client doesn't offer it as a supported dialect or the
      server chooses a lesser dialect.\\
\hline
 53 & There are a number of commands that are considered deprecated and/or
       obsolete by Microsoft (see MS-CIFS and MS-SMB).  If the preprocessor
       detects the use of a deprecated/obsolete command used it will alert.\\
\hline
 54 & There are some commands that can be used that can be considered unusual
      in the context they are used.  These include some of the transaction
      commands such as:
      \texttt{SMB\_COM\_TRANSACTION / TRANS\_READ\_NMPIPE}
      \texttt{SMB\_COM\_TRANSACTION / TRANS\_WRITE\_NMPIPE}
      \texttt{SMB\_COM\_TRANSACTION2 / TRANS2\_OPEN2}
      \texttt{SMB\_COM\_NT\_TRANSACT / NT\_TRANSACT\_CREATE}
      The preprocessor will alert if it detects unusual use of a command.\\
\hline
 55 & Transaction commands have a setup count field that indicates the
      number of 16bit words in the transaction setup.  The preprocessor
      will alert if the setup count is invalid for the transaction
      command / sub command.\\
\hline
 56 & There can be only one Negotiate transaction per session and it is the
      first thing a client and server do to determine the SMB dialect each
      supports.  The preprocessor will alert if the client attempts multiple
      dialect negotiations.\\
\hline
 57 & Malware will often set a file's attributes to ReadOnly/Hidden/System if
      it is successful in installing itself as a Windows service or is able to
      write an autorun.inf file since it doesn't want the user to see the file
      and the default folder options in Windows is not to display Hidden files.
      The preprocessor will alert if it detects a client attempt to set a
      file's attributes to ReadOnly/Hidden/System.\\
\hline
\end{longtable}
\end{itemize}

\textit{Connection-oriented DCE/RPC events}
\begin{itemize}
\item[]

\begin{longtable}[h]{|r|p{13.5cm}|}
\hline
SID & Description\\
\hline

\hline
 27 & The preprocessor will alert if the connection-oriented DCE/RPC major
      version contained in the header is not equal to 5.\\
\hline
 28 & The preprocessor will alert if the connection-oriented DCE/RPC minor
      version contained in the header is not equal to 0.\\
\hline
 29 & The preprocessor will alert if the connection-oriented DCE/RPC PDU type
      contained in the header is not a valid PDU type.\\
\hline
 30 & The preprocessor will alert if the fragment length defined in the header
      is less than the size of the header.\\
\hline
 31 & The preprocessor will alert if the remaining fragment length is less
      than the remaining packet size.\\
\hline
 32 & The preprocessor will alert if in a \texttt{Bind} or
      \texttt{Alter Context} request, there are no context items specified.\\
\hline
 33 & The preprocessor will alert if in a \texttt{Bind} or
      \texttt{Alter Context} request, there are no transfer syntaxes to go with
      the requested interface.\\
\hline
 34 & The preprocessor will alert if a non-last fragment is less than the size
      of the negotiated maximum fragment length. Most evasion techniques try
      to fragment the data as much as possible and usually each fragment comes
      well below the negotiated transmit size.\\
\hline
 35 & The preprocessor will alert if a fragment is larger than the maximum
      negotiated fragment length.\\
\hline
 36 & The byte order of the request data is determined by the Bind in
      connection-oriented DCE/RPC for Windows.  It is anomalous behavior
      to attempt to change the byte order mid-session.\\
\hline
 37 & The call id for a set of fragments in a fragmented request should stay
      the same (it is incremented for each complete request). The preprocessor
      will alert if it changes in a fragment mid-request.\\
\hline
 38 & The operation number specifies which function the request is calling on
      the bound interface. If a request is fragmented, this number should stay
      the same for all fragments.  The preprocessor will alert if the opnum
      changes in a fragment mid-request.\\
\hline
 39 & The context id is a handle to a interface that was bound to. If a
      request if fragmented, this number should stay the same for all
      fragments. The preprocessor will alert if the context id changes in a
      fragment mid-request.\\
\hline
\end{longtable}
\end{itemize}

\textit{Connectionless DCE/RPC events}
\begin{itemize}
\item[]
\begin{longtable}{|r|p{13.5cm}|}
\hline
SID & Description\\
\hline
\hline
 40 & The preprocessor will alert if the connectionless DCE/RPC major version
      is not equal to 4.\\
\hline
 41 & The preprocessor will alert if the connectionless DCE/RPC PDU type is
      not a valid PDU type.\\
\hline
 42 & The preprocessor will alert if the packet data length is less than the
      size of the connectionless header.\\
\hline
 43 & The preprocessor will alert if the sequence number uses in a request is
      the same or less than a previously used sequence number on the session.
      In testing, wrapping the sequence number space produces strange behavior
      from the server, so this should be considered anomalous behavior.\\
\hline
\end{longtable}
\end{itemize}

\subsubsection{Rule Options}

New rule options are supported by enabling the \texttt{dcerpc2} preprocessor:

\begin{itemize}
\item[]
\begin{verbatim}
    dce_iface
    dce_opnum
    dce_stub_data
\end{verbatim}
\end{itemize}

New modifiers to existing \texttt{byte\_test} and \texttt{byte\_jump} rule
options:

\begin{itemize}
\item[]
\begin{verbatim}
    byte_test:dce
    byte_jump:dce
\end{verbatim}
\end{itemize}

\texttt{dce\_iface}
\label{dcerpc2:dce_iface}
\begin{itemize}

\item[] For DCE/RPC based rules it has been necessary to set flow-bits based on
a client bind to a service to avoid false positives.  It is necessary for a
client to bind to a service before being able to make a call to it. When a
client sends a bind request to the server, it can, however, specify one or more
service interfaces to bind to. Each interface is represented by a UUID. Each
interface UUID is paired with a unique index (or context id) that future
requests can use to reference the service that the client is making a call to.
The server will respond with the interface UUIDs it accepts as valid and will
allow the client to make requests to those services.  When a client makes a
request, it will specify the context id so the server knows what service the
client is making a request to. Instead of using flow-bits, a rule can simply
ask the preprocessor, using this rule option, whether or not the client has
bound to a specific interface UUID and whether or not this client request is
making a request to it. This can eliminate false positives where more than one
service is bound to successfully since the preprocessor can correlate the bind
UUID to the context id used in the request.  A DCE/RPC request can specify
whether numbers are represented as big endian or little endian. The
representation of the interface UUID is different depending on the endianness
specified in the DCE/RPC previously requiring two rules - one for big endian
and one for little endian.  The preprocessor eliminates the need for two rules
by normalizing the UUID.  An interface contains a version. Some versions of an
interface may not be vulnerable to a certain exploit.  Also, a DCE/RPC request
can be broken up into 1 or more fragments. Flags (and a field in the
connectionless header) are set in the DCE/RPC header to indicate whether the
fragment is the first, a middle or the last fragment. Many checks for data in
the DCE/RPC request are only relevant if the DCE/RPC request is a first
fragment (or full request), since subsequent fragments will contain data deeper
into the DCE/RPC request. A rule which is looking for data, say 5 bytes into
the request (maybe it's a length field), will be looking at the wrong data on a
fragment other than the first, since the beginning of subsequent fragments are
already offset some length from the beginning of the request. This can be a
source of false positives in fragmented DCE/RPC traffic. By default it is
reasonable to only evaluate if the request is a first fragment (or full
request).  However, if the \texttt{any\_frag} option is used to specify
evaluating on all fragments.\\

\textit{Syntax}
\footnotesize
\begin{verbatim}
    dce_iface:<uuid>[, <operator><version>][, any_frag];

    uuid       = hexlong '-' hexshort '-' hexshort '-' 2hexbyte '-' 6hexbyte
    hexlong    = 4hexbyte
    hexshort   = 2hexbyte
    hexbyte    = 2HEXDIGIT
    operator   = '<' | '>' | '=' | '!'
    version    = 0-65535
\end{verbatim}
\normalsize
\textit{Examples}
\footnotesize
\begin{verbatim}
    dce_iface:4b324fc8-1670-01d3-1278-5a47bf6ee188;
    dce_iface:4b324fc8-1670-01d3-1278-5a47bf6ee188, <2;
    dce_iface:4b324fc8-1670-01d3-1278-5a47bf6ee188, any_frag;
    dce_iface:4b324fc8-1670-01d3-1278-5a47bf6ee188, =1, any_frag;
\end{verbatim}
\normalsize

This option is used to specify an interface UUID. Optional arguments are an
interface version and operator to specify that the version be less than
('\textless'), greater than ('\textgreater'), equal to ('=') or not equal to
('!') the version specified.  Also, by default the rule will only be evaluated
for a first fragment (or full request, i.e. not a fragment) since most rules
are written to start at the beginning of a request. The \texttt{any\_frag}
argument says to evaluate for middle and last fragments as well.  This option
requires tracking client \texttt{Bind} and \texttt{Alter Context} requests as
well as server \texttt{Bind Ack} and \texttt{Alter Context} responses for
connection-oriented DCE/RPC in the preprocessor. For each \texttt{Bind} and
\texttt{Alter Context} request, the client specifies a list of interface UUIDs
along with a handle (or context id) for each interface UUID that will be used
during the DCE/RPC session to reference the interface.  The server response
indicates which interfaces it will allow the client to make requests to - it
either accepts or rejects the client's wish to bind to a certain interface.
This tracking is required so that when a request is processed, the context id
used in the request can be correlated with the interface UUID it is a handle
for.

\texttt{hexlong} and \texttt{hexshort} will be specified and interpreted to be
in big endian order (this is usually the default way an interface UUID will be
seen and represented). As an example, the following Messenger interface UUID as
taken off the wire from a little endian \texttt{Bind} request:

\begin{verbatim}
    |f8 91 7b 5a 00 ff d0 11 a9 b2 00 c0 4f b6 e6 fc|
\end{verbatim}

must be written as:

\begin{verbatim}
    5a7b91f8-ff00-11d0-a9b2-00c04fb6e6fc
\end{verbatim}

The same UUID taken off the wire from a big endian \texttt{Bind} request:

\begin{verbatim}
    |5a 7b 91 f8 ff 00 11 d0 a9 b2 00 c0 4f b6 e6 fc|
\end{verbatim}

must be written the same way:

\begin{verbatim}
    5a7b91f8-ff00-11d0-a9b2-00c04fb6e6fc
\end{verbatim}

This option matches if the specified interface UUID matches the interface UUID
(as referred to by the context id) of the DCE/RPC request and if supplied, the
version operation is true. This option will not match if the fragment is not a
first fragment (or full request) unless the \texttt{any\_frag} option is
supplied in which case only the interface UUID and version need match.  Note
that a defragmented DCE/RPC request will be considered a full request.

\begin{note}
Using this rule option will automatically insert fast pattern contents into
the fast pattern matcher.  For UDP rules, the interface UUID, in both big and
little endian format will be inserted into the fast pattern matcher.  For TCP
rules, (1) if the rule option \texttt{flow:to\_server|from\_client} is used, $|$05 00 00$|$
will be inserted into the fast pattern matcher, (2) if the rule option
\texttt{flow:from\_server|to\_client} is used, $|$05 00 02$|$ will be inserted into the
fast pattern matcher and (3) if the flow isn't known, $|$05 00$|$ will be inserted
into the fast pattern matcher.  Note that if the rule already has content rule
options in it, the best (meaning longest) pattern will be used.  If a content
in the rule uses the \texttt{fast\_pattern} rule option, it will unequivocally be used
over the above mentioned patterns.
\end{note}

\end{itemize}

\texttt{dce\_opnum}
\label{dcerpc2:dce_opnum}
\begin{itemize}

\item[] The opnum represents a specific function call to an interface. After is
has been determined that a client has bound to a specific interface and is
making a request to it (see above - \texttt{dce\_iface}) usually we want to
know what function call it is making to that service. It is likely that an
exploit lies in the particular DCE/RPC function call.\\

\textit{Syntax}
\footnotesize
\begin{verbatim}
    dce_opnum:<opnum-list>;

    opnum-list   = opnum-item | opnum-item ',' opnum-list
    opnum-item   = opnum | opnum-range
    opnum-range  = opnum '-' opnum
    opnum        = 0-65535
\end{verbatim}
\normalsize
\textit{Examples}
\footnotesize
\begin{verbatim}
    dce_opnum:15;
    dce_opnum:15-18;
    dce_opnum:15, 18-20;
    dce_opnum:15, 17, 20-22;
\end{verbatim}
\normalsize

This option is used to specify an opnum (or operation number), opnum range or
list containing either or both opnum and/or opnum-range. The opnum of a DCE/RPC
request will be matched against the opnums specified with this option.  This
option matches if any one of the opnums specified match the opnum of the
DCE/RPC request.

\end{itemize}

\texttt{dce\_stub\_data}
\label{dcerpc2:dce_stub_data}
\begin{itemize}

\item[] Since most netbios rules were doing protocol decoding only to get to
the DCE/RPC stub data, i.e. the remote procedure call or function call data,
this option will alleviate this need and place the cursor at the beginning of
the DCE/RPC stub data. This reduces the number of rule option checks and the
complexity of the rule.

This option takes no arguments.\\

\textit{Example}
\footnotesize
\begin{verbatim}
    dce_stub_data;
\end{verbatim}
\normalsize

This option is used to place the cursor (used to walk the packet payload in
rules processing) at the beginning of the DCE/RPC stub data, regardless of
preceding rule options. There are no arguments to this option.  This option
matches if there is DCE/RPC stub data.

The cursor is moved to the beginning of the stub data.  All ensuing rule
options will be considered "sticky" to this buffer.  The first rule option
following \texttt{dce\_stub\_data} should use absolute location modifiers if it is
position-dependent.  Subsequent rule options should use a relative modifier if
they are meant to be relative to a previous rule option match in the stub data
buffer.  Any rule option that does not specify a relative modifier will be
evaluated from the start of the stub data buffer.  To leave the stub data buffer
and return to the main payload buffer, use the \texttt{pkt\_data} rule option -
see section \ref{sub:pkt_data} for details).

\end{itemize}
\texttt{byte\_test} and \texttt{byte\_jump} with \texttt{dce}\label{dcerpc2:byte_test_jump}
\begin{itemize}

\item[] A DCE/RPC request can specify whether numbers are represented in big or
little endian. These rule options will take as a new argument \texttt{dce} and
will work basically the same as the normal
\texttt{byte\_test}/\texttt{byte\_jump}, but since the DCE/RPC preprocessor
will know the endianness of the request, it will be able to do the correct
conversion.

\texttt{byte\_test}
\begin{itemize}
\item[] \textit{Syntax}
\footnotesize
\begin{verbatim}
    byte_test:<convert>, [!]<operator>, <value>, <offset> [, relative], dce;

    convert    = 1 | 2 | 4 (only with option "dce")
    operator   = '<' | '=' | '>' | '<=' | '>=' | '&' | '^'
    value      = 0 - 4294967295
    offset     = -65535 to 65535
\end{verbatim}
\normalsize

\textit{Examples}
\footnotesize
\begin{verbatim}
    byte_test:4, >, 35000, 0, relative, dce;
    byte_test:2, !=, 2280, -10, relative, dce;
\end{verbatim}
\normalsize

When using the \texttt{dce} argument to a \texttt{byte\_test}, the following
normal \texttt{byte\_test} arguments will not be allowed: \texttt{big},
\texttt{little}, \texttt{string}, \texttt{hex}, \texttt{dec} and \texttt{oct}.
\end{itemize}

\texttt{byte\_jump}
\begin{itemize}
\item[] \textit{Syntax}
\footnotesize
\begin{verbatim}
    byte_jump:<convert>, <offset>[, relative][, multiplier <mult_value>] \
        [, align][, post_offset <adjustment_value>], dce;

    convert           = 1 | 2 | 4 (only with option "dce")
    offset            = -65535 to 65535
    mult_value        = 0 - 65535
    adjustment_value  = -65535 to 65535
\end{verbatim}
\normalsize

\textit{Example}
\footnotesize
\begin{verbatim}
    byte_jump:4,-4,relative,align,multiplier 2,post_offset -4,dce;
\end{verbatim}
\normalsize

When using the \texttt{dce} argument to a \texttt{byte\_jump}, the following
normal \texttt{byte\_jump} arguments will not be allowed: \texttt{big},
\texttt{little}, \texttt{string}, \texttt{hex}, \texttt{dec}, \texttt{oct} and
\texttt{from\_beginning}.

\end{itemize}
\end{itemize}

\textit{Example of rule complexity reduction}
\begin{itemize}

\item[] The following two rules using the new rule options replace 64 (set and
isset flowbit) rules that are necessary if the new rule options are not used:

\footnotesize
\begin{verbatim}
    alert tcp $EXTERNAL_NET any -> $HOME_NET [135,139,445,593,1024:] \
        (msg:"dns R_Dnssrv funcs2 overflow attempt"; flow:established,to_server; \
        dce_iface:50abc2a4-574d-40b3-9d66-ee4fd5fba076; dce_opnum:0-11; dce_stub_data; \
        pcre:"/^.{12}(\x00\x00\x00\x00|.{12})/s"; byte_jump:4,-4,relative,align,dce; \
        byte_test:4,>,256,4,relative,dce; reference:bugtraq,23470; reference:cve,2007-1748; \
        classtype:attempted-admin; sid:1000068;)

    alert udp $EXTERNAL_NET any -> $HOME_NET [135,1024:] \
        (msg:"dns R_Dnssrv funcs2 overflow attempt"; flow:established,to_server; \
        dce_iface:50abc2a4-574d-40b3-9d66-ee4fd5fba076; dce_opnum:0-11; dce_stub_data; \
        pcre:"/^.{12}(\x00\x00\x00\x00|.{12})/s"; byte_jump:4,-4,relative,align,dce; \
        byte_test:4,>,256,4,relative,dce; reference:bugtraq,23470; reference:cve,2007-1748; \
        classtype:attempted-admin; sid:1000069;)
\end{verbatim}
\normalsize
\end{itemize}

\subsection{Sensitive Data Preprocessor}
\label{sub:sensitive_data}

The Sensitive Data preprocessor is a Snort module that performs detection and
filtering of Personally Identifiable Information (PII). This information
includes credit card numbers, U.S. Social Security numbers, and email addresses.
A limited regular expression syntax is also included for defining your own PII.

\subsubsection{Dependencies}

The Stream preprocessor must be enabled for the Sensitive Data preprocessor
to work.

\subsubsection{Preprocessor Configuration}

Sensitive Data configuration is split into two parts: the preprocessor config,
and the rule options. The preprocessor config starts with:

\begin{verbatim}
preprocessor sensitive_data:
\end{verbatim}

\textit{Option syntax}
\begin{itemize}

\item[]
\begin{tabular}{|l|c|c|p{6cm}|}
\hline
Option & Argument & Required & Default\\
\hline
\hline
\texttt{alert\_threshold} & \texttt{<number>} & NO & \texttt{alert\_threshold 25}\\
\hline
\texttt{mask\_output} & NONE & NO & OFF\\
\hline
\texttt{ssn\_file} & \texttt{<filename>} & NO & OFF\\
\hline
\end{tabular}
\end{itemize}

\footnotesize
\begin{verbatim}
    alert_threshold     =  1 - 65535
\end{verbatim}
\normalsize

\textit{Option explanations}
\begin{itemize}
\item[] \texttt{alert\_threshold}
\begin{itemize}

\item[] The preprocessor will alert when any combination of PII are detected
in a session. This option specifies how many need to be detected before
alerting.
This should be set higher than the highest individual count in your
"sd\_pattern" rules.

\end{itemize}

\item[] \texttt{mask\_output}
\begin{itemize}

\item[] This option replaces all but the last 4 digits of a detected PII with
"X"s. This is only done on credit card \& Social Security numbers, where
an organization's regulations may prevent them from seeing unencrypted
numbers.

\end{itemize}

\item[] \texttt{ssn\_file}
\begin{itemize}

\item[] A Social Security number is broken up into 3 sections:
Area (3 digits), Group (2 digits), and Serial (4 digits).
On a monthly basis, the Social Security Administration publishes a list
of which Group numbers are in use for each Area.
These numbers can be updated in Snort by supplying a CSV file with the
new maximum Group numbers to use.
By default, Snort recognizes Social Security numbers issued up through
November 2009.

\end{itemize}
\end{itemize}

\textit{Example preprocessor config}
\begin{verbatim}
preprocessor sensitive_data: alert_threshold 25 \
                             mask_output \
                             ssn_file ssn_groups_Jan10.csv
\end{verbatim}

\subsubsection{Rule Options}

Snort rules are used to specify which PII the preprocessor should look for.
A new rule option is provided by the preprocessor:

\begin{verbatim}
sd_pattern
\end{verbatim}

This rule option specifies what type of PII a rule should detect.

\textit{Syntax}
\begin{verbatim}
    sd_pattern:<count>, <pattern>;
\end{verbatim}
\footnotesize
\begin{verbatim}

    count   = 1 - 255
    pattern = any string
\end{verbatim}
\normalsize

\textit{Option Explanations}

\begin{itemize}
\item[] \texttt{count}

\begin{itemize}
\item[] This dictates how many times a PII pattern must be matched for an alert
to be generated. The count is tracked across all packets in a session.
\end{itemize}

\item[] \texttt{pattern}

\begin{itemize}
\item[] This is where the pattern of the PII gets specified. There are a few
built-in patterns to choose from:

\begin{itemize}
\item[] \texttt{credit\_card}

\begin{itemize}
\item[] The "credit\_card" pattern matches 15- and 16-digit credit card
numbers. These numbers may have spaces, dashes, or nothing in
between groups. This covers Visa, Mastercard, Discover, and
American Express.
Credit card numbers matched this way have their check digits
verified using the Luhn algorithm.
\end{itemize}

\item[] \texttt{us\_social}

\begin{itemize}
\item[] This pattern matches against 9-digit U.S. Social Security numbers.
The SSNs are expected to have dashes between the Area, Group, and
Serial sections.

SSNs have no check digits, but the preprocessor will check matches
against the list of currently allocated group numbers.
\end{itemize}

\item[] \texttt{us\_social\_nodashes}

\begin{itemize}
\item[] This pattern matches U.S. Social Security numbers without dashes
separating the Area, Group, and Serial sections.
\end{itemize}

\item[] \texttt{email}

\begin{itemize}
\item[] This pattern matches against email addresses.
\end{itemize}
\end{itemize}

\item[] If the pattern specified is not one of the above built-in patterns,
then it is the definition of a custom PII pattern. Custom PII types
are defined using a limited regex-style syntax. The following
special characters and escape sequences are supported:

\item[]
\begin{tabular}{|c|p{10cm}|}

\hline
\texttt{\textbackslash d} & matches any digit\\
\hline
\texttt{\textbackslash D} & matches any non-digit\\
\hline
\texttt{\textbackslash l} & matches any letter\\
\hline
\texttt{\textbackslash L} & matches any non-letter\\
\hline
\texttt{\textbackslash w} & matches any alphanumeric character\\
\hline
\texttt{\textbackslash W} & matches any non-alphanumeric character\\
\hline
\texttt{\{num\}} & used to repeat a character or escape sequence "num" times.
example:  "\d\{3\}" matches 3 digits.\\
\hline
\texttt{?} & makes the previous character or escape sequence optional.
example:  " ?" matches an optional space.
This behaves in a greedy manner.\\
\hline
\texttt{\textbackslash\textbackslash} & matches a backslash\\
\hline
\textbackslash \{, \textbackslash \} & matches \{ and \}\\
\hline
\textbackslash ? & matches a question mark.\\
\hline
\end{tabular}

\item[] Other characters in the pattern will be matched literally.

\begin{note}
Unlike PCRE, \texttt{\textbackslash w} in this rule option does NOT match underscores.
\end{note}
\end{itemize}

\item[] \textit{Examples}
\begin{verbatim}
    sd_pattern: 2,us_social;
\end{verbatim}
    Alerts when 2 social security numbers (with dashes) appear in a session.

\begin{verbatim}
    sd_pattern: 5,(\d{3})\d{3}-\d{4};
\end{verbatim}
    Alerts on 5 U.S. phone numbers, following the format (123)456-7890

    Whole rule example:

\begin{verbatim}
    alert tcp $HOME_NET $HIGH_PORTS -> $EXTERNAL_NET $SMTP_PORTS \
    (msg:"Credit Card numbers sent over email"; gid:138; sid:1000; rev:1; \
    sd_pattern:4,credit_card; metadata:service smtp;)
\end{verbatim}


\item[] \textit{Caveats}
\begin{itemize}
\item[] \texttt{sd\_pattern} is not compatible with other rule options. Trying to use
other rule options with \texttt{sd\_pattern} will result in an error message.

Rules using \texttt{sd\_pattern} must use GID 138.
\end{itemize}
\end{itemize}

\subsection{Normalizer}

When operating Snort in inline mode, it is helpful to normalize packets to help
minimize the chances of evasion.

To enable the normalizer, use the following when configuring Snort:

\begin{verbatim}
    ./configure --enable-normalizer
\end{verbatim}

The normalize preprocessor is activated via the conf as outlined below.  There
are also many new preprocessor and decoder rules to alert on or drop packets
with "abnormal" encodings.

Note that in the following, fields are cleared only if they are non-zero.
Also, normalizations will only be enabled if the selected DAQ supports packet
replacement and is operating in inline mode.

If a policy is configured for \texttt{inline\_test} or passive mode, any
normalization statements in the policy config are ignored.

\subsubsection{IP4 Normalizations}

IP4 normalizations are enabled with:

\begin{verbatim}
    preprocessor normalize_ip4: [df], [rf], [tos], [trim]
\end{verbatim}

Base normalizations enabled with "preprocessor \texttt{normalize\_ip4}" include:

\begin{itemize} 
\item
TTL normalization if enabled (explained below).

\item
Clear the differentiated services field (formerly TOS).

\item
NOP all options octets.
\end{itemize} 

Optional normalizations include:

\begin{itemize} 
\item \texttt{df} 
don't fragment:  clear this bit on incoming packets.

\item \texttt{rf} 
reserved flag:  clear this bit on incoming packets.

\item \texttt{tos} 
type of service (differentiated services): clear this byte.

\item \texttt{trim} 
truncate packets with excess payload to the datagram length specified in the
IP header + the layer 2 header (e.g. ethernet), but don't truncate below minimum
frame length.  This is automatically disabled if the DAQ can't inject packets.
\end{itemize} 

\subsubsection{IP6 Normalizations}

IP6 normalizations are enabled with:

\begin{verbatim}
    preprocessor normalize_ip6
\end{verbatim}

Base normalizations enabled with "preprocessor \texttt{normalize\_ip6}" include:

\begin{itemize} 
\item
Hop limit normalization if enabled (explained below).

\item
NOP all options octets in hop-by-hop and destination options extension headers.
\end{itemize} 


\subsubsection{ICMP4/6 Normalizations}

ICMP4 and ICMP6 normalizations are enabled with:

\begin{verbatim}
    preprocessor normalize_icmp4
    preprocessor normalize_icmp6
\end{verbatim}

Base normalizations enabled with the above include:

\begin{itemize} 
\item
Clear the code field in echo requests and replies.
\end{itemize} 


\subsubsection{TCP Normalizations}

TCP normalizations are enabled with:

\begin{verbatim}
    preprocessor normalize_tcp: \
        [block], [rsv], [pad], \
        [req_urg], [req_pay], [req_urp], \
        [ips], [urp], [trim], \
        [trim_syn], [trim_rst], \
        [trim_win], [trim_mss], \
        [ecn <ecn_type>], \
        [opts [allow <allowed_opt>+]]

    <ecn_type> ::= stream | packet

    <allowed_opt> ::= \
        sack | echo | partial_order | conn_count | alt_checksum | md5 | <num>

    <sack> ::= { 4, 5 }
    <echo> ::= { 6, 7 }
    <partial_order> ::= { 9, 10 }
    <conn_count> ::= { 11, 12, 13 }
    <alt_checksum> ::= { 14, 15 }
    <md5> ::= { 19 }
    <num> ::= (3..255) 
\end{verbatim}

Normalizations include:

\begin{itemize} 
\item \texttt{block}
allow packet drops during TCP normalization.

\item \texttt{rsv}
clear the reserved bits in the TCP header.

\item \texttt{pad}
clear any option padding bytes.

\item \texttt{req\_urg}
clear the urgent pointer if the urgent flag is not set.

\item \texttt{req\_pay}
clear the urgent pointer and the urgent flag if there is no payload.

\item \texttt{req\_urp} 
clear the urgent flag if the urgent pointer is not set.

\item \texttt{ips} 
ensure consistency in retransmitted data (also forces reassembly policy to
"first").  Any segments that can't be properly reassembled will be dropped.

\item \texttt{trim\_syn} 
remove data on SYN.

\item \texttt{trim\_rst} 
remove any data from RST packet.

\item \texttt{trim\_win} 
trim data to window.

\item \texttt{trim\_mss} 
trim data to MSS.

\item \texttt{trim} 
enable all of the above trim options.

\item \texttt{ecn packet} 

clear ECN flags on a per packet basis (regardless of negotiation).

\item \texttt{ecn stream} 

clear ECN flags if usage wasn't negotiated.  Should also enable \texttt{require\_3whs}.

\item \texttt{opts} 

NOP all option bytes other than maximum segment size, window scaling,
timestamp, and any explicitly allowed with the allow keyword.  You can allow
options to pass by name or number.

\item \texttt{opts} 

if timestamp is present but invalid, or valid but not negotiated, NOP the
timestamp octets.

\item \texttt{opts} 

if timestamp was negotiated but not present, block the packet.

\item \texttt{opts} 

clear TS ECR if ACK flag is not set.

\item \texttt{opts} 

MSS and window scale options are NOP'd if SYN flag is not set.

\end{itemize} 

\subsubsection{TTL Normalization}

TTL normalization pertains to both IP4 TTL (time-to-live) and IP6 (hop limit)
and is only performed if both the relevant base normalization is enabled (as
described above) and the minimum and new TTL values are configured, as follows:

\begin{verbatim}
    config min_ttl: <min_ttl>
    config new_ttl: <new_ttl>

    <min_ttl> ::= (1..255)
    <new_ttl> ::= (<min_ttl>+1..255)
\end{verbatim}

If \texttt{new\_ttl }$>$ \texttt{min\_ttl}, then if a packet is received with a
TTL $<$ \texttt{min\_ttl}, the TTL will be set to \texttt{new\_ttl}.

Note that this configuration item was deprecated in 2.8.6:

\begin{verbatim}
    preprocessor stream5_tcp: min_ttl <#>
\end{verbatim}

By default \texttt{min\_ttl} = 1 (TTL normalization is disabled). 
When TTL normalization is turned on the \texttt{new\_ttl} is set
to 5 by default.

\subsection{SIP Preprocessor}
\label{sub:sip}
Session Initiation Protocol (SIP) is an application-layer control (signaling)
protocol for creating,   modifying, and terminating sessions with one or more 
participants.  These sessions include Internet telephone calls, multimedia 
distribution, and multimedia conferences. SIP Preprocessor provides ways to 
tackle Common Vulnerabilities and Exposures (CVEs) related with SIP found over 
the past few years. It also makes detecting new attacks easier.

\subsubsection{Dependency Requirements}

For proper functioning of the preprocessor:

\begin{itemize}

\item Stream session tracking must be enabled, i.e. stream5. Both TCP and UDP must be
      enabled in stream5. The preprocessor requires a session tracker to keep its 
      data. In addition, Stream API is able to provide correct support for ignoring 
      audio/video data channel.

\item  IP defragmentation should be enabled, i.e. the frag3 preprocessor should be
      enabled and configured.

\end{itemize}

\subsubsection{Configuration}
The preprocessor configuration name is \texttt{sip}.\\
\begin{verbatim}
    preprocessor sip
\end{verbatim}
\textit{Option syntax}
\begin{itemize}

\item[]
\begin{tabular}{|l|c|c|p{6cm}|}
\hline
Option & Argument & Required & Default\\
\hline
\hline
\texttt{disabled} & NONE & NO & OFF\\
\hline
\texttt{max\_sessions} & \texttt{<max\_sessions>} & NO & \texttt{max\_sessions 10000}\\
\hline
\texttt{max\_dialogs} & \texttt{<max\_dialogs>} & NO & \texttt{max\_dialogs 4}\\
\hline
\texttt{ports} & \texttt{<ports>} & NO & \texttt{ports \{ 5060 5061 \} }\\
\hline
\texttt{methods} & \texttt{<methods>} & NO & \texttt{methods \{ invite cancel ack bye 
                                                          register options \} }\\
\hline
\texttt{max\_uri\_len} & \texttt{<max\_uri\_len>} & NO & \texttt{max\_uri\_len 256 }\\
\hline
\texttt{max\_call\_id\_len} & \texttt{<max\_call\_id\_len>} & NO & \texttt{max\_call\_id\_len 256 }\\
\hline
\texttt{max\_requestName\_len} & \texttt{<max\_requestName\_len>} & NO & \texttt{max\_requestName\_len 20 }\\
\hline
\texttt{max\_from\_len} & \texttt{<max\_from\_len>} & NO & \texttt{max\_from\_len 256 }\\
\hline
\texttt{max\_to\_len} & \texttt{<max\_to\_len>} & NO & \texttt{max\_to\_len 256 }\\
\hline
\texttt{max\_via\_len} & \texttt{<max\_via\_len>} & NO & \texttt{max\_via\_len 1024 }\\
\hline
\texttt{max\_contact\_len} & \texttt{<max\_contact\_len>} & NO & \texttt{max\_contact\_len 256 }\\
\hline
\texttt{max\_content\_len} & \texttt{<max\_content\_len>} & NO & \texttt{max\_content\_len 1024 }\\
\hline
\texttt{ignore\_call\_channel} & NONE & NO & OFF\\
\hline
\end{tabular}
\end{itemize}
\footnotesize
\begin{verbatim}
     max_sessions        = 1024-4194303
     max_dialogs         = 1-4194303
     methods             = "invite"|"cancel"|"ack"|"bye"|"register"| "options"\
                           |"refer" |"subscribe"|"update"|"join"|"info"|"message"\                
                           |"notify"|"prack"
     max_uri_len         = 0-65535
     max_call_id_len     = 0-65535
     max_requestName_len = 0-65535
     max_from_len        = 0-65535
     max_to_len          = 0-65535
     max_via_len         = 0-65535
     max_contact_len     = 0-65535
     max_content_len     = 0-65535
\end{verbatim}
\normalsize
\textit{Option explanations}
\begin{itemize}
\item[] \texttt{disabled}
\begin{itemize}
\item[]  SIP dynamic preprocessor can be enabled/disabled through configuration. 
      By default this value is turned off. When the preprocessor is disabled, 
      only the max\_sessions option is applied when specified with the configuration.
\end{itemize}
\item[] \texttt{max\_sessions}
\begin{itemize}
\item[]  This specifies the maximum number of sessions that can be allocated. 
      Those sessions are stream sessions, so they are bounded by maximum number of 
      stream sessions. Default is 10000.
\end{itemize}
\item[] \texttt{max\_dialogs}
\begin{itemize}
\item[]   This specifies the maximum number of dialogs within one stream session. If exceeded,
      the oldest dialog will be dropped. Default is 4.
\end{itemize}
\item[] \texttt{ports}
\begin{itemize}
\item[]  This specifies on what ports to check for SIP messages. Typically, this will 
      include 5060, 5061.
\item[] \textit{Syntax}
\begin{verbatim}
    ports { <port> [<port>< ... >] }
\end{verbatim}
\item[] \textit{Examples}
\begin{verbatim}
    ports { 5060 5061 }
\end{verbatim}
\item[] Note: there are spaces before and after `\{' and `\}'.
\end{itemize}
 \item[] \texttt{methods}
\begin{itemize}
  \item[]  This specifies on what methods to check for SIP messages: (1) invite, 
      (2) cancel, (3) ack, (4) bye, (5) register, (6) options, (7) refer, 
      (8) subscribe, (9) update (10) join (11) info (12) message (13) notify 
      (14) prack.
      Note: those 14 methods are up to date list (Feb. 2011). New methods can be 
      added to the list. Up to 32 methods supported.

\item[] \textit{Syntax}
\begin{verbatim}
      methods { <method-list> }
      method-list = method|method method-list
      methods     = "invite"|"cancel"|"ack"|"bye"|"register"| "options"\
                    |"refer"|"subscribe"|"update"|"join"|"info"|"message"\                
                    |"notify"|"prack"
\end{verbatim}
\item[] \textit{Examples}
\begin{verbatim}
   methods { invite cancel ack bye register options }
   methods { invite cancel ack bye register options information }
\end{verbatim}    
\item[] Note: there are spaces before and after `\{' and `\}'.
\end{itemize}
\item[] \texttt{max\_uri\_len}
\begin{itemize}
  \item[] This specifies the maximum Request URI field size. If the Request URI field
      is greater than this size, an alert is generated. Default is set to 256. 
      The allowed range for this option is 0 - 65535. ``0'' means never alert.
\end{itemize}

\item[] \texttt{max\_call\_id\_len}
\begin{itemize}
  \item[] This specifies the maximum Call-ID field size. If the Call-ID field is 
      greater than this size, an alert is generated. Default is set to 256. 
      The allowed range for this option is 0 - 65535. ``0'' means never alert.
\end{itemize}

\item[] \texttt{max\_requestName\_len}
\begin{itemize}
  \item[]  This specifies the maximum request name size that is part of the CSeq ID. 
      If the request name is greater than this size, an alert is generated. 
      Default is set to 20. The allowed range for this option is 0 - 65535. 
     ``0'' means never alert.
\end{itemize}

\item[] \texttt{max\_from\_len}
\begin{itemize}
  \item[]  This specifies the maximum From field size. If the From field is greater 
      than this size, an alert is generated. Default is set to 256. The allowed
      range for this option is 0 - 65535. ``0'' means never alert.
\end{itemize}

\item[] \texttt{max\_to\_len}
\begin{itemize}
  \item[]   This specifies the maximum To field size. If the To field is greater than
      this size, an alert is generated. Default is set to 256. The allowed range 
      for this option is 0 - 65535. ``0'' means never alert.
\end{itemize}

\item[] \texttt{max\_via\_len}
\begin{itemize}
  \item[]  This specifies the maximum Via field size. If the Via field is greater than 
      this size, an alert is generated. Default is set to 1024. The allowed range 
      for this option is 0 - 65535. ``0'' means never alert.
\end{itemize}
\item[] \texttt{max\_contact\_len}
\begin{itemize}
  \item[]   This specifies the maximum Contact field size. If the Contact field is 
      greater than this size, an alert is generated. Default is set to 256. 
      The allowed range for this option is 0 - 65535. ``0''  means never alert.
\end{itemize}
\item[] \texttt{max\_content\_len}
\begin{itemize}
  \item[]   This specifies the maximum content length of the message body. If the 
      content length is greater than this number, an alert is generated. 
      Default is set to 1024. The allowed range for this option is 0 - 65535. 
      ``0'' means never alert.
\end{itemize}
\item[] \texttt{ignore\_call\_channel}
\begin{itemize}
  \item[]   This enables the support for ignoring audio/video data channel 
      (through Stream API). By default, this is disabled.
\end{itemize}

\end{itemize}
\textit{Option examples}
\footnotesize
\begin{verbatim}
     max_sessions 30000
     disabled
     ports { 5060 5061 }
     methods { invite cancel ack bye register options }
     methods { invite cancel ack bye register options information }
     max_uri_len 1024
     max_call_id_len 1024
     max_requestName_len 10
     max_from_len 1024
     max_to_len 1024
     max_via_len 1024
     max_contact_len 1024
     max_content_len 1024
     max_content_len
     ignore_call_channel    
\end{verbatim}
\normalsize

\textit{Configuration examples}
\footnotesize
\begin{verbatim}
     preprocessor sip
     preprocessor sip: max_sessions 500000
     preprocessor sip: max_contact_len 512, max_sessions 300000, methods { invite \
                       cancel ack bye register options } , ignore_call_channel
     preprocessor sip: ports { 5060 49848 36780 10270 }, max_call_id_len 200, \
                      max_from_len 100, max_to_len 200, max_via_len 1000, \
                      max_requestName_len 50, max_uri_len 100, ignore_call_channel,\
                      max_content_len 1000
     preprocessor sip: disabled
     preprocessor sip: ignore_call_channel
\end{verbatim}
\normalsize

\textit{Default configuration}
\footnotesize
\begin{verbatim}
     preprocessor sip
\end{verbatim}
\normalsize
\subsubsection{Events}
The preprocessor uses GID 140 to register events.
\begin{longtable}{|r|p{13.5cm}|}

\hline
SID & Description\\
\hline
 1  & If the memory cap is reached and the preprocessor is configured to alert, 
     this alert will be created. \\
\hline
 2  & Request URI is required. When Request URI is empty, this alert will be created. \\
\hline
 3  & The Request URI is larger than the defined length in configuration.\\
\hline
 4  & When Call-ID is empty, this alert will be created.\\
\hline
 5  & The Call-ID is larger than the defined length in configuration.\\
\hline
 6  & The sequence e number value MUST be expressible as a 32-bit unsigned integer
     and MUST be less than $2^{31}$.\\
\hline
 7  & The request name in the CSeq is larger than the defined length in configuration.\\
\hline
 8  & From field is empty.\\
\hline
 9  & From field is larger than the defined length in configuration.\\
\hline
 10 & To field is empty.\\
\hline
 11 & To field is larger than the defined length in configuration.\\
\hline
 12 & Via filed is empty.\\
\hline
 13 & Via filed is larger than the defined length in configuration.\\
\hline
 14 & Contact is empty, but it is required non-empty for the message.\\ 
\hline
 15 & The Contact is larger than the defined length in configuration. \\
\hline
 16 & The content length is larger than the defined length in configuration or is negative. \\
\hline
 17 & There are multiple requests in a single packet. Old SIP protocol supports
      multiple sip messages within one packet.\\
\hline
 18 & There are inconsistencies between Content-Length in SIP header and 
      actual body data.\\
\hline
 19 & Request name is invalid in response.\\
\hline
 20 & Authenticated invite message received, but no challenge from server received. 
      This is the case of InviteReplay billing attack.\\
\hline
 21 & Authenticated invite message received, but session information has been changed. 
      This is different from re-INVITE, where the dialog has been established.
      and authenticated. This is can prevent FakeBusy billing attack.\\
\hline
 22 &  Response status code is not a 3 digit number.\\
\hline
 23 &  Content type header field is required if the message body is not empty.\\
\hline
 24 &  SIP version other than 2.0, 1.0, and 1.1 is invalid \\
\hline
 25 &  Mismatch in Method of request and the CSEQ header\\
\hline
 26 &  The method is unknown \\
\hline
 27 &  The number of dialogs in the stream session exceeds the maximal value. \\
\hline
\end{longtable}
\subsubsection{Rule Options}
New rule options are supported by enabling the \texttt{sip} preprocessor:
\begin{itemize}
\item[]
\begin{verbatim}
  sip_method
  sip_stat_code
  sip_header
  sip_body
\end{verbatim}
\end{itemize}
Overload modifiers to existing \texttt{pcre} rule options:
\begin{itemize}
\item[] H: Match SIP request or SIP response header, Similar to \texttt{sip\_header}.
\item[] P: Match SIP request or SIP response body, Similar to \texttt{sip\_body}.
\end{itemize}
\texttt{sip\_method}
\label{sip:sip_method}
\begin{itemize}
 \item[] The  \texttt{sip\_method} keyword is used to check for specific SIP request methods.  
  The list of methods is: invite, cancel, ack, bye, register, options, refer, 
  subscribe, update, join, info, message, notify, prack.  More than one method 
  can be specified, via a comma separated list, and are OR'ed together.  
  It will be applied in fast pattern match if available. If the method used in 
  this rule is not listed in the preprocessor configuration, it will be added
  to the preprocessor configuration for the associated policy.\\

\textit{Syntax}
\footnotesize
\begin{verbatim}
    sip_method:<method-list>;
    method-list = method|method, method-list
    method      = ["!"] "invite"|"cancel"|"ack"|"bye"|"register"| "options"\
                  |"refer"|"subscribe"|"update"|"join"|"info"|"message"\                
                  |"notify"|"prack"
    Note: if "!" is used, only one method is allowed in sip_method. 
\end{verbatim}
\normalsize
\textit{Examples}
\footnotesize
\begin{verbatim}
   sip_method:invite, cancel
   sip_method:!invite
   
   Note: If a user wants to use "and", they can use something like this:
   sip_method:!invite; sip_method:!bye
\end{verbatim}
\normalsize
\end{itemize}

\texttt{sip\_stat\_code}
\label{sip:sip_stat_code}
\begin{itemize}
 \item[]  The \texttt{sip\_stat\_code} is used to check the SIP response status code.  This option
  matches if any one of the state codes specified matches the status codes of
  the SIP response.\\

\textit{Syntax}
\footnotesize
\begin{verbatim}
   sip_stat_code:<code _list> ;
   code_list = state_code|state_code, code_list
   code      = "100-999"|"1-9"   
\end{verbatim}
 \item[] Note: 1,2,3,4,5,6... mean to check for "1xx", "2xx", '3xx', '4xx', '5xx', 
          '6xx'... responses.  \\
\normalsize

\textit{Examples}
\footnotesize
\begin{verbatim}
   sip_stat_code:200  
   sip_stat_code: 2  
   sip_stat_code: 200, 180
\end{verbatim}
\normalsize
\end{itemize}

\texttt{sip\_header}
\label{sip:sip_header}
\begin{itemize}
 \item[]  The \texttt{sip\_header} keyword restricts the search to the extracted Header fields of 
  a SIP message request or a response. This works similar to \texttt{file\_data}. \\

\textit{Syntax}
\footnotesize
\begin{verbatim}
   sip_header;
\end{verbatim}
\normalsize

\textit{Examples}
\footnotesize
\begin{verbatim}
   alert udp any any -> any 5060 (sip_header; content:"CSeq"; )
\end{verbatim}
\normalsize
\end{itemize}

\texttt{sip\_body}
\label{sip:sip_body}
\begin{itemize}
 \item[]  The \texttt{sip\_body} keyword places the cursor at the beginning of the Body fields 
  of a SIP message. This works similar to \texttt{file\_data} and \texttt{dce\_stub\_data}. The message
  body includes channel information using SDP protocol (Session Description Protocol).\\

\textit{Syntax}
\footnotesize
\begin{verbatim}
   sip_body;
\end{verbatim}
\normalsize

\textit{Examples}
\footnotesize
\begin{verbatim}
   alert udp any any -> any 5060 (sip_body; content:"C=IN 0.0.0.0"; within 100;)
\end{verbatim}
\normalsize
\end{itemize}
\texttt{pcre}
\label{sip:pcre}
\begin{itemize}
 \item[]  SIP overloads two options for \texttt{pcre}:\\
\begin{itemize}
\item H: Match SIP header for request or response , Similar to \texttt{sip\_header}.\\
\item P: Match SIP body for request or  response , Similar to \texttt{sip\_body}.\\
\end{itemize}
\textit{Examples}
\footnotesize
\begin{verbatim}
    alert udp any any -> any 5060 (pcre:"/INVITE/H"; sid:1000000;)
    alert udp any any -> any 5060 (pcre:"/m=/P"; sid:2000000;)
\end{verbatim}
\normalsize
\end{itemize}

\subsection{Reputation Preprocessor}
\label{sub:reputation}
Reputation preprocessor provides basic IP blacklist/whitelist capabilities, to 
block/drop/pass traffic from IP addresses listed. In the past, we use standard 
Snort rules to implement Reputation-based IP blocking. This preprocessor will 
address the performance issue and make the IP reputation management easier. 
This preprocessor runs before other preprocessors.
\subsubsection{Configuration}
The preprocessor configuration name is \texttt{reputation}.\\

\begin{verbatim}
    preprocessor reputation
\end{verbatim}
\textit{Option syntax}
\begin{itemize}

\item[]
\begin{tabular}{|l|c|c|p{6cm}|}
\hline
Option & Argument & Required & Default\\
\hline
\hline
\texttt{memcap} & \texttt{<memcap>} & NO & \texttt{memcap 500}\\
\hline
\texttt{scan\_local} & NONE & NO & OFF\\
\hline
\texttt{blacklist} & \texttt{<list file name>} & NO & NONE\\
\hline
\texttt{whitelist} & \texttt{<list file name>} & NO & NONE\\
\hline
\texttt{priority} & [blacklist whitelist] & NO & \texttt{priority whitelist}\\
\hline
\texttt{nested\_ip} & [inner outer both] & NO & \texttt{nested\_ip inner}\\
\hline
\texttt{white} & [unblack trust] & NO & \texttt{white unblack}\\
\hline
\end{tabular}
\end{itemize}
\footnotesize
\begin{verbatim}
     memcap        = 1-4095 Mbytes  
\end{verbatim}
\normalsize
\textit{Option explanations}
\begin{itemize}

\item[] \texttt{memcap}
\begin{itemize}
\item[]   Maximum total memory supported. It can be set up to 4095 Mbytes.
\end{itemize}

\item[] \texttt{scan\_local}
\begin{itemize}
\item[]   Enable to inspect local address defined in RFC 1918:
\begin{itemize}
         \item[]  10.0.0.0        -   10.255.255.255  (10/8 prefix)
         \item[]  172.16.0.0      -   172.31.255.255  (172.16/12 prefix)
         \item[]  192.168.0.0     -   192.168.255.255 (192.168/16 prefix)
\end{itemize}
\end{itemize}

\item[] \texttt{blacklist/whitelist}
\begin{itemize}
\item[]    The IP lists are loaded from external files. It supports relative 
           paths for inclusion and \$variables for path. Multiple blacklists or 
           whitelists are supported.
          
 \item[]   Note: if the same IP is redefined later, it will overwrite the 
           previous one. In other words, IP lists always favors the last file or
           entry processed.
\end{itemize}

\item[] \texttt{priority}
\begin{itemize}
\item[]   Specify either blacklist or whitelist has higher priority when 
           source/destination is on blacklist while destination/source is on 
           whitelist. By default, whitelist has higher priority. In other words,
           the packet will be passed when either source or destination is 
           whitelisted.
           
\item[]    Note: this only defines priority when there is a decision conflict, 
           during run-time. During initialization time, if the same IP address 
           is defined in whitelist and blacklist, whoever the last one defined 
           will be the final one. Priority does not work on this case.
\end{itemize}

\item[] \texttt{nested\_ip}
\begin{itemize}
\item[]  Specify which IP address to be used when there is IP 
           encapsulation. 
\end{itemize}

\item[] \texttt{white}
\begin{itemize}
\item[]   Specify the meaning of whitelist. When white means unblack, it 
           unblacks IPs that are in blacklists; when white means trust, the 
           packet gets bypassed, without further detection by snort. You can 
           only specify either unblack or trust.
           
 \item[]   Note: when white means unblack, whitelist always has higher priority
           than blacklist.
\end{itemize}

\end{itemize}

\textit{Configuration examples}
\footnotesize
\begin{verbatim}
    preprocessor reputation:\ 
                   blacklist /etc/snort/default.blacklist, \
                   whitelist /etc/snort/default.whitelist
   
    preprocessor reputation: \
                   nested_ip both, \
                   blacklist /etc/snort/default.blacklist, \
                   whitelist /etc/snort/default.whitelist
   
    preprocessor reputation: \
                   memcap  4095, scan_local, nested_ip both, \
                   priority whitelist,  \
                   blacklist /etc/snort/default.blacklist, \
                   whitelist /etc/snort/default.whitelist,
                   white trust
   
    $REP_BLACK_FILE1 = ../dshield.list
    $REP_BLACK_FILE2 = ../snort.org.list
    preprocessor reputation: \
                blacklist $REP_BLACK_FILE1,\
                blacklist $REP_BLACK_FILE2
    
\end{verbatim}
\normalsize
\textit{IP List File Format}
\begin{itemize}

\item[] \texttt{Syntax}
\begin{itemize}
\item[]   The IP list file has 1 entry per line. The entry can be either IP entry or 
    comment.
\end{itemize}

\begin{itemize}
\item[] \texttt{IP Entry}
\begin{itemize}
\item[]   CIDR notation $<$comments$>$ line break.
\item[]   Example:
\footnotesize
\begin{verbatim}
     172.16.42.32/32
     172.33.42.32/16
\end{verbatim}
\normalsize
\end{itemize}

\item[] \texttt{Comment}
\begin{itemize}
\item[]   The comment start with \#
\item[]   \#  $<$comments$>$
\item[]  Example
\footnotesize
\begin{verbatim}
    #  This is a full line comment
    172.33.42.32/16    # This is a in-line comment
\end{verbatim}
\normalsize
\end{itemize}

\end{itemize}

\item[] \texttt{IP List File Example}
\begin{itemize}

\item[]
\footnotesize
\begin{verbatim}
    # This is a full line comment
    172.16.42.32/32    # This is an inline comment, line with single CIDR block
    172.33.42.32/16
\end{verbatim}
\normalsize
\end{itemize}
\end{itemize}

\textit{Use case}
\begin{itemize}
 \item[] A user wants to protect his/her network from unwanted/unknown IPs, only allowing 
  some trusted IPs. Here is the configuration:
\item[] \footnotesize
\begin{verbatim}  
  preprocessor reputation: \
        blacklist /etc/snort/default.blacklist
        whitelist /etc/snort/default.whitelist
  
  In file "default.blacklist"
        # These two entries will match all ipv4 addresses  
        1.0.0.0/1  
        128.0.0.0/1
  
  In file "default.whitelist"
        68.177.102.22 # sourcefire.com
        74.125.93.104 # google.com
\end{verbatim}
\end{itemize}
\normalsize

\subsubsection{Events}
Reputation preprocessor uses GID 136 to register events.
\begin{longtable}{|r|p{13.5cm}|}

\hline
SID & Description\\
\hline
 1  & Packet is blacklisted. \\
\hline
 2  & Packet is whitelisted. \\
\hline
 3  & Packet is inspected. \\
\hline
\end{longtable}
\subsubsection{Shared memory support}

\begin{itemize}
 \item[]
  In order to minimize memory consumption when multiple Snort instances are 
  running concurrently, we introduce the support of shared memory. After 
  configured, all the snort instances share the same IP tables in shared memory. 

 \item[]\textit{System requirement}
\begin{itemize}
  \item[]This feature is supported only in Linux.
\end{itemize}  

 \item[]\textit{Build configuration}

 \begin{itemize} \item[]A new option, \texttt{--enable-shared-rep} is introduced to 
   \texttt{./configure} command. 
    This option enables the support for shared memory.
\end{itemize}
 \item[]\textit{Configuration}

\begin{itemize}
  
 \item[]\texttt{shared\_mem}
\begin{itemize}
 
 \item[] If the build supports shared memory, this configuration will enable shared 
  memory. If this option isn't set, standard memory is used.  This option must 
  specify a path or directory where IP lists will be loaded in shared memory. 
  One snort instance will create and maintain the shared IP lists. 
  We use instance ID 1, specified in the snort \texttt{-G} option to be the master snort.
  All the other snort instances are clients (readers). 
 
  \item[] \textit{Syntax}
   \begin{verbatim}
    shared_mem: path
   \end{verbatim}
  \item[] \textit{Examples}  
   \begin{verbatim}          
    shared_mem  /user/reputation/iplists
   \end{verbatim}
\end{itemize}
 \item[]\texttt{shared\_refresh}

\begin{itemize}
  \item[]This option changes the period of checking new shared memory segment, in the unit of second.
  By default, the refresh rate is $60$ seconds.  
  
  \item[]\textit{Syntax}
   \begin{verbatim}    
    shared_refresh <period>
    period = "1 - 4294967295"
   \end{verbatim}  
  \item[]\textit{Examples}
   \begin{verbatim}            
    shared_refresh 60
   \end{verbatim}
\end{itemize}
\end{itemize} 

 \item[]\textit{Steps to configure shared memory}

\begin{itemize}
   
   \item When building Snort, add option \texttt{--enable-shared-rep} to \texttt{./configure}\\
         For example:
      \begin{verbatim}
       ./configure --enable-gre --enable-sourcefire --enable-flexresp3 
       --enable-pthread --enable-linux-smp-stats 
       --enable-targetbased --enable-shared-rep --enable-control-socket
      \end{verbatim}
   \item Put your IP list file into a directory, where snort has full access. \\
      For example: 

      \begin{verbatim}
       /user/reputation/iplists
      \end{verbatim}

      In order to separate whitelist with blacklist, you need to specify 
      whitelist with \texttt{.wlf} extension and blacklist with \texttt{.blf} extension.
   \item In snort config file, specify shared memory support with the path to IP files.\\
      For example: 

      \begin{verbatim}
       shared_mem  /user/reputation/iplists
      \end{verbatim}

      If you want to change the period of checking new IP lists, add refresh period.\\
      For example:

      \begin{verbatim}
       shared_refresh 300
      \end{verbatim}

   \item Start shared memory master(writer) with \texttt{-G} 0 option. Note: only one master
      should be enabled.
   \item Start shared memory clients (readers) with \texttt{-G}  1 or other IDs. Note: for 
      one ID, only one snort instance should be enabled.
   \item You will see the IP lists got loaded and shared across snort instances! 
\end{itemize}

 \item[]\textit{Reload IP lists using control socket}
\begin{itemize}
  \item Run snort using command line with option \texttt{--cs-dir <path>}
      or configure snort with:
     \begin{verbatim} 
      config cs_dir:<path>
     \end{verbatim}
  \item (Optional) you can create a version file named ``IPRVersion.dat'' in the IP list
      directory. This file helps managing reloading IP lists, by specifying a version. 
      When the version isn't changed, IP lists will not be reloaded if they are 
      already in shared memory. The version number should be a 32 bit number.\\
      For example:
     \begin{verbatim}
      VERSION=1
     \end{verbatim}
  \item  In the \texttt{<snort root>/src/tools/control} directory, you will find \texttt{snort\_control} command if 
      built with  \texttt{--enable-control-socket} option. 
  \item  Type the following command to reload IP lists. Before typing this command,
      make sure to update version file if you are using version file.
      The \texttt{<path>} is the same path in first step.\\
     \begin{verbatim}
      <snort root>/src/tools/control/snort_control  <path> 1361
     \end{verbatim}
\end{itemize}
 \item[]\textit{Using manifest file to manage loading (optional)}
\begin{itemize}
 \item[]  Using manifest file, you can control the file loading sequence, action taken,
   and support zone based detection. You can create a manifest file named ``zone.info''
   in the IP list directory.\\
    
 \item[]   When Snort is signaled to load new lists, a manifest file is read first to 
   determine which zones the IPs in each list are applicable to and what action 
   to take per list (Block, White, Monitor). \\
   
 \item[]   Files listed in manifest are loaded from top to bottom. You should put files 
   that have higher priority first. In manifest file, you can put up to 255 
   files. Without manifest file, files will be loaded in alphabet order.\\
   
 \item[]   Here's the format of the manifest file. Each line of the file has the 
   following format:\\
    \begin{verbatim} 
     <filename>, <list id>,<action>[, <zone>]+
  
     <list id> ::= 32 bit integer
     <action> ::= "monitor"|"block"|"white"
     <zone>  ::= [0-1051]
    \end{verbatim}

 \item[]   Using manifest file, you can specify a new action called ``monitor'', which
   indicates a packet needs to be inspected, but does not disable detection. This is
   different from ``block'' action, which disables further detection. This new
   action helps users evaluate their IP lists before applying it.
 \item[] An example manifest file:
    \begin{verbatim} 
     #ipreputation manifest file
     white.wlf, 111 ,white, 
     black1.blf, 1112, black,  3, 12
     black2.blf, 1113, black,  3, 12
     monitor.blf,2222, monitor, 0, 2, 8
    \end{verbatim}
\end{itemize}
\end{itemize}
\subsection{GTP Decoder and Preprocessor}
\label{sub:gtp}
GTP (GPRS Tunneling Protocol) is used in core communication networks to establish
a channel between GSNs (GPRS Serving Node). GTP decoding preprocessor provides
ways to tackle intrusion attempts to those networks through GTP. It also makes 
detecting new attacks easier.

Two components are developed: GTP decoder and GTP preprocessor. 
\begin{itemize}
\item GTP decoder extracts payload inside GTP PDU; 
\item GTP preprocessor inspects all the signaling messages and provide keywords for
further inspection
\end{itemize}

When the decoder is enabled and configured, the decoder strips the GTP headers 
and parses the underlying IP/TCP/UDP encapsulated packets. Therefore all rules
and detection work as if there was no GTP header.

Example:

\begin{itemize}
\item[]
Most GTP packets look like this
\begin{verbatim}
IP -> UDP -> GTP -> IP -> TCP -> HTTP
\end{verbatim}

If you had a standard HTTP rule:
\begin{verbatim}
alert tcp any any ->  any $HTTP_PORTS (msg:"Test HTTP"; flow:to_server,established; 
content:"SOMETHINGEVIL"; http_uri;  .... sid:X; rev:Y;)
\end{verbatim}
it would alert on the inner HTTP data that is encapsulated in GTP without any changes
 to the rule other than enabling and configuring the GTP decoder.
\end{itemize}

\subsubsection{Dependency Requirements}

For proper functioning of the preprocessor:
\begin{itemize}

\item Stream session tracking must be enabled, i.e. stream5. UDP must be
      enabled in stream5. The preprocessor requires a session tracker to keep its 
      data. 

\item  IP defragmentation should be enabled, i.e. the frag3 preprocessor should be
      enabled and configured.

\end{itemize}

\subsubsection{GTP Data Channel Decoder Configuration}
GTP decoder extracts payload from GTP PDU. The following configuration sets 
GTP decoding:
\begin{verbatim}
config enable_gtp
\end{verbatim}
By default, GTP decoder uses port number $2152$ (GTPv1) and $3386$ (GTPv0). 
If users want to change those values, they can use \texttt{portvar GTP\_PORTS}:

\begin{verbatim}
portvar GTP_PORTS [2152,3386]
\end{verbatim}

\subsubsection{GTP Control Channel Preprocessor Configuration}

Different from GTP decoder, GTP preprocessor examines all signaling messages.
The preprocessor configuration name is \texttt{gtp}.
\begin{verbatim}
preprocessor gtp
\end{verbatim}
\textit{Option syntax}
\begin{itemize}
\item[]
\begin{tabular}{|l|c|c|p{6cm}|}
\hline
Option & Argument & Required & Default\\
\hline
\hline
\texttt{ports} & \texttt{<ports>} & NO & \texttt{ports \{ 2123 3386 \} }\\
\hline
\end{tabular}
\end{itemize}
\normalsize
\textit{Option explanations}
\begin{itemize}
\item[] \texttt{ports}
\begin{itemize}
\item[]  This specifies on what ports to check for GTP messages. Typically, 
     this will include 5060, 5061.
\item[] \textit{Syntax}
\begin{verbatim}
    ports { <port> [<port>< ... >] }
\end{verbatim}
\item[] \textit{Examples}
\begin{verbatim}
    ports { 2123 3386 2152 }
\end{verbatim}
\item[] Note: there are spaces before and after `\{' and `\}'.
\end{itemize}
\end{itemize}
\normalsize

\textit{Default configuration}
\footnotesize
\begin{verbatim}
     preprocessor gtp
\end{verbatim}
\normalsize
\subsubsection{GTP Decoder Events}
\begin{longtable}{|r|p{13.5cm}|}
\hline
SID & Description\\
\hline
 297 &  Two or more GTP encapsulation layers present \\
\hline
 298 &  GTP header length is invalid  \\
\hline
\end{longtable}

\subsubsection{GTP Preprocessor Events}
\begin{longtable}{|r|p{13.5cm}|}
\hline
SID & Description\\
\hline
 1 &   Message length is invalid. \\
\hline
 2 &  Information element length is invalid.  \\
\hline
 3 &  Information elements are out of order. \\
\hline
\end{longtable}
\subsubsection{Rule Options}
New rule options are supported by enabling the \texttt{gtp} preprocessor:
\begin{itemize}
\item[]
\begin{verbatim}
  gtp_type
  gtp_info
  gtp_version
\end{verbatim}
\end{itemize}

\texttt{gtp\_type}
\label{gtp:gtp_method}
\begin{itemize}
 \item[] The \texttt{gtp\_type} keyword is used to check for specific GTP types.
  User can input message type value, an integer in [0, 255], or a string defined
  in the Table below. More than one type can be specified, via a comma separated
  list, and are OR'ed together. If the type used in a rule is not listed in the 
  preprocessor configuration, an error will be thrown. 

 \item[] A message type can have different type value in different GTP 
  versions. For example, \texttt{sgsn\_\-context\_\-request} has message type 
  value $50$ in GTPv0 and GTPv1, but $130$ in GTPv2. \texttt{gtp\_type} will 
  match to a different value depending on the version number in the packet. 
  In this example, evaluating a GTPv0 or GTPv1 packet will check whether the 
  message type value is $50$; evaluating a GTPv2 packet will check whether the 
  message type value is $130$. When a message type is not defined in a version,
  any packet in that version will always return ``No match''. 
  
 \item[] If an integer is used to specify message type, every GTP packet is 
  evaluated, no matter what version the packet is. If the message type matches
  the value in packet, it will return ``Match''. \\

\textit{Syntax}
\footnotesize
\begin{verbatim}
   gtp_type:<type-list>;
   type-list = type|type, type-list
   type      = "0-255"|
                | "echo_request" | "echo_response" ...
\end{verbatim}
\normalsize
\textit{Examples}
\footnotesize
\begin{verbatim}
   gtp_type:10, 11, echo_request;
\end{verbatim}
\normalsize

\textit{GTP message types}
\small
\begin{longtable}{|r|c|c|c|p{13.5cm}|}
\hline
Type & GTPv0 & GTPv1 & GTPv2\\
\hline
0 & N/A & N/A & N/A\\
\hline
1 & echo\_request & echo\_request & echo\_request\\
\hline
2 & echo\_response & echo\_response & echo\_response\\
\hline
3 & version\_not\_supported & version\_not\_supported & version\_not\_supported\\
\hline
4 & node\_alive\_request & node\_alive\_request & N/A\\
\hline
5 & node\_alive\_response & node\_alive\_response & N/A\\
\hline
6 & redirection\_request & redirection\_request & N/A\\
\hline
7 & redirection\_response & redirection\_response & N/A \\
\hline
16 & create\_pdp\_context\_request & create\_pdp\_context\_request & N/A\\
\hline
17 & create\_pdp\_context\_response & create\_pdp\_context\_response & N/A \\
\hline
18 & update\_pdp\_context\_request & update\_pdp\_context\_request & N/A\\
\hline
19 & update\_pdp\_context\_response & update\_pdp\_context\_response & N/A\\
\hline
20 & delete\_pdp\_context\_request & delete\_pdp\_context\_request & N/A\\
\hline
21 & delete\_pdp\_context\_response & delete\_pdp\_context\_response & N/A\\
\hline
22 & create\_aa\_pdp\_context\_request & init\_pdp\_context\_activation\_request & N/A\\
\hline
23 & create\_aa\_pdp\_context\_response & init\_pdp\_context\_activation\_response & N/A\\
\hline
24 & delete\_aa\_pdp\_context\_request & N/A & N/A\\
\hline
25 & delete\_aa\_pdp\_context\_response & N/A & N/A\\
\hline
26 & error\_indication & error\_indication & N/A\\
\hline
27 & pdu\_notification\_request & pdu\_notification\_request & N/A\\
\hline
28 & pdu\_notification\_response & pdu\_notification\_response & N/A\\
\hline
29 & pdu\_notification\_reject\_request & pdu\_notification\_reject\_request & N/A\\
\hline
30 & pdu\_notification\_reject\_response & pdu\_notification\_reject\_response & N/A\\
\hline
31 & N/A & supported\_ext\_header\_notification & N/A \\
\hline
32 & send\_routing\_info\_request & send\_routing\_info\_request & create\_session\_request \\
\hline
33 & send\_routing\_info\_response & send\_routing\_info\_response & create\_session\_response \\
\hline
34 & failure\_report\_request & failure\_report\_request & modify\_bearer\_request \\
\hline
35 & failure\_report\_response & failure\_report\_response & modify\_bearer\_response \\
\hline
36 & note\_ms\_present\_request & note\_ms\_present\_request & delete\_session\_request \\
\hline
37 & note\_ms\_present\_response & note\_ms\_present\_response & delete\_session\_response \\
\hline
38 & N/A & N/A & change\_notification\_request \\
\hline
39 & N/A & N/A & change\_notification\_response \\
\hline
48 & identification\_request & identification\_request & N/A \\
\hline
49 & identification\_response & identification\_response & N/A \\
\hline
50 & sgsn\_context\_request & sgsn\_context\_request & N/A \\
\hline
51 & sgsn\_context\_response & sgsn\_context\_response & N/A \\
\hline
52 & sgsn\_context\_ack & sgsn\_context\_ack & N/A \\
\hline
53 & N/A & forward\_relocation\_request & N/A \\
\hline
54 & N/A & forward\_relocation\_response & N/A \\
\hline
55 & N/A & forward\_relocation\_complete & N/A \\
\hline
56 & N/A & relocation\_cancel\_request & N/A \\
\hline
57 & N/A & relocation\_cancel\_response & N/A \\
\hline
58 & N/A & forward\_srns\_contex & N/A \\
\hline
59 & N/A & forward\_relocation\_complete\_ack & N/A \\
\hline
60 & N/A & forward\_srns\_contex\_ack & N/A \\
\hline
64 & N/A & N/A & modify\_bearer\_command \\
\hline
65 & N/A & N/A & modify\_bearer\_failure\_indication \\
\hline
66 & N/A & N/A & delete\_bearer\_command \\
\hline
67 & N/A & N/A & delete\_bearer\_failure\_indication \\
\hline
68 & N/A & N/A & bearer\_resource\_command \\
\hline
69 & N/A & N/A & bearer\_resource\_failure\_indication \\
\hline
70 & N/A & ran\_info\_relay & downlink\_failure\_indication \\
\hline
71 & N/A & N/A & trace\_session\_activation \\
\hline
72 & N/A & N/A & trace\_session\_deactivation \\
\hline
73 & N/A & N/A & stop\_paging\_indication \\
\hline
95 & N/A & N/A & create\_bearer\_request \\
\hline
96 & N/A & mbms\_notification\_request & create\_bearer\_response \\
\hline
97 & N/A & mbms\_notification\_response & update\_bearer\_request \\
\hline
98 & N/A & mbms\_notification\_reject\_request & update\_bearer\_response \\
\hline
99 & N/A & mbms\_notification\_reject\_response & delete\_bearer\_request \\
\hline
100 & N/A & create\_mbms\_context\_request & delete\_bearer\_response \\
\hline
101 & N/A & create\_mbms\_context\_response & delete\_pdn\_request \\
\hline
102 & N/A & update\_mbms\_context\_request & delete\_pdn\_response \\
\hline
103 & N/A & update\_mbms\_context\_response & N/A \\
\hline
104 & N/A & delete\_mbms\_context\_request & N/A \\
\hline
105 & N/A & delete\_mbms\_context\_response & N/A \\
\hline
112 & N/A & mbms\_register\_request & N/A \\
\hline
113 & N/A & mbms\_register\_response & N/A \\
\hline
114 & N/A & mbms\_deregister\_request & N/A \\
\hline
115 & N/A & mbms\_deregister\_response & N/A \\
\hline
116 & N/A & mbms\_session\_start\_request & N/A \\
\hline
117 & N/A & mbms\_session\_start\_response & N/A \\
\hline
118 & N/A & mbms\_session\_stop\_request & N/A \\
\hline
119 & N/A & mbms\_session\_stop\_response & N/A \\
\hline
120 & N/A & mbms\_session\_update\_request & N/A \\
\hline
121 & N/A & mbms\_session\_update\_response & N/A \\
\hline
128 & N/A & ms\_info\_change\_request & identification\_request \\
\hline
129 & N/A & ms\_info\_change\_response & identification\_response \\
\hline
130 & N/A & N/A & sgsn\_context\_request \\
\hline
131 & N/A & N/A & sgsn\_context\_response \\
\hline
132 & N/A & N/A & sgsn\_context\_ack \\
\hline
133 & N/A & N/A & forward\_relocation\_request \\
\hline
134 & N/A & N/A & forward\_relocation\_response \\
\hline
135 & N/A & N/A & forward\_relocation\_complete \\
\hline
136 & N/A & N/A & forward\_relocation\_complete\_ack \\
\hline
137 & N/A & N/A & forward\_access \\
\hline
138 & N/A & N/A & forward\_access\_ack \\
\hline
139 & N/A & N/A & relocation\_cancel\_request \\
\hline
140 & N/A & N/A & relocation\_cancel\_response \\
\hline
141 & N/A & N/A & configuration\_transfer\_tunnel \\
\hline
149 & N/A & N/A & detach \\
\hline
150 & N/A & N/A & detach\_ack \\
\hline
151 & N/A & N/A & cs\_paging \\
\hline
152 & N/A & N/A & ran\_info\_relay \\
\hline
153 & N/A & N/A & alert\_mme \\
\hline
154 & N/A & N/A & alert\_mme\_ack \\
\hline
155 & N/A & N/A & ue\_activity \\
\hline
156 & N/A & N/A & ue\_activity\_ack \\
\hline
160 & N/A & N/A & create\_forward\_tunnel\_request \\
\hline
161 & N/A & N/A & create\_forward\_tunnel\_response \\
\hline
162 & N/A & N/A & suspend \\
\hline
163 & N/A & N/A & suspend\_ack \\
\hline
164 & N/A & N/A & resume \\
\hline
165 & N/A & N/A & resume\_ack \\
\hline
166 & N/A & N/A & create\_indirect\_forward\_tunnel\_request \\
\hline
167 & N/A & N/A & create\_indirect\_forward\_tunnel\_response \\
\hline
168 & N/A & N/A & delete\_indirect\_forward\_tunnel\_request \\
\hline
169 & N/A & N/A & delete\_indirect\_forward\_tunnel\_response \\
\hline
170 & N/A & N/A & release\_access\_bearer\_request \\
\hline
171 & N/A & N/A & release\_access\_bearer\_response \\
\hline
176 & N/A & N/A & downlink\_data \\
\hline
177 & N/A & N/A & downlink\_data\_ack \\
\hline
178 & N/A & N/A & N/A \\
\hline
179 & N/A & N/A & pgw\_restart \\
\hline
199 & N/A & N/A & pgw\_restart\_ack \\
\hline
200 & N/A & N/A & update\_pdn\_request \\
\hline
201 & N/A & N/A & update\_pdn\_response \\
\hline
211 & N/A & N/A & modify\_access\_bearer\_request \\
\hline
212 & N/A & N/A & modify\_access\_bearer\_response \\
\hline
231 & N/A & N/A & mbms\_session\_start\_request \\
\hline
232 & N/A & N/A & mbms\_session\_start\_response \\
\hline
233 & N/A & N/A & mbms\_session\_update\_request \\
\hline
234 & N/A & N/A & mbms\_session\_update\_response \\
\hline
235 & N/A & N/A & mbms\_session\_stop\_request \\
\hline
236 & N/A & N/A & mbms\_session\_stop\_response \\
\hline
240 & data\_record\_transfer\_request & data\_record\_transfer\_request & N/A \\
\hline
241 & data\_record\_transfer\_response & data\_record\_transfer\_response & N/A \\
\hline
254 & N/A & end\_marker & N/A \\
\hline
255 & pdu & pdu & N/A \\
\hline
\end{longtable}
\end{itemize}

\texttt{gtp\_info}
\label{gtp:gtp_info}
\begin{itemize}
\item[] The \texttt{gtp\_info} keyword is used to check for specific GTP 
  information element. This keyword restricts the search to the information 
  element field. User can input information element value, an integer in 
  $[0, 255]$, or a string defined in the Table below.  If the information 
  element used in this rule is not listed in the preprocessor configuration,
  an error will be thrown.
   
\item[]  When there are several information elements with the same type in the
  message, this keyword restricts the search to the total consecutive buffer. 
  Because the standard requires same types group together, this feature will be
  available for all valid messages. In the case of ``out of order information 
  elements'', this keyword restricts the search to the last buffer.
   
\item[] Similar to message type, same information element might have different 
  information element value in different GTP versions. For example, 
  \texttt{cause} has value $1$ in GTPv0 and GTPv1, but $2$ in GTPv2. 
  \texttt{gtp\_info} will match to  a different
  value depending on the version number in the packet. When an information 
  element is not defined in a version, any packet in that version will always
  return ``No match''.
  
  If an integer is used to specify information element type, every GTP packet
  is evaluated, no matter what version the packet is. If the message type 
  matches the value in packet, it will return ``Match''.\\

\textit{Syntax}
\footnotesize
\begin{verbatim}
   gtp_info:<ie>;
   ie      = "0-255"|
             "rai" | "tmsi"...
\end{verbatim}
\normalsize
\textit{Examples}
\footnotesize
\begin{verbatim}
   gtp_info: 16;
   gtp_info: tmsi
\end{verbatim}
\normalsize
\textit{GTP information elements}
\small
\begin{longtable}{|r|c|c|c|p{13.5cm}|}
\hline
Type & GTPv0 & GTPv1 & GTPv2\\
\hline
0 & N/A & N/A & N/A \\
\hline
1 & cause & cause & imsi\\
\hline
2 & imsi & imsi & cause \\
\hline
3 & rai & rai & recovery\\
\hline
4 & tlli & tlli & N/A\\
\hline
5 & p\_tmsi & p\_tmsi & N/A\\
\hline
6 & qos & N/A & N/A\\
\hline
7 & N/A & N/A & N/A \\
\hline
8 & recording\_required & recording\_required & N/A\\
\hline
9 & authentication & authentication & N/A\\
\hline
10 & N/A & N/A & N/A\\
\hline
11 & map\_cause & map\_cause & N/A\\
\hline
12 & p\_tmsi\_sig & p\_tmsi\_sig & N/A\\
\hline
13 & ms\_validated & ms\_validated & N/A\\
\hline
14 & recovery & recovery & N/A\\
\hline
15 & selection\_mode & selection\_mode & N/A\\
\hline
16 & flow\_label\_data\_1 & teid\_1 & N/A\\
\hline
17 & flow\_label\_signalling & teid\_control & N/A\\
\hline
18 & flow\_label\_data\_2 & teid\_2 & N/A\\
\hline
19 & ms\_unreachable & teardown\_ind & N/A\\
\hline
20 & N/A & nsapi & N/A\\
\hline
21 & N/A & ranap & N/A\\
\hline
22 & N/A & rab\_context & N/A\\
\hline
23 & N/A & radio\_priority\_sms & N/A\\
\hline
24 & N/A & radio\_priority & N/A\\
\hline
25 & N/A & packet\_flow\_id & N/A\\
\hline
26 & N/A & charging\_char & N/A\\
\hline
27 & N/A & trace\_ref & N/A\\
\hline
28 & N/A & trace\_type & N/A\\
\hline
29 & N/A & ms\_unreachable & N/A\\
\hline
71 & N/A & N/A & apn\\
\hline
72 & N/A & N/A & ambr\\
\hline
73 & N/A & N/A & ebi\\
\hline
74 & N/A & N/A & ip\_addr\\
\hline
75 & N/A & N/A & mei\\
\hline
76 & N/A & N/A & msisdn\\
\hline
77 & N/A & N/A & indication\\
\hline
78 & N/A & N/A & pco\\
\hline
79 & N/A & N/A & paa\\
\hline
80 & N/A & N/A & bearer\_qos\\
\hline
81 & N/A & N/A & flow\_qos\\
\hline
82 & N/A & N/A & rat\_type\\
\hline
83 & N/A & N/A & serving\_network\\
\hline
84 & N/A & N/A & bearer\_tft\\
\hline
85 & N/A & N/A & tad\\
\hline
86 & N/A & N/A & uli\\
\hline
87 & N/A & N/A & f\_teid\\
\hline
88 & N/A & N/A & tmsi\\
\hline
89 & N/A & N/A & cn\_id\\
\hline
90 & N/A & N/A & s103pdf\\
\hline
91 & N/A & N/A & s1udf\\
\hline
92 & N/A & N/A & delay\_value\\
\hline
93 & N/A & N/A & bearer\_context\\
\hline
94 & N/A & N/A & charging\_id\\
\hline
95 & N/A & N/A & charging\_char\\
\hline
96 & N/A & N/A & trace\_info\\
\hline
97 & N/A & N/A & bearer\_flag\\
\hline
98 & N/A & N/A & N/A\\
\hline
99 & N/A & N/A & pdn\_type\\
\hline
100 & N/A & N/A & pti\\
\hline
101 & N/A & N/A & drx\_parameter\\
\hline
102 & N/A & N/A & N/A\\
\hline
103 & N/A & N/A & gsm\_key\_tri\\
\hline
104 & N/A & N/A & umts\_key\_cipher\_quin\\
\hline
105 & N/A & N/A & gsm\_key\_cipher\_quin\\
\hline
106 & N/A & N/A & umts\_key\_quin\\
\hline
107 & N/A & N/A & eps\_quad\\
\hline
108 & N/A & N/A & umts\_key\_quad\_quin\\
\hline
109 & N/A & N/A & pdn\_connection\\
\hline
110 & N/A & N/A & pdn\_number\\
\hline
111 & N/A & N/A & p\_tmsi\\
\hline
112 & N/A & N/A & p\_tmsi\_sig\\
\hline
113 & N/A & N/A & hop\_counter\\
\hline
114 & N/A & N/A & ue\_time\_zone\\
\hline
115 & N/A & N/A & trace\_ref\\
\hline
116 & N/A & N/A & complete\_request\_msg\\
\hline
117 & N/A & N/A & guti\\
\hline
118 & N/A & N/A & f\_container\\
\hline
119 & N/A & N/A & f\_cause\\
\hline
120 & N/A & N/A & plmn\_id\\
\hline
121 & N/A & N/A & target\_id\\
\hline
122 & N/A & N/A & N/A\\
\hline
123 & N/A & N/A & packet\_flow\_id\\
\hline
124 & N/A & N/A & rab\_contex\\
\hline
125 & N/A & N/A & src\_rnc\_pdcp\\
\hline
126 & N/A & N/A & udp\_src\_port\\
\hline
127 & charge\_id & charge\_id & apn\_restriction\\
\hline
128 & end\_user\_address & end\_user\_address & selection\_mode\\
\hline
129 & mm\_context & mm\_context & src\_id\\
\hline
130 & pdp\_context & pdp\_context & N/A\\
\hline
131 & apn & apn & change\_report\_action\\
\hline
132 & protocol\_config & protocol\_config & fq\_csid\\
\hline
133 & gsn & gsn & channel\\
\hline
134 & msisdn & msisdn & emlpp\_pri\\
\hline
135 & N/A & qos & node\_type\\
\hline
136 & N/A & authentication\_qu & fqdn\\
\hline
137 & N/A & tft & ti\\
\hline
138 & N/A & target\_id & mbms\_session\_duration\\
\hline
139 & N/A & utran\_trans & mbms\_service\_area\\
\hline
140 & N/A & rab\_setup & mbms\_session\_id\\
\hline
141 & N/A & ext\_header & mbms\_flow\_id\\
\hline
142 & N/A & trigger\_id & mbms\_ip\_multicast\\
\hline
143 & N/A & omc\_id & mbms\_distribution\_ack\\
\hline
144 & N/A & ran\_trans & rfsp\_index\\
\hline
145 & N/A & pdp\_context\_pri & uci\\
\hline
146 & N/A & addi\_rab\_setup & csg\_info\\
\hline
147 & N/A & sgsn\_number & csg\_id\\
\hline
148 & N/A & common\_flag & cmi\\
\hline
149 & N/A & apn\_restriction & service\_indicator\\
\hline
150 & N/A & radio\_priority\_lcs & detach\_type\\
\hline
151 & N/A & rat\_type & ldn\\
\hline
152 & N/A & user\_loc\_info & node\_feature\\
\hline
153 & N/A & ms\_time\_zone & mbms\_time\_to\_transfer\\
\hline
154 & N/A & imei\_sv & throttling\\
\hline
155 & N/A & camel & arp\\
\hline
156 & N/A & mbms\_ue\_context & epc\_timer\\
\hline
157 & N/A & tmp\_mobile\_group\_id & signalling\_priority\_indication\\
\hline
158 & N/A & rim\_routing\_addr & tmgi\\
\hline
159 & N/A & mbms\_config & mm\_srvcc\\
\hline
160 & N/A & mbms\_service\_area & flags\_srvcc\\
\hline
161 & N/A & src\_rnc\_pdcp & mmbr\\
\hline
162 & N/A & addi\_trace\_info & N/A\\
\hline
163 & N/A & hop\_counter & N/A\\
\hline
164 & N/A & plmn\_id & N/A\\
\hline
165 & N/A & mbms\_session\_id & N/A\\
\hline
166 & N/A & mbms\_2g3g\_indicator & N/A\\
\hline
167 & N/A & enhanced\_nsapi & N/A\\
\hline
168 & N/A & mbms\_session\_duration & N/A\\
\hline
169 & N/A & addi\_mbms\_trace\_info & N/A\\
\hline
170 & N/A & mbms\_session\_repetition\_num & N/A\\
\hline
171 & N/A & mbms\_time\_to\_data & N/A\\
\hline
173 & N/A & bss & N/A\\
\hline
174 & N/A & cell\_id & N/A\\
\hline
175 & N/A & pdu\_num & N/A\\
\hline
176 & N/A & N/A & N/A\\
\hline
177 & N/A & mbms\_bearer\_capab & N/A\\
\hline
178 & N/A & rim\_routing\_disc & N/A\\
\hline
179 & N/A & list\_pfc & N/A\\
\hline
180 & N/A & ps\_xid & N/A\\
\hline
181 & N/A & ms\_info\_change\_report & N/A\\
\hline
182 & N/A & direct\_tunnel\_flags & N/A\\
\hline
183 & N/A & correlation\_id & N/A\\
\hline
184 & N/A & bearer\_control\_mode & N/A\\
\hline
185 & N/A & mbms\_flow\_id & N/A\\
\hline
186 & N/A & mbms\_ip\_multicast & N/A\\
\hline
187 & N/A & mbms\_distribution\_ack & N/A\\
\hline
188 & N/A & reliable\_inter\_rat\_handover & N/A\\
\hline
189 & N/A & rfsp\_index & N/A\\
\hline
190 & N/A & fqdn & N/A\\
\hline
191 & N/A & evolved\_allocation1 & N/A\\
\hline
192 & N/A & evolved\_allocation2 & N/A\\
\hline
193 & N/A & extended\_flags & N/A\\
\hline
194 & N/A & uci & N/A\\
\hline
195 & N/A & csg\_info & N/A\\
\hline
196 & N/A & csg\_id & N/A\\
\hline
197 & N/A & cmi & N/A\\
\hline
198 & N/A & apn\_ambr & N/A\\
\hline
199 & N/A & ue\_network & N/A\\
\hline
200 & N/A & ue\_ambr & N/A\\
\hline
201 & N/A & apn\_ambr\_nsapi & N/A\\
\hline
202 & N/A & ggsn\_backoff\_timer & N/A\\
\hline
203 & N/A & signalling\_priority\_indication & N/A\\
\hline
204 & N/A & signalling\_priority\_indication\_nsapi & N/A\\
\hline
205 & N/A & high\_bitrate & N/A\\
\hline
206 & N/A & max\_mbr & N/A\\
\hline
250 & N/A & N/A & N/A\\
\hline
 & N/A & N/A & N/A\\
\hline
251 & charging\_gateway\_addr & charging\_gateway\_addr & N/A\\
\hline
255 & private\_extension & private\_extension & private\_extension\\
\hline
\end{longtable}
\end{itemize}
\texttt{gtp\_version}
\label{gtp:gtp_version}
\begin{itemize}
 \item[]  The \texttt{gtp\_version} keyword is used to check for specific 
GTP version.  
 \item[]  Because different GTP version defines different message types and
 information elements, this keyword should combine with \texttt{gtp\_type} 
and \texttt{gtp\_info.}\\

\textit{Syntax}
\footnotesize
\begin{verbatim}
   gtp_version:<version>;
   version   = "0, 1, 2'
\end{verbatim}
\normalsize
\textit{Examples}
\footnotesize
\begin{verbatim}
   gtp_version: 1;
\end{verbatim}
\normalsize
\end{itemize}

\subsection{Modbus Preprocessor}
\label{sub:modbus}
The Modbus preprocessor is a Snort module that decodes the Modbus protocol.
It also provides rule options to access certain protocol fields.
This allows a user to write rules for Modbus packets without decoding the protocol
with a series of "content" and "byte\_test" options.

Modbus is a protocol used in SCADA networks. If your network does not contain any
Modbus-enabled devices, we recommend leaving this preprocessor turned off.

\subsubsection{Dependency Requirements}

For proper functioning of the preprocessor:
\begin{itemize}

\item Stream session tracking must be enabled, i.e. stream5. TCP must be
      enabled in stream5. The preprocessor requires a session tracker to keep its 
      data.

\item Protocol Aware Flushing (PAF) must be enabled.

\item IP defragmentation should be enabled, i.e. the frag3 preprocessor should be
      enabled and configured.

\end{itemize}

\subsubsection{Preprocessor Configuration}
To get started, the Modbus preprocessor must be enabled.
The preprocessor name is \texttt{modbus}.
\begin{verbatim}
preprocessor modbus
\end{verbatim}
\textit{Option syntax}
\begin{itemize}
\item[]
\begin{tabular}{|l|c|c|p{6cm}|}
\hline
Option & Argument & Required & Default\\
\hline
\hline
\texttt{ports} & \texttt{<ports>} & NO & \texttt{ports \{ 502 \} }\\
\hline
\end{tabular}
\end{itemize}
\normalsize
\textit{Option explanations}
\begin{itemize}
\item[] \texttt{ports}
\begin{itemize}
\item[]  This specifies on what ports to check for Modbus messages. Typically, 
     this will include 502.
\item[] \textit{Syntax}
\begin{verbatim}
    ports { <port> [<port>< ... >] }
\end{verbatim}
\item[] \textit{Examples}
\begin{verbatim}
    ports { 1237 3945 5067 }
\end{verbatim}
\item[] Note: there are spaces before and after `\{' and `\}'.
\end{itemize}
\end{itemize}
\normalsize

\textit{Default configuration}
\footnotesize
\begin{verbatim}
    preprocessor modbus
\end{verbatim}
\normalsize

\subsubsection{Rule Options}
The Modbus preprocessor adds 3 new rule options. These rule options match on
various pieces of the Modbus headers:

\begin{itemize}
\item[]
\begin{verbatim}
    modbus_func
    modbus_unit
    modbus_data
\end{verbatim}
\end{itemize}

The preprocessor must be enabled for these rule option to work.

\texttt{modbus\_func}
\label{modbus:modbus_func}
\begin{itemize}
 \item[] This option matches against the Function Code inside of a Modbus
 header. The code may be a number (in decimal format), or a string from the
 list provided below.
\end{itemize}

\textit{Syntax}
\footnotesize
\begin{verbatim}
    modbus_func:<code>

    code  = 0-255 |
            "read_coils" |
            "read_discrete_inputs" |
            "read_holding_registers" |
            "read_input_registers" |
            "write_single_coil" |
            "write_single_register" |
            "read_exception_status" |
            "diagnostics" |
            "get_comm_event_counter" |
            "get_comm_event_log" |
            "write_multiple_coils" |
            "write_multiple_registers" |
            "report_slave_id" |
            "read_file_record" |
            "write_file_record" |
            "mask_write_register" |
            "read_write_multiple_registers" |
            "read_fifo_queue" |
            "encapsulated_interface_transport"
\end{verbatim}
\normalsize
\textit{Examples}
\footnotesize
\begin{verbatim}
    modbus_func:1;
    modbus_func:write_multiple_coils;
\end{verbatim}
\normalsize

\texttt{modbus\_unit}
\label{modbus:modbus_unit}
\begin{itemize}
 \item[] This option matches against the Unit ID field in a Modbus header.
\end{itemize}

\textit{Syntax}
\footnotesize
\begin{verbatim}
    modbus_unit:<unit>

    unit = 0-255
\end{verbatim}
\normalsize
\textit{Examples}
\footnotesize
\begin{verbatim}
    modbus_unit:1;
\end{verbatim}
\normalsize

\texttt{modbus\_data}
\label{modbus:modbus_data}
\begin{itemize}
 \item[] This rule option sets the cursor at the beginning of the Data field
 in a Modbus request/response.
\end{itemize}

\textit{Syntax}
\footnotesize
\begin{verbatim}
    modbus_data;
\end{verbatim}
\normalsize

\textit{Examples}
\footnotesize
\begin{verbatim}
    modbus_data; content:"badstuff";
\end{verbatim}
\normalsize

\subsubsection{Preprocessor Events}
The Modbus preprocessor uses GID 144 for its preprocessor events.
\begin{longtable}{|r|p{13.5cm}|}

\hline
SID & Description\\
\hline
 1  & The length in the Modbus header does not match the length needed \\
    & by the Modbus function code. \\
&\\
    & Each Modbus function has an expected format for requests and responses. \\
    & If the length of the message does not match the expected format, this \\
    & alert is generated. \\
\hline
 2  & Modbus protocol ID is non-zero. \\
&\\
    & The protocol ID field is used for multiplexing other protocols with \\
    & Modbus. Since the preprocessor cannot handle these other protocols, \\
    & this alert is generated instead. \\
\hline
 3  & Reserved Modbus function code in use. \\
\hline
\end{longtable}

\subsection{DNP3 Preprocessor}
\label{sub:dnp3}
The DNP3 preprocessor is a Snort module that decodes the DNP3 protocol.
It also provides rule options to access certain protocol fields.
This allows a user to write rules for DNP3 packets without decoding the protocol
with a series of "content" and "byte\_test" options.

DNP3 is a protocol used in SCADA networks. If your network does not contain any
DNP3-enabled devices, we recommend leaving this preprocessor turned off.

\subsubsection{Dependency Requirements}

For proper functioning of the preprocessor:
\begin{itemize}

\item Stream session tracking must be enabled, i.e. stream5. TCP or UDP must be
      enabled in stream5. The preprocessor requires a session tracker to keep its 
      data.

\item Protocol Aware Flushing (PAF) must be enabled.

\item IP defragmentation should be enabled, i.e. the frag3 preprocessor should be
      enabled and configured.

\end{itemize}

\subsubsection{Preprocessor Configuration}
To get started, the DNP3 preprocessor must be enabled.
The preprocessor name is \texttt{dnp3}.
\begin{verbatim}
preprocessor dnp3
\end{verbatim}
\textit{Option syntax}
\begin{itemize}
\item[]
\begin{tabular}{|l|c|c|p{6cm}|}
\hline
Option & Argument & Required & Default\\
\hline
\hline
\texttt{ports} & \texttt{<ports>} & NO & \texttt{ports \{ 20000 \} }\\
\texttt{memcap} & \texttt{<number} & NO & \texttt{memcap 262144}\\
\texttt{check\_crc} & NONE & NO & OFF\\
\texttt{disabled} & NONE & NO & OFF\\
\hline
\end{tabular}
\end{itemize}
\normalsize
\textit{Option explanations}
\begin{itemize}
\item[] \texttt{ports}
\begin{itemize}
\item[]  This specifies on what ports to check for DNP3 messages. Typically, 
     this will include 20000.
\item[] \textit{Syntax}
\begin{verbatim}
    ports { <port> [<port>< ... >] }
\end{verbatim}
\item[] \textit{Examples}
\begin{verbatim}
    ports { 1237 3945 5067 }
\end{verbatim}
\item[] Note: there are spaces before and after `\{' and `\}'.
\end{itemize}

\item[] \texttt{memcap}
\begin{itemize}
\item[]  This sets a maximum to the amount of memory allocated to the DNP3
     preprocessor for session-tracking purposes. The argument is given
     in bytes.
     Each session requires about 4 KB to track, and the default is 256 kB.
     This gives the preprocessor the ability to track 63 DNP3 sessions
     simultaneously.
     Setting the memcap below 4144 bytes will cause a fatal error.
     When multiple configs are used, the memcap in the non-default configs
     will be overwritten by the memcap in the default config. If the default
     config isn't intended to inspect DNP3 traffic, use the "disabled"
     keyword.
\end{itemize}

\item[] \texttt{check\_crc}
\begin{itemize}
\item[]  This option makes the preprocessor validate the checksums contained in
     DNP3 Link-Layer Frames. Frames with invalid checksums will be ignored.
     If the corresponding preprocessor rule is enabled, invalid checksums
     will generate alerts.
     The corresponding rule is GID 145, SID 1.
\end{itemize}

\item[] \texttt{disabled}
\begin{itemize}
\item[]  This option is used for loading the preprocessor without inspecting
     any DNP3 traffic. The \texttt{disabled} keyword is only useful when the DNP3
     preprocessor is turned on in a separate policy.
\end{itemize}
\end{itemize}
\normalsize

\textit{Default configuration}
\footnotesize
\begin{verbatim}
    preprocessor dnp3
\end{verbatim}
\normalsize

\subsubsection{Rule Options}
The DNP3 preprocessor adds 4 new rule options. These rule options match on
various pieces of the DNP3 headers:

\begin{itemize}
\item[]
\begin{verbatim}
    dnp3_func
    dnp3_obj
    dnp3_ind
    dnp3_data
\end{verbatim}
\end{itemize}

The preprocessor must be enabled for these rule option to work.

\texttt{dnp3\_func}
\label{dnp3:dnp3_func}
\begin{itemize}
 \item[] This option matches against the Function Code inside of a DNP3
     Application-Layer request/response header. The code may be a number
     (in decimal format), or a string from the list provided below.
\end{itemize}

\textit{Syntax}
\footnotesize
\begin{verbatim}
    dnp3_func:<code>

    code  = 0-255 |
            "confirm" |
            "read" |
            "write" |
            "select" |
            "operate" |
            "direct_operate" |
            "direct_operate_nr" |
            "immed_freeze" |
            "immed_freeze_nr" |
            "freeze_clear" |
            "freeze_clear_nr" |
            "freeze_at_time" |
            "freeze_at_time_nr" |
            "cold_restart" |
            "warm_restart" |
            "initialize_data" |
            "initialize_appl" |
            "start_appl" |
            "stop_appl" |
            "save_config" |
            "enable_unsolicited" |
            "disable_unsolicited" |
            "assign_class" |
            "delay_measure" |
            "record_current_time" |
            "open_file" |
            "close_file" |
            "delete_file" |
            "get_file_info" |
            "authenticate_file" |
            "abort_file" |
            "activate_config" |
            "authenticate_req" |
            "authenticate_err" |
            "response" |
            "unsolicited_response" |
            "authenticate_resp"

\end{verbatim}
\normalsize
\textit{Examples}
\footnotesize
\begin{verbatim}
    dnp3_func:1;
    dnp3_func:delete_file;
\end{verbatim}
\normalsize

\texttt{dnp3\_ind}
\label{dnp3:dnp3_ind}
\begin{itemize}
 \item[] This option matches on the Internal Indicators flags present in a
     DNP3 Application Response Header. Much like the TCP flags rule option,
     providing multiple flags in one option will cause the rule to fire if \emph{ANY}
     one of the flags is set. To alert on a combination of flags, use multiple rule
     options.
\end{itemize}

\textit{Syntax}
\footnotesize
\begin{verbatim}
    dnp3_ind:<flag>{,<flag>...]

    flag =  "all_stations"
            "class_1_events"
            "class_2_events"
            "class_3_events"
            "need_time"
            "local_control"
            "defice_trouble"
            "device_restart"
            "no_func_code_support"
            "object_unknown"
            "parameter_error"
            "event_buffer_overflow"
            "already_executing"
            "config_corrupt"
            "reserved_2"
            "reserved_1"
\end{verbatim}
\normalsize
\textit{Examples}
\footnotesize
\begin{verbatim}
    # Alert on reserved_1 OR reserved_2
    dnp3_ind:reserved_1,reserved_2;

    # Alert on class_1 AND class_2 AND class_3 events
    dnp3_ind:class_1_events; dnp3_ind:class_2_events; dnp3_ind:class_3_events;
\end{verbatim}
\normalsize

\texttt{dnp3\_obj}
\label{dnp3:dnp3_obj}
\begin{itemize}
 \item[] This option matches on DNP3 object headers present in a request or response.
\end{itemize}

\textit{Syntax}
\footnotesize
\begin{verbatim}
    dnp3_obj:<group>,<var>

    group = 0 - 255
    var   = 0 - 255
\end{verbatim}
\normalsize
\textit{Examples}
\footnotesize
\begin{verbatim}
    # Alert on DNP3 "Date and Time" object
    dnp3_obj:50,1;
\end{verbatim}
\normalsize

\texttt{dnp3\_data}
\label{dnp3:dnp3_data}
\begin{itemize}
 \item[] As Snort processes DNP3 packets, the DNP3 preprocessor collects
     Link-Layer Frames and reassembles them back into Application-Layer Fragments.
     This rule option sets the cursor to the beginning of an Application-Layer
     Fragment, so that other rule options can work on the reassembled data.

     With the dnp3\_data rule option, you can write rules based on the data within
     Fragments without splitting up the data and adding CRCs every 16 bytes.
\end{itemize}

\textit{Syntax}
\footnotesize
\begin{verbatim}
    dnp3_data;
\end{verbatim}
\normalsize

\textit{Examples}
\footnotesize
\begin{verbatim}
    dnp3_data; content:"badstuff_longer_than_16chars";
\end{verbatim}
\normalsize

\subsubsection{Preprocessor Events}
The DNP3 preprocessor uses GID 145 for its preprocessor events.
\begin{longtable}{|r|p{13.5cm}|}

\hline
SID & Description\\
\hline
 1  & A Link-Layer Frame contained an invalid CRC. \\
    & (Enable \texttt{check\_crc} in the preprocessor config to get this alert.) \\
\hline
 2  & A DNP3 Link-Layer Frame was dropped, due to an invalid length. \\
\hline
 3  & A Transport-Layer Segment was dropped during reassembly. \\
    & This happens when segments have invalid sequence numbers. \\
\hline
 4  & The DNP3 Reassembly buffer was cleared before a complete fragment could \\
    & be reassembled. \\
    & This happens when a segment carrying the "FIR" flag appears after some \\
    & other segments have been queued. \\
\hline
 5  & A DNP3 Link-Layer Frame is larger than 260 bytes. \\
\hline
 6  & A DNP3 Link-Layer Frame uses an address that is reserved. \\
\hline
 7  & A DNP3 request or response uses a reserved function code. \\
\hline
\end{longtable}


\subsection{AppId Preprocessor}
\label{sub:appid}
With increasingly complex networks and growing network traffic, network administrators
require application awareness in managing networks. An administrator may allow only
applications that are business relevant, low bandwidth and/or deal with certain
subject matter.

AppId preprocessor adds application level view to manage networks. It does this by adding the following features
\begin{itemize}

\item Network control: The preprocessor provides simplified single point application awareness by making a set of application
identifiers (AppId) available to Snort Rule writers. 

\item Network usage awareness: the preprocessor outputs statistics to
show network bandwidth used by each application seen on network. Administrators can
monitor bandwidth usage and may decide to block applications that are wasteful.

\item Custom applications: The preprocessor enables administrators to create their own application detectors
to detect new applications. The detectors are written in Lua and interface with Snort
using a well-defined C-Lua API.

\end{itemize}

\subsubsection{Dependency Requirements}

For proper functioning of the preprocessor:
\begin{itemize}

\item Stream session tracking must be enabled, i.e. stream5. TCP or UDP must be
      enabled in stream5. The preprocessor requires a session tracker to keep its
      data.

\item Protocol Aware Flushing (PAF) must be enabled.

\item IP defragmentation should be enabled, i.e. the frag3 preprocessor should be
      enabled and configured.

\item HTTP preprocessor must be enabled and configured. The processor does not require
    any AppId specific configuration. The preprocessor provides parsed HTTP headers
    for application determination. Without HTTP preprocessor, AppId preprocessor will
    identify only non-HTTP applications.

\item LuaJIT version 2.0.2 must be installed on
    host where snort is being compiled and run. Newer versions of LuaJIT are not tested for compatibility.
\end{itemize}

\subsubsection{Preprocessor Configuration}
AppId dynamic preprocessor can be enabled during build time. The
following options must be included in ./configure:

--enable-open-appid

The configuration name is "appid":

The preprocessor name is \texttt{appid}.
\begin{verbatim}
preprocessor appid
\end{verbatim}
\textit{Option syntax}
\begin{itemize}
\item[]
\begin{tabular}{|l|c|c|p{6cm}|}
\hline
Option & Argument & Required & Default\\
\hline
\hline
\texttt{app\_detector\_dir} & \texttt{<directory>} & NO & \texttt{app\_detector\_dir \{ /usr/local/etc/appid \} }\\
\texttt{app\_stats\_filename} & \texttt{<filename>} & NO & \texttt{NULL}\\
\texttt{app\_stats\_period} & \texttt{<time in seconds>} & NO & \texttt{300 seconds}\\
\texttt{app\_stats\_rollover\_size} & \texttt{<disk size in bytes>} & NO & \texttt{20 MB}\\
\texttt{app\_stats\_rollover\_time} & \texttt{<time in seconds>} & NO & \texttt{1 day}\\
\texttt{memcap} & \texttt{<memory limit bytes>} & NO & \texttt{256 MB}\\
\texttt{debug} & \texttt{<"yes">} & NO & \texttt{disabled}\\
\texttt{dump\_ports} & \texttt{No} & NO & \texttt{disabled}\\
\hline
\end{tabular}
\end{itemize}
\normalsize
\textit{Option explanations}
\begin{itemize}
\item[] \texttt{app\_detector\_dir}
\begin{itemize}
\item[]
      specifies base path where Cisco provided detectors and application
      configuration files are installed by ODP (Open Detector Package) package.
      The package contains Lua detectors and some application metadata. Customer
      written detectors are stored in subdirectory "custom" under the same base path.
\item[] \textit{Syntax}
\begin{verbatim}
    app_detector_dir <directory name>
\end{verbatim}
\item[] \textit{Examples}
\begin{verbatim}
    app_detector_dir /usr/local/cisco/apps
\end{verbatim}
\end{itemize}

\item[] \texttt{app\_stats\_filename}
\begin{itemize}
\item[]  name of file. If this configuration is missing, application stats are disabled.
\item[] \textit{Syntax}
\begin{verbatim}
    app_stats_filename <filename>
\end{verbatim}
\item[] \textit{Examples}
\begin{verbatim}
    app_stats_filename appStats.log
\end{verbatim}
\end{itemize}

\item[] \texttt{app\_stats\_period}
\begin{itemize}
\item[]
      bucket size in seconds. Default 5 minutes.
\item[] \textit{Syntax}
\begin{verbatim}
    app_stats_period <time in seconds>
\end{verbatim}
\item[] \textit{Examples}
\begin{verbatim}
    app_stats_period 15
\end{verbatim}
\end{itemize}

\item[] \texttt{app\_stats\_rollover\_size}
\begin{itemize}
\item[]  file size which will cause file rollover. Default 20 MB.
\item[] \textit{Syntax}
\begin{verbatim}
    app_stats_rollover_size <file size in bytes>
\end{verbatim}
\item[] \textit{Examples}
\begin{verbatim}
    app_stats_rollover_size 2000000
\end{verbatim}
\end{itemize}

\item[] \texttt{app\_stats\_rollover\_time >}
\begin{itemize}
\item[]
      time since file creation which will cause rollover. Default 1 day.
\item[] \textit{Syntax}
\begin{verbatim}
    app_stats_rollover_time <time in seconds>
\end{verbatim}
\item[] \textit{Examples}
\begin{verbatim}
    app_stats_rollover_time 3600
\end{verbatim}
\end{itemize}

\item[] \texttt{memcap >}
\begin{itemize}
\item[]
      upper bound for memory used by AppId internal structures. Default 32MB.
\item[] \textit{Syntax}
\begin{verbatim}
    memcap <memory in bytes>
\end{verbatim}
\item[] \textit{Examples}
\begin{verbatim}
    memcap 100000000
\end{verbatim}
\end{itemize}

\item[] \texttt{dump\_ports >}
\begin{itemize}
\item[]
      prints port only detectors and information on active detectors. Used for troubleshooting.
\item[] \textit{Syntax}
\begin{verbatim}
    dump_ports <"yes"|"no">
\end{verbatim}
\item[] \textit{Examples}
\begin{verbatim}
    dump_ports "yes"
\end{verbatim}
\end{itemize}

\item[] \texttt{debug}
\begin{itemize}
\item[]
      Used in some old detectors for debugging.
\item[] \textit{Syntax}
\begin{verbatim}
    debug
\end{verbatim}
\item[] \textit{Examples}
\begin{verbatim}
    debug
\end{verbatim}
\end{itemize}

\end{itemize}
\normalsize

\textit{Default configuration}
\footnotesize
\begin{verbatim}
    preprocessor appid
\end{verbatim}
\normalsize

\subsubsection{Rule Options}
The AppId preprocessor adds 1 new rule option as follows:

\begin{itemize}
\item[]
\begin{verbatim}
    appid
\end{verbatim}
\end{itemize}

The preprocessor must be enabled for this rule option to work.

\texttt{appid}
\label{appid:appid}
\begin{itemize}
 \item[] 
The rule option allows users to customize
rules to specific application in a simple manner. The option can take up to 10
application names separated by spaces, tabs, or commas. Application names in
rules are the names you will see in last column in appMapping.data file.  A
rule is considered a match if one of the appId in a rule match an appId in a
session. 

For client side packets, payloadAppId in a session is matched with all AppIds
in a rule.  Thereafter miscAppId, clientAppId and serviceAppId are matched.
Since Alert Events contain one AppId, only the first match is reported. If rule
without appId option matches, then the most specific appId (in order of
payload, misc, client, server) is reported.

The same logic is followed for server side packets with one exception. Order of
matching is changed to make serviceAppId higher then clientAppId.

\end{itemize}

\textit{Syntax}
\footnotesize
\begin{verbatim}
    appid:<list of application names>
\end{verbatim}
\normalsize
\textit{Examples}
\footnotesize
\begin{verbatim}
    appid: http;
    appid: ftp, ftp-data;
    appid: cnn.com, zappos;
\end{verbatim}
\normalsize

\subsubsection{Application Rule Events}
A new event type is defined for logging application name in Snort
Alerts in unified2 format only. These events contain only one application name.
The Events can be enabled for unified2 output using 'appid\_event\_types keyword. 

For example, the following configuration will log alert in my.alert file with
application name.
\footnotesize
\begin{verbatim}
    output alert\_unified2: filename my.alert, appid\_event\_types
\end{verbatim}
\normalsize

u2spewfoo, u2openappid, u2streamer tools can be used to print alerts in new format. 
Each event will display additional application name at the end of the event.

\textit{Examples}
\footnotesize
\begin{verbatim}
#> u2spewfoo outputs the following event format
(Event)
        sensor id: 0    event id: 6     event second: 1292962302        event microsecond: 227323
        sig id: 18763   gen id: 1       revision: 4      classification: 0
        priority: 0     ip source: 98.27.88.56  ip destination: 10.4.10.79
        src port: 80    dest port: 54767        protocol: 6     impact\_flag: 0  blocked: 0
        mpls label: 0   vland id: 0     policy id: 0    appid: zappos
\end{verbatim}
\normalsize

\subsubsection{Application Usage Statistics}

AppId preprocessor prints application network usage periodically in snort
log directory in unified2 format. File name, time interval for statistic and
file rollover are controlled by appId preprocessor configuration. u2spewfoo,
u2openappid, u2streamer tools can be used to print contents of these files. An
example output from u2openappid tools is as follows:
\footnotesize
\begin{verbatim}

    statTime="1292962290",appName="firefox",txBytes="9395",rxBytes="77021"
    statTime="1292962290",appName="google\_analytic",txBytes="2024",rxBytes="928"
    statTime="1292962290",appName="http",txBytes="28954",rxBytes="238000"
    statTime="1292962290",appName="zappos",txBytes="26930",rxBytes="237072"
\end{verbatim}
\normalsize

\subsubsection{Open Detector Package (ODP) Installation}
Application detectors from Snort team will be delivered in a separate package
called Open Detector Package. ODP is a package that contains the following
artifacts:
\begin{enumerate}

\item Application detectors in Lua language. 

\item Port detectors, which are port only application detectors, in  meta-data in YAML format.

\item appMapping.data file containing application metadata. This file should not
   be modified.  The first column contains application identifier and second column 
   contains application name.  Other columns contain internal information.

\item Lua library files DetectorCommon.lua, flowTrackerModule.lua and
   hostServiceTrackerModule.lua 
\end{enumerate}
User can install ODP package in any directory of its choosing and configure
this directory in app\_detector\_dir option in appId preprocessor configuration.
Installing ODP will not modify any subdirectory named custom, where
user-created detectors are located. 

When installed, ODP will create following sub-directories:
\footnotesize
\begin{verbatim}
    odp/port    //Cisco port-only detectors
    odp/lua     //Cisco Lua detectors
    odp/libs    //Cisco Lua modules
\end{verbatim}
\normalsize

\subsubsection{User Created Application Detectors}

Users can create new applications by coding detectors in Lua language.  Users can also
copy Snort team provided detectors into custom subdirectory and customize the detector. A
document will be posted on Snort Website with details on API usage.

\par \noindent Users must organize their Lua detectors and libraries by creating the following
directory structure, under ODP installation directory.
\footnotesize
\begin{verbatim}

    custom/port    //port-only detectors
    custom/lua     //Lua detectors
    custom/libs    //Lua modules
\end{verbatim}
\normalsize

\section{Decoder and Preprocessor Rules}

Decoder and preprocessor rules allow one to enable and disable decoder and
preprocessor events on a rule by rule basis.  They also allow one to specify
the rule type or action of a decoder or preprocessor event on a rule by rule
basis.

Decoder config options will still determine whether or not to generate decoder
events.  For example, if \texttt{config disable\_decode\_alerts} is in
\texttt{snort.conf}, decoder events will not be generated regardless of whether
or not there are corresponding rules for the event.  Also note that if the
decoder is configured to enable drops, e.g. \texttt{config
enable\_decode\_drops}, these options will take precedence over the event type
of the rule.  A packet will be dropped if either a decoder config drop option
is in \texttt{snort.conf} or the decoder or preprocessor rule type is
\texttt{drop}.  Of course, the drop cases only apply if Snort is running
inline.  See \texttt{doc/README.decode} for config options that control decoder
events.

\subsection{Configuring}

The decoder and preprocessor rules are located in the \texttt{preproc\_rules/}
directory in the top level source tree, and have the names
\texttt{decoder.rules} and \texttt{preprocessor.rules} respectively.  These
files are updated as new decoder and preprocessor events are added to Snort.
The \texttt{gen-msg.map} under \texttt{etc} directory is also updated with 
new decoder and preprocessor rules.

To enable these rules in \texttt{snort.conf}, define the path to where the
rules are located and uncomment the \texttt{include} lines in
\texttt{snort.conf} that reference the rules files.

\begin{verbatim}
    var PREPROC_RULE_PATH /path/to/preproc_rules
    ...
    include $PREPROC_RULE_PATH/preprocessor.rules
    include $PREPROC_RULE_PATH/decoder.rules
\end{verbatim}

To disable any rule, just comment it with a \texttt{\#} or remove the rule
completely from the file (commenting is recommended).

To change the rule type or action of a decoder/preprocessor rule, just replace
\texttt{alert} with the desired rule type.  Any one of the following rule types
can be used:

\begin{verbatim}
    alert
    log
    pass
    drop
    sdrop
    reject
\end{verbatim}

For example one can change:

\begin{verbatim}
    alert ( msg: "DECODE_NOT_IPV4_DGRAM"; sid: 1; gid: 116; rev: 1; \
        metadata: rule-type decode ; classtype:protocol-command-decode;)
\end{verbatim}

to

\begin{verbatim}
    drop ( msg: "DECODE_NOT_IPV4_DGRAM"; sid: 1; gid: 116; rev: 1; \
        metadata: rule-type decode ; classtype:protocol-command-decode;)
\end{verbatim}

to drop (as well as alert on) packets where the Ethernet protocol is IPv4 but
version field in IPv4 header has a value other than 4.

See \texttt{README.decode}, \texttt{README.gre} and the various preprocessor
READMEs for descriptions of the rules in \texttt{decoder.rules} and
\texttt{preprocessor.rules}.

The generator ids ( gid ) for different preprocessors and the decoder are as follows:

\begin{table}[h]
\begin{center}
\begin{tabular}{| l | l |}

\hline
\textbf{Generator Id} & \textbf{Module}\\
\hline

\hline
\texttt{105} & Back Orifice preprocessor \\

\hline
\texttt{106} & RPC Decode preprocessor \\

\hline
\texttt{112} & Arpspoof preprocessor \\

\hline
\texttt{116} & Snort Decoder \\

\hline
\texttt{119} & HTTP Inspect preprocessor ( Client ) \\

\hline
\texttt{120} & HTTP Inspect preprocessor ( Server ) \\

\hline
\texttt{122} & Portscan preprocessor \\

\hline
\texttt{123} & Frag3 preprocessor \\

\hline
\texttt{124} & SMTP preprocessor \\

\hline
\texttt{125} & FTP (FTP) preprocessor \\

\hline
\texttt{126} & FTP (Telnet) preprocessor \\

\hline
\texttt{127} & ISAKMP preprocessor \\

\hline
\texttt{128} & SSH preprocessor \\

\hline
\texttt{129} & Stream preprocessor \\

\hline
\texttt{131} & DNS preprocessor \\

\hline
\texttt{132} & Skype preprocessor \\

\hline
\texttt{133} & DceRpc2 preprocessor \\

\hline
\texttt{134} & PPM preprocessor \\

\hline
\texttt{136} & Reputation preprocessor \\

\hline
\texttt{137} & SSL preprocessor \\

\hline
\texttt{139} & SDF preprocessor \\

\hline
\texttt{140} & SIP preprocessor \\

\hline
\texttt{141} & IMAP preprocessor \\

\hline
\texttt{142} & POP preprocessor \\

\hline
\texttt{143} & GTP preprocessor \\
\hline
\end{tabular}
\end{center}
\end{table}


\subsection{Reverting to original behavior}

The following config option in \texttt{snort.conf} will make Snort revert to the
old behavior:

\begin{verbatim}
    config autogenerate_preprocessor_decoder_rules
\end{verbatim}

Note that if you want to revert to the old behavior, you also have to remove
the decoder and preprocessor rules and any reference to them from
\texttt{snort.conf}, otherwise they will be loaded.  This option applies to
rules not specified and the default behavior is to alert.

\section{Event Processing}

Snort provides a variety of mechanisms to tune event processing to suit your
needs:

\begin{itemize} 
\item \texttt{Detection Filters} 

You can use detection filters to specify a threshold that must be exceeded
before a rule generates an event.  This is covered in section
\ref{detection_filter}.

\item \texttt{Rate Filters} 

You can use rate filters to change a rule action when the number or rate of
events indicates a possible attack.

\item \texttt{Event Filters} 

You can use event filters to reduce the number of logged events for noisy
rules.  This can be tuned to significantly reduce false alarms.

\item \texttt{Event Suppression} 

You can completely suppress the logging of uninteresting events.

\end{itemize} 

\subsection{Rate Filtering}
\label{rate_filtering}

\texttt{rate\_filter} provides rate based attack prevention by allowing users
to configure a new action to take for a specified time when a given rate is
exceeded.  Multiple rate filters can be defined on the same rule, in which case
they are evaluated in the order they appear in the configuration file, and the
first applicable action is taken.  

\subsubsection{Format}

Rate filters are used as standalone configurations (outside of a rule) and have
the following format:

\begin{verbatim}
    rate_filter \
        gen_id <gid>, sig_id <sid>, \
        track <by_src|by_dst|by_rule>, \
        count <c>, seconds <s>, \
        new_action alert|drop|pass|log|sdrop|reject, \
        timeout <seconds> \
        [, apply_to <ip-list>]
\end{verbatim}

The options are described in the table below - all are required except
\texttt{apply\_to}, which is optional.

% fixthis this table should appear here with the [h] but floats away!
\begin{table}[h]
\begin{center}
\begin{tabular}{| p{2in} | p{4in} |}

\hline
\textbf{Option} & \textbf{Description}\\
\hline

\hline
\texttt{track by\_src | by\_dst | by\_rule} &

rate is tracked either by source IP address, destination IP address, or by
rule.  This means the match statistics are maintained for each unique source IP
address, for each unique destination IP address, or they are aggregated at rule
level.  For rules related to Stream sessions, source and destination means
client and server respectively.  \texttt{track by\_rule} and \texttt{apply\_to}
may not be used together.\\

\hline
\texttt{count c} &

the maximum number of rule matches in \texttt{s} seconds before the rate filter
limit to is exceeded.  \texttt{c} must be nonzero value.\\

\hline
\texttt{seconds s} &

the time period over which \texttt{count} is accrued. 0 seconds means
\texttt{count} is a total count instead of a specific rate.  For example,
\texttt{rate\_filter} may be used to detect if the number of connections to a
specific server exceed a specific count.  0 seconds only applies to internal
rules (gen\_id 135) and other use will produce a fatal error by Snort.\\

\hline
\texttt{new\_action alert | drop | pass | log | sdrop | reject} &

\texttt{new\_action} replaces rule action for \texttt{t} seconds.
\texttt{drop}, \texttt{reject}, and \texttt{sdrop} can be used only when snort
is used in inline mode.  \texttt{sdrop} and \texttt{reject} are conditionally
compiled with GIDS.\\

\hline
\texttt{timeout t} &

revert to the original rule action after \texttt{t} seconds.  If \texttt{t} is
0, then rule action is never reverted back.  An \texttt{event\_filter} may be
used to manage number of alerts after the rule action is enabled by
\texttt{rate\_filter}.\\

\hline
\texttt{apply\_to <ip-list>} &

restrict the configuration to only to source or destination IP address
(indicated by track parameter) determined by \texttt{<ip-list>}.  \texttt{track
by\_rule} and \texttt{apply\_to} may not be used together.  Note that events
are generated during the timeout period, even if the rate falls below the
configured limit.\\

\hline
\end{tabular}
\end{center}
\end{table}

\subsubsection{Examples}

Example 1 - allow a maximum of 100 connection attempts per second from any one
IP address, and block further connection attempts from that IP address for 10
seconds:

\begin{verbatim}
    rate_filter \
        gen_id 135, sig_id 1, \
        track by_src, \
        count 100, seconds 1, \
        new_action drop, timeout 10
\end{verbatim}

Example 2 - allow a maximum of 100 successful simultaneous connections from any
one IP address, and block further connections from that IP address for 10
seconds:

\begin{verbatim}
    rate_filter \
        gen_id 135, sig_id 2, \
        track by_src, \
        count 100, seconds 0, \
        new_action drop, timeout 10
\end{verbatim}

\subsection{Event Filtering}
\label{event_filtering}

Event filtering can be used to reduce the number of logged alerts for noisy
rules by limiting the number of times a particular event is logged during a
specified time interval.  This can be tuned to significantly reduce false
alarms.

There are 3 types of event filters:

\begin{itemize} 
\item \texttt{limit} 

Alerts on the 1st \textit{m} events during the time interval, then ignores
events for the rest of the time interval.

\item \texttt{threshold} 

Alerts every \textit{m} times we see this event during the time interval.

\item \texttt{both} 

Alerts once per time interval after seeing \textit{m} occurrences of the event,
then ignores any additional events during the time interval.

\end{itemize}

\subsubsection{Format}

\begin{verbatim}
    event_filter \
        gen_id <gid>, sig_id <sid>, \
        type <limit|threshold|both>, \
        track <by_src|by_dst>, \
        count <c>, seconds <s>  

    threshold \
        gen_id <gid>, sig_id <sid>, \
        type <limit|threshold|both>, \
        track <by_src|by_dst>, \
        count <c>, seconds <s>  
\end{verbatim}

\texttt{threshold} is an alias for \texttt{event\_filter}.  Both formats are
equivalent and support the options described below - all are required.
\texttt{threshold} is deprecated and will not be supported in future releases.

\begin{table}[h]
\begin{center}
\begin{tabular}{| l | p{4.5in} |}

\hline
\textbf{Option} & \textbf{Description}\\
\hline

\hline
\texttt{gen\_id <gid>} &

Specify the generator ID of an associated rule.  \texttt{gen\_id 0, sig\_id 0}
can be used to specify a "global" threshold that applies to all rules.\\

\hline \texttt{sig\_id <sid>} &

Specify the signature ID of an associated rule.  \texttt{sig\_id 0} specifies a
"global" filter because it applies to all \texttt{sig\_id}s for the given
\texttt{gen\_id}.\\

\hline
\texttt{type limit|threshold|both} &

type \texttt{limit} alerts on the 1st m events during the time interval, then
ignores events for the rest of the time interval.  Type \texttt{threshold}
alerts every m times we see this event during the time interval.  Type
\texttt{both} alerts once per time interval after seeing m occurrences of the
event, then ignores any additional events during the time interval.\\

\hline
\texttt{track by\_src|by\_dst} &

rate is tracked either by source IP address, or destination IP address.  This
means count is maintained for each unique source IP addresses, or for each
unique destination IP addresses.  Ports or anything else are not tracked.\\

\hline
\texttt{count c} &

number of rule matching in s seconds that will cause \texttt{event\_filter}
limit to be exceeded.  \texttt{c} must be nonzero value.  A value of -1 disables
the event filter and can be used to override the global \texttt{event\_filter}.\\

\hline
\texttt{seconds s} &

time period over which \texttt{count} is accrued.  \texttt{s} must be nonzero
value.\\

\hline
\end{tabular}
\end{center}
\end{table}

\begin{note}

Only one \texttt{event\_filter} may be defined for a given \texttt{gen\_id,
sig\_id}.  If more than one \texttt{event\_filter} is applied to a specific
\texttt{gen\_id, sig\_id} pair, Snort will terminate with an error while
reading the configuration information. 

\end{note}

\texttt{event\_filter}s with \texttt{sig\_id} 0 are considered "global" because
they apply to all rules with the given \texttt{gen\_id}.  If \texttt{gen\_id}
is also 0, then the filter applies to all rules.  (\texttt{gen\_id 0, sig\_id
!= 0} is not allowed).  Standard filtering tests are applied first, if they do
not block an event from being logged, the global filtering test is applied.
Thresholds in a rule (deprecated) will override a global
\texttt{event\_filter}.  Global \texttt{event\_filter}s do not override what's
in a signature or a more specific stand-alone \texttt{event\_filter}.

\begin{note}
\texttt{event\_filters} can be used to suppress excessive \texttt{rate\_filter}
alerts, however, the first \texttt{new\_action} event of the timeout period is
never suppressed.  Such events indicate a change of state that are significant
to the user monitoring the network.
\end{note}

\subsubsection{Examples}

Limit logging to 1 event per 60 seconds:

\begin{verbatim}
    event_filter \
        gen_id 1, sig_id 1851, \
        type limit, track by_src,  \
        count 1, seconds 60
\end{verbatim}

Limit logging to every 3rd event:

\begin{verbatim}
    event_filter \
        gen_id 1, sig_id 1852,  \
        type threshold, track by_src, \
        count 3, seconds 60
\end{verbatim}

Limit logging to just 1 event per 60 seconds, but only if we exceed 30 events
in 60 seconds:

\begin{verbatim}
    event_filter \
        gen_id 1, sig_id 1853, \
        type both, track by_src, \
        count 30, seconds 60
\end{verbatim}

Limit to logging 1 event per 60 seconds per IP triggering each rule (rule
gen\_id is 1):

\begin{verbatim}
    event_filter \
        gen_id 1, sig_id 0, \
        type limit, track by_src, \
        count 1, seconds 60
\end{verbatim}

Limit to logging 1 event per 60 seconds per IP, triggering each rule for each
event generator:

\begin{verbatim}
    event_filter \
        gen_id 0, sig_id 0, \
        type limit, track by_src, \
        count 1, seconds 60
\end{verbatim}

Events in Snort are generated in the usual way, event filters are handled as
part of the output system. Read gen-msg.map for details on gen ids.

Users can also configure a memcap for threshold with a ``config:'' option:

\begin{verbatim}
    config event_filter: memcap <bytes>

    # this is deprecated:
    config threshold: memcap <bytes>
\end{verbatim}

\subsection{Event Suppression}
\label{event_suppression}

Event suppression stops specified events from firing without removing the rule
from the rule base. Suppression uses an IP list to select specific networks and
users for suppression.  Suppression tests are performed prior to either
standard or global thresholding tests.

Suppression are standalone configurations that reference generators, SIDs, and
IP addresses via an IP list . This allows a rule to be completely suppressed,
or suppressed when the causative traffic is going to or coming from a specific
IP or group of IP addresses.

You may apply multiple suppressions to a non-zero SID.  You may also combine
one \texttt{event\_filter} and several suppressions to the same non-zero SID.  

\subsubsection{Format}

The suppress configuration has two forms:

\begin{verbatim}
    suppress \
        gen_id <gid>, sig_id <sid>
\end{verbatim}

\begin{verbatim}
    suppress \
        gen_id <gid>, sig_id <sid>, \
        track <by_src|by_dst>, ip <ip-list>
\end{verbatim}

\begin{table}[h]
\begin{center}
\begin{tabular}{| l | p{4.5in} |}

\hline
\textbf{Option} & \textbf{Description}\\
\hline

\hline
\texttt{gen\_id <gid>} &

Specify the generator ID of an associated rule.  \texttt{gen\_id 0, sig\_id 0}
can be used to specify a "global" threshold that applies to all rules.\\

\hline \texttt{sig\_id <sid>} &

Specify the signature ID of an associated rule.  \texttt{sig\_id 0} specifies a
"global" filter because it applies to all \texttt{sig\_id}s for the given
\texttt{gen\_id}.\\

\hline
\texttt{track by\_src|by\_dst} &

Suppress by source IP address or destination IP address.  This is optional, but
if present, \texttt{ip} must be provided as well.\\

\hline
\texttt{ip <list>} &

Restrict the suppression to only source or destination IP addresses (indicated
by \texttt{track} parameter) determined by $<$list$>$.  If track is provided, ip
must be provided as well.\\

\hline
\end{tabular}
\end{center}
\end{table}

\subsubsection{Examples}

Suppress this event completely:

\begin{verbatim}
    suppress gen_id 1, sig_id 1852:
\end{verbatim}

Suppress this event from this IP:

\begin{verbatim}
    suppress gen_id 1, sig_id 1852, track by_src, ip 10.1.1.54
\end{verbatim}

Suppress this event to this CIDR block:

\begin{verbatim}
    suppress gen_id 1, sig_id 1852, track by_dst, ip 10.1.1.0/24
\end{verbatim}

\subsection{Event Logging}
\label{eventqueue}

Snort supports logging multiple events per packet/stream that are prioritized
with different insertion methods, such as max content length or event ordering
using the event queue.

The general configuration of the event queue is as follows:

\begin{verbatim}
    config event_queue: [max_queue [size]] [log [size]] [order_events [TYPE]]
\end{verbatim}

\subsubsection{Event Queue Configuration Options}

There are three configuration options to the configuration parameter
'event\_queue'.

\begin{slist}
\item \texttt{max\_queue}

This determines the maximum size of the event queue.  For example, if the event
queue has a max size of 8, only 8 events will be stored for a single packet or
stream.

The default value is 8.

\item \texttt{log}

This determines the number of events to log for a given packet or stream.  You
can't log more than the max\_event number that was specified.

The default value is 3.

\item \texttt{order\_events}

This argument determines the way that the incoming events are ordered.  We
currently have two different methods:

\begin{itemize}

\item \texttt{priority} - The highest priority (1 being the highest) events are
ordered first.

\item \texttt{content\_length} - Rules are ordered before decode or
preprocessor alerts, and rules that have a longer content are ordered before
rules with shorter contents.

\end{itemize}

The method in which events are ordered does not affect rule types such as pass,
alert, log, etc.

The default value is content\_length.
\end{slist}

\subsubsection{Event Queue Configuration Examples}

The default configuration:

\begin{verbatim}
    config event_queue: max_queue 8 log 3 order_events content_length
\end{verbatim}

Example of a reconfigured event queue:

\begin{verbatim}
    config event_queue: max_queue 10 log 3 order_events content_length
\end{verbatim}

Use the default event queue values, but change event order:

\begin{verbatim}
    config event_queue: order_events priority
\end{verbatim}

Use the default event queue values but change the number of logged events:

\begin{verbatim}
    config event_queue: log 2
\end{verbatim}

\subsection{Event Trace}
\label{eventtrace}

Snort supports logging additional information to a file about the events it is
generating relative to specific blocks of data that are matching the rule.
The blocks of data logged include information about the event, the GID, SID,
and other data related to the event itself, plus packet data including sizes,
timestamps, raw, normalized, and decompressed buffers extracted from the packet
that may have been used in evaluating the rule.  The amount of packet data
written is limited with each entry.  This is useful in debugging rules.

The \texttt{config} option \texttt{event\_trace} to \texttt{snort.conf}
provides this control.

The general configuration for event tracing is as follows:

\begin{verbatim}
    config event_trace: [file <filename>] [max_data <int>]
\end{verbatim}

The are two configuration options for \texttt{event\_trace}.

\begin{slist}
\item \texttt{file}

This sets the file name into which the trace data is written, within
Snort's log directory (see -l command line option).

The default is event\_trace.txt.

\item \texttt{max\_data}

This specifies the maximum number of bytes from each buffer of data to write into
the file.

The default is 64 bytes and valid values range from 1 to 65535 bytes.

\end{slist}

\subsubsection{Event Trace Examples}

The default configuration:

\begin{verbatim}
    config event_trace: file event_trace.txt max_data 64
\end{verbatim}

Use the default file, but change the amount of data logged:

\begin{verbatim}
    config event_trace: max_data 128
\end{verbatim}

Change the file name to which event traces are logged:

\begin{verbatim}
    config event_trace: file snort_event_trace.out
\end{verbatim}

\section{Performance Profiling}

Snort can provide statistics on rule and preprocessor performance.  Each
require only a simple \texttt{config} option to \texttt{snort.conf} and Snort
will print statistics on the worst (or all) performers on exit. When a file
name is provided in \texttt{profile\_rules} or \texttt{profile\_preprocs}, the
statistics will be saved in these files.  If \texttt{append} is not specified,
a new file will be created each time Snort is run. The filenames will have
timestamps appended to them.  These files will be found in the logging
directory.

To use this feature, you must build snort with the \texttt{--enable-perfprofiling}
option to the configure script.

\subsection{Rule Profiling}
\label{rule profiling}

\subsubsection{Format}

\begin{verbatim}
    config profile_rules: \
        print [all | <num>], \
        sort <sort_option> \
        [,filename <filename> [append]]
\end{verbatim}

\begin{itemize}
\item \texttt{<num>} is the number of rules to print
\item \texttt{<sort\_option>} is one of:

    \subitem \texttt{checks}
    \subitem \texttt{matches}
    \subitem \texttt{nomatches}
    \subitem \texttt{avg\_ticks}
    \subitem \texttt{avg\_ticks\_per\_match}
    \subitem \texttt{avg\_ticks\_per\_nomatch}
    \subitem \texttt{total\_ticks}

\item \texttt{<filename>} is the output filename

\item \texttt{[append]} dictates that the output will go to the same file each
time (optional)

\end{itemize}

\subsubsection{Examples}
\begin{itemize}

\item Print all rules, sort by avg\_ticks (default configuration if option is turned on)
\subitem \texttt{config profile\_rules}

\item Print all rules, sort by avg\_ticks, and append to file \texttt{rules\_stats.txt} 
\subitem \texttt{config profile\_rules: filename \texttt{rules\_stats.txt} append}

\item Print the top 10 rules, based on highest average time
\subitem \texttt{config profile\_rules: print 10, sort avg\_ticks}

\item Print all rules, sorted by number of checks
\subitem \texttt{config profile\_rules: print all, sort checks}

\item Print top 100 rules, based on total time
\subitem \texttt{config profile\_rules: print 100, sort total\_ticks}

\item Print with default options, save results to performance.txt each time
\subitem \texttt{config profile\_rules: filename performance.txt append}

\item Print top 20 rules, save results to perf.txt with timestamp in filename
\subitem \texttt{config profile\_rules: print 20, filename perf.txt}

\end{itemize}

\subsubsection{Output}

Snort will print a table much like the following at exit.  

\begin{figure}
\footnotesize{
\begin{verbatim}
Rule Profile Statistics (worst 4 rules)
==========================================================
   Num      SID GID Rev     Checks   Matches    Alerts               Ticks Avg/Check  Avg/Match Avg/Nonmatch
   ===      === === ===     ======   =======    ======               ===== =========  ========= ============
     1     2389   1  12          1         1         1              385698  385698.0   385698.0          0.0
     2     2178   1  17          2         0         0              107822   53911.0        0.0      53911.0
     3     2179   1   8          2         0         0               92458   46229.0        0.0      46229.0
     4     1734   1  37          2         0         0               90054   45027.0        0.0      45027.0
\end{verbatim}
}
\caption{\label{rule profiling example output}Rule Profiling Example Output}
\end{figure}

Configuration line used to print the above table: 

\subitem \texttt{config profile\_rules: print 4, sort total\_ticks}

The columns represent:

\begin{itemize}
\item Number (rank)
\item Sig ID
\item Generator ID
\item Checks (number of times rule was evaluated after fast pattern match
  within portgroup or any-$>$any rules)
\item Matches (number of times ALL rule options matched, will be high for
  rules that have no options)
\item Alerts (number of alerts generated from this rule)
\item CPU Ticks 
\item Avg Ticks per Check
\item Avg Ticks per Match
\item Avg Ticks per Nonmatch
\end{itemize}

Interpreting this info is the key.  The Microsecs (or Ticks) column is
important because that is the total time spent evaluating a given rule.  But,
if that rule is causing alerts, it makes sense to leave it alone.

A high Avg/Check is a poor performing rule, that most likely contains PCRE.
High Checks and low Avg/Check is usually an any-$>$any rule with few rule
options and no content.  Quick to check, the few options may or may not match.
We are looking at moving some of these into code, especially those with low
SIDs.

By default, this information will be printed to the console when Snort exits.
You can use the "filename" option in snort.conf to specify a file where this
will be written. If "append" is not specified, a new file will be created each
time Snort is run. The filenames will have timestamps appended to them. These
files will be found in the logging directory.

\subsection{Preprocessor Profiling}
\label{preproc profiling}

\subsubsection{Format}

\begin{verbatim}
    config profile_preprocs: \
        print [all | <num>], \
        sort <sort_option> \
        [, filename <filename> [append]]
\end{verbatim}

\begin{itemize}
\item \texttt{<num>} is the number of preprocessors to print
\item \texttt{<sort\_option>} is one of:
\subitem \texttt{checks}
\subitem \texttt{avg\_ticks}
\subitem \texttt{total\_ticks}
\item \texttt{<filename>} is the output filename
\item \texttt{[append]} dictates that the output will go to the same file each time (optional)
\end{itemize}

\subsubsection{Examples}

\begin{itemize}

\item Print all preprocessors, sort by avg\_ticks (default configuration if
option is turned on)
\subitem \texttt{config profile\_preprocs}

\item Print all preprocessors, sort by avg\_ticks, and append to file
\texttt{preprocs\_stats.txt} 
\subitem \texttt{config profile\_preprocs: filename \texttt{preprocs\_stats.txt} append}

\item Print the top 10 preprocessors, based on highest average time
\subitem \texttt{config profile\_preprocs: print 10, sort avg\_ticks}

\item Print all preprocessors, sorted by number of checks
\subitem \texttt{config profile\_preprocs: print all, sort checks}

\end{itemize}

\subsubsection{Output}

Snort will print a table much like the following at exit.

\begin{figure}
\footnotesize{
\begin{verbatim}
Preprocessor Profile Statistics (worst 10)
==========================================================
 Num            Preprocessor Layer     Checks      Exits           Microsecs  Avg/Check Pct of Caller Pct of Total
 ===            ============ =====     ======      =====           =========  ========= ============= ============
  1                   detect     0     338181     338181             9054573      26.77         64.62        64.62
   1               rule eval     1     256978     256978             2570596      10.00         28.39        18.35
    1         rule tree eval     2     399860     399860             2520629       6.30         98.06        17.99
     1                  pcre     3      51328      51328              505636       9.85         20.06         3.61
     2             byte_jump     3          6          6                   7       1.30          0.00         0.00
     3               content     3    1077588    1077588             1123373       1.04         44.57         8.02
     4            uricontent     3     106498     106498               79685       0.75          3.16         0.57
     5             byte_test     3       9951       9951                5709       0.57          0.23         0.04
     6              isdataat     3       8486       8486                3192       0.38          0.13         0.02
     7              flowbits     3     135739     135739               35365       0.26          1.40         0.25
     8                 flags     3          2          2                   0       0.20          0.00         0.00
     9  preproc_rule_options     3      15499      15499                1939       0.13          0.08         0.01
    10                  flow     3     394817     394817               36420       0.09          1.44         0.26
    11             file_data     3      15957      15957                1264       0.08          0.05         0.01
    12                   ack     3          4          4                   0       0.07          0.00         0.00
    2               rtn eval     2      36928      36928               17500       0.47          0.68         0.12
   2                    mpse     1     646528     646528             5840244       9.03         64.50        41.68
  2                       s5     0     310080     310080             3270702      10.55         23.34        23.34
   1                   s5tcp     1     310080     310080             2993020       9.65         91.51        21.36
    1             s5TcpState     2     304484     304484             2559085       8.40         85.50        18.26
     1            s5TcpFlush     3      22148      22148               70681       3.19          2.76         0.50
      1  s5TcpProcessRebuilt     4      22132      22132             2018748      91.21       2856.11        14.41
      2     s5TcpBuildPacket     4      22132      22132               34965       1.58         49.47         0.25
     2             s5TcpData     3     184186     184186              120794       0.66          4.72         0.86
      1       s5TcpPktInsert     4      46249      46249               89299       1.93         73.93         0.64
    2           s5TcpNewSess     2       5777       5777               37958       6.57          1.27         0.27
  3              httpinspect     0     204751     204751             1814731       8.86         12.95        12.95
  4                      ssl     0      10780      10780               16283       1.51          0.12         0.12
  5                   decode     0     312638     312638              437860       1.40          3.12         3.12
  6               DceRpcMain     0     155358     155358              186061       1.20          1.33         1.33
   1           DceRpcSession     1     155358     155358              156193       1.01         83.95         1.11
  7              backorifice     0         77         77                  42       0.55          0.00         0.00
  8                     smtp     0      45197      45197               17126       0.38          0.12         0.12
  9                      ssh     0      26453      26453                7195       0.27          0.05         0.05
 10                      dns     0         28         28                   5       0.18          0.00         0.00
 total                 total     0     311202     311202            14011946      45.03          0.00         0.00
\end{verbatim}
}
\caption{Preprocessor Profiling Example Output}
\label{preprocessor profiling example output}
\end{figure}

Configuration line used to print the above table: 

\begin{verbatim}
    config profile_preprocs: \
        print 10, sort total_ticks
\end{verbatim}

The columns represent:

\begin{itemize}

\item Number (rank) - The number is indented for each layer.  Layer 1
preprocessors are listed under their respective caller (and sorted similarly).

\item Preprocessor Name

\item Layer - When printing a specific number of preprocessors all subtasks
info for a particular preprocessor is printed for each layer 0 preprocessor
stat.

\item Checks (number of times preprocessor decided to look at a packet, ports
matched, app layer header was correct, etc)

\item Exits  (number of corresponding exits -- just to verify code is
instrumented correctly, should ALWAYS match Checks, unless an exception was
trapped)

\item CPU Ticks

\item Avg Ticks per Check

\item Percent of caller - For non layer 0 preprocessors, i.e. subroutines
within preprocessors, this identifies the percent of the caller's ticks that is
spent for this subtask.

\end{itemize}

Because of task swapping, non-instrumented code, and other factors, the Pct of
Caller field will not add up to 100\% of the caller's time.  It does give a
reasonable indication of how much relative time is spent within each subtask.

By default, this information will be printed to the console when Snort exits.
You can use the "filename" option in snort.conf to specify a file where this
will be written. If "append" is not specified, a new file will be created each
time Snort is run. The filenames will have timestamps appended to them. These
files will be found in the logging directory.

\subsection{Packet Performance Monitoring (PPM)}
\label{ppm}
PPM provides thresholding mechanisms that can be used to provide a basic
level of latency control for snort.  It does not provide a hard and fast
latency guarantee but should in effect provide a good average latency
control.  Both rules and packets can be checked for latency.  The action
taken upon detection of excessive latency is configurable.  The following
sections describe configuration, sample output, and some implementation
details worth noting.

To use PPM, you must build with the --enable-ppm or the --enable-sourcefire
option to configure.

PPM is configured as follows:

\begin{verbatim}
    # Packet configuration:
    config ppm: max-pkt-time <micro-secs>, \
        fastpath-expensive-packets, \
        pkt-log, \
        debug-pkts
    
    # Rule configuration:
    config ppm: max-rule-time <micro-secs>, \
        threshold count, \
        suspend-expensive-rules, \
        suspend-timeout <seconds>, \
        rule-log [log] [alert]
\end{verbatim}

Packets and rules can be configured separately, as above, or together in just
one config ppm statement.  Packet and rule monitoring is independent, so one or
both or neither may be enabled.

\subsubsection{Configuration}

Packet Configuration Options

\texttt{max-pkt-time <micro-secs>}
\begin{itemize}
\item enables packet latency thresholding using 'micros-secs' as the limit.
\item default is 0 (packet latency thresholding disabled)
\item reasonable starting defaults: 100/250/1000 for 1G/100M/5M nets
\end{itemize}

\texttt{fastpath-expensive-packets}
\begin{itemize}
\item enables stopping further inspection of a packet if the max time is
      exceeded
\item default is off
\end{itemize}

\texttt{pkt-log}
\begin{itemize}
\item enables logging packet event if packet exceeds max-pkt-time
\item default is no logging
\item if no option is given for 'pkt-log', 'pkt-log log' is implied
\item the log option enables output to syslog or console depending 
      upon snort configuration
\end{itemize}

\texttt{debug-pkts}
\begin{itemize}
\item must build with the --enable-debug option to configure
\item enables per packet timing stats to be printed after each packet
\item default is off
\end{itemize}

Rule Configuration Options

\texttt{max-rule-time <micro-secs>}
\begin{itemize}
\item enables rule latency thresholding using 'micros-secs' as the limit.
\item default is 0 (rule latency thresholding disabled)
\item reasonable starting defaults: 100/250/1000 for 1G/100M/5M nets
\end{itemize}

\texttt{threshold <count>}
\begin{itemize}
\item sets the number of cumulative rule time excesses before disabling
      a rule
\item default is 5
\end{itemize}

\texttt{suspend-expensive-rules}
\begin{itemize}
\item enables suspending rule inspection if the max rule time is exceeded
\item default is off
\end{itemize}

\texttt{suspend-timeout <seconds>}
\begin{itemize}
\item rule suspension time in seconds
\item default is 60 seconds
\item set to zero to permanently disable expensive rules
\end{itemize}

\texttt{rule-log [log] [alert]}
\begin{itemize}
\item enables event logging output for rules
\item default is no logging
\item one or both of the options 'log' and 'alert' must be used with
      'rule-log'
\item the log option enables output to syslog or console depending 
      upon snort configuration
\end{itemize}

\subsubsection{Examples}

Example 1:
The following enables packet tracking:

\begin{verbatim}
    config ppm: max-pkt-time 100
\end{verbatim}

The following enables rule tracking:

\begin{verbatim}
    config ppm: max-rule-time 50, threshold 5
\end{verbatim}

If fastpath-expensive-packets or suspend-expensive-rules is not used, then
no action is taken other than to increment the count of the number of
packets that should be fastpath'd or the rules that should be suspended. A
summary of this information is printed out when snort exits.

Example 2:

The following suspends rules and aborts packet inspection.  These rules were
used to generate the sample output that follows.

\begin{verbatim}
    config ppm: \
        max-pkt-time 50, fastpath-expensive-packets, \
        pkt-log, debug-pkts
    
    config ppm: \
        max-rule-time 50, threshold 5, suspend-expensive-rules, \
        suspend-timeout 300, rule-log log alert
\end{verbatim}

\subsubsection{Sample Snort Output}

Sample Snort Startup Output

\begin{verbatim}
    Packet Performance Monitor Config:
      ticks per usec  : 1600 ticks
      max packet time : 50 usecs
      packet action   : fastpath-expensive-packets
      packet logging  : log
      debug-pkts      : disabled
    
    Rule Performance Monitor Config:
      ticks per usec  : 1600 ticks
      max rule time   : 50 usecs
      rule action     : suspend-expensive-rules
      rule threshold  : 5 
      suspend timeout : 300 secs
      rule logging    : alert log 
\end{verbatim}

Sample Snort Run-time Output

\begin{verbatim}
    ...
    PPM: Process-BeginPkt[61] caplen=60
    PPM: Pkt[61] Used= 8.15385 usecs
    PPM: Process-EndPkt[61]
    
    PPM: Process-BeginPkt[62] caplen=342
    PPM: Pkt[62] Used= 65.3659 usecs
    PPM: Process-EndPkt[62]
    
    PPM: Pkt-Event Pkt[63] used=56.0438 usecs, 0 rules, 1 nc-rules tested, packet fastpathed
         (10.4.12.224:0 -> 10.4.14.108:54321).
    PPM: Process-BeginPkt[63] caplen=60
    PPM: Pkt[63] Used= 8.394 usecs
    PPM: Process-EndPkt[63]
    
    PPM: Process-BeginPkt[64] caplen=60
    PPM: Pkt[64] Used= 8.21764 usecs
    PPM: Process-EndPkt[64]
    ...
\end{verbatim}

Sample Snort Exit Output

\begin{verbatim}
    Packet Performance Summary:
       max packet time       : 50 usecs
       packet events         : 1
       avg pkt time          : 0.633125 usecs
    Rule Performance Summary:
       max rule time         : 50 usecs
       rule events           : 0
       avg nc-rule time      : 0.2675 usecs
\end{verbatim}

\subsubsection{Implementation Details}

\begin{itemize}

\item Enforcement of packet and rule processing times is done after processing
each rule.  Latency control is not enforced after each preprocessor.  

\item This implementation is software based and does not use an interrupt
driven timing mechanism and is therefore subject to the granularity of the
software based timing tests. Due to the granularity of the timing measurements
any individual packet may exceed the user specified packet or rule processing
time limit.  Therefore this implementation cannot implement a precise latency
guarantee with strict timing guarantees.  Hence the reason this is considered a
best effort approach.

\item Since this implementation depends on hardware based high performance
frequency counters, latency thresholding is presently only available on Intel
and PPC platforms. 

\item Time checks are made based on the total system time, not processor usage
by Snort.  This was a conscious design decision because when a system is
loaded, the latency for a packet is based on the total system time, not just
the processor time the Snort application receives.  Therefore, it is
recommended that you tune your thresholding to operate optimally when your
system is under load.

\end{itemize}

\section{Output Modules}
\label{output config}

Output modules are new as of version 1.6. They allow Snort to be much more
flexible in the formatting and presentation of output to its users. The output
modules are run when the alert or logging subsystems of Snort are called, after
the preprocessors and detection engine.  The format of the directives in the
config file is very similar to that of the preprocessors.

Multiple output plugins may be specified in the Snort configuration file. When
multiple plugins of the same type (log, alert) are specified, they are stacked
and called in sequence when an event occurs. As with the standard logging and
alerting systems, output plugins send their data to /var/log/snort by default
or to a user directed directory (using the -l command line switch).

Output modules are loaded at runtime by specifying the output keyword in the
config file:

\begin{verbatim}
    output <name>: <options>
\end{verbatim}

\begin{verbatim}
    output alert_syslog: log_auth log_alert
\end{verbatim}

\subsection{alert\_syslog}
\label{alert syslog label}

This module sends alerts to the syslog facility (much like the -s command line
switch). This module also allows the user to specify the logging facility and
priority within the Snort config file, giving users greater flexibility in
logging alerts.

\subsubsection{Available Keywords}

\paragraph{Facilities}

\begin{itemize}
\item \texttt{log\_auth} 
\item \texttt{log\_authpriv} 
\item \texttt{log\_daemon}
\item \texttt{log\_local0} 
\item \texttt{log\_local1}
\item \texttt{log\_local2}
\item \texttt{log\_local3}
\item \texttt{log\_local4}
\item \texttt{log\_local5}
\item \texttt{log\_local6}
\item \texttt{log\_local7}
\item \texttt{log\_user}
\end{itemize}

\paragraph{Priorities}

\begin{itemize}
\item \texttt{log\_emerg} 
\item \texttt{log\_alert}
\item \texttt{log\_crit}
\item \texttt{log\_err}
\item \texttt{log\_warning} 
\item \texttt{log\_notice}
\item \texttt{log\_info}
\item \texttt{log\_debug}
\end{itemize}

\paragraph{Options}

\begin{itemize}
\item \texttt{log\_cons} 
\item \texttt{log\_ndelay}
\item \texttt{log\_perror}
\item \texttt{log\_pid}
\end{itemize}

\subsubsection{Format}

\begin{verbatim}
    alert_syslog: \
        <facility> <priority> <options>
\end{verbatim}

\begin{note}

As WIN32 does not run syslog servers locally by default, a hostname and port
can be passed as options.  The default host is 127.0.0.1.  The default port is
514.

\end{note}

\begin{verbatim}
    output alert_syslog: \
        [host=<hostname[:<port>],] \
        <facility> <priority> <options>
\end{verbatim}

\subsubsection{Example}

\begin{verbatim}
    output alert_syslog: host=10.1.1.1:514, <facility> <priority> <options>
\end{verbatim}

\subsection{alert\_fast}

This will print Snort alerts in a quick one-line format to a specified output
file. It is a faster alerting method than full alerts because it doesn't need
to print all of the packet headers to the output file and because it logs to
only 1 file.

\subsubsection{Format}

\begin{verbatim}
    output alert_fast: [<filename> ["packet"] [<limit>]]
    <limit> ::= <number>[('G'|'M'|K')]
\end{verbatim}

\begin{itemize}
\item \texttt{filename}: the name of the log file.  The default name is
$<$logdir$>$/alert.  You may specify "stdout" for terminal output.  The name may
include an absolute or relative path.

\item \texttt{packet}: this option will cause multiline entries with full
packet headers to be logged.  By default, only brief single-line entries are
logged.

\item \texttt{limit}: an optional limit on file size which defaults to 128 MB.  
The minimum is 1 KB.  See \ref{Log Limits} for more information.
\end{itemize}

\subsubsection{Example}

\begin{verbatim}
    output alert_fast: alert.fast
\end{verbatim}

\subsection{alert\_full}

This will print Snort alert messages with full packet headers. The alerts will
be written in the default logging directory (/var/log/snort) or in the logging
directory specified at the command line.

Inside the logging directory, a directory will be created per IP.  These files
will be decoded packet dumps of the packets that triggered the alerts. The
creation of these files slows Snort down considerably.  This output method is
discouraged for all but the lightest traffic situations.

\subsubsection{Format}

\begin{verbatim}
    output alert_full: [<filename> [<limit>]]
    <limit> ::= <number>[('G'|'M'|K')]
\end{verbatim}

\begin{itemize}
\item \texttt{filename}: the name of the log file.  The default name is
$<$logdir$>$/alert.  You may specify "stdout" for terminal output.  The name may
include an absolute or relative path.

\item \texttt{limit}: an optional limit on file size which defaults to 128 MB.  
The minimum is 1 KB.  See \ref{Log Limits} for more information.
\end{itemize}

\subsubsection{Example}

\begin{verbatim}
    output alert_full: alert.full
\end{verbatim}

\subsection{alert\_unixsock}

Sets up a UNIX domain socket and sends alert reports to it. External
programs/processes can listen in on this socket and receive Snort alert and
packet data in real time.

\subsubsection{Format}

\begin{verbatim}
    alert_unixsock
\end{verbatim}

\subsubsection{Example}

\begin{verbatim}
    output alert_unixsock
\end{verbatim}

\begin{note}
On FreeBSD, the default \texttt{sysctl} value for \texttt{net.local.dgram.recvspace}
is too low for \texttt{alert\_unixsock} datagrams and you will likely not receive any
data.  You can change this value after booting by running:
\begin{verbatim}

$ sudo sysctl net.local.dgram.recvspace=100000

\end{verbatim}
To have this value set on each boot automatically, add the following to \texttt{/etc/sysctl.conf}:
\begin{verbatim}

net.local.dgram.recvspace=100000

\end{verbatim}
Note that the value of 100000 may be slightly generous, but the value should be at least 65864.
\end{note}

\subsection{log\_tcpdump}

The log\_tcpdump module logs packets to a tcpdump-formatted file.  This is
useful for performing post-process analysis on collected traffic with the vast
number of tools that are available for examining tcpdump-formatted files.

\subsubsection{Format}

\begin{verbatim}
    output log_tcpdump: [<filename> [<limit>]]
    <limit> ::= <number>[('G'|'M'|K')]
\end{verbatim}

\begin{itemize}
\item \texttt{filename}: the name of the log file.  The default name is
$<$logdir$>$/snort.log.  The name may include an absolute or relative path.  A
UNIX timestamp is appended to the filename.

\item \texttt{limit}: an optional limit on file size which defaults to 128 MB.
When a sequence of packets is to be logged, the aggregate size is used to test
the rollover condition.  See \ref{Log Limits} for more information.
\end{itemize}

\subsubsection{Example}

\begin{verbatim}
    output log_tcpdump: snort.log
\end{verbatim}

\subsection{csv}

The csv output plugin allows alert data to be written in a format easily
importable to a database.  The output fields and their order may be customized.

\subsubsection{Format}

\begin{verbatim}
    output alert_csv: [<filename> [<format> [<limit>]]]
    <format> ::= "default"|<list>
    <list> ::= <field>(,<field>)*
    <field> ::= "dst"|"src"|"ttl" ...
    <limit> ::= <number>[('G'|'M'|K')]
\end{verbatim}

\begin{itemize}
\item \texttt{filename}: the name of the log file.  The default name is
$<$logdir$>$/alert.csv.  You may specify "stdout" for terminal output.  The name
may include an absolute or relative path.

\item \texttt{format}: The list of formatting options is below. If the
formatting option is "default", the output is in the order of the formatting
options listed.

\begin{itemize}
\item \texttt{timestamp}
\item \texttt{sig\_generator}
\item \texttt{sig\_id}
\item \texttt{sig\_rev}
\item \texttt{msg}
\item \texttt{proto}
\item \texttt{src}
\item \texttt{srcport}
\item \texttt{dst}
\item \texttt{dstport}
\item \texttt{ethsrc}
\item \texttt{ethdst}
\item \texttt{ethlen}
\item \texttt{tcpflags}
\item \texttt{tcpseq}
\item \texttt{tcpack}
\item \texttt{tcplen}
\item \texttt{tcpwindow}
\item \texttt{ttl}
\item \texttt{tos}
\item \texttt{id}
\item \texttt{dgmlen}
\item \texttt{iplen}
\item \texttt{icmptype}
\item \texttt{icmpcode}
\item \texttt{icmpid}
\item \texttt{icmpseq}
\end{itemize}

\item \texttt{limit}: an optional limit on file size which defaults to 128 MB.  
The minimum is 1 KB.  See \ref{Log Limits} for more information.
\end{itemize}

\subsubsection{Example}

\begin{verbatim}
    output alert_csv: /var/log/alert.csv default

    output alert_csv: /var/log/alert.csv timestamp, msg
\end{verbatim}

\subsection{unified 2}

Unified2 can work in one of three modes, packet logging, alert logging, or true
unified logging.  Packet logging includes a capture of the entire packet and is
specified with \texttt{log\_unified2}.  Likewise, alert logging will only log
events and is specified with \texttt{alert\_unified2}.  To include both logging
styles in a single, unified file, simply specify \texttt{unified2}.

When MPLS support is turned on, MPLS labels can be included in unified2 events.
Use option \texttt{mpls\_event\_types} to enable this. If option
\texttt{mpls\_event\_types} is not used, then MPLS labels will be not be
included in unified2 events.

\begin{note}

By default, unified 2 files have the file creation time (in Unix Epoch format)
appended to each file when it is created.  

\end{note}

\subsubsection{Format}

\begin{verbatim}
    output alert_unified2: \
        filename <base filename> [, <limit <size in MB>] [, nostamp] [, mpls_event_types] \
        [, vlan_event_types]

    output log_unified2: \
        filename <base filename> [, <limit <size in MB>] [, nostamp]

    output unified2: \
        filename <base file name> [, <limit <size in MB>] [, nostamp] [, mpls_event_types] \
        [, vlan_event_types]
\end{verbatim}

\subsubsection{Example}

\begin{verbatim}
    output alert_unified2: filename snort.alert, limit 128, nostamp
    output log_unified2: filename snort.log, limit 128, nostamp
    output unified2: filename merged.log, limit 128, nostamp
    output unified2: filename merged.log, limit 128, nostamp, mpls_event_types
    output unified2: filename merged.log, limit 128, nostamp, vlan_event_types
\end{verbatim}

\subsubsection{Extra Data Configurations}

  Unified2 also has logging support for various extra data. The
  following configuration items will enable these extra data logging
  facilities.

\begin{verbatim}
config log_ipv6_extra_data
\end{verbatim}

This option enables Snort to log IPv6 source and destination
address as unified2 extra data events.

See section \ref{Config} for more information

\begin{verbatim}
enable_xff
\end{verbatim}

This option enables HTTP Inspect to parse and log the original
client IP present in the X-Forwarded-For, True-Client-IP, or custom
HTTP request headers along with the generated events.

See section \ref{sub:http-inspect} for more information

\begin{verbatim}
log_uri
\end{verbatim}

This option enables HTTP Inspect to parse and log the URI data
from the HTTP request and log it along with all the generated
events for that session.

See section \ref{sub:http-inspect} for more information

\begin{verbatim}
log_hostname 
\end{verbatim}

This option enables HTTP Inspect to parse and log the Host header
data from the HTTP request and log it along with all the generated
events for that session.

See section \ref{sub:http-inspect} for more information

\begin{verbatim}
log_hostname 
\end{verbatim}

This option enables HTTP Inspect to parse and log the Host header
data from the HTTP request and log it along with all the generated
events for that session.

See section \ref{sub:http-inspect} for more information

\begin{verbatim}
log_mailfrom
\end{verbatim}

This option enables SMTP preprocessor to parse and log the senders
email address extracted from the "MAIL FROM" command along with
all the generated events for that session.

See section \ref{SMTP} for more information

\begin{verbatim}
log_rcptto
\end{verbatim}

This option enables SMTP preprocessor to parse and log the
recipients email address extracted from the "RCPT FROM" command
along with all the generated events for that session.

See section \ref{SMTP} for more information

\begin{verbatim}
log_rcptto
\end{verbatim}

This option enables SMTP preprocessor to parse and log the MIME
attachment filenames extracted from the Content-Disposition header
within the MIME body along with all the generated events for that
session.

See section \ref{SMTP} for more information

\begin{verbatim}
log_email_hdrs
\end{verbatim}

This option enables SMTP preprocessor to parse and log the SMTP
email headers extracted from the SMTP data along with all the
generated events for that session.

See section \ref{SMTP} for more information

\subsubsection{Reading Unified2 Files}

\subsubsection{U2SpewFoo}

U2SpewFoo is a lightweight tool for dumping the contents of unified2 files to stdout.
  
\textbf Example usage:

\begin{verbatim}    
    u2spewfoo snort.log
\end{verbatim}


\textbf Example Output:

\begin{verbatim}
(Event)
    sensor id: 0    event id: 4 event second: 1299698138    event microsecond: 146591
    sig id: 1   gen id: 1   revision: 0  classification: 0
    priority: 0 ip source: 10.1.2.3 ip destination: 10.9.8.7
    src port: 60710 dest port: 80   protocol: 6 impact_flag: 0  blocked: 0

Packet
    sensor id: 0    event id: 4 event second: 1299698138
    packet second: 1299698138   packet microsecond: 146591
    linktype: 1 packet_length: 54
[    0] 02 09 08 07 06 05 02 01 02 03 04 05 08 00 45 00  ..............E.
[   16] 00 28 00 06 00 00 40 06 5C B7 0A 01 02 03 0A 09  .(....@.\.......
[   32] 08 07 ED 26 00 50 00 00 00 62 00 00 00 2D 50 10  ...&.P...b...-P.
[   48] 01 00 A2 BB 00 00                                ......

(ExtraDataHdr)
    event type: 4   event length: 33

(ExtraData)
    sensor id: 0    event id: 2 event second: 1299698138
    type: 9 datatype: 1 bloblength: 9   HTTP URI: /

(ExtraDataHdr)
    event type: 4   event length: 78

(ExtraData)
    sensor id: 0    event id: 2 event second: 1299698138
    type: 10    datatype: 1 bloblength: 12  HTTP Hostname: example.com 
\end{verbatim}

\subsubsection{U2Boat}

U2boat is a tool for converting unified2 files into different formats.

Currently supported conversion formats are: pcap

\textbf Example usage:
\begin{verbatim}
    u2boat -t pcap <infile> <outfile>
\end{verbatim}

\subsection{log null}

Sometimes it is useful to be able to create rules that will alert to certain
types of traffic but will not cause packet log entries.  In Snort 1.8.2, the
log\_null plugin was introduced. This is equivalent to using the -n command
line option but it is able to work within a ruletype.

\subsubsection{Format}

\begin{verbatim}
    output log_null
\end{verbatim}

\subsubsection{Example}

\begin{verbatim}
    output log_null  # like using snort -n
    
    ruletype info {
        type alert
        output alert_fast: info.alert
        output log_null
    }
\end{verbatim}

\subsection{Log Limits}
\label{Log Limits}

This section pertains to logs produced by \texttt{alert\_fast},
\texttt{alert\_full}, \texttt{alert\_csv}, and \texttt{log\_tcpdump}.
\texttt{unified2} also may be given limits.  Those limits
are described in the respective sections.

When a configured limit is reached, the current log is closed and a new log is
opened with a UNIX timestamp appended to the configured log name.

Limits are configured as follows:

\begin{verbatim}
    <limit> ::= <number>[(<gb>|<mb>|<kb>)]
    <gb> ::= 'G'|'g'
    <mb> ::= 'M'|'m'
    <kb> ::= 'K'|'k'
\end{verbatim}

Rollover will occur at most once per second so if limit is too small for
logging rate, limit will be exceeded.  Rollover works correctly if snort is
stopped/restarted.

\section{Host Attribute Table}
\label{targetbased}

Starting with version 2.8.1, Snort has the capability to use information from
an outside source to determine both the protocol for use with Snort rules, and
IP-Frag policy (see section \ref{frag3 section}) and TCP Stream reassembly
policies (see section \ref{stream5 section}).  This information is stored in an
attribute table, which is loaded at startup.  The table is re-read during run
time upon receipt of signal number 30.

Snort associates a given packet with its attribute data from the table, if
applicable.

For rule evaluation, service information is used instead of the ports when the
protocol metadata in the rule matches the service corresponding to the traffic.
If the rule doesn't have protocol metadata, or the traffic doesn't have any
matching service information, the rule relies on the port information.

\begin{note}

To use a host attribute table, Snort must be configured with the
--enable-targetbased flag.

\end{note}

\subsection{Configuration Format}

\begin{verbatim}
    attribute_table filename <path to file>
\end{verbatim}

\subsection{Attribute Table File Format}

The attribute table uses an XML format and consists of two sections, a mapping
section, used to reduce the size of the file for common data elements, and the
host attribute section.  The mapping section is optional.

An example of the file format is shown below.

\begin{verbatim}
<SNORT_ATTRIBUTES>
    <ATTRIBUTE_MAP>
        <ENTRY>
            <ID>1</ID>
            <VALUE>Linux</VALUE>
        </ENTRY>
        <ENTRY>
            <ID>2</ID>
            <VALUE>ssh</VALUE>
        </ENTRY>
    </ATTRIBUTE_MAP>
    <ATTRIBUTE_TABLE>
        <HOST>
            <IP>192.168.1.234</IP>
            <OPERATING_SYSTEM>
                <NAME>
                    <ATTRIBUTE_ID>1</ATTRIBUTE_ID>
                    <CONFIDENCE>100</CONFIDENCE>
                </NAME>
                <VENDOR>
                    <ATTRIBUTE_VALUE>Red Hat</ATTRIBUTE_VALUE>
                    <CONFIDENCE>99</CONFIDENCE>
                </VENDOR>
                <VERSION>
                    <ATTRIBUTE_VALUE>2.6</ATTRIBUTE_VALUE>
                    <CONFIDENCE>98</CONFIDENCE>
                </VERSION>
                <FRAG_POLICY>linux</FRAG_POLICY>
                <STREAM_POLICY>linux</STREAM_POLICY>
            </OPERATING_SYSTEM>
            <SERVICES>
                <SERVICE>
                    <PORT>
                        <ATTRIBUTE_VALUE>22</ATTRIBUTE_VALUE>
                        <CONFIDENCE>100</CONFIDENCE>
                    </PORT>
                    <IPPROTO>
                        <ATTRIBUTE_VALUE>tcp</ATTRIBUTE_VALUE>
                        <CONFIDENCE>100</CONFIDENCE>
                    </IPPROTO>
                    <PROTOCOL>
                        <ATTRIBUTE_ID>2</ATTRIBUTE_ID>
                        <CONFIDENCE>100</CONFIDENCE>
                    </PROTOCOL>
                    <APPLICATION>
                        <ATTRIBUTE_VALUE>OpenSSH</ATTRIBUTE_VALUE>
                        <CONFIDENCE>100</CONFIDENCE>
                        <VERSION>
                            <ATTRIBUTE_VALUE>3.9p1</ATTRIBUTE_VALUE>
                            <CONFIDENCE>93</CONFIDENCE>
                        </VERSION>
                    </APPLICATION>
                </SERVICE>
                <SERVICE>
                    <PORT>
                        <ATTRIBUTE_VALUE>2300</ATTRIBUTE_VALUE>
                        <CONFIDENCE>100</CONFIDENCE>
                    </PORT>
                    <IPPROTO>
                        <ATTRIBUTE_VALUE>tcp</ATTRIBUTE_VALUE>
                        <CONFIDENCE>100</CONFIDENCE>
                    </IPPROTO>
                    <PROTOCOL>
                        <ATTRIBUTE_VALUE>telnet</ATTRIBUTE_VALUE>
                        <CONFIDENCE>100</CONFIDENCE>
                    </PROTOCOL>
                    <APPLICATION>
                        <ATTRIBUTE_VALUE>telnet</ATTRIBUTE_VALUE>
                        <CONFIDENCE>50</CONFIDENCE>
                    </APPLICATION>
                </SERVICE>
            </SERVICES>
            <CLIENTS>
                <CLIENT>
                    <IPPROTO>
                        <ATTRIBUTE_VALUE>tcp</ATTRIBUTE_VALUE>
                        <CONFIDENCE>100</CONFIDENCE>
                    </IPPROTO>
                    <PROTOCOL>
                        <ATTRIBUTE_VALUE>http</ATTRIBUTE_VALUE>
                        <CONFIDENCE>91</CONFIDENCE>
                    </PROTOCOL>
                    <APPLICATION>
                        <ATTRIBUTE_VALUE>IE Http Browser</ATTRIBUTE_VALUE>
                        <CONFIDENCE>90</CONFIDENCE>
                        <VERSION>
                            <ATTRIBUTE_VALUE>6.0</ATTRIBUTE_VALUE>
                            <CONFIDENCE>89</CONFIDENCE>
                        </VERSION>
                    </APPLICATION>
                </CLIENT>
            </CLIENTS>
        </HOST>
    </ATTRIBUTE_TABLE>
</SNORT_ATTRIBUTES>
\end{verbatim}

\begin{note}

With Snort 2.8.1, for a given host entry, the stream and IP frag information
are both used.  Of the service attributes, only the IP protocol (tcp, udp,
etc), port, and protocol (http, ssh, etc) are used.  The application and
version for a given service attribute, and any client attributes are ignored.
They may be used in a future release.

\end{note}

A DTD for verification of the Host Attribute Table XML file is provided with
the snort packages.

The confidence metric may be used to indicate the validity of a given service
or client application and its respective elements.  That field is not
currently used by Snort, but may be in future releases.

\subsection{Attribute Table Example}

In the example above, a host running Red Hat 2.6 is described. This host has
an IP address of 192.168.1.234.  On that host, TCP port 22 is ssh (running
Open SSH), and TCP port 2300 is telnet.

The IP stack fragmentation and stream reassembly is mimicked by the "linux"
configuration (see sections \ref{frag3 section} and \ref{stream5 section}).

\subsubsection{Attribute Table Affect on preprocessors}

\begin{itemize}
\item{Network Layer Preprocessors}

Each of the network layer preprocessors (frag3 and stream5) make use of the
respective \texttt{FRAG\_POLICY} and \texttt{STREAM\_POLICY} in terms of
how data is handled for reassembly for packets being received by that host.

\item{Application Layer Preprocessors}

The application layer preprocessors (HTTP, SMTP, FTP, Telnet, etc) make
use of the \texttt{SERVICE} information for connections destined to that
host on that port.  

For example, even if the telnet portion of the FTP/Telnet preprocessor is
only configured to inspect port 23, Snort will inspect packets for a connection
to 192.168.1.234 port 2300 as telnet.

Conversely, if, for example, HTTP Inspect is configured to inspect traffic
on port 2300, HTTP Inspect will NOT process the packets on a connection
to 192.168.1.234 port 2300 because it is identified as telnet.

Below is a list of the common services used by Snort's application layer
preprocessors and Snort rules (see below).

\begin{table}[h]
\label{attribute:service names}
\begin{center}
\begin{tabular}{| l | l | l | l | l | l | l |}
\hline 
http & ftp & ftp-data & telnet & smtp & ssh & tftp \\
\hline 
dcerpc & netbios-dgm & netbios-ns & netbios-ssn & nntp & finger & sunrpc \\
\hline 
dns & isakmp & mysql & oracle & cvs & shell & x11 \\
\hline 
imap & pop2 & pop3 & snmp & & & \\
\hline 
\end{tabular}
\end{center}
\end{table}
\end{itemize}

\subsubsection{Attribute Table Affect on rules}

Similar to the application layer preprocessors, rules configured for specific
ports that have a service metadata will be processed based on the service
identified by the attribute table.

When both service metadata is present in the rule and in the connection, Snort uses
the service rather than the port.  If there are rules that use the service
and other rules that do not but the port matches, Snort will ONLY inspect the rules
that have the service that matches the connection.

The following few scenarios identify whether a rule will be inspected or not.

\begin{itemize}
\item{Alert: Rule Has Service Metadata, Connection Service Matches}

The following rule will be inspected and alert on traffic to host 192.168.1.234
port 2300 because it is identified as telnet.

\begin{verbatim}
alert tcp any any -> any 23 (msg:"Telnet traffic"; flow:to_server,established;
sid:10000001; metadata: service telnet;)
\end{verbatim}

\item{Alert: Rule Has Multiple Service Metadata, Connection Service Matches One of them}

The following rule will be inspected and alert on traffic to host 192.168.1.234
port 2300 because it is identified as telnet.

\begin{verbatim}
alert tcp any any -> any 23 (msg:"Telnet traffic"; flow:to_server,established;
sid:10000002; metadata: service telnet, service smtp;)
\end{verbatim}

\item{No Alert: Rule Has Service Metadata, Connection Service Does Not Match, Port Matches}

The following rule will NOT be inspected and NOT alert on traffic to host 192.168.1.234
port 2300 because that traffic is identified as telnet, but the service is ssh.

\begin{verbatim}
alert tcp any any -> any 2300 (msg:"SSH traffic"; flow:to_server,established;
sid:10000003; metadata: service ssh;)
\end{verbatim}

\item{Alert: Rule Has No Service Metadata, Port Matches}

The following rule will be inspected and alert on traffic to host 192.168.1.234
port 2300 because the port matches.

\begin{verbatim}
alert tcp any any -> any 2300 (msg:"Port 2300 traffic"; flow:to_server,established;
sid:10000004;)
\end{verbatim}

\item{Alert: Rule Has No Service Metadata, Packet has service + other rules with service}

The first rule will NOT be inspected and NOT alert on traffic to host 192.168.1.234
port 2300 because the service is identified as telnet and there are other rules with
that service.

\begin{verbatim}
alert tcp any any -> any 2300 (msg:"Port 2300 traffic"; flow:to_server,established;
sid:10000005;)
alert tcp any any -> any 2300 (msg:"Port 2300 traffic"; flow:to_server,established;
sid:10000006; metadata: service telnet;)
\end{verbatim}

\item{No Alert: Rule Has No Service Metadata, Port Does Not Match}

The following rule will NOT be inspected and NOT alert on traffic to host 192.168.1.234
port 2300 because the port does not match.

\begin{verbatim}
alert tcp any any -> any 23 (msg:"Port 23 traffic"; flow:to_server,established;
sid:10000007;)
\end{verbatim}

\end{itemize}

\section{Dynamic Modules}

Dynamically loadable modules were introduced with Snort 2.6.  They can be
loaded via directives in \texttt{snort.conf} or via command-line options.

\subsection{Format}

\begin{verbatim}
    <directive> <parameters>
\end{verbatim}

\subsection{Directives}

\begin{longtable}{| p{2in} | p{4in} |}
\hline
{\bf Syntax} & {\bf Description}\\
\hline

\hline
\texttt{dynamicpreprocessor $[$ file $<$shared library path$>$ $|$ directory
$<$directory of shared libraries$>$ $]$} &

Tells snort to load the dynamic preprocessor shared library (if file is used)
or all dynamic preprocessor shared libraries (if directory is used).  Specify
\texttt{file}, followed by the full or relative path to the shared library.  Or,
specify \texttt{directory}, followed by the full or relative path to a directory of
preprocessor shared libraries.  (Same effect as
\texttt{--dynamic-preprocessor-lib} or \texttt{--dynamic-preprocessor-lib-dir}
options).  See chapter \ref{Dynamic Modules} for more information on dynamic
preprocessor libraries.\\

\hline
\texttt{dynamicengine $[$ file $<$shared library path$>$ $|$ directory
$<$directory of shared libraries$>$ $]$} &

Tells snort to load the dynamic engine shared library (if file is used) or all
dynamic engine shared libraries (if directory is used).  Specify \texttt{file},
followed by the full or relative path to the shared library.  Or, specify
\texttt{directory}, followed by the full or relative path to a directory of
preprocessor shared libraries.  (Same effect as \texttt{--dynamic-engine-lib}
or \texttt{--dynamic-preprocessor-lib-dir} options).  See chapter \ref{Dynamic
Modules} for more information on dynamic engine libraries.\\

\hline
\texttt{dynamicdetection $[$ file $<$shared library path$>$ $|$ directory
$<$directory of shared libraries$>$ $]$} &

Tells snort to load the dynamic detection rules shared library (if file is
used) or all dynamic detection rules shared libraries (if directory is used).
Specify \texttt{file}, followed by the full or relative path to the shared library.
Or, specify \texttt{directory}, followed by the full or relative path to a directory
of detection rules shared libraries.  (Same effect as
\texttt{--dynamic-detection-lib} or \texttt{--dynamic-detection-lib-dir}
options).  See chapter \ref{Dynamic Modules} for more information on dynamic
detection rules libraries.\\

\hline
\end{longtable}


\section{Reloading a Snort Configuration}

Snort now supports reloading a configuration in lieu of restarting Snort in
so as to provide seamless traffic inspection during a configuration change.
A separate thread will parse and create a swappable configuration object while
the main Snort packet processing thread continues inspecting traffic under the
current configuration.  When a swappable configuration object is ready for use,
the main Snort packet processing thread will swap in the new configuration to
use and will continue processing under the new configuration.  Note that for
some preprocessors, existing session data will continue to use the configuration
under which they were created in order to continue with proper state for that
session.  All newly created sessions will, however, use the new configuration.


\subsection{Enabling support}
\label{reload:enable}
To enable support for reloading a configuration, add \texttt{--enable-reload} to
configure when compiling.

There is also an ancillary option that determines how Snort should behave
if any non-reloadable options are changed (see section \ref{reload:nonreloadable} below).
This option is enabled by default and the behavior is for
Snort to restart if any non-reloadable options are added/modified/removed.
To disable this behavior and have Snort exit instead of restart, add
\texttt{--disable-reload-error-restart} in addition to \texttt{--enable-reload} to configure
when compiling.

\begin{note}
This functionality is not currently supported in Windows.

Caveat : When Snort is run on the primary network interface of an OpenBSD system, the reload and failopen operations may not function as expected.

\end{note}


\subsection{Reloading a configuration}
\label{reload:reload}
First modify your snort.conf (the file passed to the \texttt{-c} option on the
command line).

Then, to initiate a reload, send Snort a \texttt{SIGHUP} signal, e.g.

\begin{verbatim}
$ kill -SIGHUP <snort pid>
\end{verbatim}

\begin{note}
If reload support is not enabled, Snort will restart (as it always has)
upon receipt of a SIGHUP.
\end{note}

\begin{note}
An invalid configuration will still result in a fatal error, so
you should test your new configuration before issuing a reload, e.g.
\texttt{\$ snort -c snort.conf -T}
\end{note}


\subsection{Non-reloadable configuration options}
\label{reload:nonreloadable}
There are a number of option changes that are currently non-reloadable because
they require changes to output, startup memory allocations, etc.  Modifying any
of these options will cause Snort to restart (as a \texttt{SIGHUP} previously did) or
exit (if \texttt{--disable-reload-error-restart} was used to configure Snort).

Reloadable configuration options of note:
\begin{itemize}
\item Adding/modifying/removing text rules and variables are reloadable.
\item Adding/modifying/removing preprocessor configurations are reloadable (except
  as noted below).
\end{itemize}

Non-reloadable configuration options of note:
\begin{itemize}
\item Adding/modifying/removing shared objects via dynamicdetection, dynamicengine
  and dynamicpreprocessor are not reloadable, i.e. any new/modified/removed
  shared objects will require a restart.
\item Any changes to output will require a restart.
\end{itemize}

Changes to the following options are not reloadable:

\begin{verbatim}
attribute_table
config alertfile
config asn1
config chroot
config daemon
config detection_filter
config flowbits_size
config interface
config logdir
config max_attribute_hosts
config max_attribute_services_per_host
config nolog
config no_promisc
config pkt_count
config rate_filter
config response
config set_gid
config set_uid
config snaplen
config threshold
dynamicdetection
dynamicengine
dynamicpreprocessor
output
\end{verbatim}

In certain cases, only some of the parameters to a config option or
preprocessor configuration are not reloadable.  Those parameters are
listed below the relevant config option or preprocessor.

\begin{verbatim}
config ppm: max-rule-time <int>
 rule-log
config profile_rules
 filename
 print
 sort
config profile_preprocs
 filename
 print
 sort
preprocessor dcerpc2
 memcap
preprocessor frag3_global
 max_frags
 memcap
 prealloc_frags
 prealloc_memcap
 disabled
preprocessor perfmonitor
 file
 snortfile
preprocessor sfportscan
 memcap
 logfile
 disabled
preprocessor stream5_global
 memcap
 max_tcp
 max_udp
 max_icmp
 track_tcp
 track_udp
 track_icmp
\end{verbatim}

\section{Multiple Configurations}

Snort now supports multiple configurations based on VLAN Id or IP subnet within 
a single instance of Snort. This will allow administrators to specify multiple 
snort configuration files and bind each configuration to one or more VLANs or 
subnets rather than running one Snort for each configuration required. Each 
unique snort configuration file will create a new configuration instance within 
snort. VLANs/Subnets not bound to any specific configuration will use the default 
configuration. Each configuration can have different preprocessor settings and 
detection rules.

\subsection{Creating Multiple Configurations}
Default configuration for snort is specified using the existing -c option. A 
default configuration binds multiple vlans or networks to non-default configurations, 
using the following configuration line:

\begin{verbatim}
config binding: <path_to_snort.conf> vlan <vlanIdList>
config binding: <path_to_snort.conf> net <ipList>
config binding: <path_to_snort.conf> policy_id <id>
\end{verbatim}

\begin{description}{}

\item [\texttt{path\_to\_snort.conf}] - Refers to the absolute or relative path to 
the snort.conf for specific configuration.

\item [\texttt{vlanIdList}] - Refers to the comma separated list of vlandIds and 
vlanId ranges. The format for ranges is two vlanId separated by a "-". Spaces are 
allowed within ranges. Valid vlanId is any number in 0-4095 range. Negative vland 
Ids and alphanumeric are not supported.

\item [\texttt{ipList}] - Refers to ip subnets. Subnets can be CIDR blocks for 
IPV6 or IPv4.  A maximum of 512 individual IPv4 or IPv6 addresses or CIDRs can be
specified.

\item [\texttt{policy\_id}] - Refers to the specific policyi\_id to be applied. Valid policyi\_id is any number in 0-4095 range.
\end{description}

\begin{note}
Vlan and Subnets can not be used in the same line. Configurations can be applied 
based on either Vlans or Subnets not both.
\end{note}

\begin{note}
Even though Vlan Ids 0 and 4095 are reserved, they are included as valid in terms
of configuring Snort.
\end{note}

\subsection{Configuration Specific Elements}

\subsubsection{Config Options}
Generally config options defined within the default configuration are global by 
default i.e. their value applies to all other configurations. The following config 
options are specific to each configuration.

\begin{verbatim}
policy_id
policy_mode
policy_version
\end{verbatim}

The following config options are specific to each configuration. If not defined in 
a configuration, the default values of the option (not the default configuration 
values) take effect.

\begin{verbatim}
config checksum_drop
config disable_decode_alerts
config disable_decode_drops
config disable_ipopt_alerts
config disable_ipopt_drops
config disable_tcpopt_alerts
config disable_tcpopt_drops
config disable_tcpopt_experimental_alerts
config disable_tcpopt_experimental_drops
config disable_tcpopt_obsolete_alerts
config disable_tcpopt_obsolete_drops
config disable_ttcp_alerts
config disable_tcpopt_ttcp_alerts
config disable_ttcp_drops
\end{verbatim}

\subsubsection{Rules}
Rules are specific to configurations but only some parts of a rule can be customized 
for performance reasons. If a rule is not specified in a configuration then the rule 
will never raise an event for the configuration. A rule shares all parts of the rule 
options, including the general options, payload detection options, non-payload detection 
options, and post-detection options.  Parts of the rule header can be specified differently 
across configurations, limited to:

\begin{verbatim}
Source IP address and port
Destination IP address and port
Action
\end{verbatim}

A higher revision of a rule in one configuration will override other revisions of 
the same rule in other configurations.

\subsubsection{Variables}
Variables defined using "var", "portvar" and "ipvar" are specific to configurations. 
If the rules in a configuration use variables, those variables must be defined in 
that configuration. 

\subsubsection{Preprocessors}
Preprocessors configurations can be defined within each vlan or subnet specific 
configuration. Options controlling specific preprocessor memory usage, through specific 
limit on memory usage or number of instances, are processed only in default policy. 
The options control total memory usage for a preprocessor across all policies. These 
options are ignored in non-default policies without raising an error. A preprocessor 
must be configured in default configuration before it can be configured in non-default 
configuration. This is required as some mandatory preprocessor configuration options 
are processed only in default configuration.

\subsubsection{Events and Output}
An unique policy id can be assigned by user, to each configuration using the following 
config line:

\begin{verbatim}
config policy_id: <id>
\end{verbatim}

\begin{description}{}
\item [\texttt{id}] - Refers to a 16-bit unsigned value. This policy id will be used to 
identify alerts from a specific configuration in the unified2 records.
\end{description}

\begin{note}
If no policy id is specified, snort assigns 0 (zero) value to the configuration.
\end{note}

To enable vlanId logging in unified2 records the following option can be used.

\begin{verbatim}
output alert_unified2: vlan_event_types (alert logging only)
output unified2: filename <filename>, vlan_event_types (true unified logging)
\end{verbatim}

\begin{description}{}
\item [\texttt{filename}] - Refers to the absolute or relative filename.
\item [\texttt{vlan\_event\_types}] - When this option is set, snort will use unified2 event 
type 104 and 105 for IPv4 and IPv6 respectively. 
\end{description}

\begin{note}
Each event logged will have the vlanId from the packet if vlan headers are present 
otherwise 0 will be used.
\end{note}

\subsection{How Configuration is applied?}
Snort assigns every incoming packet to a unique configuration based on the following criteria. 
If VLANID is present, then the innermost VLANID is used to find bound configuration. If the 
bound configuration is the default configuration, then destination IP address is searched to 
the most specific subnet that is bound to a non-default  configuration. The packet is assigned 
non-default configuration if found otherwise the check is repeated using source IP address. 
In the end, default configuration is used if no other matching configuration is found.

For addressed based configuration binding, this can lead to conflicts between configurations 
if source address is bound to one configuration and destination address is bound to another. 
In this case, snort will use the first configuration in the order of definition, that can be 
applied to the packet. 

\section{Active Response}

Snort 2.9 includes a number of changes to better handle inline operation,
including:

\begin{itemize}
\item a single mechanism for all responses
\item fully encoded reset or icmp unreachable packets
\item updated flexible response rule option
\item updated react rule option
\item added block and sblock rule actions
\end{itemize}

These changes are outlined below.

\subsection{Enabling Active Response}

This enables active responses (snort will send TCP RST or ICMP
unreachable/port) when dropping a session.

\begin{verbatim}
    ./configure --enable-active-response / -DACTIVE_RESPONSE

    preprocessor stream5_global: \
        max_active_responses <max_rsp>, \
        min_response_seconds <min_sec> 

    <max_rsp> ::= (0..25)
    <min_sec> ::= (1..300)
\end{verbatim}

Active responses will be encoded based on the triggering packet.  TTL will be
set to the value captured at session pickup.

\subsection{Configure Sniping}

Configure the number of attempts to land a TCP RST within the session's current
window (so that it is accepted by the receiving TCP).  This sequence "strafing"
is really only useful in passive mode.  In inline mode the reset is put
straight into the stream in lieu of the triggering packet so strafing is not
necessary.

Each attempt (sent in rapid succession) has a different sequence number.  Each
active response will actually cause this number of TCP resets to be sent.  TCP
data (sent for react) is multiplied similarly.  At most 1 ICMP unreachable is
sent, if and only if attempts $>$ 0.

\begin{verbatim}
    ./configure --enable-active-response

    config response: [device <dev>] [dst_mac <MAC address>] attempts <att>

    <dev> ::= ip | eth0 | etc.
    <att> ::= (1..20)
    <MAC address> ::= nn:nn:nn:nn:nn:nn    
     (n is a hex number from 0-F)
\end{verbatim}

device ip will perform network layer injection.  It is probably a better choice
to specify an interface and avoid kernel routing tables, etc.

dst\_mac will change response destination MAC address, if the device is eth0, eth1, eth2 etc.
Otherwise, response destination MAC address is derived from packet.
Example:
\begin{verbatim}
    config response: device eth0 dst_mac 00:06:76:DD:5F:E3 attempts 2 
\end{verbatim}

\subsection{Flexresp}
\label{resp section}

Flexresp and flexresp2 are replaced with flexresp3.

* Flexresp is deleted; these features are no longer available:

\begin{verbatim}
    ./configure --enable-flexresp / -DENABLE_RESPOND -DENABLE_RESPONSE
    config flexresp: attempts 1
\end{verbatim}

* Flexresp2 is deleted; these features are deprecated, non-functional, and will
  be deleted in a future release:

\begin{verbatim}
    ./configure --enable-flexresp2 / -DENABLE_RESPOND -DENABLE_RESPONSE2

    config flexresp2_interface: eth0
    config flexresp2_attempts: 4
    config flexresp2_memcap: 1000000
    config flexresp2_rows: 1000
\end{verbatim}

* Flexresp3 is new: the resp rule option keyword is used to configure active
  responses for rules that fire.

\begin{verbatim}
    ./configure --enable-flexresp3 / -DENABLE_RESPOND -DENABLE_RESPONSE3

    alert tcp any any -> any 80 (content:"a"; resp:<resp_t>; sid:1;)
\end{verbatim}

* \texttt{resp\_t} includes all flexresp and flexresp2 options:

\begin{verbatim}
    <resp_t> ::= \
        rst_snd | rst_rcv | rst_all | \
        reset_source | reset_dest | reset_both | icmp_net | \
        icmp_host | icmp_port | icmp_all
\end{verbatim}

\subsection{React}
\label{react section}

react is a rule option keyword that enables sending an HTML page on a session
and then resetting it.  This is built with:

\begin{verbatim}
    ./configure --enable-react / -DENABLE_REACT
\end{verbatim}

The page to be sent can be read from a file:

\begin{verbatim}
    config react: <block.html>
\end{verbatim}

or else the default is used:

\begin{verbatim}
    <default_page> ::= \
        "HTTP/1.1 403 Forbidden\r\n"
        "Connection: close\r\n"
        "Content-Type: text/html; charset=utf-8\r\n"
        "\r\n"
        "<!DOCTYPE html PUBLIC \"-//W3C//DTD XHTML 1.1//EN\"\r\n" \
        "    \"http://www.w3.org/TR/xhtml11/DTD/xhtml11.dtd\">\r\n" \
        "<html xmlns=\"http://www.w3.org/1999/xhtml\" xml:lang=\"en\">\r\n" \
        "<head>\r\n" \
        "<meta http-equiv=\"Content-Type\" content=\"text/html; charset=UTF-8\" />\r\n" \
        "<title>Access Denied</title>\r\n" \
        "</head>\r\n" \
        "<body>\r\n" \
        "<h1>Access Denied</h1>\r\n" \
        "<p>%s</p>\r\n" \
        "</body>\r\n" \
        "</html>\r\n";
\end{verbatim}

Note that the file must contain the entire response, including any HTTP headers.
In fact, the response isn't strictly limited to HTTP.  You could craft a binary
payload of arbitrary content.

Be aware of size when creating such responses. While it may be possible
to respond with arbitrarily large responses, responses for TCP sessions will need
to take into account that the receiver's window may only accept up to a certain
amount of data. Sending past this limit will result in truncated data. In general,
the smaller the response, the more likely it will be successful.

When the rule is configured, the page is loaded and the %s is replaced with the
selected message, which defaults to:

\begin{verbatim}
    <default_msg> ::= \
        "You are attempting to access a forbidden site.<br />" \
        "Consult your system administrator for details.";
\end{verbatim}

Additional formatting operators beyond a single %s are prohibited, including
%d, %x, %s, as well as any URL encodings such as as %20 (space) that may be
within a reference URL.

This is an example rule:

\begin{verbatim}
    drop tcp any any -> any $HTTP_PORTS ( \
        content: "d"; msg:"Unauthorized Access Prohibited!"; \
        react: <react_opts>; sid:4;)

    <react_opts> ::= [msg] [, <dep_opts>]
\end{verbatim}

These options are deprecated:

\begin{verbatim}
    <dep_opts> ::= [block|warn], [proxy <port#>]
\end{verbatim}

The original version sent the web page to one end of the session only if the
other end of the session was port 80 or the optional proxy port.  The new
version always sends the page to the client.  If no page should be sent, a resp
option can be used instead.  The deprecated options are ignored.

\subsection{Rule Actions}

The block and sblock actions have been introduced as synonyms for drop and
sdrop to help avoid confusion between packets dropped due to load (e.g. lack of
available buffers for incoming packets) and packets blocked due to Snort's
analysis.


\chapter{Writing Snort Rules}
\label{Writing Snort Rules}

\section{The Basics}

Snort uses a simple, lightweight rules description language that is flexible
and quite powerful. There are a number of simple guidelines to remember when
developing Snort rules that will help safeguard your sanity.

Most Snort rules are written in a single line. This was required in versions
prior to 1.8. In current versions of Snort, rules may span multiple lines by
adding a backslash \textbackslash{} to the end of the line. 

Snort rules are divided into two logical sections, the rule header and the rule
options. The rule header contains the rule's action, protocol, source and
destination IP addresses and netmasks, and the source and destination ports
information. The rule option section contains alert messages and information on
which parts of the packet should be inspected to determine if the rule action
should be taken.

Figure \ref{Sample Snort Rule} illustrates a sample Snort rule.

\begin{center}
\begin{figure}
\begin{verbatim}
    alert tcp any any -> 192.168.1.0/24 111 \
        (content:"|00 01 86 a5|"; msg:"mountd access";)
\end{verbatim}

\caption{Sample Snort Rule}
\label{Sample Snort Rule}
\end{figure}
\end{center}

The text up to the first parenthesis is the rule header and the section
enclosed in parenthesis contains the rule options. The words before the colons
in the rule options section are called option \emph{keywords}. 

\begin{note}

Note that the rule options section is not specifically required by any rule,
they are just used for the sake of making tighter definitions of packets to
collect or alert on (or drop, for that matter). 

\end{note}

All of the elements in that make up a rule must be true for the indicated rule
action to be taken. When taken together, the elements can be considered to form
a logical \textsc{and} statement. At the same time, the various rules in a
Snort rules library file can be considered to form a large logical \textsc{or}
statement. 

\section{Rules Headers}

\subsection{Rule Actions}
\label{rules action section}

The rule header contains the information that defines the who, where, and what
of a packet, as well as what to do in the event that a packet with all the
attributes indicated in the rule should show up. The first item in a rule is
the rule action. The rule action tells Snort what to do when it finds a packet
that matches the rule criteria. There are 5 available default actions in Snort,
alert, log, pass, activate, and dynamic. In addition, if you are running Snort
in inline mode, you have additional options which include drop, reject, and
sdrop. 

\begin{enumerate}

\item alert - generate an alert using the selected alert method, and then
log the packet 

\item log - log the packet 

\item pass - ignore the packet 

\item activate - alert and then turn on another dynamic rule  

\item dynamic - remain idle until activated by an activate rule , then act as a
log rule

\item drop - block and log the packet

\item reject - block the packet, log it, and then send a TCP reset
if the protocol is TCP or an ICMP port unreachable message if the protocol is
UDP.

\item sdrop - block the packet but do not log it.

\end{enumerate}

You can also define your own rule types and associate one or more output
plugins with them. You can then use the rule types as actions in Snort rules.

This example will create a type that will log to just tcpdump:

\begin{verbatim}
    ruletype suspicious
    {
        type log 
        output log_tcpdump: suspicious.log
    }
\end{verbatim}

This example will create a rule type that will log to syslog and tcpdump:
database:

\begin{verbatim}
    ruletype redalert
    {
          type alert 
          output alert_syslog: LOG_AUTH LOG_ALERT 
          output log_tcpdump: suspicious.log
    }
\end{verbatim}

\subsection{Protocols}

The next field in a rule is the protocol. There are four protocols that Snort
currently analyzes for suspicious behavior -- TCP, UDP, ICMP, and IP. In the
future there may be more, such as ARP, IGRP, GRE, OSPF, RIP, IPX, etc.

\subsection{IP Addresses}

The next portion of the rule header deals with the IP address and port
information for a given rule. The keyword any may be used to define any
address. Snort does not have a mechanism to provide host name lookup for the IP
address fields in the config file. The addresses are formed by a straight
numeric IP address and a CIDR\cite{cidrnotation} block. The CIDR block
indicates the netmask that should be applied to the rule's address and any
incoming packets that are tested against the rule. A CIDR block mask of /24
indicates a Class C network, /16 a Class B network, and /32 indicates a
specific machine address. For example, the address/CIDR combination
192.168.1.0/24 would signify the block of addresses from 192.168.1.1 to
192.168.1.255.  Any rule that used this designation for, say, the destination
address would match on any address in that range. The CIDR designations give us
a nice short-hand way to designate large address spaces with just a few
characters.

In Figure \ref{Sample Snort Rule}, the source IP address was set to match for
any computer talking, and the destination address was set to match on the
192.168.1.0 Class C network.

There is an operator that can be applied to IP addresses, the negation
operator. This operator tells Snort to match any IP address except the one
indicated by the listed IP address. The negation operator is indicated with a
!. For example, an easy modification to the initial example is to make it alert
on any traffic that originates outside of the local net with the negation
operator as shown in Figure \ref{Example Negation}.

\begin{center}
\begin{figure}
\begin{verbatim}
    alert tcp !192.168.1.0/24 any -> 192.168.1.0/24 111 \
        (content:"|00 01 86 a5|"; msg:"external mountd access";)
\end{verbatim}

\caption{\label{Example Negation} Example IP Address Negation Rule}
\end{figure}
\end{center}

This rule's IP addresses indicate any tcp packet with a source IP address not
originating from the internal network and a destination address on the internal
network.

You may also specify lists of IP addresses. An IP list is specified by
enclosing a comma separated list of IP addresses and CIDR blocks within square
brackets. For the time being, the IP list may not include spaces between the
addresses. See Figure \ref{IP list usage} for an example of an IP list in
action.

\begin{center}
\begin{figure}
\begin{verbatim}
    alert tcp ![192.168.1.0/24,10.1.1.0/24] any -> \
        [192.168.1.0/24,10.1.1.0/24] 111 (content:"|00 01 86 a5|"; \
        msg:"external mountd access";)
\end{verbatim}

\caption{\label{IP list usage}IP Address Lists}
\end{figure}
\end{center}

\subsection{Port Numbers}

Port numbers may be specified in a number of ways, including any ports, static
port definitions, ranges, and by negation. Any ports are a wildcard value,
meaning literally any port. Static ports are indicated by a single port number,
such as 111 for portmapper, 23 for telnet, or 80 for http, etc. Port ranges are
indicated with the range operator :. The range operator may be applied in a
number of ways to take on different meanings, such as in Figure \ref{port range
examples}.

\begin{center}
\begin{figure}
\begin{verbatim}
    log udp any any -> 192.168.1.0/24 1:1024
\end{verbatim}

log udp traffic coming from any port and destination ports ranging from 1 to 1024

\begin{verbatim}
    log tcp any any -> 192.168.1.0/24 :6000 
\end{verbatim}

log tcp traffic from any port going to ports less than or equal to 6000

\begin{verbatim}
    log tcp any :1024 -> 192.168.1.0/24 500: 
\end{verbatim}

log tcp traffic from privileged ports less than or equal to 1024 going to ports
greater than or equal to 500

\caption{\label{port range examples}Port Range Examples}
\end{figure}
\end{center}

Port negation is indicated by using the negation operator !.  The negation
operator may be applied against any of the other rule types (except any, which
would translate to none, how Zen...). For example, if for some twisted reason
you wanted to log everything except the X Windows ports, you could do something
like the rule in Figure \ref{example port negation}.

\begin{figure}
\begin{verbatim}
    log tcp any any -> 192.168.1.0/24 !6000:6010
\end{verbatim}

\caption{\label{example port negation}Example of Port Negation}
\end{figure}

\subsection{The Direction Operator}

The direction operator -$>$ indicates the orientation, or direction, of the
traffic that the rule applies to. The IP address and port numbers on the left
side of the direction operator is considered to be the traffic coming from the
source host, and the address and port information on the right side of the
operator is the destination host. There is also a bidirectional operator, which
is indicated with a $<>$ symbol. This tells Snort to consider the address/port
pairs in either the source or destination orientation. This is handy for
recording/analyzing both sides of a conversation, such as telnet or POP3
sessions. An example of the bidirectional operator being used to record both
sides of a telnet session is shown in Figure \ref{bidirectional operator}.

Also, note that there is no $<$- operator. In Snort versions before 1.8.7, the
direction operator did not have proper error checking and many people used an
invalid token. The reason the $<$- does not exist is so that rules always read
consistently.

\begin{figure}
\begin{verbatim}
    log tcp !192.168.1.0/24 any <> 192.168.1.0/24 23
\end{verbatim}

\caption{\label{bidirectional operator}Snort rules using the Bidirectional
Operator}
\end{figure}

\subsection{Activate/Dynamic Rules}
\label{dynamic rules}

\begin{note}

Activate and Dynamic rules are being phased out in favor of a combination of
tagging (\ref{tag section}) and flowbits (\ref{flowbits}). 

\end{note}

Activate/dynamic rule pairs give Snort a powerful capability. You can now have
one rule activate another when its action is performed for a set number of
packets. This is very useful if you want to set Snort up to perform follow on
recording when a specific rule goes off. Activate rules act just like alert
rules, except they have a {*}required{*} option field: activates.  Dynamic
rules act just like log rules, but they have a different option field:
activated\_by. Dynamic rules have a second required field as well, count. 

Activate rules are just like alerts but also tell Snort to add a rule when a
specific network event occurs. Dynamic rules are just like log rules except are
dynamically enabled when the activate rule id goes off. 

Put 'em together and they look like Figure \ref{activate/dynamic rule example}.

\begin{figure}
\begin{verbatim}
    activate tcp !$HOME_NET any -> $HOME_NET 143 (flags:PA; \
        content:"|E8C0FFFFFF|/bin"; activates:1;  \
        msg:"IMAP buffer overflow!";)
    dynamic tcp !$HOME_NET any -> $HOME_NET 143 (activated_by:1; count:50;)
\end{verbatim}

\caption{Activate/Dynamic Rule Example}
\label{activate/dynamic rule example}
\end{figure}

These rules tell Snort to alert when it detects an IMAP buffer overflow and
collect the next 50 packets headed for port 143 coming from outside \$HOME\_NET
headed to \$HOME\_NET. If the buffer overflow happened and was successful,
there's a very good possibility that useful data will be contained within the
next 50 (or whatever) packets going to that same service port on the network,
so there's value in collecting those packets for later analysis.

\section{Rule Options}

Rule options form the heart of Snort's intrusion detection engine, combining
ease of use with power and flexibility. All Snort rule options are separated
from each other using the semicolon (;) character. Rule option keywords are
separated from their arguments with a colon (:) character. 

There are four major categories of rule options.  

\begin{description}

\item [general] These options provide information about the rule but do not
have any affect during detection 

\item [payload] These options all look for data inside the packet payload and
can be inter-related

\item [non-payload] These options look for non-payload data

\item [post-detection] These options are rule specific triggers that happen
after a rule has ``fired.''

\end{description}

\section{General Rule Options}

\subsection{msg}

The msg rule option tells the logging and alerting engine the message to print
along with a packet dump or to an alert. It is a simple text string that
utilizes the \textbackslash{} as an escape character to indicate a discrete
character that might otherwise confuse Snort's rules parser (such as the
semi-colon ; character).

\subsubsection{Format}

\begin{verbatim}
    msg:"<message text>";
\end{verbatim}

\subsection{reference}

The reference keyword allows rules to include references to external attack
identification systems. The plugin currently supports several specific systems
as well as unique URLs. This plugin is to be used by output plugins to provide
a link to additional information about the alert produced.

Make sure to also take a look at
\url{http://www.snort.org/pub-bin/sigs-search.cgi/} for a system that is
indexing descriptions of alerts based on of the sid (See Section \ref{keyword
sid}).

\begin{table}[h]
\begin{center}
\caption{Supported Systems}
\label{references systems}
\begin{tabular}{|c|c|}

\hline 
System&
URL Prefix\\
\hline

\hline 
bugtraq&
http://www.securityfocus.com/bid/\\

\hline 
cve&
http://cve.mitre.org/cgi-bin/cvename.cgi?name=\\

\hline 
nessus &
http://cgi.nessus.org/plugins/dump.php3?id=\\

\hline 
arachnids&
(currently down) http://www.whitehats.com/info/IDS\\

\hline 
mcafee&
http://vil.nai.com/vil/content/v\_\\

\hline 
osvdb&
http://osvdb.org/show/osvdb/\\

\hline 
msb&
http://technet.microsoft.com/en-us/security/bulletin/\\

\hline 
url&
http://\\

\hline
\end{tabular}
\end{center}
\end{table}

\subsubsection{Format}

\begin{verbatim}
    reference:<id system>, <id>; [reference:<id system>, <id>;]
\end{verbatim}

\subsubsection{Examples}

\begin{verbatim}
    alert tcp any any -> any 7070 (msg:"IDS411/dos-realaudio"; \
        flags:AP; content:"|fff4 fffd 06|"; reference:arachnids,IDS411;)
    
    alert tcp any any -> any 21 (msg:"IDS287/ftp-wuftp260-venglin-linux"; \
        flags:AP; content:"|31c031db 31c9b046 cd80 31c031db|"; \
        reference:arachnids,IDS287; reference:bugtraq,1387; \
        reference:cve,CAN-2000-1574;)
\end{verbatim}

\subsection{gid}
\label{keyword gid}

The \texttt{gid} keyword (generator id) is used to identify what part of Snort
generates the event when a particular rule fires.  For example gid 1 is
associated with the rules subsystem and various gids over 100 are designated
for specific preprocessors and the decoder.  See etc/generators in the source
tree for the current generator ids in use.  Note that the gid keyword is
optional and if it is not specified in a rule, it will default to 1 and the
rule will be part of the general rule subsystem.  To avoid potential conflict
with gids defined in Snort (that for some reason aren't noted it
etc/generators), it is recommended that values starting at 1,000,000 be used.
For general rule writing, it is not recommended that the \texttt{gid} keyword
be used.  This option should be used with the \texttt{sid} keyword.  (See
section \ref{keyword sid})

The file etc/gen-msg.map contains contains more information on preprocessor and
decoder gids.

\subsubsection{Format}

\begin{verbatim}
    gid:<generator id>;
\end{verbatim}

\subsubsection{Example}

This example is a rule with a generator id of 1000001. 

\begin{verbatim}
    alert tcp any any -> any 80 (content:"BOB"; gid:1000001; sid:1; rev:1;)
\end{verbatim}

\subsection{sid}
\label{keyword sid}

The \texttt{sid} keyword is used to uniquely identify Snort rules. This
information allows output plugins to identify rules easily.  This option should
be used with the \texttt{rev} keyword.  (See section \ref{keyword rev})

\begin{itemize}
\item $<$100 Reserved for future use
\item 100-999,999 Rules included with the Snort distribution
\item $>=$1,000,000 Used for local rules
\end{itemize}

The file sid-msg.map contains a mapping of alert messages to Snort rule IDs.
This information is useful when post-processing alert to map an ID to an alert
message.  

\subsubsection{Format}

\begin{verbatim}
    sid:<snort rules id>;
\end{verbatim}

\subsubsection{Example}

This example is a rule with the Snort Rule ID of 1000983.

\begin{verbatim}
    alert tcp any any -> any 80 (content:"BOB"; sid:1000983; rev:1;)
\end{verbatim}

\subsection{rev}
\label{keyword rev}

The \texttt{rev} keyword is used to uniquely identify revisions of Snort rules.
Revisions, along with Snort rule id's, allow signatures and descriptions to be
refined and replaced with updated information.  This option should be used with
the \texttt{sid} keyword.  (See section \ref{keyword sid})

\subsubsection{Format}

\begin{verbatim}
    rev:<revision integer>;
\end{verbatim}

\subsubsection{Example}

This example is a rule with the Snort Rule Revision of 1.

\begin{verbatim}
    alert tcp any any -> any 80 (content:"BOB"; sid:1000983; rev:1;)
\end{verbatim}

\subsection{classtype}

The \texttt{classtype} keyword is used to categorize a rule as detecting an
attack that is part of a more general type of attack class. Snort provides a
default set of attack classes that are used by the default set of rules it
provides. Defining classifications for rules provides a way to better organize
the event data Snort produces.

\subsubsection{Format}

\begin{verbatim}
    classtype:<class name>;
\end{verbatim}

\subsubsection{Example}

\begin{verbatim}
    alert tcp any any -> any 25 (msg:"SMTP expn root"; flags:A+; \
        content:"expn root"; nocase; classtype:attempted-recon;)
\end{verbatim}

Attack classifications defined by Snort reside in the
\texttt{classification.config} file. The file uses the following syntax:

\begin{verbatim}
    config classification:  <class name>,<class description>,<default priority>
\end{verbatim}

These attack classifications are listed in Table \ref{Snort Default
Classifications}. They are currently ordered with 4 default priorities. A
priority of 1 (high) is the most severe and 4 (very low) is the least severe.

\begin{center}
\begin{longtable}[h]{|p{2in}|p{2.5in}|c|}
\caption{Snort Default Classifications}
\label{Snort Default Classifications} \\
\hline 
Classtype & Description & Priority \\
\hline
\hline 
attempted-admin&
Attempted Administrator Privilege Gain & high \\
\hline 
attempted-user&
Attempted User Privilege Gain & high\\
\hline
inappropriate-content&
Inappropriate Content was Detected & high\\
\hline 
policy-violation&
Potential Corporate Privacy Violation & high\\
\hline
shellcode-detect&
Executable code was detected & high\\
\hline 
successful-admin&
Successful Administrator Privilege Gain & high\\
\hline 
successful-user&
Successful User Privilege Gain & high\\
\hline 
trojan-activity&
A Network Trojan was detected & high\\
\hline 
unsuccessful-user&
Unsuccessful User Privilege Gain & high\\
\hline 
web-application-attack&
Web Application Attack & high\\
\hline
attempted-dos&
Attempted Denial of Service & medium\\
\hline 
attempted-recon&
Attempted Information Leak & medium\\
\hline 
bad-unknown&
Potentially Bad Traffic & medium\\
\hline
default-login-attempt&
Attempt to login by a default username and password & medium\\
\hline 
denial-of-service&
Detection of a Denial of Service Attack & medium\\
\hline 
misc-attack&
Misc Attack & medium\\
\hline 
non-standard-protocol&
Detection of a non-standard protocol or event & medium\\
\hline 
rpc-portmap-decode&
Decode of an RPC Query & medium\\
\hline 
successful-dos&
Denial of Service & medium\\
\hline 
successful-recon-largescale&
Large Scale Information Leak & medium\\
\hline 
successful-recon-limited&
Information Leak & medium\\
\hline 
suspicious-filename-detect&
A suspicious filename was detected & medium\\
\hline 
suspicious-login&
An attempted login using a suspicious username was detected & medium\\
\hline 
system-call-detect&
A system call was detected & medium\\
\hline 
unusual-client-port-connection&
A client was using an unusual port & medium\\
\hline 
web-application-activity&
Access to a potentially vulnerable web application & medium\\
\hline
icmp-event&
Generic ICMP event & low\\
\hline 
misc-activity&
Misc activity & low\\
\hline 
network-scan&
Detection of a Network Scan & low\\
\hline 
not-suspicious&
Not Suspicious Traffic & low\\
\hline 
protocol-command-decode&
Generic Protocol Command Decode & low\\
\hline 
string-detect&
A suspicious string was detected & low\\
\hline 
unknown&
Unknown Traffic & low\\
\hline
tcp-connection&
A TCP connection was detected & very low\\
\hline
\end{longtable}
\end{center}

\subsubsection{Warnings}

The \texttt{classtype} option can only use classifications that have been
defined in \texttt{snort.conf} by using the \texttt{config classification}
option.  Snort provides a default set of classifications in
\texttt{classification.config} that are used by the rules it provides.

\subsection{priority}

The \texttt{priority} tag assigns a severity level to rules. A
\texttt{classtype} rule assigns a default priority (defined by the
\texttt{config classification} option) that may be overridden with a priority
rule.  Examples of each case are given below.

\subsubsection{Format}

\begin{verbatim}
    priority:<priority integer>;
\end{verbatim}

\subsubsection{Examples}

\begin{verbatim}
    alert tcp any any -> any 80 (msg:"WEB-MISC phf attempt"; flags:A+; \
        content:"/cgi-bin/phf"; priority:10;)

    alert tcp any any -> any 80 (msg:"EXPLOIT ntpdx overflow"; \ 
        dsize:>128; classtype:attempted-admin; priority:10 );
\end{verbatim}

\subsection{metadata}

The \texttt{metadata} tag allows a rule writer to embed additional information
about the rule, typically in a key-value format.  Certain metadata keys and
values have meaning to Snort and are listed in Table \ref{Snort Metadata Keys}.
Keys other than those listed in the table are effectively ignored by Snort and
can be free-form, with a key and a value.  Multiple keys are separated by a
comma, while keys and values are separated by a space.

\begin{table}[h]
\begin{center}
\caption{Snort Metadata Keys}
\label{Snort Metadata Keys}
\begin{tabular}{|p{1in}|p{2.5in}|c|}

\hline 
Key & Description & Value Format \\
\hline

\hline
\texttt{engine} &
Indicate a Shared Library Rule & "shared" \\

\hline 
\texttt{soid} &
Shared Library Rule Generator and SID & gid$|$sid \\

\hline 
\texttt{service} &
Target-Based Service Identifier & "http" \\

\hline 
\end{tabular}
\end{center}
\end{table}

\begin{note}

The \texttt{service} Metadata Key is only meaningful when a Host Attribute
Table is provided.  When the value exactly matches the service ID as specified
in the table, the rule is applied to that packet, otherwise, the rule is not
applied (even if the ports specified in the rule match).  See Section
\ref{targetbased} for details on the Host Attribute Table.

\end{note}.

\subsubsection{Format}

The examples below show an stub rule from a shared library rule.  The first
uses multiple metadata keywords, the second a single metadata keyword, with
keys separated by commas.

\begin{verbatim}
    metadata:key1 value1;
    metadata:key1 value1, key2 value2;
\end{verbatim}

\subsubsection{Examples}

\begin{verbatim}
    alert tcp any any -> any 80 (msg:"Shared Library Rule Example"; \
        metadata:engine shared; metadata:soid 3|12345;)

    alert tcp any any -> any 80 (msg:"Shared Library Rule Example"; \
        metadata:engine shared, soid 3|12345;)

    alert tcp any any -> any 80 (msg:"HTTP Service Rule Example"; \
        metadata:service http;)
\end{verbatim}

\subsection{General Rule Quick Reference}
\begin{center}
\begin{longtable}[h]{| p{1in} | p{4.5in} |}
\caption{General rule option keywords} \\

\hline
Keyword & Description \\
\hline

\hline
\texttt{msg} &

The msg keyword tells the logging and alerting engine the message to print with
the packet dump or alert. \\

\hline
\texttt{reference} &

The reference keyword allows rules to include references to external attack
identification systems. \\

\hline
\texttt{gid} &

The gid keyword (generator id) is used to identify what part of Snort generates
the event when a particular rule fires. \\

\hline
\texttt{sid} &

The sid keyword is used to uniquely identify Snort rules. \\

\hline
\texttt{rev} &

The rev keyword is used to uniquely identify revisions of Snort rules. \\

\hline
\texttt{classtype} &

The classtype keyword is used to categorize a rule as detecting an attack that
is part of a more general type of attack class. \\

\hline
\texttt{priority} &

The priority keyword assigns a severity level to rules. \\

\hline
\texttt{metadata} &

The metadata keyword allows a rule writer to embed additional information about
the rule, typically in a key-value format. \\

\hline
\end{longtable}
\end{center}

\section{Payload Detection Rule Options}
\subsection{content}
\label{sub:content}

The content keyword is one of the more important features of Snort.  It allows
the user to set rules that search for specific content in the packet payload
and trigger response based on that data. Whenever a content option pattern
match is performed, the Boyer-Moore pattern match function is called and the
(rather computationally expensive) test is performed against the packet
contents. If data exactly matching the argument data string is contained
anywhere within the packet's payload, the test is successful and the remainder
of the rule option tests are performed. Be aware that this test is case
sensitive.

The option data for the content keyword is somewhat complex; it can contain
mixed text and binary data. The binary data is generally enclosed within the
pipe ($|$) character and represented as bytecode. Bytecode represents binary
data as hexadecimal numbers and is a good shorthand method for describing
complex binary data.  The example below shows use of mixed text and binary data
in a Snort rule. 

Note that multiple content rules can be specified in one rule. This allows
rules to be tailored for less false positives. 

If the rule is preceded by a \texttt{!}, the alert will be triggered on packets
that do not contain this content. This is useful when writing rules that want
to alert on packets that do not match a certain pattern

\begin{note}

Also note that the following characters must be escaped inside a content rule:

\begin{verbatim}
    ; \ "
\end{verbatim}
\end{note}

\subsubsection{Format}

\begin{verbatim}
    content:[!]"<content string>";
\end{verbatim}

\subsubsection{Examples}

\begin{verbatim}
    alert tcp any any -> any 139 (content:"|5c 00|P|00|I|00|P|00|E|00 5c|";)
\end{verbatim}

\begin{verbatim}
    alert tcp any any -> any 80 (content:!"GET";)
\end{verbatim}

\begin{note}

A \texttt{!} modifier negates the results of the entire content search,
modifiers included.  For example, if using \texttt{content:!"A"; within:50;}
and there are only 5 bytes of payload and there is no "A" in those 5 bytes, the
result will return a match.  If there must be 50 bytes for a valid match, use
\texttt{isdataat} as a pre-cursor to the content.

\end{note}

\subsubsection{Changing content behavior}

The \texttt{content} keyword has a number of modifier keywords.  The modifier
keywords change how the previously specified content works.  These modifier
keywords are:

\begin{table}[h]
\begin{center}
\caption{Content Modifiers}
\label{Content Modifiers}
\begin{tabular}{|p{1in}|p{1in}|}

\hline 
Modifier & Section \\
\hline

\hline 
nocase & \ref{sub:nocase} \\

\hline
rawbytes & \ref{sub:rawbytes} \\

\hline
depth & \ref{sub:depth} \\

\hline 
offset & \ref{sub:offset} \\

\hline
distance & \ref{sub:Distance} \\

\hline 
within & \ref{sub:Within} \\

\hline
http\_client\_body & \ref{sub:HttpClientBody} \\

\hline 
http\_cookie & \ref{sub:HttpCookie} \\

\hline
http\_raw\_cookie & \ref{sub:RawHttpCookie} \\

\hline 
http\_header & \ref{sub:HttpHeader} \\

\hline 
http\_raw\_header & \ref{sub:RawHttpHeader} \\

\hline 
http\_method & \ref{sub:HttpMethod} \\

\hline 
http\_uri & \ref{sub:HttpUri} \\

\hline 
http\_raw\_uri & \ref{sub:RawHttpUri} \\

\hline
http\_stat\_code & \ref{sub:HttpStatCode} \\

\hline
http\_stat\_msg & \ref{sub:HttpStatMsg} \\

\hline
fast\_pattern & \ref{sub:FastPattern} \\

\hline
\end{tabular}
\end{center}
\end{table}

\subsection{protected\_content}
\label{sub:protectedcontent}

The protected\_content keyword provides much of the functionality of the content keyword, however it performs and is utilized in a very different manner. The primary advantage protected\_content has over content is that protected allows one to hide the target contents by only revealing secure hash digests of said content. As with the content keyword, its primary purpose is to match strings of specific bytes. The search is performed by hashing portions of incoming packets and comparing the results against the hash provided, and as such, it is computationally expensive.

Currently, it is possible to utilize the MD5, SHA256, and SHA512 hash algorithms with the protected\_content keyword. A hashing algorithm must be specified in the rule using \texttt{hash} if a default has not be set in the Snort configuration. Additionally, a \texttt{length} modifier must be specified with protected to indicate the length of the raw data.

As with content, it is possible to use multiple protected\_content rules can in one rule. Additionally, it is possible to mix multiple protected\_content rules with multiple content rules.

If the rule is preceded by a \texttt{!}, the alert will be triggered on packets
that do not contain the target content. This is useful when writing rules that want
to alert on packets that do not match a certain pattern

\begin{note}

The protected\_content keyword can be used with some (but not all) of the content modifiers. Those not supported include:
\begin{verbatim}
	nocase
	fast_pattern
	depth
	within
\end{verbatim}

\end{note}

\subsubsection{Format}

\begin{verbatim}
    protected_content:[!]"<content hash>", length:orig_len[, hash:md5|sha256|sha512];
\end{verbatim}

\subsubsection{Examples}
The following alert on the string "HTTP":

\begin{verbatim}
    alert tcp any any <> any 80 (msg:"MD5 Alert"; 
    protected_content:"293C9EA246FF9985DC6F62A650F78986"; hash:md5; offset:0; length:4;)
\end{verbatim}

\begin{verbatim}
    alert tcp any any <> any 80 (msg:"SHA256 Alert"; 
    protected_content:"56D6F32151AD8474F40D7B939C2161EE2BBF10023F4AF1DBB3E13260EBDC6342"; 
    hash:sha256; offset:0; length:4;)
\end{verbatim}

\begin{note}

A \texttt{!} modifier negates the results of the entire content search,
modifiers included.  For example, if using \texttt{content:!"A"; within:50;}
and there are only 5 bytes of payload and there is no "A" in those 5 bytes, the
result will return a match.  If there must be 50 bytes for a valid match, use
\texttt{isdataat} as a pre-cursor to the content.

\end{note}

\subsection{hash}
\label{sub:hash}

The hash keyword is used to specify the hashing algorithm to use when matching a protected\_content rule. If a default algorithm is not specified in the Snort configuration, a protected\_content rule must specify the algorithm used. Currently, MD5, SHA256, and SHA512 are supported.

\subsubsection{Format}

\begin{verbatim}
    hash:[md5|sha256|sha512];
\end{verbatim}

\subsection{length}
\label{sub:length}

The length keyword is used to specify the original length of the content specified in a protected\_content rule digest. The value provided must be greater than 0 and less than 65536.

\subsubsection{Format}

\begin{verbatim}
    length:[<original_length>];
\end{verbatim}

\subsection{nocase}
\label{sub:nocase}

The nocase keyword allows the rule writer to specify that the Snort should look
for the specific pattern, ignoring case.  nocase modifies the previous
\texttt{content} keyword in the rule.

\subsubsection{Format}

\begin{verbatim}
    nocase;
\end{verbatim}

\subsubsection{Example}

\begin{verbatim}
    alert tcp any any -> any 21 (msg:"FTP ROOT"; content:"USER root"; nocase;)
\end{verbatim}

\subsection{rawbytes}
\label{sub:rawbytes}

The rawbytes keyword allows rules to look at the raw packet data, ignoring any
decoding that was done by preprocessors.  This acts as a modifier to the
previous content \ref{sub:content} option.

HTTP Inspect has a set of keywords to use raw data, such as 
\texttt{http\_raw\_cookie}, \texttt{http\_raw\_header}, \texttt{http\_raw\_uri} etc
that match on specific portions of the raw HTTP requests and responses.

Most other preprocessors use decoded/normalized data for content match by default, if
\texttt{rawbytes} is not specified explicitly.  Therefore, \texttt{rawbytes} should
be specified in order to inspect arbitrary raw data from the packet.

\subsubsection{format}

\begin{verbatim}
    rawbytes;
\end{verbatim}

\subsubsection{Example}

This example tells the content pattern matcher to look at the raw traffic,
instead of the decoded traffic provided by the Telnet decoder.

\begin{verbatim}
    alert tcp any any -> any 21 (msg:"Telnet NOP"; content:"|FF F1|"; rawbytes;)
\end{verbatim}

\subsection{depth}
\label{sub:depth}

The depth keyword allows the rule writer to specify how far into a packet Snort
should search for the specified pattern.  depth modifies the previous `content'
keyword in the rule.

A depth of 5 would tell Snort to only look for the specified pattern within the
first 5 bytes of the payload.

As the depth keyword is a modifier to the previous \texttt{content} keyword, there
must be a content in the rule before \texttt{depth} is specified.

This keyword allows values greater than or equal to the pattern length being
searched. The minimum allowed value is 1.  The maximum allowed value for this
keyword is 65535.

The value can also be set to a string value referencing a variable extracted by the
\texttt{byte\_extract} keyword in the same rule.

\subsubsection{Format}

\begin{verbatim}
    depth:[<number>|<var_name>];
\end{verbatim}

\subsection{offset}
\label{sub:offset}

The offset keyword allows the rule writer to specify where to start searching
for a pattern within a packet.  offset modifies the previous 'content' keyword
in the rule.

An offset of 5 would tell Snort to start looking for the specified pattern
after the first 5 bytes of the payload.

As this keyword is a modifier to the previous \texttt{content} keyword, there must be
a content in the rule before \texttt{offset} is specified.

This keyword allows values from -65535 to 65535.

The value can also be set to a string value referencing a variable extracted by the
\texttt{byte\_extract} keyword in the same rule.

\subsubsection{Format}

\begin{verbatim}
    offset:[<number>|<var_name>];
\end{verbatim}

\subsubsection{Example}

The following example shows use of a combined content, offset, and depth search
rule.

\begin{verbatim}
    alert tcp any any -> any 80 (content:"cgi-bin/phf"; offset:4; depth:20;)
\end{verbatim}

\subsection{distance}
\label{sub:Distance}

The distance keyword allows the rule writer to specify how far into a packet
Snort should ignore before starting to search for the specified pattern
relative to the end of the previous pattern match.  

This can be thought of as exactly the same thing as offset (See Section
\ref{sub:offset}), except it is relative to the end of the last pattern match
instead of the beginning of the packet.

This keyword allows values from -65535 to 65535.

The value can also be set to a string value referencing a variable extracted by the
\texttt{byte\_extract} keyword in the same rule.

\subsubsection{Format}

\begin{verbatim}
    distance:[<byte_count>|<var_name>];
\end{verbatim}

\subsubsection{Example}

The rule below maps to a regular expression of /ABC.\{1\}DEF/.

\begin{verbatim}
    alert tcp any any -> any any (content:"ABC"; content:"DEF"; distance:1;)
\end{verbatim}

\subsection{within}
\label{sub:Within}

The within keyword is a content modifier that makes sure that at most N bytes
are between pattern matches using the content keyword ( See Section
\ref{sub:content} ).  It's designed to be used in conjunction with the distance
(Section \ref{sub:Distance}) rule option.

This keyword allows values greater than or equal to pattern length being searched. 
The maximum allowed value for this keyword is 65535.

The value can also be set to a string value referencing a variable extracted by the
\texttt{byte\_extract} keyword in the same rule.

\subsubsection{Format}

\begin{verbatim}
    within:[<byte_count>|<var_name>];
\end{verbatim}

\subsubsection{Examples}

This rule constrains the search of EFG to not go past 10 bytes past the ABC match. 

\begin{verbatim}
    alert tcp any any -> any any (content:"ABC"; content:"EFG"; within:10;)
\end{verbatim}

\subsection{http\_client\_body}
\label{sub:HttpClientBody}

The http\_client\_body keyword is a content modifier that restricts the search
to the body of an HTTP client request.

As this keyword is a modifier to the previous \texttt{content} keyword, there must be
a content in the rule before 'http\_client\_body' is specified. 

The amount of data that is inspected with this option depends on the \texttt{post\_depth}
config option of HttpInspect. Pattern matches with this keyword wont work when 
 \texttt{post\_depth} is set to -1.

\subsubsection{Format}

\begin{verbatim}
    http_client_body;
\end{verbatim}

\subsubsection{Examples}

This rule constrains the search for the pattern "EFG" to the raw body of
an HTTP client request.

\begin{verbatim}
    alert tcp any any -> any 80 (content:"ABC"; content:"EFG"; http_client_body;)
\end{verbatim}

\begin{note}
The \texttt{http\_client\_body} modifier is not allowed to be used with
the \texttt{rawbytes} modifier for the same content.
\end{note}

\subsection{http\_cookie}
\label{sub:HttpCookie}

The http\_cookie keyword is a content modifier that restricts the search to the
extracted Cookie Header field (excluding the header name itself and the CRLF terminating 
the header line) of a HTTP client request or a HTTP server response (per the configuration 
of HttpInspect \ref{sub:http-inspect}). The Cookie buffer does not include the header 
names (\texttt{Cookie:} for HTTP requests or \texttt{Set-Cookie:} for HTTP responses) 
or leading spaces and the CRLF terminating the header line. These are included in the HTTP
header buffer. 

As this keyword is a modifier to the previous \texttt{content} keyword, there must be
a content in the rule before \texttt{http\_cookie} is specified. This keyword is dependent
on the \texttt{enable\_cookie} config option. The Cookie Header field will be extracted only
when this option is configured. If \texttt{enable\_cookie} is not specified, the cookie 
still ends up in HTTP header.  When \texttt{enable\_cookie} is not specified, using 
\texttt{http\_cookie} is the same as using \texttt{http\_header}.

The extracted Cookie Header field may be NORMALIZED, per the configuration of
HttpInspect (see \ref{sub:http-inspect}).

\subsubsection{Format}

\begin{verbatim}
    http_cookie;
\end{verbatim}

\subsubsection{Examples}

This rule constrains the search for the pattern "EFG" to the extracted Cookie
Header field of a HTTP client request.

\begin{verbatim}
    alert tcp any any -> any 80 (content:"ABC"; content:"EFG"; http_cookie;)
\end{verbatim}

\begin{note}

The \texttt{http\_cookie} modifier is not allowed to be used with the
\texttt{rawbytes} or \texttt{fast\_pattern} modifiers for the same content.

\end{note}

\subsection{http\_raw\_cookie}
\label{sub:RawHttpCookie}

The http\_raw\_cookie keyword is a content modifier that restricts the search to the
extracted UNNORMALIZED Cookie Header field of a HTTP client request or a HTTP server 
response (per the configuration of HttpInspect \ref{sub:http-inspect}).

As this keyword is a modifier to the previous \texttt{content} keyword, there must be
a content in the rule before \texttt{http\_raw\_cookie} is specified. This keyword is dependent
on the \texttt{enable\_cookie} config option. The Cookie Header field will be extracted only
when this option is configured.

\subsubsection{Format}

\begin{verbatim}
    http_raw_cookie;
\end{verbatim}

\subsubsection{Examples}

This rule constrains the search for the pattern "EFG" to the extracted Unnormalized
Cookie Header field of a HTTP client request.

\begin{verbatim}
    alert tcp any any -> any 80 (content:"ABC"; content:"EFG"; http_raw_cookie;)
\end{verbatim}

\begin{note}

The \texttt{http\_raw\_cookie} modifier is not allowed to be used with the
\texttt{rawbytes}, \texttt{http\_cookie} or \texttt{fast\_pattern} modifiers for the same 
content.

\end{note}

\subsection{http\_header}
\label{sub:HttpHeader}

The http\_header keyword is a content modifier that restricts the search to the
extracted Header fields of a HTTP client request or a HTTP server response (per the 
configuration of HttpInspect \ref{sub:http-inspect}).

As this keyword is a modifier to the previous \texttt{content} keyword, there must be
a content in the rule before \texttt{http\_header} is specified.

The extracted Header fields may be NORMALIZED, per the configuration of
HttpInspect (see \ref{sub:http-inspect}).

\subsubsection{Format}

\begin{verbatim}
    http_header;
\end{verbatim}

\subsubsection{Examples}

This rule constrains the search for the pattern "EFG" to the extracted Header
fields of a HTTP client request or a HTTP server response.

\begin{verbatim}
    alert tcp any any -> any 80 (content:"ABC"; content:"EFG"; http_header;)
\end{verbatim}

\begin{note}

The \texttt{http\_header} modifier is not allowed to be used with the
\texttt{rawbytes} modifier for the same content.

\end{note}

\subsection{http\_raw\_header}
\label{sub:RawHttpHeader}

The http\_raw\_header keyword is a content modifier that restricts the search to the
extracted UNNORMALIZED Header fields of a HTTP client request or a HTTP server
response (per the configuration of HttpInspect \ref{sub:http-inspect}).

As this keyword is a modifier to the previous \texttt{content} keyword, there must be
a content in the rule before \texttt{http\_raw\_header} is specified.

\subsubsection{Format}

\begin{verbatim}
    http_raw_header;
\end{verbatim}

\subsubsection{Examples}

This rule constrains the search for the pattern "EFG" to the extracted Header fields
of a HTTP client request or a HTTP server response.

\begin{verbatim}
    alert tcp any any -> any 80 (content:"ABC"; content:"EFG"; http_raw_header;)
\end{verbatim}

\begin{note}

The \texttt{http\_raw\_header} modifier is not allowed to be used with the
\texttt{rawbytes}, \texttt{http\_header} or \texttt{fast\_pattern} modifiers for the same
content.

\end{note}

\subsection{http\_method}
\label{sub:HttpMethod}

The http\_method keyword is a content modifier that restricts the search to the
extracted Method from a HTTP client request.

As this keyword is a modifier to the previous \texttt{content} keyword, there must be
a content in the rule before \texttt{http\_method} is specified.

\subsubsection{Format}

\begin{verbatim}
    http_method;
\end{verbatim}

\subsubsection{Examples}

This rule constrains the search for the pattern "GET" to the extracted Method
from a HTTP client request.

\begin{verbatim}
    alert tcp any any -> any 80 (content:"ABC"; content:"GET"; http_method;)
\end{verbatim}

\begin{note}

The \texttt{http\_method} modifier is not allowed to be used with the
\texttt{rawbytes} or \texttt{fast\_pattern} modifiers for the same content.

\end{note}

\subsection{http\_uri}
\label{sub:HttpUri}

The http\_uri keyword is a content modifier that restricts the search to the
NORMALIZED request \textsc{URI} field .  Using a content rule option followed
by a http\_uri modifier is the same as using a uricontent by itself (see:
\ref{sub:UriContent}).

As this keyword is a modifier to the previous \texttt{content} keyword, there must be
a content in the rule before \texttt{http\_uri} is specified.

\subsubsection{Format}

\begin{verbatim}
    http_uri;
\end{verbatim}

\subsubsection{Examples}

This rule constrains the search for the pattern "EFG" to the NORMALIZED URI.

\begin{verbatim}
    alert tcp any any -> any 80 (content:"ABC"; content:"EFG"; http_uri;)
\end{verbatim}

\begin{note}

The \texttt{http\_uri} modifier is not allowed to be used with the
\texttt{rawbytes} modifier for the same content.

\end{note}

\subsection{http\_raw\_uri}
\label{sub:RawHttpUri}
The http\_raw\_uri keyword is a content modifier that restricts the search to the
UNNORMALIZED request \textsc{URI} field . 

As this keyword is a modifier to the previous \texttt{content} keyword, there must be
a content in the rule before \texttt{http\_raw\_uri} is specified.

\subsubsection{Format}

\begin{verbatim}
    http_raw_uri;
\end{verbatim}

\subsubsection{Examples}

This rule constrains the search for the pattern "EFG" to the UNNORMALIZED URI.

\begin{verbatim}
    alert tcp any any -> any 80 (content:"ABC"; content:"EFG"; http_raw_uri;)
\end{verbatim}
\begin{note}

The \texttt{http\_raw\_uri} modifier is not allowed to be used with the
\texttt{rawbytes}, \texttt{http\_uri} or \texttt{fast\_pattern} modifiers for the same
content.

\end{note}

\subsection{http\_stat\_code}
\label{sub:HttpStatCode}

The http\_stat\_code keyword is a content modifier that restricts the search to the
extracted Status code field from a HTTP server response.

As this keyword is a modifier to the previous \texttt{content} keyword, there must be
a content in the rule before \texttt{http\_stat\_code} is specified. 

The Status Code field will be extracted only if the extended\_response\_inspection is 
configured for the HttpInspect (see \ref{sub:http-inspect}).

\subsubsection{Format}

\begin{verbatim}
    http_stat_code;
\end{verbatim}

\subsubsection{Examples}

This rule constrains the search for the pattern "200" to the extracted Status Code field 
of a HTTP server response.

\begin{verbatim}
    alert tcp any any -> any 80 (content:"ABC"; content:"200"; http_stat_code;)
\end{verbatim}

\begin{note}

The \texttt{http\_stat\_code} modifier is not allowed to be used with the
\texttt{rawbytes} or \texttt{fast\_pattern} modifiers for the same content.

\end{note}

\subsection{http\_stat\_msg}
\label{sub:HttpStatMsg}

The http\_stat\_msg keyword is a content modifier that restricts the search to the
extracted Status Message field from a HTTP server response.

As this keyword is a modifier to the previous \texttt{content} keyword, there must be
a content in the rule before \texttt{http\_stat\_msg} is specified.

The Status Message field will be extracted only if the extended\_response\_inspection is
configured for the HttpInspect (see \ref{sub:http-inspect}).

\subsubsection{Format}

\begin{verbatim}
    http_stat_msg;
\end{verbatim}

\subsubsection{Examples}

This rule constrains the search for the pattern "Not Found" to the extracted Status 
Message field of a HTTP server response.

\begin{verbatim}
    alert tcp any any -> any 80 (content:"ABC"; content:"Not Found"; http_stat_msg;)
\end{verbatim}

\begin{note}

The \texttt{http\_stat\_msg} modifier is not allowed to be used with the
\texttt{rawbytes} or \texttt{fast\_pattern} modifiers for the same content.

\end{note}

\subsection{http\_encode}
\label{sub:HttpEncode}

The \texttt{http\_encode} keyword will enable alerting based on encoding type present
in a HTTP client request or a HTTP server response (per the configuration of 
HttpInspect \ref{sub:http-inspect}).

There are several keywords associated with \texttt{http\_encode}. The keywords
'uri', 'header' and 'cookie' determine the HTTP fields used to search for a
particular encoding type.  The keywords 'utf8', 'double\_encode', 'non\_ascii',
'uencode', 'iis\_encode', 'ascii' and 'bare\_byte' determine the encoding 
type which would trigger the alert. These keywords can be combined using a OR operation.
Negation is allowed on these keywords.

The config option 'normalize\_headers' needs to be turned on for rules to work
with the keyword 'header'.  The keyword 'cookie' is dependent on config options
'enable\_cookie' and 'normalize\_cookies' (see \ref{sub:http-inspect}).  This
rule option will not be able to detect encodings if the specified HTTP fields
are not NORMALIZED.

\begin{tabular}{| l | p{4.5in} |}
\hline
{\bf Option} & {\bf Description}\\
\hline
\hline
\texttt{uri} & Check for the specified encoding type in HTTP client request URI field.\\
\hline
\texttt{header} & Check for the specified encoding type in HTTP request or HTTP response header fields
(depending on the packet flow)\\
\hline
\texttt{cookie} & Check for the specified encoding type in HTTP request or HTTP response cookie 
header fields (depending on the packet flow)\\
\hline
\texttt{utf8} & Check for utf8 encoding in the specified buffer\\
\hline
\texttt{double\_encode} & Check for double encoding in the specified buffer\\
\hline
\texttt{non\_ascii} & Check for non-ASCII encoding in the specified buffer\\
\hline
\texttt{uencode} & Check for u-encoding in the specified buffer\\
\hline
\texttt{bare\_byte} & Check for bare byte encoding in the specified buffer\\
\hline
\texttt{ascii} & Check for ascii encoding in the specified buffer\\
\hline
\texttt{iis\_encode} & Check for IIS Unicode encoding in the specified buffer\\
\hline
\end{tabular}

\subsubsection{Format}

\begin{verbatim}
    http_encode:<http buffer type>, [!]<encoding type>
    http_encode:[uri|header|cookie], [!][<utf8|double_encode|non_ascii|uencode|bare_byte|ascii|iis_encode>];
\end{verbatim}

\subsubsection{Examples}
\begin{verbatim}
    alert tcp any any -> any any (msg:"UTF8/UEncode Encoding present"; http_encode:uri,utf8|uencode;)
    alert tcp any any -> any any (msg:"No UTF8"; http_encode:uri,!utf8;)
\end{verbatim}


\begin{note}

Negation(!) and OR({\tt |}) operations cannot be used in conjunction with each other for the 
\texttt{http\_encode} keyword. The OR and negation operations work only on the encoding type
field and not on http buffer type field.

\end{note}

\subsection{fast\_pattern}
\label{sub:FastPattern}

The \texttt{fast\_pattern} keyword is a content modifier that sets the content
within a rule to be used with the fast pattern matcher. The default behavior of
fast pattern determination is to use the longest HTTP buffer content. If no HTTP
buffer is present, then the fast pattern is the longest content. Given this behavior,
it is useful if a shorter content is more "unique" than the longer content, 
meaning the shorter content is less likely to be found in a packet than the longer content.

The fast pattern matcher is used to select only those rules that have a
chance of matching by using a content in the rule for selection and only
evaluating that rule if the content is found in the payload.  Though this
may seem to be overhead, it can significantly reduce the number of rules
that need to be evaluated and thus increases performance.  The better the
content used for the fast pattern matcher, the less likely the rule will
needlessly be evaluated.

As this keyword is a modifier to the previous \texttt{content} keyword, there must be
a \texttt{content} rule option in the rule before \texttt{fast\_pattern} is specified.
The \texttt{fast\_pattern} option may be specified only once per rule.

\begin{note}
The \texttt{fast\_pattern} modifier cannot be used with the following http
content modifiers: \texttt{http\_cookie}, \texttt{http\_raw\_uri},
\texttt{http\_raw\_header}, \texttt{http\_raw\_cookie}, \texttt{http\_method},
\texttt{http\_stat\_code}, \texttt{http\_stat\_msg}.
\end{note}

\begin{note}
The \texttt{fast\_pattern} modifier can be used with negated contents only if
those contents are not modified with \texttt{offset}, \texttt{depth},
\texttt{distance} or \texttt{within}.
\end{note}

\begin{note}
The fast pattern matcher is always case insensitive.
\end{note}

\subsubsection{Format}
The \texttt{fast\_pattern} option can be used alone or optionally take arguments.
When used alone, the meaning is simply to use the specified content as the
fast pattern content for the rule.

\begin{verbatim}
fast_pattern;
\end{verbatim}

The optional argument \texttt{only} can be used to specify that the content
should only be used for the fast pattern matcher and should not be evaluated
as a rule option.  This is useful, for example, if a known content must be
located in the payload independent of location in the payload, as it saves
the time necessary to evaluate the rule option.  
Note that (1) the modified content must be case insensitive since
patterns are inserted into the pattern matcher in a case insensitive manner,
(2) negated contents cannot be used and (3) contents cannot have any positional
modifiers such as \texttt{offset}, \texttt{depth}, \texttt{distance}
or \texttt{within}.
\begin{verbatim}
fast_pattern:only;
\end{verbatim}

The optional argument \texttt{<offset>,<length>} can be used to specify that
only a portion of the content should be used for the fast pattern matcher.
This is useful if the pattern is very long and only a portion of the pattern
is necessary to satisfy "uniqueness" thus reducing the memory required to
store the entire pattern in the fast pattern matcher.
\begin{verbatim}
fast_pattern:<offset>,<length>;
\end{verbatim}

\begin{note}
The optional arguments \texttt{only} and \texttt{<offset>,<length>} are
mutually exclusive.
\end{note}

\subsubsection{Examples}

This rule causes the pattern "IJKLMNO" to be used with the fast pattern matcher,
even though it is shorter than the earlier pattern "ABCDEFGH".

\begin{verbatim}
alert tcp any any -> any 80 (content:"ABCDEFGH"; content:"IJKLMNO"; fast_pattern;)
\end{verbatim}

This rule says to use the content "IJKLMNO" for the fast pattern matcher and that
the content should only be used for the fast pattern matcher and not evaluated
as a \texttt{content} rule option.
\begin{verbatim}
alert tcp any any -> any 80 (content:"ABCDEFGH"; content:"IJKLMNO"; nocase; fast_pattern:only;)
\end{verbatim}

This rule says to use "JKLMN" as the fast pattern content, but still evaluate
the \texttt{content} rule option as "IJKLMNO".
\begin{verbatim}
alert tcp any any -> any 80 (content:"ABCDEFGH"; content:"IJKLMNO"; fast_pattern:1,5;)
\end{verbatim}

\subsection{uricontent}
\label{sub:UriContent}

The \texttt{uricontent} keyword in the Snort rule language searches the
NORMALIZED request \textsc{URI} field.  This is equivalent to using the
\texttt{http\_uri} modifier to a \texttt{content} keyword.  As such if you
are writing rules that include things that are normalized, such as \%2f or
directory traversals, these rules will not alert.  The reason is that the
things you are looking for are normalized out of the URI buffer.  

For example, the URI: 

\begin{verbatim}
    /scripts/..%c0%af../winnt/system32/cmd.exe?/c+ver
\end{verbatim}

will get normalized into:

\begin{verbatim}
    /winnt/system32/cmd.exe?/c+ver
\end{verbatim}

Another example, the URI:

\begin{verbatim}
    /cgi-bin/aaaaaaaaaaaaaaaaaaaaaaaaaa/..%252fp%68f?
\end{verbatim}

will get normalized into:

\begin{verbatim}
    /cgi-bin/phf?
\end{verbatim}

When writing a \texttt{uricontent} rule, write the content that you want to
find in the context that the URI will be normalized.  For example, if Snort
normalizes directory traversals, do not include directory traversals.  

You can write rules that look for the non-normalized content by using the
content option.  (See Section \ref{sub:content})

\texttt{uricontent} can be used with several of the modifiers available to the
\texttt{content} keyword.  These include:

\begin{table}[h]
\begin{center}
\caption{Uricontent Modifiers}
\label{Uricontent Modifiers}
\begin{tabular}{|p{1in}|p{1in}|}

\hline 
Modifier & Section \\
\hline

\hline 
nocase & \ref{sub:nocase} \\

\hline
depth & \ref{sub:depth} \\

\hline 
offset & \ref{sub:offset} \\

\hline
distance & \ref{sub:Distance} \\

\hline 
within & \ref{sub:Within} \\

\hline
fast\_pattern & \ref{sub:FastPattern} \\

\hline
\end{tabular}
\end{center}
\end{table}


This option works in conjunction with the HTTP Inspect preprocessor specified
in Section \ref{sub:http-inspect}.

\subsubsection{Format}

\begin{verbatim}
    uricontent:[!]"<content string>";
\end{verbatim}

\begin{note}

\texttt{uricontent} cannot be modified by a \texttt{rawbytes} modifier or any
of the other HTTP modifiers.  If you wish to search the UNNORMALIZED
request \textsc{URI} field, use the \texttt{http\_raw\_uri} modifier with a
\texttt{content} option.

\end{note}

\subsection{urilen}

The \texttt{urilen} keyword in the Snort rule language specifies the exact
length, the minimum length, the maximum length, or range of URI lengths to
match.  By default the raw uri buffer will be used.  With the optional
\texttt{<uribuf>} argument, you can specify whether the raw or normalized
buffer are used.

\subsubsection{Format}

\begin{verbatim}
    urilen:min<>max[,<uribuf>];
    urilen:[<|>]<number>[,<uribuf>];

    <uribuf> : "norm" | "raw"
\end{verbatim}

The following example will match URIs that are 5 bytes long:

\begin{verbatim}
    urilen:5;
\end{verbatim}

The following example will match URIs that are shorter than 5 bytes:

\begin{verbatim}
    urilen:<5;
\end{verbatim}

The following example will match URIs that are greater than 5 bytes and less
than 10 bytes (inclusive):

\begin{verbatim}
    urilen:5<>10;
\end{verbatim}

The following example will match URIs that are greater than 500 bytes using the
normalized URI buffer:

\begin{verbatim}
    urilen:>500,norm;
\end{verbatim}

The following example will match URIs that are greater than 500 bytes explicitly
stating to use the raw URI buffer:

\begin{verbatim}
    urilen:>500,raw;
\end{verbatim}

This option works in conjunction with the HTTP Inspect preprocessor specified
in Section \ref{sub:http-inspect}.

\subsection{isdataat}

Verify that the payload has data at a specified location, optionally looking
for data relative to the end of the previous content match.

\subsubsection{Format}

\begin{verbatim}
    isdataat:[!]<int>[, relative|rawbytes];
\end{verbatim}

\subsubsection{Example}

\begin{verbatim}
    alert tcp any any -> any 111 (content:"PASS"; isdataat:50,relative; \ 
        content:!"|0a|"; within:50;)
\end{verbatim}

This rule looks for the string PASS exists in the packet, then verifies there
is at least 50 bytes after the end of the string PASS, then verifies that there
is not a newline character within 50 bytes of the end of the PASS string.

When the \texttt{rawbytes} modifier is specified with \texttt{isdataat}, it 
looks at the raw packet data, ignoring any decoding that was done by the 
preprocessors. This modifier will work with the \texttt{relative} modifier 
as long as the previous content match was in the raw packet data.

A \texttt{!} modifier negates the results of the isdataat test. It will alert 
if a certain amount of data is not present within the payload. For example,
the rule with modifiers \texttt{content:"foo"; isdataat:!10,relative;}  would 
alert if there were not 10 bytes after "foo" before the payload ended. 

\subsection{pcre}
\label{pcre}

The pcre keyword allows rules to be written using perl compatible regular
expressions.   For more detail on what can be done via a pcre regular
expression, check out the PCRE web site \url{http://www.pcre.org}

\subsubsection{Format}

\begin{verbatim}
    pcre:[!]"(/<regex>/|m<delim><regex><delim>)[ismxAEGRUBPHMCOIDKYS]";
\end{verbatim}

The post-re modifiers set compile time flags for the regular expression.  See
tables \ref{pcre-mod_perl}, \ref{pcre-mod_pcre}, and \ref{pcre-mod_snort} for
descriptions of each modifier.

\begin{table}[ht]
\begin{center}
\caption{Perl compatible modifiers for \texttt{pcre}}
\label{pcre-mod_perl}
\begin{tabular}{|c|p{4.5in}|}

\hline
i & case insensitive \\

\hline
s & include newlines in the dot metacharacter \\

\hline
m &

By default, the string is treated as one big line of characters.  \^\ and \$
match at the beginning and ending of the string. When m is set, \^\ and \$
match immediately following or immediately before any newline in the buffer, as
well as the very start and very end of the buffer. \\

\hline
x &

whitespace  data characters in the pattern are ignored except when escaped or
inside a character class \\

\hline
\end{tabular}
\end{center}
\end{table}

\begin{table}[ht]
\begin{center}
\caption{PCRE compatible modifiers for \texttt{pcre}}
\label{pcre-mod_pcre}
\begin{tabular}{|c|p{4.5in}|}

\hline
A &

the pattern must match only at the start of the buffer (same as \^\ ) \\

\hline
E &

Set \$ to match only at the end of the subject string.  Without E, \$ also
matches immediately before the final character if it is a newline (but not
before any other newlines). \\

\hline
G &

Inverts the "greediness" of the quantifiers so that they are not greedy by
default, but become greedy if followed by "?". \\

\hline
\end{tabular}
\end{center}
\end{table}

\begin{table}[ht]
\begin{center}
\caption{Snort specific modifiers for \texttt{pcre}}
\label{pcre-mod_snort}
\begin{tabular}{|c|p{4.5in}|}

\hline
R &
Match relative to the end of the last pattern match.  (Similar to distance:0;) \\

\hline
U &
Match the decoded URI buffers (Similar to \texttt{uricontent} and \texttt{http\_uri}).
This modifier is not allowed with the unnormalized HTTP request uri buffer modifier(I) 
for the same content. \\

\hline
I &
Match the unnormalized HTTP request uri buffer (Similar to \texttt{http\_raw\_uri}).
This modifier is not allowed with the HTTP request uri buffer modifier(U) for the
same content. \\

\hline
P &
Match unnormalized HTTP request body (Similar to \texttt{http\_client\_body}).\\

 & For SIP message, match SIP body for request or response (Similar to \texttt{sip\_body}). \\

\hline
H &
Match normalized HTTP request or HTTP response header (Similar to \texttt{http\_header}).
This modifier is not allowed with the unnormalized HTTP request or HTTP response header 
modifier(D) for the same content. \\
 & For SIP message, match SIP header for request or response (Similar to \texttt{sip\_header}). \\

\hline
D &
Match unnormalized HTTP request or HTTP response header (Similar to \texttt{http\_raw\_header}). 
This modifier is not allowed with the normalized HTTP request or HTTP response header 
modifier(H) for the same content. \\

\hline
M &
Match normalized HTTP request method (Similar to \texttt{http\_method}) \\

\hline
C &
Match normalized HTTP request or HTTP response cookie (Similar to \texttt{http\_cookie}).
This modifier is not allowed with the unnormalized HTTP request or HTTP response cookie 
modifier(K) for the same content. \\

\hline
K &
Match unnormalized HTTP request or HTTP response cookie (Similar to \texttt{http\_raw\_cookie}). 
This modifier is not allowed with the normalized HTTP request or HTTP response cookie 
modifier(C) for the same content. \\

\hline
S &
Match HTTP response status code (Similar to \texttt{http\_stat\_code}) \\

\hline
Y &
Match HTTP response status message (Similar to \texttt{http\_stat\_msg}) \\

\hline
B &
Do not use the decoded buffers (Similar to rawbytes) \\

\hline
O &
Override the configured pcre match limit and pcre match limit recursion for 
this expression (See section \ref{Config}). It completely ignores the limits 
while evaluating the pcre pattern specified. \\

\hline
\end{tabular}
\end{center}
\end{table}

\begin{note}
The modifiers R (relative) and B (rawbytes) are not allowed with any of the HTTP modifiers such as U, I, 
P, H, D, M, C, K, S and Y. 
\end{note}

\subsubsection{Example}

This example performs a case-insensitive search for the HTTP URI \texttt{foo.php?id=<some numbers>}

\begin{verbatim}
    alert tcp any any -> any 80 (content:"/foo.php?id="; pcre:"/\/foo.php?id=[0-9]{1,10}/iU";)
\end{verbatim}

\begin{note}
It is wise to have at least one \texttt{content} keyword in a rule that uses \texttt{pcre}.  This
allows the fast-pattern matcher to filter out non-matching packets so that the pcre evaluation is
not performed on each and every packet coming across the wire.
\end{note}

\begin{note}

Snort's handling of multiple URIs with PCRE does not work as expected.  PCRE
when used without a \texttt{uricontent} only evaluates the first URI.  In order
to use pcre to inspect all URIs, you must use either a content or a uricontent.

\end{note}

\subsection{pkt\_data}
\label{sub:pkt_data}
This option sets the cursor used for detection to the raw transport payload.

Any relative or absolute content matches (without HTTP modifiers or rawbytes) and other 
payload detecting rule options that follow \texttt{pkt\_data} in a rule will apply to the 
raw TCP/UDP payload or the normalized buffers (in case of telnet, smtp normalization) until 
the cursor (used for detection) is set again.

This rule option can be used several times in a rule.

\subsubsection{Format}

\begin{verbatim}
        pkt_data;
\end{verbatim}

\subsubsection{Example}

\begin{verbatim}
        
        alert tcp any any -> any any(msg:"Absolute Match"; pkt_data; content:"BLAH"; offset:0; depth:10;)
        alert tcp any any -> any any(msg:"PKT DATA"; pkt_data; content:"foo"; within:10;)
        alert tcp any any -> any any(msg:"PKT DATA"; pkt_data; content:"foo";)
        alert tcp any any -> any any(msg:"PKT DATA"; pkt_data; pcre:"/foo/i";)

\end{verbatim}

\subsection{file\_data}
\label{sub:file_data}
This option sets the cursor used for detection to one of the following buffers:
1. When the traffic being detected is HTTP it sets the buffer to,
	a. HTTP response body (without chunking/compression/normalization)
	b. HTTP de-chunked response body
	c. HTTP decompressed response body (when \texttt{inspect\_gzip} is turned on)
	d. HTTP normalized response body (when \texttt{normalized\_javascript} is turned on)
	e. HTTP UTF normalized response body (when \texttt{normalize\_utf} is turned on)
	f. All of the above
2. When the traffic being detected is SMTP/POP/IMAP it sets the buffer to,
	a. SMTP/POP/IMAP data body (including Email headers and MIME when decoding
	   is turned off)
        b. Base64 decoded MIME attachment (when \texttt{b64\_decode\_depth} is greater than -1)
	c. Non-Encoded MIME attachment (when \texttt{bitenc\_decode\_depth} is greater than -1)
	d. Quoted-Printable decoded MIME attachment (when \texttt{qp\_decode\_depth} is greater than -1) 
	e. Unix-to-Unix decoded attachment (when \texttt{uu\_decode\_depth} is greater than -1)
3. If it is not set by 1 and 2, it will be set to the payload.

Any relative or absolute content matches (without HTTP modifiers or rawbytes) and payload detecting
rule options that follow \texttt{file\_data} in a rule will apply to this buffer until explicitly reset 
by other rule options.

This rule option can be used several time in a rule. 

The argument \texttt{mime} to \texttt{file\_data} is deprecated. The rule options \texttt{file\_data} will 
itself point to the decoded MIME attachment.

\subsubsection{Format}

\begin{verbatim}
	file_data;
\end{verbatim}

\subsubsection{Example}

\begin{verbatim}
	alert tcp any any -> any any(msg:"Absolute Match"; file_data; content:"BLAH"; offset:0; depth:10;)
	alert tcp any any -> any any(msg:"FILE DATA"; file_data; content:"foo"; within:10;)
	alert tcp any any -> any any(msg:"FILE DATA"; file_data; content:"foo";)
	alert tcp any any -> any any(msg:"FILE DATA"; file_data; pcre:"/foo/i";)
	
	The following rule searches for content "foo" within the file_data buffer and content "bar" within the 
	entire packet payload. The rule option pkt_data will reset the cursor used for detection to the 
	TCP payload.
	alert tcp any any -> any any(msg:"FILE DATA"; file_data; content:"foo"; pkt_data; content:"bar";)

\end{verbatim}


\subsection{base64\_decode}
\label{sub:base64_decode}

This option is used to decode the base64 encoded data. This option is particularly useful
in case of HTTP headers such as HTTP authorization headers. This option unfolds the data
before decoding it.

\subsubsection{Format}

\begin{verbatim}
	base64_decode[:[bytes <bytes_to_decode>][, ][offset <offset>[, relative]]];
\end{verbatim}

\begin{tabular}{| l | p{4.5in} |}
\hline
{\bf Option} & {\bf Description}\\
\hline
\hline
\texttt{bytes} & 

Number of base64 encoded bytes to decode. This argument takes positive and 
non-zero values only. When this option is not specified we look for base64
encoded data till either the end of header line is reached or end of packet
payload is reached.\\
\hline
\texttt{offset} & 

Determines the offset relative to the doe\_ptr when the option \texttt{relative} 
is specified or relative to the start of the packet payload to begin inspection 
of base64 encoded data. This argument takes positive and non-zero values only.\\
\hline
\texttt{relative} &

Specifies the inspection for base64 encoded data is relative to the doe\_ptr.\\

\hline
\end{tabular}

The above arguments to \texttt{base64\_decode} are optional.

\begin{note}

This option can be extended to protocols with folding similar to HTTP. If folding is not
present the search for base64 encoded data will end when we see a carriage return or line feed
or both without a following space or tab.

This option needs to be used in conjunction with \texttt{base64\_data} for any other 
payload detecting rule options to work on base64 decoded buffer.
\end{note}

\subsubsection{Examples}

\begin{verbatim}
    alert tcp $EXTERNAL_NET any -> $HOME_NET any \
	(msg:"Base64 Encoded Data"; base64_decode; base64_data; \
	content:"foo bar"; within:20;)

    alert tcp $EXTERNAL_NET any -> $HOME_NET any \
	(msg:"Authorization NTLM"; content:"Authorization: NTLM";
	base64_decode:relative; base64_data; content:"NTLMSSP"; )

    alert tcp any any -> any any (msg:"Authorization NTLM"; \
	content:"Authorization:"; http_header; \
	base64_decode:bytes 12, offset 6, relative; base64_data; \
	content:"NTLMSSP"; within:8;)
\end{verbatim}

\subsection{base64\_data}
\label{sub:base64_data}
This option is similar to the rule option \texttt{file\_data} and is used 
to set the cursor used for detection to the beginning of the base64 decoded 
buffer if present.

This option does not take any arguments. The rule option \texttt{base64\_decode}
needs to be specified before the \texttt{base64\_data} option. 

\subsubsection{Format}

\begin{verbatim}
	base64_data;
\end{verbatim}

This option matches if there is base64 decoded buffer.

\begin{note}

Fast pattern content matches are not allowed with this buffer.
\end{note}


\subsubsection{Example}

\begin{verbatim}
    alert tcp any any -> any any (msg:"Authorization NTLM"; \
        content:"Authorization:"; http_header; \
        base64_decode:bytes 12, offset 6, relative; base64_data; \
        content:"NTLMSSP"; within:8;)
\end{verbatim}


\subsection{byte\_test}
\label{sub:byte_test}

Test a byte field against a specific value (with operator).  Capable of testing
binary values or converting representative byte strings to their binary
equivalent and testing them.

For a more detailed explanation, please read Section \ref{testing numerical
values}.

\subsubsection{Format}

\footnotesize
\begin{verbatim}
    byte_test:<bytes to convert>, [!]<operator>, <value>, <offset>  \
        [, relative][, <endian>][, string, <number type>][, dce];

    bytes      = 1 - 10
    operator   = '<' | '=' | '>' | '<=' | '>=' | '&' | '^'
    value      = 0 - 4294967295
    offset     = -65535 to 65535
\end{verbatim}
\normalsize

\begin{tabular}{| l | p{4.5in} |}
\hline
{\bf Option} & {\bf Description}\\
\hline
\hline
\texttt{bytes\_to\_convert} & 

Number of bytes to pick up from the packet. The allowed values are 1 to 10 when 
used without \texttt{dce}. If used with \texttt{dce} allowed values are 1, 2 and 4.\\

\hline
\texttt{operator} & Operation to perform to test the value:
\begin{itemize}
\item \textless{} - less than
\item \textgreater{} - greater than
\item \textless{}= - less than or equal
\item \textgreater{}= - greater than or equal
\item = - equal
\item \& - bitwise AND
\item \textasciicircum{} - bitwise OR
\end{itemize}\\
\hline
\texttt{value} & Value to test the converted value against\\
\hline
\texttt{offset} & Number of bytes into the payload to start processing\\
\hline
\texttt{relative} & Use an offset relative to last pattern match\\
\hline
\texttt{endian} & Endian type of the number being read: 
\begin{itemize}
\item \texttt{big} - Process data as big endian (default)
\item \texttt{little} - Process data as little endian
\end{itemize}\\
\hline
\texttt{string} & Data is stored in string format in packet\\
\hline
\texttt{number type} & Type of number being read:
\begin{itemize}
\item \texttt{hex} - Converted string data is represented in hexadecimal
\item \texttt{dec} - Converted string data is represented in decimal
\item \texttt{oct} - Converted string data is represented in octal
\end{itemize}\\

\hline
\texttt{dce} &

Let the DCE/RPC 2 preprocessor determine the byte order of the value to be
converted.  See section \ref{sub:dcerpc2} for a description and examples
(\ref{dcerpc2:byte_test_jump} for quick reference).\\

\hline
\end{tabular}

Any of the operators can also include \emph{!} to check if the operator is not
true.  If \emph{!} is specified without an operator, then the operator is set
to \emph{=}.

\begin{note}

Snort uses the C operators for each of these operators.  If the \emph{\&}
operator is used, then it would be the same as using \emph{if (data \& value)
\{ do\_something();\} }

\end{note}

\subsubsection{Examples}

\begin{verbatim}
    alert udp $EXTERNAL_NET any -> $HOME_NET any \
        (msg:"AMD procedure 7 plog overflow"; \
        content:"|00 04 93 F3|"; \
        content:"|00 00 00 07|"; distance:4; within:4; \
        byte_test:4, >, 1000, 20, relative;)
    
    alert tcp $EXTERNAL_NET any -> $HOME_NET any \
        (msg:"AMD procedure 7 plog overflow"; \
        content:"|00 04 93 F3|"; \
        content:"|00 00 00 07|"; distance:4; within:4; \
        byte_test:4, >, 1000, 20, relative;)
    
    alert udp any any -> any 1234 \
        (byte_test:4, =, 1234, 0, string, dec; \
        msg:"got 1234!";)
    
    alert udp any any -> any 1235 \
        (byte_test:3, =, 123, 0, string, dec; \
        msg:"got 123!";)
    
    alert udp any any -> any 1236 \
        (byte_test:2, =, 12, 0, string, dec; \
        msg:"got 12!";)
    
    alert udp any any -> any 1237 \
        (byte_test:10, =, 1234567890, 0, string, dec; \
        msg:"got 1234567890!";)
    
    alert udp any any -> any 1238 \
        (byte_test:8, =, 0xdeadbeef, 0, string, hex; \
        msg:"got DEADBEEF!";)
\end{verbatim}

\subsection{byte\_jump}
\label{sub:byte_jump}

The \texttt{byte\_jump} keyword allows rules to be written for length encoded
protocols trivially. By having an option that reads the length of a portion of
data, then skips that far forward in the packet, rules can be written that skip
over specific portions of length-encoded protocols and perform detection in
very specific locations.

The \texttt{byte\_jump} option does this by reading some number of bytes,
convert them to their numeric representation, move that many bytes forward and
set a pointer for later detection.  This pointer is known as the detect offset
end pointer, or doe\_ptr.

For a more detailed explanation, please read Section \ref{testing numerical
values}.

\subsubsection{Format}

\begin{verbatim}
    byte_jump:<bytes_to_convert>, <offset> \
        [, relative][, multiplier <mult_value>][, <endian>][, string, <number_type>]\
        [, align][, from_beginning][, post_offset <adjustment value>][, dce];

    bytes       = 1 - 10
    offset      = -65535 to 65535
    mult_value  = 0 - 65535
    post_offset = -65535 to 65535
    
\end{verbatim}

\begin{tabular}{| l | p{4.5in} |}
\hline
{\bf Option} & {\bf Description}\\
\hline
\hline
\texttt{bytes\_to\_convert} &

Number of bytes to pick up from the packet. The allowed values are 1 to 10 when
used without \texttt{dce}. If used with \texttt{dce} allowed values are 1, 2 and 4.\\

\hline
\texttt{offset} & Number of bytes into the payload to start processing\\
\hline
\texttt{relative} & Use an offset relative to last pattern match\\
\hline
\texttt{multiplier $<$value$>$} & Multiply the number of calculated bytes by
\texttt{$<$value$>$} and skip forward that number of bytes.\\
\hline
\texttt{big} & Process data as big endian (default)\\
\hline
\texttt{little} & Process data as little endian\\
\hline
\texttt{string} & Data is stored in string format in packet\\
\hline
\texttt{hex} & Converted string data is represented in hexadecimal\\
\hline
\texttt{dec} & Converted string data is represented in decimal\\
\hline
\texttt{oct} & Converted string data is represented in octal\\
\hline
\texttt{align} & Round the number of converted bytes up to the next 32-bit boundary\\
\hline
\texttt{from\_beginning} & Skip forward from the beginning of the packet
payload instead of from the current position in the packet.\\
\hline
\texttt{post\_offset $<$value$>$} & Skip forward or backwards (positive of
negative value) \texttt{by $<$value$>$} number of bytes after the other jump
options have been applied.\\
\hline
\texttt{dce} & Let the DCE/RPC 2 preprocessor determine the byte order of the
value to be converted.  See section \ref{sub:dcerpc2} for a description and
examples (\ref{dcerpc2:byte_test_jump} for quick reference).\\ \hline
\end{tabular}

\subsubsection{Example}

\begin{verbatim}
    alert udp any any -> any 32770:34000 (content:"|00 01 86 B8|"; \
        content:"|00 00 00 01|"; distance:4; within:4; \
        byte_jump:4, 12, relative, align; \
        byte_test:4, >, 900, 20, relative; \
        msg:"statd format string buffer overflow";)
\end{verbatim}

\subsection{byte\_extract}
\label{byte_extract}

The \texttt{byte\_extract} keyword is another useful option for writing rules
against length-encoded protocols. It reads in some number of bytes from the
packet payload and saves it to a variable. These variables can be referenced
later in the rule, instead of using hard-coded values.

\begin{note}

Only two \texttt{byte\_extract} variables may be created per rule. They can be
re-used in the same rule any number of times.

\end{note}

\subsubsection{Format}

\begin{verbatim}
    byte_extract:<bytes_to_extract>, <offset>, <name> \
        [, relative][, multiplier <multiplier value>][, <endian>]\ 
        [, string][, hex][, dec][, oct][, align <align value>][, dce]
\end{verbatim}

\begin{tabular}{| l | p{4.5in} |}
\hline
{\bf Option} & {\bf Description}\\
\hline
\hline
\texttt{bytes\_to\_convert} & Number of bytes to pick up from the packet\\
\hline
\texttt{offset} & Number of bytes into the payload to start processing\\
\hline
\texttt{name} & Name of the variable. This will be used to reference the
variable in other rule options.\\
\hline
\texttt{relative} & Use an offset relative to last pattern match\\
\hline
\texttt{multiplier $<$value$>$} & Multiply the bytes read from the packet by
\texttt{$<$value$>$} and save that number into the variable.\\
\hline
\texttt{big} & Process data as big endian (default)\\
\hline
\texttt{little} & Process data as little endian\\
\hline
\texttt{dce} & Use the DCE/RPC 2 preprocessor to determine the byte-ordering.
The DCE/RPC 2 preprocessor must be enabled for this option to work.\\
\hline
\texttt{string} & Data is stored in string format in packet\\
\hline
\texttt{hex} & Converted string data is represented in hexadecimal\\
\hline
\texttt{dec} & Converted string data is represented in decimal\\
\hline
\texttt{oct} & Converted string data is represented in octal\\
\hline
\texttt{align $<$value$>$} & Round the number of converted bytes up to the next
\texttt{$<$value$>$-byte} boundary. \texttt{$<$value$>$} may be \texttt{2} or
\texttt{4}.\\
\hline
\end{tabular}

\subsubsection{Other options which use byte\_extract variables}

A \texttt{byte\_extract} rule option detects nothing by itself. Its use is in
extracting packet data for use in other rule options. Here is a list of places
where \texttt{byte\_extract} variables can be used:

\begin{tabular}{| l | p{4.5in} |}
\hline
{\bf Rule Option} & {\bf Arguments that Take Variables}\\
\hline
\hline
\texttt{content}/\texttt{uricontent} & \texttt{offset}, \texttt{depth},
\texttt{distance}, \texttt{within}\\
\hline
\texttt{byte\_test} & \texttt{offset}, \texttt{value}\\
\hline
\texttt{byte\_jump} & \texttt{offset}\\
\hline
\texttt{isdataat} & \texttt{offset}\\
\hline
\end{tabular}

\subsubsection{Examples}

This example uses two variables to:
\begin{itemize}
\item Read the offset of a string from a byte at offset 0.
\item Read the depth of a string from a byte at offset 1.
\item Use these values to constrain a pattern match to a smaller area.
\end{itemize}

\begin{verbatim}
    alert tcp any any -> any any (byte_extract:1, 0, str_offset; \
        byte_extract:1, 1, str_depth; \
        content:"bad stuff"; offset:str_offset; depth:str_depth; \
        msg:"Bad Stuff detected within field";)
\end{verbatim}

\subsection{ftpbounce}

The ftpbounce keyword detects FTP bounce attacks.

\subsubsection{Format}
                                                                                
\begin{verbatim}
    ftpbounce;
\end{verbatim}

\subsubsection{Example}

\begin{verbatim}
    alert tcp $EXTERNAL_NET any -> $HOME_NET 21 (msg:"FTP PORT bounce attempt"; \
        flow:to_server,established; content:"PORT"; nocase; ftpbounce; pcre:"/^PORT/smi";\
        classtype:misc-attack; sid:3441; rev:1;)
\end{verbatim}

\subsection{asn1}
\label{asn1}

The ASN.1 detection plugin decodes a packet or a portion of a packet, and looks
for various malicious encodings.

Multiple options can be used in an 'asn1' option and the implied logic is
boolean OR.  So if any of the arguments evaluate as true, the whole option
evaluates as true.

The ASN.1 options provide programmatic detection capabilities as well as some
more dynamic type detection.  If an option has an argument, the option and the
argument are separated by a space or a comma.  The preferred usage is to use a
space between option and argument.

\subsubsection{Format}

\begin{verbatim}
    asn1:[bitstring_overflow][, double_overflow][, oversize_length <value>][, absolute_offset <value>|relative_offset <value>];
\end{verbatim}

\begin{tabular}{| l | p{4.5in} |}
\hline
{\bf Option} & {\bf Description}\\
\hline

\hline
\texttt{bitstring\_overflow} &

Detects invalid bitstring encodings that are known to be remotely exploitable.
\\

\hline
\texttt{double\_overflow} &

Detects a double ASCII encoding that is larger than a standard buffer.  This is
known to be an exploitable function in Microsoft, but it is unknown at this
time which services may be exploitable. \\

\hline
\texttt{oversize\_length $<$value$>$} &

Compares ASN.1 type lengths with the supplied argument.  The syntax looks like,
``oversize\_length 500''.  This means that if an ASN.1 type is greater than
500, then this keyword is evaluated as true.  This keyword must have one
argument which specifies the length to compare against. \\

\hline
\texttt{absolute\_offset $<$value$>$} &

This is the absolute offset from the beginning of the packet.  For example, if
you wanted to decode snmp packets, you would say ``absolute\_offset 0''.
\texttt{absolute\_offset} has one argument, the offset value.  Offset may be
positive or negative. \\  

\hline
\texttt{relative\_offset $<$value$>$} &

This is the relative offset from the last content match, pcre or byte\_jump.
\texttt{relative\_offset} has one argument, the offset number.  So if you
wanted to start decoding an ASN.1 sequence right after the content ``foo'',
you would specify \verb!'content:"foo"; asn1:bitstring_overflow, relative_offset 0'!.
Offset values may be positive or negative. \\ 

\hline
\end{tabular}

\subsubsection{Examples}

\begin{verbatim}
    alert udp any any -> any 161 (msg:"Oversize SNMP Length"; \
        asn1:oversize_length 10000, absolute_offset 0;)
        
    alert tcp any any -> any 80 (msg:"ASN1 Relative Foo"; content:"foo"; \
        asn1:bitstring_overflow, relative_offset 0;)
\end{verbatim}

\subsection{cvs}

The CVS detection plugin aids in the detection of: Bugtraq-10384,
CVE-2004-0396: "Malformed Entry Modified and Unchanged flag insertion".
Default CVS server ports are 2401 and 514 and are included in the default ports
for stream reassembly.

\begin{note}

This plugin cannot do detection over encrypted sessions, e.g. SSH (usually port
22).

\end{note}

\subsubsection{Format}

\begin{verbatim}
    cvs:<option>;
\end{verbatim}

\begin{tabular}{| l | p{4.5in} |}
\hline
{\bf Option} & {\bf Description}\\
\hline

\hline
\texttt{invalid-entry} &

Looks for an invalid Entry string, which is a way of causing a heap overflow
(see CVE-2004-0396) and bad pointer dereference in versions of CVS 1.11.15 and
before.\\

\hline
\end{tabular}

\subsubsection{Examples}

\begin{verbatim}
    alert tcp any any -> any 2401 (msg:"CVS Invalid-entry"; \
        flow:to_server,established; cvs:invalid-entry;)
\end{verbatim}

\subsection{dce\_iface}

See the DCE/RPC 2 Preprocessor section \ref{sub:dcerpc2} for a description and
examples of using this rule option.

\subsection{dce\_opnum}

See the DCE/RPC 2 Preprocessor section \ref{sub:dcerpc2} for a description and
examples of using this rule option.

\subsection{dce\_stub\_data}

See the DCE/RPC 2 Preprocessor section \ref{sub:dcerpc2} for a description and
examples of using this rule option.

\subsection{sip\_method}

See the SIP Preprocessor section \ref{sub:sip} for a description and
examples of using this rule option.

\subsection{sip\_stat\_code}

See the SIP Preprocessor section \ref{sub:sip} for a description and
examples of using this rule option.

\subsection{sip\_header}

See the SIP Preprocessor section \ref{sub:sip} for a description and
examples of using this rule option.

\subsection{sip\_body}

See the SIP Preprocessor section \ref{sub:sip} for a description and
examples of using this rule option.

\subsection{gtp\_type}

See the GTP Preprocessor section \ref{sub:gtp} for a description and
examples of using this rule option.

\subsection{gtp\_info}

See the GTP Preprocessor section \ref{sub:gtp} for a description and
examples of using this rule option.

\subsection{gtp\_version}

See the GTP Preprocessor section \ref{sub:gtp} for a description and
examples of using this rule option.

%\subsection{content-list}
%
%The content-list keyword is broken and should not be used.

%The content-list keyword allows multiple content strings to be specified
%in the place of a single content option. The patterns to be searched
%for must each be on a single line of content-list file as shown in
%Figure \ref{content-list example}, but they are treated otherwise
%identically to content strings specified as an argument to a standard
%content directive. This option is the basis for the react keyword.
%
%%
%\begin{figure}
%\begin{verbatim}
%# adult sites 
%\end{verbatim}
%\caption{\label{content-list example}Content-list adults
%file example}
%\end{figure}
%
%
%\subsubsection{Format}
%
%\begin{verbatim}
%content-list: <file_name>;
%\end{verbatim}

\subsection{ssl\_version}

See the SSL/TLS Preprocessor section \ref{sub:SSL/TLS} for a description and examples of
using this rule option.

\subsection{ssl\_state}

See the SSL/TLS Preprocessor section \ref{sub:SSL/TLS} for a description and examples of
using this rule option.

\subsection{Payload Detection Quick Reference}
\begin{center}
\begin{longtable}[h]{| p{1in} | p{4.5in} |}
\caption{Payload detection rule option keywords} \\

\hline
Keyword & Description \\
\hline

\hline
\texttt{content} &

The content keyword allows the user to set rules that search for specific
content in the packet payload and trigger response based on that data. \\

\hline
\texttt{rawbytes} &

The rawbytes keyword allows rules to look at the raw packet data, ignoring any
decoding that was done by preprocessors. \\

\hline
\texttt{depth} &

The depth keyword allows the rule writer to specify how far into a packet Snort
should search for the specified pattern. \\

\hline
\texttt{offset} &

The offset keyword allows the rule writer to specify where to start searching
for a pattern within a packet. \\

\hline
\texttt{distance} &

The distance keyword allows the rule writer to specify how far into a packet
Snort should ignore before starting to search for the specified pattern
relative to the end of the previous pattern match. \\

\hline
\texttt{within} &

The within keyword is a content modifier that makes sure that at most N bytes
are between pattern matches using the content keyword. \\

\hline
\texttt{uricontent} &

The uricontent keyword in the Snort rule language searches the normalized
request URI field. \\

\hline
\texttt{isdataat} &

The isdataat keyword verifies that the payload has data at a specified
location. \\

\hline
\texttt{pcre} &

The pcre keyword allows rules to be written using perl compatible regular
expressions. \\

\hline
\texttt{byte\_test} &

The byte\_test keyword tests a byte field against a specific value (with
operator). \\

\hline
\texttt{byte\_jump} &

The byte\_jump keyword allows rules to read the length of a portion of data,
then skip that far forward in the packet. \\

\hline
\texttt{ftpbounce} &

The ftpbounce keyword detects FTP bounce attacks. \\

%\hline
%\texttt{content-list} &
%The content-list keyword is broken and should not be used. \\

\hline
\texttt{asn1} &

The asn1 detection plugin decodes a packet or a portion of a packet, and looks
for various malicious encodings. \\

\hline
\texttt{cvs} &

The cvs keyword detects invalid entry strings. \\

\hline
\texttt{dce\_iface} &

See the DCE/RPC 2 Preprocessor section \ref{dcerpc2:dce_iface}. \\

\hline
\texttt{dce\_opnum} &

See the DCE/RPC 2 Preprocessor section \ref{dcerpc2:dce_opnum}. \\

\hline
\texttt{dce\_stub\_data} &

See the DCE/RPC 2 Preprocessor section \ref{dcerpc2:dce_stub_data}. \\

\hline
\texttt{sip\_method} &

See the SIP Preprocessor section \ref{sub:sip}. \\

\hline
\texttt{sip\_stat\_code} &

See the SIP Preprocessor section \ref{sub:sip}.\\

\hline
\texttt{sip\_header} &

See the SIP Preprocessor section \ref{sub:sip}.\\

\hline
\texttt{sip\_body} &

See the SIP Preprocessor section \ref{sub:sip}.\\

\hline
\texttt{gtp\_type} &

See the GTP Preprocessor section \ref{sub:gtp}.\\

\hline
\texttt{gtp\_info} &

See the GTP Preprocessor section \ref{sub:gtp}.\\

\hline
\texttt{gtp\_version} &

See the GTP Preprocessor section \ref{sub:gtp}.\\

\hline
\end{longtable}
\end{center}

\section{Non-Payload Detection Rule Options}
\subsection{fragoffset}
\label{fragoffset section}

The fragoffset keyword allows one to compare the IP fragment offset field
against a decimal value. To catch all the first fragments of an IP session, you
could use the fragbits keyword and look for the More fragments option in
conjunction with a fragoffset of 0.

\subsubsection{Format}

\begin{verbatim}
    fragoffset:[!|<|>]<number>;
\end{verbatim}

\subsubsection{Example}

\begin{verbatim}
    alert ip any any -> any any \
        (msg:"First Fragment"; fragbits:M; fragoffset:0;)
\end{verbatim}

\subsection{ttl}

The ttl keyword is used to check the IP time-to-live value.  This option
keyword was intended for use in the detection of traceroute attempts. This
 keyword takes numbers from 0 to 255.

\subsubsection{Format}

\begin{verbatim}
    ttl:[<, >, =, <=, >=]<number>;
    ttl:[<number>]-[<number>];
\end{verbatim}

\subsubsection{Example}

This example checks for a time-to-live value that is less than 3.

\begin{verbatim}
    ttl:<3;
\end{verbatim}

This example checks for a time-to-live value that between 3 and 5.

\begin{verbatim}
    ttl:3-5;
\end{verbatim}

This example checks for a time-to-live value that between 0 and 5.

\begin{verbatim}
    ttl:-5;
\end{verbatim}

This example checks for a time-to-live value that between 5 and 255.

\begin{verbatim}
    ttl:5-;
\end{verbatim}

Few other examples are as follows:

\begin{verbatim}
    ttl:<=5;
    ttl:>=5;
    ttl:=5;
\end{verbatim}

The following examples are NOT allowed by ttl keyword:

\begin{verbatim}
    ttl:=>5;
    ttl:=<5;
    ttl:5-3;
\end{verbatim}



\subsection{tos}

The tos keyword is used to check the IP TOS field for a specific value. 

\subsubsection{Format}

\begin{verbatim}
    tos:[!]<number>;
\end{verbatim}

\subsubsection{Example}

This example looks for a tos value that is not 4

\begin{verbatim}
    tos:!4;
\end{verbatim}

\subsection{id}

The id keyword is used to check the IP ID field for a specific value.  Some
tools (exploits, scanners and other odd programs) set this field specifically
for various purposes, for example, the value 31337 is very popular with some
hackers. 

\subsubsection{Format}

\begin{verbatim}
    id:<number>;
\end{verbatim}

\subsubsection{Example}

This example looks for the IP ID of 31337.

\begin{verbatim}
    id:31337;
\end{verbatim}

\subsection{ipopts}

The ipopts keyword is used to check if a specific IP option is present.

The following options may be checked:

\begin{description}
\item [rr] - Record Route 
\item [eol] - End of list 
\item [nop] - No Op 
\item [ts] - Time Stamp 
\item [sec] - IP Security
\item [esec] - IP Extended Security
\item [lsrr] - Loose Source Routing
\item [lsrre] - Loose Source Routing (For MS99-038 and CVE-1999-0909) 
\item [ssrr] - Strict Source Routing 
\item [satid] - Stream identifier
\item [any] - any IP options are set
\end{description}

The most frequently watched for IP options are strict and loose source routing
which aren't used in any widespread internet applications.

\subsubsection{Format}

\begin{verbatim}
    ipopts:<rr|eol|nop|ts|sec|esec|lsrr|lsrre|ssrr|satid|any>;
\end{verbatim}

\subsubsection{Example}

This example looks for the IP Option of Loose Source Routing.

\begin{verbatim}
    ipopts:lsrr;
\end{verbatim}

\subsubsection{Warning}

Only a single ipopts keyword may be specified per rule.

\subsection{fragbits}

The \texttt{fragbits} keyword is used to check if fragmentation and reserved
bits are set in the IP header.

The following bits may be checked:

\begin{description}
\item [M] - More Fragments
\item [D] - Don't Fragment
\item [R] - Reserved Bit
\end{description}

The following modifiers can be set to change the match criteria:

\begin{description}
\item [+] match on the specified bits, plus any others
\item [*] match if any of the specified bits are set
\item [!] match if the specified bits are not set
\end{description}

\subsubsection{Format}

\begin{verbatim}
    fragbits:[+*!]<[MDR]>;
\end{verbatim}

\subsubsection{Example}

This example checks if the More Fragments bit and the Do not Fragment bit are
set.

\begin{verbatim}
    fragbits:MD+;
\end{verbatim}

\subsection{dsize}

The dsize keyword is used to test the packet payload size.  This may be used to
check for abnormally sized packets that might cause buffer overflows.

\subsubsection{Format}

\begin{verbatim}
    dsize:min<>max;
    dsize:[<|>]<number>;
\end{verbatim}

\subsubsection{Example}

This example looks for a dsize that is between 300 and 400 bytes (inclusive).

\begin{verbatim}
    dsize:300<>400;
\end{verbatim}

\subsubsection{Warning}

Note that segmentation makes dsize less reliable for TCP based protocols such
as HTTP.  Furthermore, dsize will fail on stream rebuilt packets, regardless of
the size of the payload, unless protocol aware flushing (PAF) marks this packet
as the start of a message.

\subsection{flags}

The flags keyword is used to check if specific TCP flag bits are present.

The following bits may be checked:

\begin{description}
\item [F] - FIN - Finish (LSB in TCP Flags byte)
\item [S] - SYN - Synchronize sequence numbers
\item [R] - RST - Reset
\item [P] - PSH - Push
\item [A] - ACK - Acknowledgment
\item [U] - URG - Urgent
\item [C] - CWR - Congestion Window Reduced (MSB in TCP Flags byte)
\item [E] - ECE - ECN-Echo (If SYN, then ECN capable.  Else, CE flag in IP header is set)
\item [0] - No TCP Flags Set
\end{description}

The following modifiers can be set to change the match criteria:

\begin{description}
\item [+] - match on the specified bits, plus any others
\item [*] - match if any of the specified bits are set
\item [!] - match if the specified bits are not set
\end{description}

To handle writing rules for session initiation packets such as ECN where a SYN
packet is sent with CWR and ECE set, an option mask
may be specified by preceding the mask with a comma. A rule could check for a flags value of S,CE if one wishes to
find packets with just the syn bit, regardless of the values of the reserved
bits.

\subsubsection{Format}

\begin{verbatim}
    flags:[!|*|+]<FSRPAUCE0>[,<FSRPAUCE>];
\end{verbatim}

\subsubsection{Example}

This example checks if just the SYN and the FIN bits are set, ignoring CWR (reserved
bit 1) and ECN (reserved bit 2).

\begin{verbatim}
    alert tcp any any -> any any (flags:SF,CE;)
\end{verbatim}
	
\begin{note}

The reserved bits '1' and '2' have been replaced with 'C' and 'E', respectively, 
to match RFC 3168, "The Addition of Explicit Congestion Notification (ECN) to IP".
The old values of '1' and '2' are still valid for the \texttt{flag} keyword, but
are now deprecated.

\end{note}
\subsection{flow}
\label{flow section}

The flow keyword is used in conjunction with session tracking (see Section
\ref{session section}). It allows rules to only apply to certain directions of
the traffic flow. 

This allows rules to only apply to clients or servers. This allows packets
related to \$HOME\_NET clients viewing web pages to be distinguished from
servers running in the \$HOME\_NET.

The established keyword will replace the \texttt{flags:+A} used in many places
to show established TCP connections.

\subsubsection*{Options}

\begin{tabular}{| l | p{4.5in} |}
\hline
{\bf Option} & {\bf Description}\\
\hline
\hline
\texttt{to\_client} & Trigger on server responses from A to B\\
\hline
\texttt{to\_server} & Trigger on client requests from A to B\\
\hline
\texttt{from\_client} & Trigger on client requests from A to B\\
\hline
\texttt{from\_server} & Trigger on server responses from A to B\\
\hline
\texttt{established} & Trigger only on established TCP connections\\
\hline
\texttt{not\_established} & Trigger only when no TCP connection is established\\
\hline
\texttt{stateless} & Trigger regardless of the state of the stream processor
(useful for packets that are designed to cause machines to crash)\\
\hline
\texttt{no\_stream} & Do not trigger on rebuilt stream packets (useful 
for dsize and stream5)\\
\hline
\texttt{only\_stream} & Only trigger on rebuilt stream
packets\\
\hline
\texttt{no\_frag} & Do not trigger on rebuilt frag packets\\
\hline
\texttt{only\_frag} & Only trigger on rebuilt frag packets\\
\hline
\end{tabular}

\subsubsection{Format}

\begin{verbatim}
    flow:[(established|not_established|stateless)] 
        [,(to_client|to_server|from_client|from_server)]
        [,(no_stream|only_stream)]
        [,(no_frag|only_frag)];
\end{verbatim}

\subsubsection{Examples}

\begin{verbatim}
    alert tcp !$HOME_NET any -> $HOME_NET 21 (msg:"cd incoming detected"; \
        flow:from_client; content:"CWD incoming"; nocase;)

    alert tcp !$HOME_NET 0 -> $HOME_NET 0 (msg:"Port 0 TCP traffic"; \
        flow:stateless;)
\end{verbatim}

\subsection{flowbits}
\label{flowbits}

The \texttt{flowbits} keyword is used in conjunction with conversation tracking
from the Session preprocessor (see Section\ref{session section}).  It allows
rules to track states during a transport protocol session.  The flowbits option
is most useful for TCP sessions, as it allows rules to generically track the
state of an application protocol.

There are several keywords associated with flowbits. Most of the options need a
user-defined name for the specific state that is being checked.  Some keyword 
uses group name. When no group name is specified the flowbits will belong to a 
default group. A particular flowbit can belong to more than one group.
Flowbit name and group name should be limited to any alphanumeric string 
including periods, dashes, and underscores.

\subsubsection{General Format}

\begin{verbatim}
    flowbits:[set|setx|unset|toggle|isset|isnotset|noalert|reset][, <bits/bats>][, <GROUP_NAME>];
    bits ::= bit[|bits]
    bats ::= bit[&bats]
\end{verbatim}

\begin{tabular}{| l | p{4.5in} |}
\hline
{\bf Option} & {\bf Description}\\
\hline
\hline
\texttt{set} & Sets the specified states for the current flow and assign them to a group when a GROUP\_NAME is specified.\\
\hline
\texttt{setx} & Sets the specified states for the current flow and clear other states in the group\\
\hline
\texttt{unset} & Unsets the specified states for the current flow.\\
\hline
\texttt{toggle} & For every state specified, sets the specified state if the state is unset and unsets it if the state is set.\\
\hline
\texttt{isset} & Checks if the specified states are set.\\
\hline
\texttt{isnotset} & Checks if the specified states are not set.\\
\hline
\texttt{noalert} & Cause the rule to not generate an alert, regardless of the rest of the detection options.\\
\hline
\texttt{reset} & Reset all states on a given flow.\\
\hline
\end{tabular}

\subsubsection{set}
This keyword sets bits to group for a particular flow. When no group specified, set the default group. This keyword always returns true.

\begin{verbatim}
Syntax:
    flowbits:set,bats[,group]
Usage:  
    flowbits:set,bit1,doc;
    flowbits:set,bit2&bit3,doc;
    First rule sets bit1 in doc group, second rule sets bit2 and bit3 in doc group. 
    So doc group has bit 1, bit2 and bit3 set  
\end{verbatim}

\subsubsection{setx}
This keyword sets bits to group exclusively. This clears other bits in group. 
Group must present.This keyword always returns true.
\begin{verbatim}
Syntax:
    flowbits:setx,bats,group
Usage:  
    flowbits: setx, bit1, doc
    flowbits: setx, bit2&bit3, doc
    First rule sets bit1 in doc group, second rule sets bit2 and bit3 in doc group.
    So doc group has bit2 and bit3 set, because bit1 is cleared by rule 2.
 \end{verbatim}
  
\subsubsection{unset}
This keyword clears bits specified for a particular flow or clears all bits in the 
group (Group must present). This keyword always returns true.

\begin{verbatim}
Syntax:
   flowbits:unset,bats
   flowbits:unset,all,group
   
Usage:  
   flowbits: unset, bit1  
   Clear bit1.
   
   flowbits: unset, bit1&bit2
   Clear bit1 and bit2
   
   flowbits: unset, all, doc
   This clears all bits in the doc group.
\end{verbatim}


\subsubsection{toggle}
If flowbit is set, unset it. If it is unset, set it. Toggle every bit specified or
toggle all the bits in group (Group must be present). This keyword always returns true.

\begin{verbatim}
Syntax:  
   flowbits:toggle,bats
   flowbits:toggle,all,group
   
Usage:
   flowbits: toggle, bit1&bit2
   If bit1 is 0 and bit2 is 1 before, after this rule, bit1 is 1 and bit2 is 0.
   
   flowbits:toggle,all,doc
   Toggle all the bits in group doc as described above. 
\end{verbatim}

\subsubsection{isset}
This keyword checks a bit or several bits to see if it is set.  It returns true or
false based on the following syntax.

\begin{verbatim}
Syntax:  
   flowbits:isset, bits => Check whether any bit is set
   flowbits:isset, bats => Check whether all bits are set
   flowbits:isset, any, group => Check whether any bit in the group is set.
   flowbits:isset, all, group => Check whether all bits in the group are set. 
   
Usage
   flowbits:isset, bit1|bit2 => If either bit1 or bit2 is set, return true
   flowbits:isset, bit1&bit2 => If both bit1 and bit2 are set, return true, otherwise false
   flowbits:isset, any, doc => If any bit in group doc is set, return true
   flowbits:isset, all, doc => If all the bits in doc group are set, return true 
\end{verbatim}

\subsubsection{isnotset}
This keyword is the reverse of isset.  It returns true if isset is false, it returns
false if isset is true. Isnotset works on the final result, not on individual bits.

\begin{verbatim}
Syntax:  
   flowbits:isnotset, bits => Check whether not any bit is set
   flowbits:isnotset, bats => Check whether not all bits are set
   flowbits:isnotset, any, group => Check whether not bit in the group is set.
   flowbits:isnotset, all, group => Check whether not all bits in the group are set. 
   
Usage
   flowbits:isnotset, bit1|bit2 => If either bit1 or bit2 is set, return true
   flowbits:isnotset, bit1&bit2 => If both bit1 and bit2 are set, return true, otherwise false
   flowbits:isnotset, any, doc => If any bit in group doc is set, return true
   flowbits:isnotset, all, doc => If all the bits in doc group are set, return true 
\end{verbatim}


\subsubsection{noalert}
This keyword always returns false.  It allows users to write rules that set,
unset, and toggle bit without generating an alert.  This is most useful
for writing flowbit rules that set bit on normal traffic and significantly reduces 
unwanted alerts.  There is no bit specified with this keyword.

\begin{verbatim}
  flowbits:noalert;
\end{verbatim}

\subsubsection{reset}

This keyword resets all of the states on a given flow if no group specified, otherwise, 
reset all the bits in a group. This always returns true.  There is no bit specified with
 this keyword.

\begin{verbatim}
Syntax: 
   flowbits:reset[,group]

Usage:  
   flowbits:reset => reset all the bits in the flow
   flowbits: reset, doc => reset all the bits in the doc group
\end{verbatim}

\subsubsection{Examples}

\begin{verbatim}
    alert tcp any 143 -> any any (msg:"IMAP login"; 
        content:"OK LOGIN"; flowbits:set,logged_in; 
        flowbits:noalert;)
      
    alert tcp any any -> any 143 (msg:"IMAP LIST"; content:"LIST"; 
        flowbits:isset,logged_in;)
	
\end{verbatim}

\subsection{seq}

The seq keyword is used to check for a specific TCP sequence number.

\subsubsection{Format}

\begin{verbatim}
    seq:<number>;
\end{verbatim}

\subsubsection{Example}

This example looks for a TCP sequence number of 0.

\begin{verbatim}
    seq:0;
\end{verbatim}

\subsection{ack}

The ack keyword is used to check for a specific TCP acknowledge number.

\subsubsection{Format}

\begin{verbatim}
    ack:<number>;
\end{verbatim}

\subsubsection{Example}

This example looks for a TCP acknowledge number of 0.

\begin{verbatim}
    ack:0;
\end{verbatim}

\subsection{window}

The window keyword is used to check for a specific TCP window size.

\subsubsection{Format}

\begin{verbatim}
    window:[!]<number>;
\end{verbatim}

\subsubsection{Example}

This example looks for a TCP window size of 55808.

\begin{verbatim}
    window:55808;
\end{verbatim}

\subsection{itype}

The itype keyword is used to check for a specific ICMP type value.

\subsubsection{Format}

\begin{verbatim}
    itype:min<>max;
    itype:[<|>]<number>;
\end{verbatim}

\subsubsection{Example}

This example looks for an ICMP type greater than 30.

\begin{verbatim}
    itype:>30;
\end{verbatim}

\subsection{icode}

The icode keyword is used to check for a specific ICMP code value.

\subsubsection{Format}

\begin{verbatim}
    icode:min<>max;
    icode:[<|>]<number>;
\end{verbatim}

The \texttt{<>} operator in the first format checks for an ICMP code within a specified range (exclusive).  That is, strictly greater than the min value and strictly less than the max value.  Note that the min value can a \texttt{-1} allowing an ICMP code of zero to be included in the range.

Numerical values are validated with respect to permissible ICMP code values between 0 and 255 and other criteria. 

\begin{verbatim}
    icode:min<>max
        -1 <= min <= 254
         1 <= max <= 256
         (max - min) > 1

    icode:number
        0 <= number <= 255

    icode:<number
        1 <= number <= 256

    icode:>number
        0 <= number <= 254
\end{verbatim}

\subsubsection{Examples}

This example looks for an ICMP code greater than 30.

\begin{verbatim}
    icode:>30;
\end{verbatim}

This example looks for an ICMP code greater than zero and less than 30.

\begin{verbatim}
    icode:-1<>30;
\end{verbatim}

\subsection{icmp\_id}

The icmp\_id keyword is used to check for a specific ICMP ID value.

This is useful because some covert channel programs use static ICMP fields when
they communicate.  This particular plugin was developed to detect the
stacheldraht DDoS agent.

\subsubsection{Format}

\begin{verbatim}
    icmp_id:<number>;
\end{verbatim}

\subsubsection{Example}

This example looks for an ICMP ID of 0.

\begin{verbatim}
    icmp_id:0;
\end{verbatim}

\subsection{icmp\_seq}

The icmp\_seq keyword is used to check for a specific ICMP sequence value.

This is useful because some covert channel programs use static ICMP fields when
they communicate.  This particular plugin was developed to detect the
stacheldraht DDoS agent.

\subsubsection{Format}

\begin{verbatim}
    icmp_seq:<number>;
\end{verbatim}

\subsubsection{Example}

This example looks for an ICMP Sequence of 0.

\begin{verbatim}
    icmp_seq:0;
\end{verbatim}

\subsection{rpc}

The rpc keyword is used to check for a RPC application, version, and procedure
numbers in SUNRPC CALL requests.

Wildcards are valid for both version and procedure numbers by using '*';

\subsubsection{Format}

\begin{verbatim}
    rpc:<application number>, [<version number>|*], [<procedure number>|*]>;
\end{verbatim}

\subsubsection{Example}

The following example looks for an RPC portmap GETPORT request.

\begin{verbatim}
    alert tcp any any -> any 111 (rpc:100000, *, 3;);
\end{verbatim}

\subsubsection{Warning}

Because of the fast pattern matching engine, the RPC keyword is slower than
looking for the RPC values by using normal content matching.

\subsection{ip\_proto}

The ip\_proto keyword allows checks against the IP protocol header.  For a list
of protocols that may be specified by name, see /etc/protocols. 

\subsubsection{Format}

\begin{verbatim}
    ip_proto:[!|>|<] <name or number>;
\end{verbatim}

\subsubsection{Example}

This example looks for IGMP traffic.

\begin{verbatim}
    alert ip any any -> any any (ip_proto:igmp;)
\end{verbatim}

\subsection{sameip}

The sameip keyword allows rules to check if the source ip is the same as the
destination IP.

\subsubsection{Format}

\begin{verbatim}
    sameip;
\end{verbatim}

\subsubsection{Example}

This example looks for any traffic where the Source IP and the Destination IP
is the same.

\begin{verbatim}
    alert ip any any -> any any (sameip;)
\end{verbatim}

\subsection{stream\_reassemble}

The stream\_reassemble keyword allows a rule to enable or disable TCP stream reassembly
on matching traffic.

\begin{note}

The stream\_reassemble option is only available when the Stream preprocessor is
enabled.

\end{note}

\subsubsection{Format}

\begin{verbatim}
    stream_reassemble:<enable|disable>, <server|client|both>[, noalert][, fastpath];
\end{verbatim}

\begin{itemize}
\item The optional \texttt{noalert} parameter causes the rule to not generate
an alert when it matches.
\item The optional \texttt{fastpath} parameter causes Snort to ignore the rest of the
connection.
\end{itemize}

\subsubsection{Example}

For example, to disable TCP reassembly for client traffic when we see a
HTTP 200 Ok Response message, use:

\begin{verbatim}
    alert tcp any 80 -> any any (flow:to_client, established; content:"200 OK";
       stream_reassemble:disable,client,noalert;)
\end{verbatim}

\subsection{stream\_size}

The stream\_size keyword allows a rule to match traffic according to the number
of bytes observed, as determined by the TCP sequence numbers.  

\begin{note}

The stream\_size option is only available when the Stream preprocessor is
enabled.

\end{note}

\subsubsection{Format}

\begin{verbatim}
    stream_size:<server|client|both|either>, <operator>, <number>;
\end{verbatim}

Where the operator is one of the following:

\begin{itemize}
\item $<$ - less than
\item $>$ - greater than
\item = - equal
\item != - not equal
\item $<$= - less than or equal
\item $>$= - greater than or equal
\end{itemize}

\subsubsection{Example}

For example, to look for a session that is less that 6 bytes from the client
side, use:

\begin{verbatim}
    alert tcp any any -> any any (stream_size:client,<,6;)
\end{verbatim}

\subsection{Non-Payload Detection Quick Reference}
\begin{center}
\begin{longtable}[h]{| p{1in} | p{4.5in} |}
\caption{Non-payload detection rule option keywords} \\

\hline
Keyword & Description \\
\hline

\hline
\texttt{fragoffset} &

The fragoffset keyword allows one to compare the IP fragment offset field
against a decimal value. \\

\hline
\texttt{ttl} &

The ttl keyword is used to check the IP time-to-live value. \\

\hline
\texttt{tos} &

The tos keyword is used to check the IP TOS field for a specific value. \\

\hline
\texttt{id} &

The id keyword is used to check the IP ID field for a specific value. \\

\hline
\texttt{ipopts} &

The ipopts keyword is used to check if a specific IP option is present. \\

\hline
\texttt{fragbits} &

The fragbits keyword is used to check if fragmentation and reserved bits are
set in the IP header. \\

\hline
\texttt{dsize} &

The dsize keyword is used to test the packet payload size. \\

\hline
\texttt{flags} &

The flags keyword is used to check if specific TCP flag bits are present. \\

\hline
\texttt{flow} &

The flow keyword allows rules to only apply to certain directions of the
traffic flow. \\

\hline
\texttt{flowbits} &

The flowbits keyword allows rules to track states during a transport protocol
session. \\

\hline
\texttt{seq} &

The seq keyword is used to check for a specific TCP sequence number. \\

\hline
\texttt{ack} &

The ack keyword is used to check for a specific TCP acknowledge number. \\

\hline
\texttt{window} &

The window keyword is used to check for a specific TCP window size. \\

\hline
\texttt{itype} &

The itype keyword is used to check for a specific ICMP type value. \\

\hline
\texttt{icode} &

The icode keyword is used to check for a specific ICMP code value. \\

\hline
\texttt{icmp\_id} &

The icmp\_id keyword is used to check for a specific ICMP ID value. \\

\hline
\texttt{icmp\_seq} &

The icmp\_seq keyword is used to check for a specific ICMP sequence value. \\

\hline
\texttt{rpc} &

The rpc keyword is used to check for a RPC application, version, and procedure
numbers in SUNRPC CALL requests. \\

\hline
\texttt{ip\_proto} &

The ip\_proto keyword allows checks against the IP protocol header. \\

\hline
\texttt{sameip} &

The sameip keyword allows rules to check if the source ip is the same as the
destination IP. \\

\hline
\end{longtable}
\end{center}

\section{Post-Detection Rule Options}
\subsection{logto}

The logto keyword tells Snort to log all packets that trigger this rule to a
special output log file. This is especially handy for combining data from
things like NMAP activity, HTTP CGI scans, etc. It should be noted that this
option does not work when Snort is in binary logging mode.

\subsubsection{Format}

\begin{verbatim}
    logto:"filename";
\end{verbatim}

\subsection{session}

The session keyword is built to extract user data from TCP Sessions.  There are
many cases where seeing what users are typing in telnet, rlogin, ftp, or even
web sessions is very useful. 

There are three available argument keywords for the session rule option:
\texttt{printable}, \texttt{binary}, or \texttt{all}.

The \texttt{printable} keyword only prints out data that the user
would normally see or be able to type.  The \texttt{binary} keyword prints out data in a
binary format. The \texttt{all} keyword substitutes non-printable characters with
their hexadecimal equivalents. 

\subsubsection{Format}

\begin{verbatim}
    session:[printable|binary|all];
\end{verbatim}

\subsubsection{Example}

The following example logs all printable strings in a telnet packet.

\begin{verbatim}
    log tcp any any <> any 23 (session:printable;)
\end{verbatim}

Given an FTP data session on port 12345, this example logs the payload bytes
in binary form.

\begin{verbatim}
    log tcp any any <> any 12345 (metadata:service ftp-data; session:binary;)
\end{verbatim}

\subsubsection{Warnings}

Using the session keyword can slow Snort down considerably, so it should not be
used in heavy load situations. The session keyword is best suited for
post-processing binary (pcap) log files.

The \texttt{binary} keyword does not log any protocol headers below the
application layer, and Stream reassembly will cause duplicate data when
the reassembled packets are logged.

\subsection{resp}

The resp keyword enables an active response that kills the offending session.
Resp can be used in both passive or inline modes.  See \ref{resp section} for
details.

\subsection{react}

The react keyword enables an active response that includes sending a web page
or other content to the client and then closing the connection.  React can be
used in both passive and inline modes.  See \ref{react section} for details.

\subsection{tag}
\label{tag section}

The tag keyword allow rules to log more than just the single packet that
triggered the rule. Once a rule is triggered, additional traffic involving the
source and/or destination host is \emph{tagged}. Tagged traffic is logged to
allow analysis of response codes and post-attack traffic.  \emph{tagged} alerts
will be sent to the same output plugins as the original alert, but it is the
responsibility of the output plugin to properly handle these special alerts.

\subsubsection{Format}

\begin{verbatim}
    tag:host, <count>, <metric>, <direction>;
    tag:session[, <count>, <metric>][, exclusive];
\end{verbatim}

\begin{description}{}
\item [\texttt{type}]~

\begin{itemize}{}
\item \texttt{session} - Log packets in the session that set off the rule 

\item \texttt{host} - Log packets from the host that caused the tag to activate
(uses {[}direction{]} modifier)

\end{itemize}

\item [\texttt{count}]~
\begin{itemize}{}

\item \texttt{<integer>} - Count is specified as a number of units. Units are
specified in the $<$metric$>$ field.

\end{itemize}{}

\item [\texttt{metric}]~

\begin{itemize}{}
\item \texttt{packets} - Tag the host/session for $<$count$>$ packets 
\item \texttt{seconds} - Tag the host/session for $<$count$>$ seconds
\item \texttt{bytes}   - Tag the host/session for $<$count$>$ bytes
\end{itemize}

\item [\texttt{other}]

\begin{itemize}{}

\item \texttt{src} - Tag packets containing the source IP address of the packet
that generated the initial event.  Only relevant if host type is used.

\item \texttt{dst} - Tag packets containing the destination IP address of the
packet that generated the initial event.  Only relevant if host type is used.

\item \texttt{exclusive} - Tag packets only in the first matching session.
Only relevant if session type is used.

\end{itemize}
\end{description}

Note that neither subsequent alerts nor event filters will prevent a tagged
packet from being logged.  Subsequent tagged alerts will cause the limit to
reset.

\begin{verbatim}
    alert tcp any any <> 10.1.1.1 any \
        (flowbits:isnotset,tagged; content:"foobar"; nocase; \
         flowbits:set,tagged; tag:host,600,seconds,src;)
\end{verbatim}

Also note that if you have a tag option in a rule that uses a metric other than
\texttt{packets}, a \texttt{tagged\_packet\_limit} will be used to limit the
number of tagged packets regardless of whether the \texttt{seconds} or
\texttt{bytes} count has been reached.  The default
\texttt{tagged\_packet\_limit} value is 256 and can be modified by using a
config option in your snort.conf file (see Section \ref{Config} on how to use
the \texttt{tagged\_packet\_limit} config option).  You can disable this packet
limit for a particular rule by adding a \texttt{packets} metric to your tag
option and setting its count to 0 (This can be done on a global scale by
setting the \texttt{tagged\_packet\_limit} option in snort.conf to 0).  Doing
this will ensure that packets are tagged for the full amount of
\texttt{seconds} or \texttt{bytes} and will not be cut off by the
\texttt{tagged\_packet\_limit}.  (Note that the \texttt{tagged\_packet\_limit}
was introduced to avoid DoS situations on high bandwidth sensors for tag rules
with a high \texttt{seconds} or \texttt{bytes} counts.)

\begin{verbatim}
    alert tcp 10.1.1.4 any -> 10.1.1.1 any \
        (content:"TAGMYPACKETS"; tag:host,0,packets,600,seconds,src;)
\end{verbatim}

\subsubsection{Example}

This example logs the first 10 seconds or the \texttt{tagged\_packet\_limit}
(whichever comes first) of any telnet session.

\begin{verbatim}
    alert tcp any any -> any 23 (flags:S,CE; tag:session,10,seconds;)
\end{verbatim}

While at least one count and metric is required for tag:host, tag:session
with exclusive without any metrics can be used to get a full session like
this:

\begin{verbatim}
    pass tcp any any -> 192.168.1.1 80 (flags:S; tag:session,exclusive;)
\end{verbatim}

\subsection{activates}

The \texttt{activates} keyword allows the rule writer to specify a rule to add
when a specific network event occurs. See Section \ref{dynamic rules} for more
information.

\subsubsection{Format}

\begin{verbatim}
    activates:1;
\end{verbatim}

\subsection{activated\_by}

The \texttt{activated\_by} keyword allows the rule writer to dynamically enable
a rule when a specific activate rule is triggered.  See Section \ref{dynamic
rules} for more information.

\subsubsection{Format}

\begin{verbatim}
    activated_by:1;
\end{verbatim}

\subsection{count}

The \texttt{count} keyword must be used in combination with the
\texttt{activated\_by} keyword.  It allows the rule writer to specify how many
packets to leave the rule enabled for after it is activated.  See Section
\ref{dynamic rules} for more information.

\subsubsection{Format}

\begin{verbatim}
    activated_by:1; count:50;
\end{verbatim}

\subsection{replace}
\label{replace}

The \texttt{replace} keyword is a feature available in inline mode which will
cause Snort to replace the prior matching content with the given string.  Both
the new string and the content it is to replace must have the same length.  You
can have multiple replacements within a rule, one per content.

\begin{verbatim}
    replace:"<string>";
\end{verbatim}

\subsection{detection\_filter}
\label{detection_filter}

detection\_filter defines a rate which must be exceeded by a source or
destination host before a rule can generate an event.  detection\_filter has
the following format:

\begin{verbatim}
    detection_filter: \
        track <by_src|by_dst>, \
        count <c>, seconds <s>;
\end{verbatim}

\begin{table}[h]
\begin{center}
\begin{tabular}{| p{1in} | p{4.5in} |}

\hline
Option & Description \\
\hline

\hline
\texttt{track by\_src|by\_dst} &

Rate is tracked either by source IP address or destination IP address.  This
means count is maintained for each unique source IP address or each unique
destination IP address.\\

\hline
\texttt{count c} &

The maximum number of rule matches in s seconds allowed before the detection
filter limit to be exceeded.  C must be nonzero.\\

\hline
\texttt{seconds s} &

Time period over which count is accrued.  The value must be nonzero.\\

\hline
\end{tabular}
\end{center}
\end{table}

Snort evaluates a \texttt{detection\_filter} as the last step of the detection
phase, after evaluating all other rule options (regardless of the position of
the filter within the rule source).  At most one \texttt{detection\_filter} is
permitted per rule.

Example - this rule will fire on every failed login attempt from 10.1.2.100
during one sampling period of 60 seconds, after the first 30 failed login
attempts:

\begin{verbatim}
    drop tcp 10.1.2.100 any > 10.1.1.100 22 ( \
        msg:"SSH Brute Force Attempt";
        flow:established,to_server; \
        content:"SSH"; nocase; offset:0; depth:4; \
        detection_filter:track by_src, count 30, seconds 60; \
        sid:1000001; rev:1;)
\end{verbatim}

Since potentially many events will be generated, a \texttt{detection\_filter}
would normally be used in conjunction with an \texttt{event\_filter} to reduce
the number of logged events.

\begin{note}
As mentioned above, Snort evaluates \texttt{detection\_filter} as the last step of 
the detection and not in post-detection.
\end{note}

\subsection{Post-Detection Quick Reference}

\begin{center}
\begin{longtable}[h]{| p{1in} | p{4.5in} |}
\caption{Post-detection rule option keywords} \\

\hline
Keyword & Description \\
\hline

\hline
\texttt{logto} &

The logto keyword tells Snort to log all packets that trigger this rule to a
special output log file. \\

\hline
\texttt{session} &

The session keyword is built to extract user data from TCP Sessions. \\

\hline
\texttt{resp} &

The resp keyword is used attempt to close sessions when an alert is triggered.
\\

\hline
\texttt{react} &

This keyword implements an ability for users to react to traffic that matches a
Snort rule by closing connection and sending a notice. \\

\hline
\texttt{tag} &

The tag keyword allow rules to log more than just the single packet that
triggered the rule. \\

\hline
\texttt{activates} &

This keyword  allows the rule writer to specify a rule to add when a specific
network event occurs. \\

\hline
\texttt{activated\_by} &

This keyword allows the rule writer to dynamically enable a rule when a
specific activate rule is triggered. \\

\hline
\texttt{count} &

This keyword must be used in combination with the \texttt{activated\_by}
keyword.  It allows the rule writer to specify how many packets to leave the
rule enabled for after it is activated. \\

\hline
\texttt{replace} & 

Replace the prior matching content with the given string of the same length.
Available in inline mode only.\\

\hline
\texttt{detection\_filter} & 

Track by source or destination IP address and if the rule otherwise matches
more than the configured rate it will fire.\\

\hline
\end{longtable}
\end{center}

\section{Rule Thresholds}

\begin{note}
Rule thresholds are deprecated and will not be supported in a future release.
Use \texttt{detection\_filter}s (\ref{detection_filter}) within rules, or
\texttt{event\_filter}s (\ref{event_filtering}) as standalone configurations
instead.
\end{note}

\texttt{threshold} can be included as part of a rule, or you can use standalone
thresholds that reference the generator and SID they are applied to. There is
no functional difference between adding a threshold to a rule, or using a
standalone threshold applied to the same rule.   There is a logical difference.
Some rules may only make sense with a threshold.  These should incorporate the
threshold into the rule.  For instance, a rule for detecting a too many login
password attempts may require more than 5 attempts.  This can be done using the
`limit' type of threshold.  It makes sense that the threshold feature is an
integral part of this rule.

\paragraph{Format}

\begin{verbatim}
    threshold: \
        type <limit|threshold|both>, \
        track <by_src|by_dst>, \
        count <c>, seconds <s>;
\end{verbatim}

\begin{table}[h]
\begin{center}
\begin{tabular}{| l | p{4.5in} |}

\hline
\textbf{Option} & \textbf{Description}\\
\hline

\hline
\texttt{type limit|threshold|both} &

type \texttt{limit} alerts on the 1st m events during the time interval, then
ignores events for the rest of the time interval.  Type \texttt{threshold}
alerts every m times we see this event during the time interval.  Type
\texttt{both} alerts once per time interval after seeing m occurrences of the
event, then ignores any additional events during the time interval.\\

\hline
\texttt{track by\_src|by\_dst} &

rate is tracked either by source IP address, or destination IP address.  This
means count is maintained for each unique source IP addresses, or for each
unique destination IP addresses.  Ports or anything else are not tracked.\\

\hline
\texttt{count c} &

number of rule matching in s seconds that will cause \texttt{event\_filter}
limit to be exceeded.  \texttt{c} must be nonzero value.\\

\hline
\texttt{seconds s} &

time period over which \texttt{count} is accrued.  \texttt{s} must be nonzero
value.\\

\hline
\end{tabular}
\end{center}
\end{table}

\subsubsection{Examples}

This rule logs the first event of this SID every 60 seconds.

\begin{verbatim}
    alert tcp $external_net any -> $http_servers $http_ports \
        (msg:"web-misc robots.txt access"; flow:to_server, established; \
        uricontent:"/robots.txt"; nocase; reference:nessus,10302; \
        classtype:web-application-activity; threshold:type limit, track \
        by_src, count 1 , seconds 60; sid:1000852; rev:1;)
\end{verbatim}

This rule logs every 10th event on this SID during a 60 second interval. So if
less than 10 events occur in 60 seconds, nothing gets logged.  Once an event is
logged, a new time period starts for type=threshold.

\begin{verbatim}
    alert tcp $external_net any -> $http_servers $http_ports \
        (msg:"web-misc robots.txt access"; flow:to_server, established; \
        uricontent:"/robots.txt"; nocase; reference:nessus,10302; \
        classtype:web-application-activity; threshold:type threshold, \
        track by_dst, count 10 , seconds 60 ; sid:1000852; rev:1;)
\end{verbatim}

This rule logs at most one event every 60 seconds if at least 10 events on this
SID are fired.

\begin{verbatim}
    alert tcp $external_net any -> $http_servers $http_ports \
        (msg:"web-misc robots.txt access"; flow:to_server, established; \
        uricontent:"/robots.txt"; nocase; reference:nessus,10302; \
        classtype:web-application-activity; threshold:type both, track \
        by_dst, count 10, seconds 60; sid:1000852; rev:1;)
\end{verbatim}

\section{Writing Good Rules}

There are some general concepts to keep in mind when developing Snort rules to
maximize efficiency and speed.

\subsection{Content Matching}

Snort groups rules by protocol (ip, tcp, udp, icmp), then by ports
(ip and icmp use slightly different logic), then by those with \texttt{content}
and those without.  For rules with \texttt{content}, a multi-pattern matcher is
used to select rules that have a chance at matching based on a single content.
Selecting rules for evaluation via this "fast" pattern matcher was found to
increase performance, especially when applied to large rule groups like HTTP.
The longer and more unique a \texttt{content} is, the less likely that rule and
all of its rule options will be evaluated unnecessarily - it's safe to say
there is generally more "good" traffic than "bad".  Rules without
\texttt{content} are always evaluated (relative to the protocol and port group
in which they reside), potentially putting a drag on performance.
While some detection options, such as \texttt{pcre} and \texttt{byte\_test},
perform detection in the payload section of the packet, they are not used by the
fast pattern matching engine.  If at all possible, try and have at least one
\texttt{content} (or \texttt{uricontent}) rule option in your rule.

\subsection{Catch the Vulnerability, Not the Exploit}

Try to write rules that target the vulnerability, instead of a specific
exploit.

For example, look for a the vulnerable command with an argument that is too
large, instead of shellcode that binds a shell.

By writing rules for the vulnerability, the rule is less vulnerable to evasion
when an attacker changes the exploit slightly.

\subsection{Catch the Oddities of the Protocol in the Rule}

Many services typically send the commands in upper case letters.  FTP is a good
example.  In FTP, to send the username, the client sends:

\begin{verbatim}
    user username_here
\end{verbatim}

A simple rule to look for FTP root login attempts could be:

\begin{verbatim}
    alert tcp any any -> any any 21 (content:"user root";)
\end{verbatim}

While it may \emph{seem} trivial to write a rule that looks for the username
root, a good rule will handle all of the odd things that the protocol might
handle when accepting the user command.

For example, each of the following are accepted by most FTP servers:

\begin{verbatim}
    user root
    user root
    user root
    user root
    user<tab>root
\end{verbatim}

To handle all of the cases that the FTP server might handle, the rule needs
more smarts than a simple string match.

A good rule that looks for root login on ftp would be:

\begin{verbatim}
    alert tcp any any -> any 21 (flow:to_server,established; \
        content:"root"; pcre:"/user\s+root/i";)
\end{verbatim}

There are a few important things to note in this rule:

\begin{itemize}

\item The rule has a \emph{flow} option, verifying this is traffic going to the
server on an established session.

\item The rule has a \emph{content} option, looking for \emph{root}, which is
the longest, most unique string in the attack.  This option is added to allow
the fast pattern matcher to select this rule for evaluation only if the
content \emph{root} is found in the payload.

\item The rule has a \emph{pcre} option, looking for user, followed at least
one space character (which includes tab), followed by root, ignoring case.

\end{itemize}

% optimizing :)
\subsection{Optimizing Rules}

The content matching portion of the detection engine has recursion to handle a
few evasion cases.  Rules that are not properly written can cause Snort to
waste time duplicating checks.

The way the recursion works now is if a pattern matches, and if any of the
detection options after that pattern fail, then look for the pattern again
after where it was found the previous time.  Repeat until the pattern is not
found again or the opt functions all succeed.

On first read, that may not sound like a smart idea, but it is needed.  For
example, take the following rule:

\begin{verbatim}
    alert ip any any -> any any (content:"a"; content:"b"; within:1;)
\end{verbatim}

This rule would look for ``a'', immediately followed by ``b''.  Without
recursion, the payload ``aab'' would fail, even though it is obvious that the
payload ``aab'' has ``a'' immediately followed by ``b'', because the first "a"
is not immediately followed by ``b''.

While recursion is important for detection, the recursion implementation is not
very smart.

For example, the following rule options are not optimized:

\begin{verbatim}
    content:"|13|"; dsize:1;
\end{verbatim}

By looking at this rule snippet, it is obvious the rule looks for a packet with
a single byte of 0x13.  However, because of recursion, a packet with 1024 bytes
of 0x13 could cause 1023 too many pattern match attempts and 1023 too many
dsize checks.  Why?  The content 0x13 would be found in the first byte, then
the dsize option would fail, and because of recursion, the content 0x13 would
be found again starting after where the previous 0x13 was found, once it is
found, then check the dsize again, repeating until 0x13 is not found in the
payload again.

Reordering the rule options so that discrete checks (such as dsize) are moved
to the beginning of the rule speed up Snort.  

The optimized rule snipping would be:
\begin{verbatim}
    dsize:1; content:"|13|";
\end{verbatim}

A packet of 1024 bytes of 0x13 would fail immediately, as the dsize check is
the first option checked and dsize is a discrete check without recursion.

The following rule options are discrete and should generally be placed at the
beginning of any rule:

\begin{itemize}
\item \texttt{dsize}
\item \texttt{flags}
\item \texttt{flow}
\item \texttt{fragbits}
\item \texttt{icmp\_id}
\item \texttt{icmp\_seq} 
\item \texttt{icode} 
\item \texttt{id}
\item \texttt{ipopts}
\item \texttt{ip\_proto} 
\item \texttt{itype} 
\item \texttt{seq}
\item \texttt{session} 
\item \texttt{tos}
\item \texttt{ttl}
\item \texttt{ack}
\item \texttt{window}
\item \texttt{resp} 
\item \texttt{sameip}
\end{itemize}

\subsection{Testing Numerical Values}
\label{testing numerical values}

The rule options \emph{byte\_test} and \emph{byte\_jump} were written to
support writing rules for protocols that have length encoded data.  RPC was the
protocol that spawned the requirement for these two rule options, as RPC uses
simple length based encoding for passing data.

In order to understand \emph{why} byte\_test and byte\_jump are useful, let's
go through an exploit attempt against the sadmind service.

This is the payload of the exploit:

\begin{verbatim}
89 09 9c e2 00 00 00 00 00 00 00 02 00 01 87 88  ................
00 00 00 0a 00 00 00 01 00 00 00 01 00 00 00 20  ...............
40 28 3a 10 00 00 00 0a 4d 45 54 41 53 50 4c 4f  @(:.....metasplo
49 54 00 00 00 00 00 00 00 00 00 00 00 00 00 00  it..............
00 00 00 00 00 00 00 00 40 28 3a 14 00 07 45 df  ........@(:...e.
00 00 00 00 00 00 00 00 00 00 00 00 00 00 00 00  ................
00 00 00 00 00 00 00 06 00 00 00 00 00 00 00 00  ................
00 00 00 00 00 00 00 04 00 00 00 00 00 00 00 04  ................
7f 00 00 01 00 01 87 88 00 00 00 0a 00 00 00 04  ................
7f 00 00 01 00 01 87 88 00 00 00 0a 00 00 00 11  ................
00 00 00 1e 00 00 00 00 00 00 00 00 00 00 00 00  ................
00 00 00 00 00 00 00 3b 4d 45 54 41 53 50 4c 4f  .......;metasplo
49 54 00 00 00 00 00 00 00 00 00 00 00 00 00 00  it..............
00 00 00 00 00 00 00 00 00 00 00 00 00 00 00 00  ................
00 00 00 00 00 00 00 00 00 00 00 00 00 00 00 00  ................
00 00 00 00 00 00 00 06 73 79 73 74 65 6d 00 00  ........system..
00 00 00 15 2e 2e 2f 2e 2e 2f 2e 2e 2f 2e 2e 2f  ....../../../../
2e 2e 2f 62 69 6e 2f 73 68 00 00 00 00 00 04 1e  ../bin/sh.......
<snip>
\end{verbatim}

Let's break this up, describe each of the fields, and figure out how to write a
rule to catch this exploit.  

There are a few things to note with RPC:

\begin{itemize}

\item Numbers are written as uint32s, taking four bytes.  The number 26 would
show up as 0x0000001a.

\item Strings are written as a uint32 specifying the length of the string, the
string, and then null bytes to pad the length of the string to end on a 4 byte
boundary.  The string ``bob'' would show up as 0x00000003626f6200.

\end{itemize}

\begin{verbatim}
89 09 9c e2     - the request id, a random uint32, unique to each request
00 00 00 00     - rpc type (call = 0, response = 1)
00 00 00 02     - rpc version (2)
00 01 87 88     - rpc program (0x00018788 = 100232 = sadmind)
00 00 00 0a     - rpc program version (0x0000000a = 10)
00 00 00 01     - rpc procedure (0x00000001 = 1)
00 00 00 01     - credential flavor (1 = auth\_unix)
00 00 00 20     - length of auth\_unix data (0x20 = 32

## the next 32 bytes are the auth\_unix data
40 28 3a 10 - unix timestamp (0x40283a10 = 1076378128 = feb 10 01:55:28 2004 gmt)
00 00 00 0a - length of the client machine name (0x0a = 10)
4d 45 54 41 53 50 4c 4f 49 54 00 00  - metasploit

00 00 00 00 - uid of requesting user (0)
00 00 00 00 - gid of requesting user (0)
00 00 00 00 - extra group ids (0)

00 00 00 00     - verifier flavor (0 = auth\_null, aka none)
00 00 00 00     - length of verifier (0, aka none)
\end{verbatim}

The rest of the packet is the request that gets passed to procedure 1 of
sadmind.

However, we know the vulnerability is that sadmind trusts the uid coming from
the client.  sadmind runs any request where the client's uid is 0 as root.  As
such, we have decoded enough of the request to write our rule.  

First, we need to make sure that our packet is an RPC call.
    
\begin{verbatim}
    content:"|00 00 00 00|"; offset:4; depth:4;
\end{verbatim}

Then, we need to make sure that our packet is a call to sadmind.
    
\begin{verbatim}
    content:"|00 01 87 88|"; offset:12; depth:4;
\end{verbatim}

Then, we need to make sure that our packet is a call to the procedure 1, the
vulnerable procedure.  
   
\begin{verbatim}
    content:"|00 00 00 01|"; offset:20; depth:4;
\end{verbatim}

Then, we need to make sure that our packet has auth\_unix credentials.  
    
\begin{verbatim}
    content:"|00 00 00 01|"; offset:24; depth:4;
\end{verbatim}

We don't care about the hostname, but we want to skip over it and check a
number value after the hostname.  This is where byte\_test is useful.  Starting
at the length of the hostname, the data we have is:

\begin{verbatim}
00 00 00 0a 4d 45 54 41 53 50 4c 4f 49 54 00 00
00 00 00 00 00 00 00 00 00 00 00 00 00 00 00 00
00 00 00 00
\end{verbatim}

We want to read 4 bytes, turn it into a number, and jump that many bytes
forward, making sure to account for the padding that RPC requires on strings.
If we do that, we are now at:

\begin{verbatim}
00 00 00 00 00 00 00 00 00 00 00 00 00 00 00 00
00 00 00 00 
\end{verbatim}

which happens to be the exact location of the uid, the value we want to check.

In English, we want to read 4 bytes, 36 bytes from the beginning of the packet,
and turn those 4 bytes into an integer and jump that many bytes forward,
aligning on the 4 byte boundary.  To do that in a Snort rule, we use:

\begin{verbatim}
    byte_jump:4,36,align;
\end{verbatim}

then we want to look for the uid of 0.

\begin{verbatim}
    content:"|00 00 00 00|"; within:4;
\end{verbatim}

Now that we have all the detection capabilities for our rule, let's put them
all together.  

\begin{verbatim}
    content:"|00 00 00 00|"; offset:4; depth:4;
    content:"|00 01 87 88|"; offset:12; depth:4;
    content:"|00 00 00 01|"; offset:20; depth:4;
    content:"|00 00 00 01|"; offset:24; depth:4;
    byte_jump:4,36,align;
    content:"|00 00 00 00|"; within:4;
\end{verbatim}

The 3rd and fourth string match are right next to each other, so we should
combine those patterns.  We end up with:

\begin{verbatim}
    content:"|00 00 00 00|"; offset:4; depth:4;
    content:"|00 01 87 88|"; offset:12; depth:4;
    content:"|00 00 00 01 00 00 00 01|"; offset:20; depth:8;
    byte_jump:4,36,align;
    content:"|00 00 00 00|"; within:4;
\end{verbatim}

If the sadmind service was vulnerable to a buffer overflow when reading the
client's hostname, instead of reading the length of the hostname and jumping
that many bytes forward, we would check the length of the hostname to make sure
it is not too large.  

To do that, we would read 4 bytes, starting 36 bytes into the packet, turn it
into a number, and then make sure it is not too large (let's say bigger than
200 bytes).  In Snort, we do: 

\begin{verbatim}
    byte_test:4,>,200,36;
\end{verbatim}
    
Our full rule would be:    

\begin{verbatim}
    content:"|00 00 00 00|"; offset:4; depth:4;
    content:"|00 01 87 88|"; offset:12; depth:4;
    content:"|00 00 00 01 00 00 00 01|"; offset:20; depth:8;
    byte_test:4,>,200,36;
\end{verbatim}

\chapter{Dynamic Modules}\label{Dynamic Modules}

Preprocessors, detection capabilities, and rules can now be developed as
dynamically loadable modules to snort.  The dynamic API presents a means for
loading dynamic libraries and allowing the module to utilize certain functions
within the main snort code.

The remainder of this chapter will highlight the data structures and API
functions used in developing preprocessors, detection engines, and rules as a
dynamic plugin to snort.

Beware:  the definitions herein may be out of date; check the appropriate
header files for the current definitions.

\section{Data Structures}

A number of data structures are central to the API.  The definition of each is
defined in the following sections.

\subsection{DynamicPluginMeta}

The {\em DynamicPluginMeta} structure defines the type of dynamic module
(preprocessor, rules, or detection engine), the version information, and path
to the shared library.  A shared library can implement all three types, but
typically is limited to a single functionality such as a preprocessor.  It is
defined in \texttt{sf\_dynamic\_meta.h} as:

\begin{verbatim}
#define MAX_NAME_LEN 1024

#define TYPE_ENGINE 0x01
#define TYPE_DETECTION 0x02
#define TYPE_PREPROCESSOR 0x04

typedef struct _DynamicPluginMeta
{
    int type;
    int major;
    int minor;
    int build;
    char uniqueName[MAX_NAME_LEN];
    char *libraryPath;
} DynamicPluginMeta;
\end{verbatim}

\subsection{DynamicPreprocessorData}

The {\em DynamicPreprocessorData} structure defines the interface the
preprocessor uses to interact with snort itself.  This includes functions to
register the preprocessor's configuration parsing, restart, exit, and
processing functions.  It includes function to log messages, errors, fatal
errors, and debugging info.  It also includes information for setting alerts,
handling Inline drops, access to the StreamAPI, and it provides access to the
normalized http and alternate data buffers.  This data structure should be
initialized when the preprocessor shared library is loaded.  It is defined in
\texttt{sf\_dynamic\_preprocessor.h}.  Check the header file for the current
definition.

\subsection{DynamicEngineData}

The {\em DynamicEngineData} structure defines the interface a detection engine
uses to interact with snort itself.  This includes functions for logging
messages, errors, fatal errors, and debugging info as well as a means to
register and check flowbits.  It also includes a location to store rule-stubs
for dynamic rules that are loaded, and it provides access to the normalized
http and alternate data buffers.  It is defined in
\texttt{sf\_dynamic\_engine.h} as:

\begin{verbatim}
typedef struct _DynamicEngineData
{
    int version;
    u_int8_t *altBuffer;
    UriInfo *uriBuffers[MAX_URIINFOS];
    RegisterRule ruleRegister;
    RegisterBit flowbitRegister;
    CheckFlowbit flowbitCheck;
    DetectAsn1 asn1Detect;
    LogMsgFunc logMsg;
    LogMsgFunc errMsg;
    LogMsgFunc fatalMsg;
    char *dataDumpDirectory;

    GetPreprocRuleOptFuncs getPreprocOptFuncs;

    SetRuleData setRuleData;
    GetRuleData getRuleData;

    DebugMsgFunc debugMsg;
#ifdef HAVE_WCHAR_H
    DebugWideMsgFunc debugWideMsg;
#endif

    char **debugMsgFile;
    int *debugMsgLine;

    PCRECompileFunc pcreCompile;
    PCREStudyFunc pcreStudy;
    PCREExecFunc pcreExec;

} DynamicEngineData;
\end{verbatim}

\subsection{SFSnortPacket}

The {\em SFSnortPacket} structure mirrors the snort Packet structure and
provides access to all of the data contained in a given packet.

It and the data structures it incorporates are defined in
\texttt{sf\_snort\_packet.h}.  Additional data structures may be defined to
reference other protocol fields.  Check the header file for the current
definitions.

\subsection{Dynamic Rules}

A dynamic rule should use any of the following data structures.  The following
structures are defined in \texttt{sf\_snort\_plugin\_api.h}.

\subsubsection{Rule}

The {\em Rule} structure defines the basic outline of a rule and contains the
same set of information that is seen in a text rule.  That includes protocol,
address and port information and rule information (classification, generator
and signature IDs, revision, priority, classification, and a list of
references).  It also includes a list of rule options and an optional
evaluation function.

\begin{verbatim}
#define RULE_MATCH 1
#define RULE_NOMATCH 0

typedef struct _Rule
{
    IPInfo ip;
    RuleInformation info;

    RuleOption **options; /* NULL terminated array of RuleOption union */

    ruleEvalFunc evalFunc;

    char initialized;     /* Rule Initialized, used internally */
    u_int32_t numOptions; /* Rule option count, used internally */
    char noAlert;         /* Flag with no alert, used internally */
    void *ruleData;    /* Hash table for dynamic data pointers */
} Rule;
\end{verbatim}

The rule evaluation function is defined as

\begin{verbatim}
typedef int (*ruleEvalFunc)(void *);
\end{verbatim}

where the parameter is a pointer to the SFSnortPacket structure.

\subsubsection{RuleInformation}

The {\em RuleInformation} structure defines the meta data for a rule and
includes generator ID, signature ID, revision, classification, priority,
message text, and a list of references.

\begin{verbatim}
typedef struct _RuleInformation
{
    u_int32_t genID;
    u_int32_t sigID;
    u_int32_t revision;
    char     *classification; /* String format of classification name */
    u_int32_t priority;
    char     *message;
    RuleReference **references; /* NULL terminated array of references */
    RuleMetaData **meta; /* NULL terminated array of references */
} RuleInformation;
\end{verbatim}

\subsubsection{RuleReference}

The {\em RuleReference} structure defines a single rule reference, including
the system name and rereference identifier.

\begin{verbatim}
typedef struct _RuleReference
{
    char *systemName;
    char *refIdentifier;
} RuleReference;
\end{verbatim}

\subsubsection{IPInfo}

The {\em IPInfo} structure defines the initial matching criteria for a rule and
includes the protocol, src address and port, destination address and port, and
direction.  Some of the standard strings and variables are predefined - any,
HOME\_NET, HTTP\_SERVERS, HTTP\_PORTS, etc.

\begin{verbatim}
typedef struct _IPInfo
{
    u_int8_t protocol;
    char *   src_addr;
    char *   src_port; /* 0 for non TCP/UDP */
    char     direction;     /* non-zero is bi-directional */
    char *   dst_addr;
    char *   dst_port; /* 0 for non TCP/UDP */
} IPInfo;

#define ANY_NET         "any"
#define HOME_NET        "$HOME_NET"
#define EXTERNAL_NET    "$EXTERNAL_NET"
#define ANY_PORT        "any"
#define HTTP_SERVERS    "$HTTP_SERVERS"
#define HTTP_PORTS      "$HTTP_PORTS"
#define SMTP_SERVERS    "$SMTP_SERVERS"
\end{verbatim}

\subsubsection{RuleOption}

The {\em RuleOption} structure defines a single rule option as an option type
and a reference to the data specific to that option.  Each option has a flags
field that contains specific flags for that option as well as a "Not" flag.
The "Not" flag is used to negate the results of evaluating that option.

\begin{verbatim}
typedef enum DynamicOptionType {
     OPTION_TYPE_PREPROCESSOR,
     OPTION_TYPE_CONTENT,
     OPTION_TYPE_PCRE,
     OPTION_TYPE_FLOWBIT,
     OPTION_TYPE_FLOWFLAGS,
     OPTION_TYPE_ASN1,
     OPTION_TYPE_CURSOR,
     OPTION_TYPE_HDR_CHECK,
     OPTION_TYPE_BYTE_TEST,
     OPTION_TYPE_BYTE_JUMP,
     OPTION_TYPE_BYTE_EXTRACT,
     OPTION_TYPE_SET_CURSOR,
     OPTION_TYPE_LOOP,
     OPTION_TYPE_MAX
};

typedef struct _RuleOption
{
    int optionType;
    union
    {   
        void *ptr;
        ContentInfo *content;
        CursorInfo *cursor;
        PCREInfo *pcre;
        FlowBitsInfo *flowBit;
        ByteData *byte;
        ByteExtract *byteExtract;
        FlowFlags *flowFlags;
        Asn1Context *asn1;
        HdrOptCheck *hdrData;
        LoopInfo    *loop;
        PreprocessorOption *preprocOpt;
    } option_u;
} RuleOption;

#define NOT_FLAG                0x10000000
\end{verbatim}

Some options also contain information that is initialized at run time, such as
the compiled PCRE information, Boyer-Moore content information, the integer ID
for a flowbit, etc.

The option types and related structures are listed below.

\begin{itemize}
%\item {OptionType: Preprocessor defined \& Structure: {\em PreprocessorOption}}
%
%The {\em PreprocessorOption} structure defines an option for a
%preprocessor defined option. 
%
%\begin{verbatim}
%typedef struct _PreprocessorOption
%{
%    char *optionName;
%    char *optionParameters;
%    u_int32_t flags;
%    void *optionInit;
%    void *optionEval;
%    void *dataPtr;
%} PreprocessorOption;
%\end{verbatim}

\item {OptionType: Content \& Structure: {\em ContentInfo}}

The {\em ContentInfo} structure defines an option for a content search.  It
includes the pattern, depth and offset, and flags (one of which must specify
the buffer -- raw, URI or normalized -- to search).  Additional flags include
nocase, relative, unicode, and a designation that this content is to be used
for snorts fast pattern evaluation.  The most unique content, that which
distinguishes this rule as a possible match to a packet, should be marked for
fast pattern evaluation.  In the dynamic detection engine provided with Snort,
if no {\em ContentInfo} structure in a given rules uses that flag, the one with
the longest content length will be used.

\begin{verbatim}
typedef struct _ContentInfo
{
    u_int8_t *pattern;
    u_int32_t depth;
    int32_t   offset;
    u_int32_t flags;        /* must include a CONTENT_BUF_X */
    void     *boyer_ptr;
    u_int8_t *patternByteForm;
    u_int32_t patternByteFormLength;
    u_int32_t incrementLength;
} ContentInfo;

#define CONTENT_NOCASE          0x01
#define CONTENT_RELATIVE        0x02
#define CONTENT_UNICODE2BYTE    0x04
#define CONTENT_UNICODE4BYTE    0x08
#define CONTENT_FAST_PATTERN    0x10
#define CONTENT_END_BUFFER      0x20

#define CONTENT_BUF_NORMALIZED  0x100
#define CONTENT_BUF_RAW         0x200
#define CONTENT_BUF_URI         0x400
\end{verbatim}

\item {OptionType: PCRE \& Structure: {\em PCREInfo}}

The {\em PCREInfo} structure defines an option for a PCRE search.  It includes
the PCRE expression, pcre\_flags such as caseless, as defined in PCRE.h, and
flags to specify the buffer.

\begin{verbatim}
/*
pcre.h provides flags:

PCRE_CASELESS
PCRE_MULTILINE
PCRE_DOTALL
PCRE_EXTENDED
PCRE_ANCHORED
PCRE_DOLLAR_ENDONLY
PCRE_UNGREEDY
*/

typedef struct _PCREInfo
{
    char     *expr;
    void     *compiled_expr;
    void     *compiled_extra;
    u_int32_t compile_flags;
    u_int32_t flags; /* must include a CONTENT_BUF_X */
} PCREInfo;
\end{verbatim}

\item {OptionType: Flowbit \& Structure: {\em FlowBitsInfo}}

The {\em FlowBitsInfo} structure defines a flowbits option.  It includes the
name of the flowbit and the operation (set, setx, unset, toggle, isset, isnotset).

\begin{verbatim}
#define FLOWBIT_SET       0x01
#define FLOWBIT_UNSET     0x02
#define FLOWBIT_TOGGLE    0x04
#define FLOWBIT_ISSET     0x08
#define FLOWBIT_ISNOTSET  0x10
#define FLOWBIT_RESET     0x20
#define FLOWBIT_NOALERT   0x40
#define FLOWBIT_SETX      0x80

typedef struct _FlowBitsInfo
{
    char     *flowBitsName;
    uint8_t   operation;
    uint16_t  id;
    uint32_t  flags;
    char     *groupName;
    uint8_t   eval;
    uint16_t *ids;
    uint8_t   num_ids;
    
} FlowBitsInfo;
\end{verbatim}

\item {OptionType: Flow Flags \& Structure: {\em FlowFlags}}

The {\em FlowFlags} structure defines a flow option.  It includes the flags,
which specify the direction (from\_server, to\_server), established session,
etc. 

\begin{verbatim}
#define FLOW_ESTABLISHED 0x10
#define FLOW_IGNORE_REASSEMBLED 0x1000
#define FLOW_ONLY_REASSMBLED    0x2000
#define FLOW_FR_SERVER   0x40
#define FLOW_TO_CLIENT   0x40 /* Just for redundancy */
#define FLOW_TO_SERVER   0x80
#define FLOW_FR_CLIENT   0x80 /* Just for redundancy */

typedef struct _FlowFlags
{
    u_int32_t   flags;
} FlowFlags;
\end{verbatim}

\item {OptionType: ASN.1 \& Structure: {\em Asn1Context}}

The {\em Asn1Context} structure defines the information for an ASN1 option.  It
mirrors the ASN1 rule option and also includes a flags field.

\begin{verbatim}
#define ASN1_ABS_OFFSET 1
#define ASN1_REL_OFFSET 2

typedef struct _Asn1Context
{
    int bs_overflow;
    int double_overflow;
    int print;
    int length;
    unsigned int max_length;
    int offset;
    int offset_type;
    u_int32_t  flags;
} Asn1Context;
\end{verbatim}

\item {OptionType: Cursor Check \& Structure: {\em CursorInfo}}

The {\em CursorInfo} structure defines an option for a cursor evaluation.  The
cursor is the current position within the evaluation buffer, as related to
content and PCRE searches, as well as byte tests and byte jumps.  It includes
an offset and flags that specify the buffer.  This can be used to verify there
is sufficient data to continue evaluation, similar to the isdataat rule option.  

\begin{verbatim}
typedef struct _CursorInfo
{
    int32_t   offset;
    u_int32_t flags;        /* specify one of CONTENT_BUF_X */
} CursorInfo;
\end{verbatim}

\item {OptionType: Protocol Header \& Structure: {\em HdrOptCheck}}

The {\em HdrOptCheck} structure defines an option to check a protocol header
for a specific value.  It includes the header field, the operation ($<$,$>$,=,etc),
a value, a mask to ignore that part of the header field, and flags.

\begin{verbatim}
#define IP_HDR_ID           0x0001  /* IP Header ID */
#define IP_HDR_PROTO        0x0002  /* IP Protocol */
#define IP_HDR_FRAGBITS     0x0003  /* Frag Flags set in IP Header */
#define IP_HDR_FRAGOFFSET   0x0004  /* Frag Offset set in IP Header */
#define IP_HDR_OPTIONS      0x0005  /* IP Options -- is option xx included */
#define IP_HDR_TTL          0x0006  /* IP Time to live */
#define IP_HDR_TOS          0x0007  /* IP Type of Service */
#define IP_HDR_OPTCHECK_MASK 0x000f

#define TCP_HDR_ACK         0x0010  /* TCP Ack Value */
#define TCP_HDR_SEQ         0x0020  /* TCP Seq Value */
#define TCP_HDR_FLAGS       0x0030  /* Flags set in TCP Header */
#define TCP_HDR_OPTIONS     0x0040  /* TCP Options -- is option xx included */
#define TCP_HDR_WIN         0x0050  /* TCP Window */
#define TCP_HDR_OPTCHECK_MASK 0x00f0

#define ICMP_HDR_CODE       0x1000  /* ICMP Header Code */
#define ICMP_HDR_TYPE       0x2000  /* ICMP Header Type */
#define ICMP_HDR_ID         0x3000  /* ICMP ID for ICMP_ECHO/ICMP_ECHO_REPLY */
#define ICMP_HDR_SEQ        0x4000  /* ICMP ID for ICMP_ECHO/ICMP_ECHO_REPLY */
#define ICMP_HDR_OPTCHECK_MASK 0xf000

typedef struct _HdrOptCheck
{
    u_int16_t hdrField;   /* Field to check */
    u_int32_t op;         /* Type of comparison */
    u_int32_t value;      /* Value to compare value against */
    u_int32_t mask_value; /* bits of value to ignore */
    u_int32_t flags;
} HdrOptCheck;
\end{verbatim}

\item {OptionType: Byte Test \& Structure: {\em ByteData}}

The {\em ByteData} structure defines the information for both ByteTest and
ByteJump operations.  It includes the number of bytes, an operation (for
ByteTest, $<$,$>$,=,etc), a value, an offset, multiplier, and flags.  The flags
must specify the buffer.

\begin{verbatim}
#define CHECK_EQ            0
#define CHECK_NEQ           1
#define CHECK_LT            2
#define CHECK_GT            3
#define CHECK_LTE           4
#define CHECK_GTE           5
#define CHECK_AND           6
#define CHECK_XOR           7
#define CHECK_ALL           8
#define CHECK_ATLEASTONE    9
#define CHECK_NONE          10

typedef struct _ByteData
{
    u_int32_t bytes;      /* Number of bytes to extract */
    u_int32_t op;         /* Type of byte comparison, for checkValue */
    u_int32_t value;      /* Value to compare value against, for checkValue, or extracted value */
    int32_t   offset;     /* Offset from cursor */
    u_int32_t multiplier; /* Used for byte jump -- 32bits is MORE than enough */
    u_int32_t flags;      /* must include a CONTENT_BUF_X */
} ByteData;
\end{verbatim}

\item {OptionType: Byte Jump \& Structure: {\em ByteData}}

See {\em Byte Test} above.

\item {OptionType: Set Cursor \& Structure: {\em CursorInfo}}

See {\em Cursor Check} above.

\item {OptionType: Loop \& Structures: {\em LoopInfo,ByteExtract,DynamicElement}}

The {\em LoopInfo} structure defines the information for a set of options that
are to be evaluated repeatedly.  The loop option acts like a FOR loop and
includes start, end, and increment values as well as the comparison operation
for termination.  It includes a cursor adjust that happens through each
iteration of the loop, a reference to a RuleInfo structure that defines the
RuleOptions are to be evaluated through each iteration.  One of those options
may be a ByteExtract.

\begin{verbatim}
typedef struct _LoopInfo
{
    DynamicElement *start;      /* Starting value of FOR loop (i=start) */
    DynamicElement *end;        /* Ending value of FOR loop (i OP end) */
    DynamicElement *increment;  /* Increment value of FOR loop (i+= increment) */
    u_int32_t op;               /* Type of comparison for loop termination */
    CursorInfo *cursorAdjust;   /* How to move cursor each iteration of loop */
    struct _Rule *subRule;      /* Pointer to SubRule & options to evaluate within
                                 * the loop */
    u_int8_t initialized;       /* Loop initialized properly (safeguard) */
    u_int32_t flags;            /* can be used to negate loop results, specifies                                 * relative. */
} LoopInfo;
\end{verbatim}

The {\em ByteExtract} structure defines the information to use when extracting
bytes for a DynamicElement used a in Loop evaluation.  It includes the number of
bytes, an offset, multiplier, flags specifying the buffer, and a reference to
the DynamicElement.

\begin{verbatim}
typedef struct _ByteExtract
{
    u_int32_t bytes;      /* Number of bytes to extract */
    int32_t   offset;     /* Offset from cursor */
    u_int32_t multiplier; /* Multiply value by this (similar to byte jump) */
    u_int32_t flags;      /* must include a CONTENT_BUF_X */
    char *refId;          /* To match up with a DynamicElement refId */
    void *memoryLocation; /* Location to store the data extracted */
} ByteExtract;
\end{verbatim}

The {\em DynamicElement} structure is used to define the values for a looping
evaluation.  It includes whether the element is static (an integer) or dynamic
(extracted from a buffer in the packet) and the value.  For a dynamic element,
the value is filled by a related ByteExtract option that is part of the loop.

\begin{verbatim}
#define DYNAMIC_TYPE_INT_STATIC 1
#define DYNAMIC_TYPE_INT_REF    2

typedef struct _DynamicElement
{
    char dynamicType;           /* type of this field - static or reference */
    char *refId;                /* reference ID (NULL if static) */
    union
    {
        void *voidPtr;          /* Holder */
        int32_t staticInt;        /* Value of static */
        int32_t *dynamicInt;  /* Pointer to value of dynamic */
    } data;
} DynamicElement;
\end{verbatim}

\end{itemize}

\section{Required Functions}

Each dynamic module must define a set of functions and data objects to work
within this framework.

\subsection{Preprocessors}

Each dynamic preprocessor must define the following items.  These must be defined
in the global scope of a source file (e.g. spp\_example.c).

\begin{itemize}
\item {\em const int MAJOR\_VERSION}

This specifies the major version of the preprocessor.

\item {\em const int MINOR\_VERSION}

This specifies the minor version of the preprocessor.

\item {\em const int BUILD\_VERSION}

This specifies the build version of the preprocessor.

\item {\em const char *PREPROC\_NAME}

This specifies the display name of the preprocessor.

\item {\em void DYNAMIC\_PREPROC\_SETUP(void)}

This function is called to register the preprocessor to be called with packets data.

\end{itemize}

The preprocessor must be built with the same macros defined as the Snort binary and
linked with the dynamic preprocessor library that was created during the Snort build.
A package configuration file is exported as part of the Snort build and can be accessed
using the following commands with PKG\_CONFIG\_PATH=$<$snort build prefix/lib/pkgconfig$>$:

\begin{itemize}
\item {\em pkg-config --cflags snort\_preproc}

Returns the macros and include path needed to compile the dynamic preprocessor.

\item {\em pkg-config --libs snort\_preproc}

Returns the library and library path needed to link the dynamic preprocessor.

\end{itemize}

\subsection{Detection Engine}

Each dynamic detection engine library must define the following functions.

\begin{itemize}
\item {\em int LibVersion(DynamicPluginMeta *)}

This function returns the metadata for the shared library.

\item {\em int InitializeEngineLib(DynamicEngineData *)}

This function initializes the data structure for use by the engine.

\end{itemize}

The sample code provided with Snort predefines those functions and defines the
following APIs to be used by a dynamic rules library.

\begin{itemize}
\item {\em int RegisterRules(Rule **)}

This is the function to iterate through each rule in the list, initialize it to
setup content searches, PCRE evaluation data, and register flowbits.

\item {\em int DumpRules(char *,Rule **)}

This is the function to iterate through each rule in the list and write a
rule-stop to be used by snort to control the action of the rule (alert, log,
drop, etc).

\item {\em int ruleMatch(void *p, Rule *rule)}

This is the function to evaluate a rule if the rule does not have its own Rule
Evaluation Function.  This uses the individual functions outlined below for
each of the rule options and handles repetitive content issues.  

Each of the functions below returns RULE\_MATCH if the option matches based on
the current criteria (cursor position, etc).

\begin{itemize}

\item {\em int contentMatch(void *p, ContentInfo* content, u\_int8\_t **cursor)}

This function evaluates a single content for a given packet, checking for the
existence of that content as delimited by ContentInfo and cursor.  Cursor
position is updated and returned in *cursor.

With a text rule, the with option corresponds to depth, and the distance option
corresponds to offset.

\item {\em int checkFlow(void *p, FlowFlags *flowflags)}

This function evaluates the flow for a given packet.

\item {\em int extractValue(void *p, ByteExtract *byteExtract, u\_int8\_t *cursor)}

This function extracts the bytes from a given packet, as specified by
ByteExtract and delimited by cursor.  Value extracted is stored in ByteExtract
memoryLocation parameter.

\item {\em int processFlowbits(void *p, FlowBitsInfo *flowbits)}

This function evaluates the flowbits for a given packet, as specified by
FlowBitsInfo.  It will interact with flowbits used by text-based rules.

\item {\em int setCursor(void *p, CursorInfo *cursorInfo, u\_int8\_t **cursor)}

This function adjusts the cursor as delimited by CursorInfo.  New cursor
position is returned in *cursor.  It handles bounds checking for the specified
buffer and returns RULE\_NOMATCH if the cursor is moved out of bounds.  

It is also used by contentMatch, byteJump, and pcreMatch to adjust the cursor
position after a successful match.

\item {\em int checkCursor(void *p, CursorInfo *cursorInfo, u\_int8\_t *cursor)}

This function validates that the cursor is within bounds of the specified
buffer.

\item {\em int checkValue(void *p, ByteData *byteData, u\_int32\_t value, u\_int8\_t *cursor)}

This function compares the {\em value} to the value stored in ByteData.

\item {\em int byteTest(void *p, ByteData *byteData, u\_int8\_t *cursor)}

This is a wrapper for extractValue() followed by checkValue().

\item {\em int byteJump(void *p, ByteData *byteData, u\_int8\_t **cursor)}

This is a wrapper for extractValue() followed by setCursor().

\item {\em int pcreMatch(void *p, PCREInfo *pcre, u\_int8\_t **cursor)}

This function evaluates a single pcre for a given packet, checking for the
existence of the expression as delimited by PCREInfo and cursor.  Cursor
position is updated and returned in *cursor.

\item {\em int detectAsn1(void *p, Asn1Context *asn1, u\_int8\_t *cursor)}

This function evaluates an ASN.1 check for a given packet, as delimited by
Asn1Context and cursor.

\item {\em int checkHdrOpt(void *p, HdrOptCheck *optData)}

This function evaluates the given packet's protocol headers, as specified by
HdrOptCheck.

\item {\em int loopEval(void *p, LoopInfo *loop, u\_int8\_t **cursor)}

This function iterates through the SubRule of LoopInfo, as delimited by
LoopInfo and cursor.  Cursor position is updated and returned in *cursor.

\item {\em int preprocOptionEval(void *p, PreprocessorOption *preprocOpt, u\_int8\_t **cursor)}

This function evaluates the preprocessor defined option, as specified by
PreprocessorOption.  Cursor position is updated and returned in *cursor.

\item {\em void setTempCursor(u\_int8\_t **temp\_cursor, u\_int8\_t **cursor)}

This function is used to handled repetitive contents to save off a cursor
position temporarily to be reset at later point.

\item {\em void revertTempCursor(u\_int8\_t **temp\_cursor, u\_int8\_t **cursor)}

This function is used to revert to a previously saved temporary cursor
position.

\end{itemize}

\begin{note}

If you decide to write your own rule evaluation function, patterns that occur
more than once may result in false negatives.  Take extra care to handle this
situation and search for the matched pattern again if subsequent rule options
fail to match.  This should be done for both content and PCRE options.

\end{note}
\end{itemize}

\subsection{Rules}

Each dynamic rules library must define the following functions.  Examples are
defined in the file \texttt{sfnort\_dynamic\_detection\_lib.c}.  The metadata
and setup function for the preprocessor should be defined in
\texttt{sfsnort\_dynamic\_detection\_lib.h}.

\begin{itemize}
\item {\em int LibVersion(DynamicPluginMeta *)}

This function returns the metadata for the shared library.

\item {\em int EngineVersion(DynamicPluginMeta *)}

This function defines the version requirements for the corresponding detection
engine library.

\item {\em int DumpSkeletonRules()}

This functions writes out the rule-stubs for rules that are loaded.

\item {\em int InitializeDetection()}

This function registers each rule in the rules library.  It should set up fast
pattern-matcher content, register flowbits, etc.  \end{itemize}

The sample code provided with Snort predefines those functions and uses the
following data within the dynamic rules library.

\begin{itemize}

\item {\em Rule *rules[]}

A NULL terminated list of Rule structures that this library defines.

\end{itemize}

\section{Examples}

This section provides a simple example of a dynamic preprocessor and a dynamic
rule.

\subsection{Preprocessor Example}

The following is an example of a simple preprocessor.  This preprocessor always
alerts on a packet if the TCP port matches the one configured.

The following code is defined in {\em spp\_example.c} and is compiled
together with {\em libsf\_dynamic\_preproc.a}, using pkg-config, into
lib\_sfdynamic\_preprocessor\_example.so.

Define the required meta data variables.

\begin{verbatim}
#define GENERATOR_EXAMPLE 256
extern DynamicPreprocessorData _dpd;

const int MAJOR_VERSION = 1;
const int MINOR_VERSION = 0;
const int BUILD_VERSION = 0;
const char *PREPROC_NAME = "SF_Dynamic_Example_Preprocessor";

#define ExampleSetup DYNAMIC_PREPROC_SETUP
\end{verbatim}

Define the Setup function to register the initialization function.

\begin{verbatim}
void ExampleInit(unsigned char *);
void ExampleProcess(void *, void *);

void ExampleSetup()
{
    _dpd.registerPreproc("dynamic_example", ExampleInit);

    DEBUG_WRAP(_dpd.debugMsg(DEBUG_PLUGIN, "Preprocessor: Example is setup\n"););
}
\end{verbatim}

The initialization function to parse the keywords from \texttt{snort.conf}.

\begin{verbatim}
u_int16_t portToCheck;

void ExampleInit(unsigned char *args)
{
    char *arg;
    char *argEnd;
    unsigned long port;

    _dpd.logMsg("Example dynamic preprocessor configuration\n");

    arg = strtok(args, " \t\n\r");

    if(!strcasecmp("port", arg))
    {
        arg = strtok(NULL, "\t\n\r");
        if (!arg)
        {
            _dpd.fatalMsg("ExamplePreproc: Missing port\n");
        }
        
        port = strtoul(arg, &argEnd, 10);
        if (port < 0 || port > 65535)
        {
            _dpd.fatalMsg("ExamplePreproc: Invalid port %d\n", port);
        }
        portToCheck = port;
    
        _dpd.logMsg("    Port: %d\n", portToCheck);
    }
    else
    {
        _dpd.fatalMsg("ExamplePreproc: Invalid option %s\n", arg);
    }

    /* Register the preprocessor function, Transport layer, ID 10000 */
    _dpd.addPreproc(ExampleProcess, PRIORITY_TRANSPORT, 10000);

    DEBUG_WRAP(_dpd.debugMsg(DEBUG_PLUGIN, "Preprocessor: Example is initialized\n"););
}
\end{verbatim}

The function to process the packet and log an alert if the either port matches.

\begin{verbatim}
#define SRC_PORT_MATCH  1
#define SRC_PORT_MATCH_STR "example_preprocessor: src port match"
#define DST_PORT_MATCH  2
#define DST_PORT_MATCH_STR "example_preprocessor: dest port match"
void ExampleProcess(void *pkt, void *context)
{
    SFSnortPacket *p = (SFSnortPacket *)pkt;
    if (!p->ip4_header || p->ip4_header->proto != IPPROTO_TCP || !p->tcp_header)
    {
        /* Not for me, return */
        return;
    }

    if (p->src_port == portToCheck)
    {
        /* Source port matched, log alert */
        _dpd.alertAdd(GENERATOR_EXAMPLE, SRC_PORT_MATCH,
                      1, 0, 3, SRC_PORT_MATCH_STR, 0);
        return;
    }

    if (p->dst_port == portToCheck)
    {
        /* Destination port matched, log alert */
        _dpd.alertAdd(GENERATOR_EXAMPLE, DST_PORT_MATCH,
                      1, 0, 3, DST_PORT_MATCH_STR, 0);
        return;
    }
}
\end{verbatim}

\subsection{Rules}

The following is an example of a simple rule, take from the current rule set,
SID 109.  It is implemented to work with the detection engine provided with
snort.

The snort rule in normal format:

\begin{verbatim}
alert tcp $HOME_NET 12345:12346 -> $EXTERNAL_NET any \
(msg:"BACKDOOR netbus active"; flow:from_server,established; \
content:"NetBus"; reference:arachnids,401; classtype:misc-activity; \
sid:109; rev:5;)
\end{verbatim}

This is the metadata for this rule library, defined in {\em
detection\_lib\_meta.h}.

\begin{verbatim}
/* Version for this rule library */
#define DETECTION_LIB_MAJOR_VERSION 1
#define DETECTION_LIB_MINOR_VERSION 0
#define DETECTION_LIB_BUILD_VERSION 1
#define DETECTION_LIB_NAME "Snort_Dynamic_Rule_Example"

/* Required version and name of the engine */
#define REQ_ENGINE_LIB_MAJOR_VERSION 1
#define REQ_ENGINE_LIB_MINOR_VERSION 0
#define REQ_ENGINE_LIB_NAME "SF_SNORT_DETECTION_ENGINE"

\end{verbatim}

The definition of each data structure for this rule is in {\em sid109.c}.

Declaration of the data structures.

\begin{itemize}

\item{Flow option}

Define the {\em FlowFlags} structure and its corresponding {\em RuleOption}.
Per the text version, flow is from\_server,established.

\begin{verbatim}
static FlowFlags sid109flow =
{
    FLOW_ESTABLISHED|FLOW_TO_CLIENT
};

static RuleOption sid109option1 =
{
    OPTION_TYPE_FLOWFLAGS,
    {
        &sid109flow
    }
};
\end{verbatim}

\item{Content Option}

Define the {\em ContentInfo} structure and its corresponding {\em RuleOption}.
Per the text version, content is "NetBus", no depth or offset, case sensitive,
and non-relative.  Search on the normalized buffer by default.  {\bf NOTE}:
This content will be used for the fast pattern matcher since it is the longest
content option for this rule and no contents have a flag of {\em
CONTENT\_FAST\_PATTERN}.

\begin{verbatim}
static ContentInfo sid109content =
{
    "NetBus",               /* pattern to search for */
    0,                      /* depth */
    0,                      /* offset */
    CONTENT_BUF_NORMALIZED, /* flags */
    NULL,                   /* holder for boyer/moore info */
    NULL,                   /* holder for byte representation of "NetBus" */
    0,                      /* holder for length of byte representation */
    0                       /* holder for increment length */
};

static RuleOption sid109option2 =
{
    OPTION_TYPE_CONTENT,
    {
        &sid109content
    }
};
\end{verbatim}

\item{Rule and Meta Data}

Define the references.

\begin{verbatim}
static RuleReference sid109ref_arachnids =
{
    "arachnids",    /* Type */
    "401"           /* value */
};

static RuleReference *sid109refs[] =
{
    &sid109ref_arachnids,
    NULL
};
\end{verbatim}

The list of rule options.  Rule options are evaluated in the order specified.

\begin{verbatim}
RuleOption *sid109options[] =
{
    &sid109option1,
    &sid109option2,
    NULL
};
\end{verbatim}

The rule itself, with the protocol header, meta data (sid, classification,
message, etc).  

\begin{verbatim}
Rule sid109 =
{
    /* protocol header, akin to => tcp any any -> any any               */
    {
        IPPROTO_TCP,        /* proto */
        HOME_NET,           /* source IP     */
        "12345:12346",      /* source port(s) */
        0,                  /* Direction */
        EXTERNAL_NET,       /* destination IP     */
        ANY_PORT,           /* destination port   */
    },
    /* metadata */
    {
        3,                  /* genid -- use 3 to distinguish a C rule */
        109,                /* sigid */
        5,                  /* revision */
        "misc-activity",    /* classification */
        0,                  /* priority */
        "BACKDOOR netbus active",     /* message */
        sid109refs          /* ptr to references */
    },
    sid109options,          /* ptr to rule options */
    NULL,                   /* Use internal eval func */
    0,                      /* Holder, not yet initialized, used internally */
    0,                      /* Holder, option count, used internally */
    0,                      /* Holder, no alert, used internally for flowbits */
    NULL                    /* Holder, rule data, used internally */
\end{verbatim}

\item{The List of rules defined by this rules library}

The NULL terminated list of rules.  The InitializeDetection iterates through
each Rule in the list and initializes the content, flowbits, pcre, etc.

\begin{verbatim}
extern Rule sid109;
extern Rule sid637;

Rule *rules[] = 
{
    &sid109,
    &sid637,
    NULL
};
\end{verbatim}

\end{itemize}

\chapter{Snort Development}

Currently, this chapter is here as a place holder. It will someday contain
references on how to create new detection plugins and preprocessors.  End users
don't really need to be reading this section. This is intended to help
developers get a basic understanding of whats going on quickly.

If you are going to be helping out with Snort development, please use the
\textsc{head} branch of cvs. We've had problems in the past of people
submitting patches only to the stable branch (since they are likely writing
this stuff for their own IDS purposes). Bug fixes are what goes into
\textsc{stable}. Features go into \textsc{head}.

\section{Submitting Patches}

Patches to Snort should be sent to the
\verb!snort-devel@lists.sourceforge.net!
mailing list.  Patches should done with the command
\verb!diff -nu snort-orig snort-new!.

\section{Snort Data Flow}

First, traffic is acquired from the network link via libpcap. Packets are
passed through a series of decoder routines that first fill out the packet
structure for link level protocols then are further decoded for things like TCP
and UDP ports.

Packets are then sent through the registered set of preprocessors.  Each
preprocessor checks to see if this packet is something it should look at.

Packets are then sent through the detection engine. The detection engine checks
each packet against the various options listed in the Snort config files. Each
of the keyword options is a plugin. This allows this to be easily extensible.

\subsection{Preprocessors}

For example, a TCP analysis preprocessor could simply return if the packet does
not have a TCP header. It can do this by checking: 

\begin{verbatim}
if (p->tcph==null)
   return;
\end{verbatim}

Similarly, there are a lot of packet\_flags available that can be used to mark
a packet as ``reassembled'' or logged. Check out src/decode.h for the list of
pkt\_{*} constants.

\subsection{Detection Plugins}

Basically, look at an existing output plugin and copy it to a new item and
change a few things. Later, we'll document what these few things are.

\subsection{Output Plugins}

Generally, new output plugins should go into the barnyard project rather than
the Snort project. We are currently cleaning house on the available output
options.

\section{Unified2 File Format}

  Unified 2 records should not be assumed to be in any order.  All
  values are stored in network byte order.

    An example structure of unified2 files

\begin{verbatim}
    [ Serial Unified2 Header    ]
    [ Unified2 IDS Event        ]
    [ Unified2 Packet           ]
    [ Unified2 Extra Data       ]
    .
    .
    .
    [ Serial Unified2 Header    ]
    [ Unified2 IDS Event        ]
    [ Unified2 Packet           ]
    [ Unified2 Extra Data       ]
\end{verbatim}

\subsection{Serial Unified2 Header}

\begin{verbatim}
    record type             4 bytes
    record length           4 bytes
\end{verbatim}

  All unified2 records are preceded by a Serial Unified2 header.  This
  unified2 record allows an interpreting application to skip past and
  apply simple heuristics against records.
  
  The Record Type indicates one of the following unified2 records
  follows the Serial Unified2 Header:

\begin{verbatim}
    Value           Record Type
    ----------      -----------
    2               Unified2 Packet
    7               Unified2 IDS Event
    72              Unified2 IDS Event IP6
    104             Unified2 IDS Event      (Version 2)
    105             Unified2 IDS Event IP6  (Version 2)
    110             Unified2 Extra Data
\end{verbatim}

  The record length field specifies the entire length of the record 
  (not including the Serial Unified2 Header itself) up to the next 
  Serial Unified2 Header or EOF.

\subsection{Unified2 Packet}
    
\begin{verbatim}
    sensor id               4 bytes
    event id                4 bytes
    event seconds           4 bytes
    event microseconds      4 bytes
    linktype                4 bytes
    packet length           4 bytes
    packet data             <variable length>
\end{verbatim}

  A Unified2 Packet is provided with each Unified2 Event record. This
  packet is the `alerting' packet that caused a given event.

  Unified2 Packet records contain contain a copy of the packet that
  caused an alert (Packet Data) and is packet length octets long.

\subsection{Unified2 IDS Event}

\begin{verbatim}
    sensor id               4 bytes
    event id                4 bytes
    event second            4 bytes
    event microsecond       4 bytes
    signature id            4 bytes
    generator id            4 bytes
    signature revision      4 bytes
    classification id       4 bytes
    priority id             4 bytes
    ip source               4 bytes
    ip destination          4 bytes
    source port/icmp type   2 bytes
    dest. port/icmp code    2 bytes
    protocol                1 byte
    impact flag             1 byte
    impact                  1 byte
    blocked                 1 byte
\end{verbatim}

  Unified2 IDS Event is logged for IPv4 Events without VLAN or MPLS
  tagging. 

\subsection{Unified2 IDS Event IP6}

\begin{verbatim}
    sensor id               4 bytes
    event id                4 bytes
    event second            4 bytes
    event microsecond       4 bytes
    signature id            4 bytes
    generator id            4 bytes
    signature revision      4 bytes
    classification id       4 bytes
    priority id             4 bytes
    ip source               16 bytes
    ip destination          16 bytes
    source port/icmp type   2 bytes
    dest. port/icmp code    2 bytes
    protocol                1 byte
    impact flag             1 byte
    impact                  1 byte
    blocked                 1 byte
\end{verbatim}

  Unified2 IDS Event IP6 is logged for IPv6 Events without VLAN or 
  MPLS tagging. 

\subsection{Unified2 IDS Event (Version 2)}

\begin{verbatim}
    sensor id               4 bytes
    event id                4 bytes
    event second            4 bytes
    event microsecond       4 bytes
    signature id            4 bytes
    generator id            4 bytes
    signature revision      4 bytes
    classification id       4 bytes
    priority id             4 bytes
    ip source               4 bytes
    ip destination          4 bytes
    source port/icmp type   2 bytes
    dest. port/icmp code    2 bytes
    protocol                1 byte
    impact flag             1 byte
    impact                  1 byte
    blocked                 1 byte
    mpls label              4 bytes
    vlan id                 2 bytes
    padding                 2 bytes
\end{verbatim}

  Unified2 IDS Event (Version 2) are logged for IPv4 packets which
  contain either  MPLS or VLAN headers. Otherwise a Unified2 IDS Event
  is logged.

\begin{note}
\begin{itemize}
\item Note that you'll need to pass --enable-mpls to configure in order to have Snort fill in the mpls label field.

\item Note that you'll need to configure unified2 logging with either mpls\_event\_types or vlan\_event\_types to get this record type.
\end{itemize}
\end{note}

\subsection{Unified2 IDS Event IP6 (Version 2)}

\begin{verbatim}
    sensor id               4 bytes
    event id                4 bytes
    event second            4 bytes
    event microsecond       4 bytes
    signature id            4 bytes
    generator id            4 bytes
    signature revision      4 bytes
    classification id       4 bytes
    priority id             4 bytes
    ip source               16 bytes
    ip destination          16 bytes
    source port/icmp type   2 bytes
    dest. port/icmp code    2 bytes
    protocol                1 byte
    impact flag             1 byte
    impact                  1 byte
    blocked                 1 byte
    mpls label              4 bytes
    vlan id                 2 bytes
    padding                 2 bytes
\end{verbatim}

  Unified2 IDS Event IP6 (Version 2) are logged for IPv6 packets which
  contain either MPLS or VLAN headers. Otherwise a Unified2 IDS Event
  IP6 is logged.

\begin{note}
\begin{itemize}
\item Note that you'll need to pass --enable-mpls to configure in order to have Snort fill in the mpls label field.

\item Note that you'll need to configure unified2 logging with either mpls\_event\_types or vlan\_event\_types to get this record type.
\end{itemize}
\end{note}

\subsection{Unified2 Extra Data}

\begin{verbatim}
    sensor id               4 bytes
    event id                4 bytes
    event second            4 bytes
    type                    4 bytes
    data type               4 bytes
    data length             4 bytes
    data                    <variable length>
\end{verbatim}

\subsection{Description of Fields}

\begin{itemize}\itemsep2pt
\item Sensor ID 

Unused 

\item Event ID

The upper 2 bytes represent the snort instance, if specified by passing the -G option to Snort.

The lower 2 bytes indicate the unique id of the event.

The Event ID field is used to facilitate the task of coalescing
events with packet data.

\item Event Seconds and Event Microseconds

Timestamp represented as seconds since the epoch of when the alert was generated.

\item Link Type (Unified2 Packet)

The Datalink type of the packet, typically EN10M but could be any of the values as returned by pcap\_datalink that Snort handles.
   
\item Packet Length (Unified2 Packet) 

Length of the Packet Data.
 
\item Packet Data (Unified2 Packet)

The alerting packet, of Packet Length bytes long.

\item Type (Unified2 Extra Data)

Type specifies the type of extra data that was logged, the valid types are:

\begin{verbatim}
    Value           Description 
    ----------      -----------
    1               Original Client IPv4 
    2               Original Client IPv6
    3               UNUSED
    4               GZIP Decompressed Data
    5               SMTP Filename
    6               SMTP Mail From
    7               SMTP RCPT To
    8               SMTP Email Headers
    9               HTTP URI
    10              HTTP Hostname
    11              IPv6 Source Address
    12              IPv6 Destination Address
    13              Normalized Javascript Data
\end{verbatim}
    
\item Data Type (Unified2 Extra Data)

    The type of extra data in the record.

\begin{verbatim}
    Value           Description 
    ----------      -----------
    1               Blob  
\end{verbatim}

\item Data Length (Unified2 Extra Data)

    Length of the data stored in the extra data record

\item Data (Unified2 Extra Data)

    Raw extra event data up to Data Length bytes in size.

    All of these Extra data types, with the exception of 1, 2, 11, and
    12 (IP Addresses) are stored in plain-text. The IP Address types
    need to be interpreted as if they were coming off the wire.

\item Signature ID

    The Signature ID of the alerting rule, as specified by the sid
    keyword.

\item Generator ID

    The Generator ID of the alerting rule, as specified by the gid
    keyword.

\item Signature Revision

    Revision of the rule as specified by the rev keyword.

\item Classification ID

    Classification ID as mapped in the file classifications.conf 

\item Priority ID

    Priority of the rule as mapped in the file classifications.conf or
    overridden by the priority keyword for text rules. 

\item IP Source

    Source IP of the packet that generated the event.

\item IP Destination

    Destination IP of the packet that generated the event.

\item Source Port/ICMP Type

    If Protocol is TCP or UDP than this field contains the source port
    of the alerting packet.

    If Protocol is ICMP than this field contains the ICMP type of the
    alerting packet.

\item Destination Port/ICMP Code

    If protocol is TCP or UDP than this field contains the source port
    of the alerting packet.

    If protocol is icmp than this field contains the icmp code of the
    alerting packet.
  
\item Protocol

    Transport protocol of the alerting packet. One of: ip, tcp, udp, or
    icmp.

\item Impact flag

    Legacy field, specifies whether a packet was dropped or not.

\begin{verbatim}
    Value           Description 
    ----------      -----------
    32              Blocked 
\end{verbatim}

\item Impact

    UNUSED; deprecated.

\item Blocked

    Whether the packet was not dropped, was dropped or would have been
    dropped. 

\begin{verbatim}
    Value           Description 
    ----------      -----------
    0               Was NOT Dropped 
    1               Was Dropped
    2               Would Have Dropped*
\end{verbatim}

\begin{note}
Note that you'll obtain Would Have Dropped on rules which are set to drop
while Snort is running in inline-test mode.  Would Have Dropped is also 
obtained when a drop rule fires while pruning sessions or during shutdown.
\end{note}

\item MPLS Label (4 bytes)

    The extracted mpls label from the mpls header in the alerting
    packet.

\item VLAN ID

    The extracted vlan id from the vlan header in the alerting packet.

\item Padding

    Padding is used to keep the event structures aligned on a 4 byte
    boundary.
\end{itemize}

\section{The Snort Team}

\begin{tabular}{p{3in} p{3in}}

\textbf{Creator and Lead Architect}
& Marty Roesch\\
\\
\textbf{Lead Snort Developers}
& Steve Sturges\\
& Bhagyashree Bantwal\\
& Ed Borgoyn\\
& Hui Cao\\
& Russ Combs\\
& Victor Roemer\\
& Charles Summers\\
& Josh Rosenbaum\\
& Carter Waxman\\
& Tom Peters\\
\\
\textbf{Snort QA Team}
& Chris Spencer\\
& Jigeshwar Patel\\
& Albert Lewis\\
& Nihal Desai\\
\\
\textbf{Vulnerability Research Team}
& Matt Watchinski\\
& Aaron Benson\\
& Nathan Benson\\
& Andrew Blunck\\
& Christoph Cordes\\
& Joel Esler\\
& Douglas Goddard\\
& Ethan Gulla\\
& Nigel Houghton\\
& Pawel Janic\\
& Richard Johnson\\
& Tom Judge\\
& Alex Kambis\\
& Alex Kirk\\
& Chris Marshall\\
& Christopher McBee\\
& Alex McDonnell\\
& Kevin Miklavcic\\
& Steve Morgan\\
& Patrick Mullen\\
& Katie Nolan\\
& Matt Olney\\
& Carlos Pacho\\
& Ryan Pentney\\
& Nick Randolph\\
& Dave Raynor\\
& Marcos Rodriguez\\
& Ryan Steinmetz\\
& Brandon Stultz\\
& Andy Walker\\
& Shawn Webb\\
& Carl Wu\\
& Yves Younan\\
& Alain Zidouemba\\ 
\\
\textbf{Win32 Maintainer}
& Snort Team\\
\\
\textbf{Snort Community Manager}
& Joel Esler\\
\\
\textbf{Snort Web Team}
& Aaron Norling\\
& Mike Verbeck\\
\\
\end{tabular}

\begin{tabular}{p{3in} p{3in}}
\textbf{Major Contributors}
& Erek Adams\\
& Michael Altizer\\
& Ayushi Agarwal\\
& Andrew Baker\\
& Scott Campbell\\
& Brian Caswell\\
& Dilbagh Chahal\\
& JJ Cummings\\
& Scott Czajkowski\\
& Roman D.\\
& Michael Davis\\
& Ron Dempster\\
& Matt Donnan\\
& Chris Green\\
& Lurene Grenier\\
& Mike Guiterman\\
& Jed Haile\\
& Justin Heath\\
& Patrick Harper\\
& Jeremy Hewlett\\
& Ryan Jordan\\
& Victor Julien\\
& Glenn Mansfield Keeni\\
& Adam Keeton\\
& Keith Konecnik\\
& Veronica Kovah\\
& Chad Kreimendahl\\
& Kevin Liu\\
& Rob McMillen\\
& William Metcalf\\
& Andrew Mullican\\
& Jeff Nathan\\
& Marc Norton\\
& Judy Novak\\
& Andreas Ostling\\
& William Parker\\
& Chris Reid\\
& Daniel Roelker\\
& Dragos Ruiu\\
& Chris Sherwin\\
& Matt Smith\\
& Jennifer Steffens\\
& Todd Wease\\
& JP Vossen\\
& Leon Ward\\
& Daniel Wittenberg\\
& Phil Wood\\
& Fyodor Yarochkin\\
\end{tabular}

\appendix

\begin{thebibliography}{99}
\bibitem{phracknmaparticle}http://packetstorm.securify.com/mag/phrack/phrack49/p49-06
\bibitem{nmaphomepage}http://www.nmap.org
\bibitem{cidrnotation}http://public.pacbell.net/dedicated/cidr.html
\bibitem{whitehatsurl}http://www.whitehats.com
\bibitem{snortdb}http://www.incident.org/snortdb
\bibitem{pcre}http://www.pcre.org
\end{thebibliography}

\end{document}

